\apisummary{
    Concatenates blocks of data from multiple \acp{PE} to an array in every
    \ac{PE}.
}

\begin{apidefinition}

%% C11
{\color{Green}
\begin{C11synopsis}
int @\FuncDecl{shmem\_collect32}@(void *dest, const void *source, size_t nelems, shmem_team_t team);
int @\FuncDecl{shmem\_collect64}@(void *dest, const void *source, size_t nelems, shmem_team_t team);
int @\FuncDecl{shmem\_fcollect32}@(void *dest, const void *source, size_t nelems, shmem_team_t team);
int @\FuncDecl{shmem\_fcollect64}@(void *dest, const void *source, size_t nelems, shmem_team_t team);
\end{C11synopsis}
}

\begin{Csynopsis}
\end{Csynopsis}
{\color{Green}
\begin{CsynopsisCol}
int @\FuncDecl{shmem\_team\_collect32}@(void *dest, const void *source, size_t nelems, shmem_team_t team);
int @\FuncDecl{shmem\_team\_collect64}@(void *dest, const void *source, size_t nelems, shmem_team_t team);
int @\FuncDecl{shmem\_team\_fcollect32}@(void *dest, const void *source, size_t nelems, shmem_team_t team);
int @\FuncDecl{shmem\_team\_fcollect64}@(void *dest, const void *source, size_t nelems, shmem_team_t team);
\end{CsynopsisCol}
}
\begin{DeprecateBlock}
\begin{CsynopsisCol}
void @\FuncDecl{shmem\_collect32}@(void *dest, const void *source, size_t nelems, int PE_start, int logPE_stride, int PE_size, long *pSync);
void @\FuncDecl{shmem\_collect64}@(void *dest, const void *source, size_t nelems, int PE_start, int logPE_stride, int PE_size, long *pSync);
void @\FuncDecl{shmem\_fcollect32}@(void *dest, const void *source, size_t nelems, int PE_start, int logPE_stride, int PE_size, long *pSync);
void @\FuncDecl{shmem\_fcollect64}@(void *dest, const void *source, size_t nelems, int PE_start, int logPE_stride, int PE_size, long *pSync);
\end{CsynopsisCol}
\end{DeprecateBlock}

\begin{Fsynopsis}
INTEGER nelems
INTEGER PE_start, logPE_stride, PE_size
INTEGER pSync(SHMEM_COLLECT_SYNC_SIZE)
CALL @\FuncDecl{SHMEM\_COLLECT4}@(dest, source, nelems, PE_start, logPE_stride, PE_size, pSync)
CALL @\FuncDecl{SHMEM\_COLLECT8}@(dest, source, nelems, PE_start, logPE_stride, PE_size, pSync)
CALL @\FuncDecl{SHMEM\_COLLECT32}@(dest, source, nelems, PE_start, logPE_stride, PE_size, pSync)
CALL @\FuncDecl{SHMEM\_COLLECT64}@(dest, source, nelems, PE_start, logPE_stride, PE_size, pSync)
CALL @\FuncDecl{SHMEM\_FCOLLECT4}@(dest, source, nelems, PE_start, logPE_stride, PE_size, pSync)
CALL @\FuncDecl{SHMEM\_FCOLLECT8}@(dest, source, nelems, PE_start, logPE_stride, PE_size, pSync)
CALL @\FuncDecl{SHMEM\_FCOLLECT32}@(dest, source, nelems, PE_start, logPE_stride, PE_size, pSync)
CALL @\FuncDecl{SHMEM\_FCOLLECT64}@(dest, source, nelems, PE_start, logPE_stride, PE_size, pSync)
\end{Fsynopsis}

\begin{apiarguments}

\apiargument{OUT}{dest}{A symmetric array large enough
    to accept the concatenation of the \source{} arrays on all participating \acp{PE}.
    \newtext{See table below in this description for allowable data types.}}
\apiargument{IN}{source}{A symmetric data object that can be of any type permissible
    for the \dest{} argument.}
\apiargument{IN}{nelems}{The number of elements in the \source{} array. \VAR{nelems}
    must be of type \VAR{size\_t} for \Cstd. When using \Fortran, it must be
    a default integer value.}

\newtext{%
\apiargument{IN}{team}{A valid \openshmem team handle to a team which has been
    created without disabling support for collective operations.}
}

\apiargument{IN}{PE\_start}{The lowest \ac{PE} number of the active set of
    \acp{PE}.  \VAR{PE\_start} must be of type integer.  When using \Fortran,
    it must be a default integer value.}
\apiargument{IN}{logPE\_stride}{The log (base \CONST{2}) of the stride between
    consecutive \ac{PE} numbers in the active set. \VAR{logPE\_stride} must be of
    type integer.  When using \Fortran, it must be a default integer value.}
\apiargument{IN}{PE\_size}{The number of \acp{PE} in the active set. \VAR{PE\_size}
    must be of type integer.  When using  \Fortran, it must be a default
    integer value.}
\apiargument{IN}{pSync}{
    A symmetric work array of size \CONST{SHMEM\_COLLECT\_SYNC\_SIZE}.
    In \CorCpp, \VAR{pSync} must be an array of elements of type \CTYPE{long}.
    In \Fortran, \VAR{pSync} must be an array of elements of default integer type.
    Every element of this array must be initialized with the value
    \CONST{SHMEM\_SYNC\_VALUE} before any of the \acp{PE} in the active set
    enter \FUNC{shmem\_collect} or \FUNC{shmem\_fcollect}.}

\end{apiarguments}

\apidescription{
{\color{Green}
    \openshmem \FUNC{collect} and \FUNC{fcollect} routines concatenate \VAR{nelems}
    \CONST{64}-bit or \CONST{32}-bit data items from the \source{} array into the
    \dest{} array, over an \openshmem team or active set
    in processor number order. The resultant \dest{} array contains the contribution from
    \acp{PE} as follows:
    
    \begin{itemize}
    \item For an active set, the data from \ac{PE} \VAR{PE\_start} is first, then the
    contribution from \ac{PE} \VAR{PE\_start} + \VAR{PE\_stride} second, and so on.
    \item For a team, the data from \ac{PE} number \CONST{0} in the team is first, then the
    contribution from \ac{PE} \CONST{1} in the team, and so on.
    \end{itemize}
    
    The collected result is written to the \dest{} array for all \acp{PE}
    that participate in the collective. The same \dest{} and \source{}
    arrays must be passed by all \acp{PE} that participate in the collective.
}
    
    The \FUNC{fcollect} routines require that \VAR{nelems} be the same value in all
    participating \acp{PE}, while the \FUNC{collect} routines allow \VAR{nelems} to
    vary from \ac{PE} to \ac{PE}.

{\color{Green}
    Team-based collect routines operate over all \acp{PE} in the provided team argument. All
    \acp{PE} in the provided team must participate in the collective. If a team created without
    support for collectives is passed to this or any other team collective routine, the
    behavior is undefined.

    Active-set-based broadcast routines operate over all \acp{PE} in the active set
    defined by the \VAR{PE\_start}, \VAR{logPE\_stride}, \VAR{PE\_size} triplet.
    As with all active-set-based collective routines,
    each of these routines assumes that
    only \acp{PE} in the active set call the routine. If a \ac{PE} not in the
    active set and calls this collective routine, the behavior is undefined.
}
    
    The values of arguments \VAR{PE\_start}, \VAR{logPE\_stride}, and \VAR{PE\_size}
    must be the same value on all \acp{PE} in the active set. The same
    \oldtext{\dest{} and \source{} arrays and the same} %%
    \VAR{pSync} work array must be passed by all \acp{PE} in the active set.
    
    Upon return from a collective routine, the following are true for the local
    \ac{PE}:
    \begin{itemize}
    \item The \dest{} array is updated and the \source{} array may be safely reused. 
    \item \newtext{For active-set-based collectives,} the values in the \VAR{pSync} array are
    restored to the original values.
    \end{itemize}
}

{\color{Green}
\apidesctable{
The  \dest{}  and \source{} data  objects must conform to certain typing
constraints, which are as follows:
}{Routine}{Data type of \VAR{dest} and \VAR{source}}
\apitablerow{\FUNC{shmem\_collect8}, \FUNC{shmem\_collect64}, \FUNC{shmem\_fcollect8}, \FUNC{shmem\_fcollect64}}%
    {Any noncharacter type that has an element size of \CONST{64} bits. No \Fortran derived types nor
    \CorCpp{} structures are allowed.}
\apitablerow{\FUNC{shmem\_collect4}, \FUNC{shmem\_collect32}, \FUNC{shmem\_fcollect4}, \FUNC{shmem\_fcollect32}}%
    {Any noncharacter type that has an element size of \CONST{32} bits. No \Fortran derived types nor
    \CorCpp{} structures are allowed.}
}

\apireturnvalues{
    \newtext{Zero on successful local completion. Nonzero otherwise.}
}

\apinotes{
\newtext{%
    There are no specifically defined error codes for these routines.
    See section \ref{subsec:error_handling} for expected error checking and
    return code behavior specific to implementations. For portable
    error checking and debugging behavior, programs should do their own checks
    for invalid team handles or \LibConstRef{SHMEM\_TEAM\_NULL}.
}

    All \openshmem collective routines reset the values in \VAR{pSync} before they
    return, so a particular \VAR{pSync} buffer need only be initialized the first
    time it is used.
    
    The user must ensure that the \VAR{pSync} array is not being updated on any \ac{PE}
    in the active set while any of the \acp{PE} participate in processing of an
    \openshmem collective routine.  Be careful to avoid these situations: If the
    \VAR{pSync} array is initialized at run time, some type of synchronization is
    needed to ensure that all \acp{PE} in the working set have initialized
    \VAR{pSync} before any of them  enter an \openshmem routine called with the
    \VAR{pSync} synchronization array.  A \VAR{pSync} array can be reused on a
    subsequent \openshmem collective routine only if none of the \acp{PE} in the
    active set  are still processing a  prior \openshmem collective routine call
    that used the same \VAR{pSync} array.  In general, this may be ensured only by
    doing some type of synchronization.  
    
    The collective routines operate on active \ac{PE} sets that have a
    non-power-of-two \VAR{PE\_size} with some performance degradation.  They operate
    with no performance degradation when \VAR{nelems} is a non-power-of-two value.
}

\begin{apiexamples}

\apicexample
    {The following \FUNC{shmem\_collect} example is for \CorCpp{} programs:}
    {./example_code/shmem_collect_example.c}
    {}

\apifexample
    {The following \FUNC{SHMEM\_COLLECT} example is for \Fortran programs:}
    {./example_code/shmem_collect_example.f90}
    {}

\end{apiexamples}

\end{apidefinition}
