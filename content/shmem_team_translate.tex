\apisummary{
    Translates a given \ac{PE} number to the corresponding \ac{PE} number in another team.
}

\begin{apidefinition}

\begin{Csynopsis}
int @\FuncDecl{shmem\_team\_translate\_pe}@(shmem_team_t src_team, int src_pe,
    shmem_team_t dest_team);
\end{Csynopsis}

\begin{apiarguments}
\apiargument{IN}{src\_team}{A valid SHMEM team handle.}
\apiargument{IN}{src\_pe}{A \ac{PE} number in src\_team.}
\apiargument{IN}{dest\_team}{A valid SHMEM team handle.}
\end{apiarguments}

\apidescription{
The \FUNC{shmem\_team\_translate\_pe} function will translate a given \ac{PE} number
to the corresponding \ac{PE} number in another team.
Specifically, given the src\_pe in src\_team, this function returns that
\ac{PE}'s number in dest\_team. If src\_pe is not a member of
dest\_team, a value of -1 is returned.

If \CONST{SHMEM\_TEAM\_WORLD} is provided as the dest\_team parameter, this function
acts as a global \ac{PE} number translator and will return the corresponding
\CONST{SHMEM\_TEAM\_WORLD} number. This may be useful when performing point-to-
point operations between \acp{PE} in a subset, as point-to-point operations
that do not take a context argument require the global \CONST{SHMEM\_TEAM\_WORLD}
\ac{PE} number.

Error checking will be done to ensure valid team handles are provided.
All team handle errors are considered fatal and will result in the job
aborting with an informative error message.
}

\apireturnvalues{
The specified \ac{PE}'s number in the dest\_team.
}

\apinotes{
By default, \openshmem creates two predefined teams that will be available
for use once the routine \FUNC{shmem\_init} has been called. These teams can be
referenced in the application by the constants \CONST{SHMEM\_TEAM\_WORLD} and
\CONST{SHMEM\_TEAM\_NODE}. Every \ac{PE}process is a member of the \CONST{SHMEM\_TEAM\_WORLD}
team, and its number in \CONST{SHMEM\_TEAM\_WORLD} corresponds to the value of its
global \ac{PE} number. The \CONST{SHMEM\_TEAM\_NODE} team contains only the set of \acp{PE}
that reside on the same node as the current PE.
}

\end{apidefinition}
