\apisummary{
    Translates a given \ac{PE} number to the corresponding \ac{PE} number in another team.
}

\begin{apidefinition}

\begin{Csynopsis}
int @\FuncDecl{shmem\_team\_translate\_pe}@(shmem_team_t src_team, int src_pe,
    shmem_team_t dest_team);
\end{Csynopsis}

\begin{apiarguments}
\apiargument{IN}{src\_team}{A valid SHMEM team handle.}
\apiargument{IN}{src\_pe}{A \ac{PE} number in src\_team.}
\apiargument{IN}{dest\_team}{A valid SHMEM team handle.}
\end{apiarguments}

\apidescription{
The \FUNC{shmem\_team\_translate\_pe} function will translate a given \ac{PE} number
to the corresponding \ac{PE} number in another team.
Specifically, given the \VAR{src\_pe} in \VAR{src\_team}, this function returns that
\ac{PE}'s number in \VAR{dest\_team}. If \VAR{src\_pe} is not a member of both the
\VAR{src\_team} and \VAR{dest\_team}, a value less than \CONST{0} is returned.

If \LibHandleRef{SHMEM\_TEAM\_WORLD} is provided as the \VAR{dest\_team} parameter, this function
acts as a global \ac{PE} number translator and will return the corresponding
\LibHandleRef{SHMEM\_TEAM\_WORLD} number. This may be useful when performing point-to-
point operations between \acp{PE} in a subset, as point-to-point operations
that do not take a context argument require the global \LibHandleRef{SHMEM\_TEAM\_WORLD}
\ac{PE} number.

Error checking will be done to ensure valid team handles are provided.
Errors will result in a return value less than \CONST{0}.
}

\begin{FeedbackRequest}
\apireturnvalues{
The specified \ac{PE}'s number in the \VAR{dest\_team}, or a value less than \CONST{0} if any
team handle arguments are invalid or the \VAR{src\_pe} is not in both the source and destination teams.
}
\end{FeedbackRequest}

\apinotes{
None.
}

\end{apidefinition}
