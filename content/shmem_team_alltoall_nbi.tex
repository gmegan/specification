\apisummary{
    Initiates a nonblocking transfer of fixed-size blocks of data from each
    \ac{PE} in a team to all \acp{PE} in the team.
}

\begin{apidefinition}

\begin{Csynopsis}
void @\FuncDecl{shmem\_team\_alltoall8\_nbi}@(shmem_team_t team, void *dest, const void *source, size_t nelems);
void @\FuncDecl{shmem\_team\_alltoall16\_nbi}@(shmem_team_t team, void *dest, const void *source, size_t nelems);
void @\FuncDecl{shmem\_team\_alltoall32\_nbi}@(shmem_team_t team, void *dest, const void *source, size_t nelems);
void @\FuncDecl{shmem\_team\_alltoall64\_nbi}@(shmem_team_t team, void *dest, const void *source, size_t nelems);
\end{Csynopsis}

\begin{apiarguments}

\apiargument{IN}{team}{A valid \openshmem team handle to a team which has been
    created without disabling support for collective operations.}
\apiargument{OUT}{dest}{A symmetric data object large enough to receive the
    combined total of \VAR{nelems} elements from each \ac{PE} in the team.}
\apiargument{IN}{source}{A symmetric data object that contains \VAR{nelems}
    elements of data for each \ac{PE} in the team, ordered according to
    destination \ac{PE}.}
\apiargument{IN}{nelems}{The number of elements to transfer to each \ac{PE}.}

\end{apiarguments}

\apidescription{
    The \FUNC{shmem\_team\_alltoall\_nbi} routines are nonblocking collective
    data-exchange operations initiated by \acp{PE} in a team.
    When the call to \FUNC{shmem\_team\_alltoall\_nbi} returns, the operation
    is considered active.
    The operation is ensured to be complete after a subsequent call to
    \FUNC{shmem\_team\_sync}.

    When the operation is complete, each \ac{PE} in the team will have
    transferred \VAR{nelems} data elements to all \acp{PE} in the team.
    The data are sent from and stored to the contiguous symmetric data
    objects \source{} and \dest{}, respectively.

    Both \source{} and \dest{} objects contain \VAR{N} blocks of data,
    where \VAR{N} is the size of the team, each block contains \VAR{nelems}
    elements of size \VAR{M} bits, and \VAR{M} corresponds to the numeric
    suffix in the \FUNC{shmem\_team\_alltoall\_nbi} routine name
    (e.g., \FUNC{shmem\_team\_alltoall64\_nbi} transfers 64-bit elements;
    that is, $M = 64$).
    The total size of each \ac{PE}'s \source{} object and \dest{} object
    is $nelems \cdot N \cdot M$.
    Given a \ac{PE} \VAR{i} that is the \kth PE in the team and a \ac{PE}
    \VAR{j} that is the \lth \ac{PE} in the team,
    \ac{PE} \VAR{i} sends the \lth block of its \source{} object to
    the \kth block of the \dest{} object of \ac{PE} \VAR{j}.

    The values of arguments \VAR{nelems} must be equal on all \acp{PE}
    in the team.
    The same \dest{} and \source{} data objects must be passed by
    all \acp{PE} in the team.

    Upon entry to a \FUNC{shmem\_team\_alltoall\_nbi} routine by any \ac{PE},
    the \source{} and \dest{} data objects on all \acp{PE} in the team must
    be ready to send or receive the data to be transferred, respectively.

    Upon return from a \FUNC{shmem\_team\_alltoall\_nbi} routine,
    the operation is considered initiated (or posted).
    The operation is ensured to be complete after a subsequent call to
    \FUNC{shmem\_team\_sync}.
    Once complete, the \dest{} symmetric data object is completely updated
    and the data has been copied out of the \source{} data object.
}

\apidesctable{The \dest{} and \source{} data objects must conform
  to certain alignment constraints, which are as follows:
}{Routine}{Data type of \dest{} and \source{}}
\apitablerow{\FUNC{shmem\_alltoall8}}{\CONST{8} bits aligned}
\apitablerow{\FUNC{shmem\_alltoall16}}{\CONST{16} bits aligned}
\apitablerow{\FUNC{shmem\_alltoall32}}{\CONST{32} bits aligned}
\apitablerow{\FUNC{shmem\_alltoall64}}{\CONST{64} bits aligned}

\apireturnvalues{
    None.
}

\apinotes{
    None.
}

\end{apidefinition}

