\apisummary{
    Broadcasts a block of data from one \ac{PE} in a team to all other \acp{PE} in the team
}

\begin{apidefinition}

\begin{Csynopsis}
void @\FuncDecl{shmem\_team\_broadcast32}@(shmem_team_t team, void *dest, const void *source, size_t nelems, int PE_root);
void @\FuncDecl{shmem\_team\_broadcast64}@(shmem_team_t team, void *dest, const void *source, size_t nelems, int PE_root);
\end{Csynopsis}

\begin{apiarguments}

\apiargument{IN}{team}{A valid \openshmem team handle to a team which has been created without
    disabling support for collective operations.}
\apiargument{OUT}{dest}{A symmetric data object. See the table below in this description
    for allowable types} 
\apiargument{IN}{source}{A symmetric data object that can be of any data type
    that is permissible for the \dest{} argument.}
\apiargument{IN}{nelems}{The number of elements in \source{}.  For
    \FUNC{shmem\_team\_broadcast32}, this is the number of
    32-bit halfwords. nelems must be of type \CTYPE{size\_t}.}
\apiargument{IN}{PE\_root}{Zero-based ordinal of the \ac{PE}, with respect to
    the team, from which the data is copied. \VAR{PE\_root} must be of type \CTYPE{int}.} 

\end{apiarguments}

\apidescription{   
    \openshmem team broadcast routines are collective routines over an existing team.
    They copy data object \source{} on the processor specified by \VAR{PE\_root}
    and store the values at \dest{} on the other \acp{PE} that are members of the
    team. The data is not copied to the \dest{} area on the root \ac{PE}.
    
    If the team has been created with the \LibConstRef{SHMEM\_TEAM\_NOCOLLECTIVE} option,
    it will not have the required support structures to complete this routine. If
    such a team is passed to this or any other team collective routine, the behavior
    is undefined.

    As with all \openshmem routines where the operation occurs over a given team, \ac{PE}
    numbering is relative to the team. The specified root \ac{PE} must be a valid \ac{PE}
    number for the team, between \CONST{0} and \VAR{N-1}, where \VAR{N} is
    the size of the team.
    
    The values of the argument \VAR{PE\_root} must be the same value on all \acp{PE} in
    the team. The same \dest{} and \source{} data objects must be passed by all \acp{PE}
    in the team.

    Upon return from a broadcast routine, the following are true for the local \ac{PE}:
    \begin{itemize}
    \item If the current \ac{PE} is not the root \ac{PE},
      the \dest{} data object is updated.
    \item The \source{} data object may be safely reused.
    \end{itemize}

    Error checking will be done to detect a value of \LibConstRef{SHMEM\_TEAM\_NULL} passed
    for the team argument. In that case, the program will abort with an informative
    error message. If an invalid team handle is passed to the routine,
    the behavior is undefined.
}

\apidesctable{
The  \dest{}  and \source{} data  objects must conform to certain typing
constraints, which are as follows:
}{Routine}{Data type of \VAR{dest} and \VAR{source}}

\apitablerow{shmem\_team\_broadcast64}{Any noncharacter
    type that has an element size of \CONST{64} bits.
    \CorCpp{} structures are NOT allowed.}
\apitablerow{shmem\_team\_broadcast32}{Any noncharacter
    type that has an element size of \CONST{32} bits.
    \CorCpp{} structures are NOT allowed.}

\apireturnvalues{
    None.
}

\apinotes{
}

\end{apidefinition}
