\apisummary{
    {\color{Green}
    Registers the arrival of a \ac{PE} at a barrier and blocks the \ac{PE}
    until all other \acp{PE} in a given \openshmem team or active set
    arrive at the barrier and all local updates and remote memory updates
    on the specified context or implied default context are completed.
    }
}

\begin{apidefinition}

{\color{Green}
\begin{C11synopsis}
void @\FuncDecl{shmem\_barrier}@(shmem_ctx_t ctx);
\end{C11synopsis}
}

\begin{Csynopsis}
\end{Csynopsis}
{\color{Green}
\begin{CsynopsisCol}
void @\FuncDecl{shmem\_ctx\_barrier}@(shmem_ctx_t ctx);
\end{CsynopsisCol}
}
\begin{DeprecateBlock}
\begin{CsynopsisCol}
void @\FuncDecl{shmem\_barrier}@(int PE_start, int logPE_stride, int PE_size, long *pSync);
\end{CsynopsisCol}
\end{DeprecateBlock}

\begin{Fsynopsis}
INTEGER PE_start, logPE_stride, PE_size
INTEGER pSync(SHMEM_BARRIER_SYNC_SIZE)
CALL @\FuncDecl{SHMEM\_BARRIER}@(PE_start, logPE_stride, PE_size, pSync)
\end{Fsynopsis}

\begin{apiarguments}

\newtext{%
\apiargument{IN}{ctx}{The context associated with the team over which to perform the operation.}%
}

\begin{DeprecateBlock}
\apiargument{IN}{PE\_start}{The lowest \ac{PE} number of the active set of \acp{PE}.
    \VAR{PE\_start} must be of type integer.  When using \Fortran, it must be
    a default integer value.}
\apiargument{IN}{logPE\_stride}{The log (base 2) of the stride between consecutive
    \ac{PE} numbers in the active set.  \VAR{logPE\_stride} must be of type integer.
    When using \Fortran, it must be a default integer value.}
\apiargument{IN}{PE\_size}{The number of  \acp{PE} in the active set.  \VAR{PE\_size}
    must be of type integer.  When using  \Fortran, it must be a default
    integer value.}
\apiargument{IN}{pSync}{
    A symmetric work array of size \CONST{SHMEM\_BARRIER\_SYNC\_SIZE}.
    In \CorCpp, \VAR{pSync} must be an array of elements of type \CTYPE{long}.
    In \Fortran, \VAR{pSync} must be an array of elements of default integer type.
    Every element
    of this array must be initialized to \CONST{SHMEM\_SYNC\_VALUE} before any of
    the \acp{PE} in the active set enter \FUNC{shmem\_barrier} the first time.}
\end{DeprecateBlock}

\end{apiarguments}

\apidescription{
    \FUNC{shmem\_barrier} is a collective synchronization routine over
    \newtext{an existing \openshmem context and its associated team or an active set and the implied default context.}

    \oldtext{Control returns from \FUNC{shmem\_barrier} after all \acp{PE} in
    the active set (specified by \VAR{PE\_start}, \VAR{logPE\_stride}, and
    \VAR{PE\_size}) have called \FUNC{shmem\_barrier}.}

    {\color{Green}
    The routine registers the arrival of a \ac{PE} at a barrier in the program.
    This is a mechanism for synchronizing all \acp{PE} that participate in this
    collective call. The routine blocks the calling \ac{PE} until all \ac{PE}
    participating in the barrier have called \FUNC{shmem\_barrier}.
    In a multithreaded \openshmem program, only the calling thread is blocked.

    Prior to synchronizing with other \acp{PE}, \FUNC{shmem\_barrier} ensures
    that all previously issued stores and remote
    memory updates, including \acp{AMO} and \ac{RMA} operations, issued on a
    context by any of the participating \acp{PE} are complete before returning.
    If no context is provided, as with active-set-based routines, the default
    context is used.

    A team-based barrier operates over all \acp{PE} in the team associated with the
    provided context. This is the team that would be returned by \FUNC{shmem\_ctx\_get\_team(ctx)}.
    All \acp{PE} in that team must participate in the barrier.

    In the case of teams, the call to \FUNC{shmem\_ctx\_barrier(ctx)} is equivalent
    to calling \FUNC{shmem\_ctx\_quiet(ctx)} followed by
    \FUNC{shmem\_team\_sync}(\FUNC{shmem\_ctx\_get\_team(ctx)}).

    If the context handle \VAR{ctx} does not correspond to a valid context, the behavior is
    undefined. If a the team associated with \VAR{ctx} was
    created without support for collectives, the behavior is undefined.
    
    Active-set-based sync routines operate over all \acp{PE} in the active set
    defined by the \VAR{PE\_start}, \VAR{logPE\_stride}, \VAR{PE\_size} triplet.
    A barrier on an active set implies that the default context is used to quiet
    outstanding operations.
    }

    The values of arguments \VAR{PE\_start}, \VAR{logPE\_stride}, and \VAR{PE\_size}
    must be the same value on all \acp{PE} in the active set.  The same work array must be
    passed in \VAR{pSync} to all \acp{PE} in the active set. The same \VAR{pSync}
    array may be reused on consecutive calls to \FUNC{shmem\_barrier}
    if the same active set is used.

    As with all \oldtext{\openshmem} \newtext{active-set-based} collective routines,
    each of these routines assumes that only \acp{PE} in the active set call the routine.
    If a \ac{PE} not  in  the
    active set calls an \openshmem collective routine, the behavior is undefined.
}

\apireturnvalues{
        None.
}

\apinotes{
    If the \VAR{pSync} array is initialized at the run time, all
    \acp{PE} must be synchronized before the first call to \FUNC{shmem\_barrier}
    (e.g., by \FUNC{shmem\_barrier\_all}) to ensure the array has been initialized
    by all \acp{PE} before it is used.
    
    If  the active set does not change, \FUNC{shmem\_barrier} can  be called
    repeatedly with the same \VAR{pSync} array.  No additional synchronization
    beyond that implied by \FUNC{shmem\_barrier} itself is necessary in this case.

    The \FUNC{shmem\_barrier} routine can be used to
    portably ensure that memory access operations observe remote updates in the order
    enforced by initiator \acp{PE}.

    Calls to \FUNC{shmem\_ctx\_quiet} can be performed prior
    to calling the barrier routine to ensure completion of operations issued on
    additional contexts.
}

\begin{apiexamples}

\apicexample
	{The following barrier example is for \Cstd[11] programs:}
	{./example_code/shmem_barrier_example.c}
	{}

\end{apiexamples}

\end{apidefinition}
