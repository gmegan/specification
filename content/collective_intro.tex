\emph{Collective routines} are defined as \newtext{coordinated} communication or synchronization
operations \oldtext{on} \newtext{performed by} a group of \acp{PE} \oldtext{called an active set}.

{\color{Green}
\openshmem provides two types of collective routines:

\begin{enumerate}
\item Collective routines that operate on teams use a team handle parameter to determine
which \acp{PE} will participate in the routine, and use resources encapsulated by the team object
to perform operations. See Section~\ref{subsec:team} for details on team management.
These routines will be the standard for \openshmem moving forward.
\item Collective routines that operate on active sets use a set of parameters to determine
which \acp{PE} will participate and what resources are used to perform operations. These routines
are the legacy API for collectives which will be deprecated and phased out of
implementations moving forward.
\end{enumerate}

Collective routines with no team or active set parameters are deprecated,
and implicitly operate on the team consisting of all \acp{PE} in the computation,
\LibHandleRef{SHMEM\_TEAM\_WORLD}

The team-based collective routines are performed with respect to a valid
\openshmem team, which is specified by a team handle argument.
Team-based collective operations require all \acp{PE} in the team to call
the routine in order for the operation to complete. Team-based collective routines
should not be passed team handles to teams created with a configuration
that disables support for collective operations. If such a team
or \LibConstRef{SHMEM\_TEAM\_NULL} is passed to a team-based collective
routine, the behavior is undefined.

Team objects encapsulate the system resources required to complete team-based collective routines.
On completion of a team-based collective call, the team resources on the calling
\ac{PE} will be ready for the next collective call. However, other \acp{PE} in the
team may still be participating in the collective call, and therefore team
resources may still be in use on some \acp{PE} in the team after others have returned from
the collective routine. Before a subsequent call to a collective routine by the team,
the previous collective operation must be complete on all \acp{PE} in the team,
which can be ensured by a call to a synchronization routine, like \FUNC{shmem\_sync},
by the team.

The team-based collective routines defined in the \openshmem Specification are:

\begin{itemize}
\item \FUNC{shmem\_team\_sync}
\item \FUNC{shmem\_team\_broadcast\{32, 64\}}
\item \FUNC{shmem\_team\_collect\{32, 64\}}
\item \FUNC{shmem\_team\_fcollect\{32, 64\}}
\item Reductions for the following operations: AND, MAX, MIN, SUM, PROD, OR, XOR
\item \FUNC{shmem\_team\_alltoall\{32, 64\}}
\item \FUNC{shmem\_team\_alltoalls\{32, 64\}}
\end{itemize}

The deprecated function \FUNC{shmem\_sync\_all} is provided for backward compatibility to synchronize
all \acp{PE} in the computation. This should be replaced in applications by the equivalent
\FUNC{shmem\_sync(SHMEM\_TEAM\_WORLD)}.
}

\begin{DeprecateBlock}
The \newtext{active-set-based} collective routines require all \acp{PE}
in the active set to simultaneously call the
routine.  A \ac{PE} that is not in the active set calling the collective
routine results in undefined behavior.  \oldtext{All collective routines have an
active set as an input parameter except \FUNC{shmem\_barrier\_all} and
\FUNC{shmem\_sync\_all}. Both \FUNC{shmem\_barrier\_all} and
\FUNC{shmem\_sync\_all} must be called by all \acp{PE} of the \openshmem program.}

The active set is defined by the arguments \VAR{PE\_start}, \VAR{logPE\_stride},
and \VAR{PE\_size}.  \VAR{PE\_start} specifies the starting \ac{PE} number and
is the lowest numbered \ac{PE} in the active set.  The stride between successive
\acp{PE} in the active set is $2^{logPE\_stride}$ and \VAR{logPE\_stride} must
be greater than or equal to zero.  \VAR{PE\_size} specifies the number of
\acp{PE} in the active set and must be greater than zero.  The active set must
satisfy the requirement that its last member corresponds to a valid \ac{PE}
number, that is
$0 \le PE\_start + (PE\_size - 1) * 2^{logPE\_stride} < npes$.

All \acp{PE} participating in the \newtext{active-set-based} collective routine must provide the same
values for these arguments.  If any of these requirements are not met, the
behavior is undefined.

Another argument important to \newtext{active-set-based} collective routines is \VAR{pSync}, which is a
symmetric work array.  All \acp{PE} participating in an \newtext{active-set-based} collective must pass the
same \VAR{pSync} array.  On completion of \newtext{such} a collective call, the \VAR{pSync} is
restored to its original contents.  The user is permitted to reuse a \VAR{pSync}
array if all previous collective routines using the \VAR{pSync} array have been
completed by all participating \acp{PE}. One can use a synchronization
collective routine such as \FUNC{shmem\_barrier} to ensure completion of previous \newtext{active-set-based} collective
routines. The \FUNC{shmem\_barrier} and \FUNC{shmem\_sync} routines allow the same
\VAR{pSync} array to be used on consecutive calls as long as the \acp{PE}
in the active set do not change.

All collective routines defined in the Specification are blocking.  The
collective routines return on completion.  The \newtext{active-set-based} collective
routines defined in the \openshmem Specification are:

\begin{itemize}
\item \FUNC{shmem\_barrier\_all}
\item \FUNC{shmem\_barrier}
\item \FUNC{shmem\_sync\_all}
\item \FUNC{shmem\_sync}
\item \FUNC{shmem\_broadcast\{32, 64\}}
\item \FUNC{shmem\_collect\{32, 64\}}
\item \FUNC{shmem\_fcollect\{32, 64\}}
\item Reductions for the following operations: AND, MAX, MIN, SUM, PROD, OR, XOR
\item \FUNC{shmem\_alltoall\{32, 64\}}
\item \FUNC{shmem\_alltoalls\{32, 64\}}
\end{itemize}

{\color{Green}
The active-set-based \FUNC{shmem\_barrier} and routine has been deprecated and
no team-based barrier routines will be defined. In future, the behavior
previously provided by \FUNC{shmem\_barrier} should be realized by first calling
\FUNC{shmem\_ctx\_quiet} on any relevant communication contexts followed by a call
to \FUNC{shmem\_sync} by some \openshmem team.

Calls to \FUNC{shmem\_barrier\_all}
should be replaced with a call to quiet the default communication context followed
by a call to \FUNC{shmem\_sync} by \LibHandleRef{SHMEM\_TEAM\_WORLD}.
}
\end{DeprecateBlock}
