\apisummary{
    Performs an atomic increment operation on a remote data object.
}

\begin{apidefinition}

\begin{C11synopsis}
void @\FuncDecl{shmem\_atomic\_inc}@(TYPE *dest, int pe);
void @\FuncDecl{shmem\_atomic\_inc}@(shmem_ctx_t ctx, TYPE *dest, int pe);
\end{C11synopsis}
where \TYPE{} is one of the standard \ac{AMO} types specified by
Table~\ref{stdamotypes}.

\begin{Csynopsis}
void @\FuncDecl{shmem\_\FuncParam{TYPENAME}\_atomic\_inc}@(TYPE *dest, int pe);
void @\FuncDecl{shmem\_ctx\_\FuncParam{TYPENAME}\_atomic\_inc}@(shmem_ctx_t ctx, TYPE *dest, int pe);
\end{Csynopsis}
where \TYPE{} is one of the standard \ac{AMO} types and has a corresponding
\TYPENAME{} specified by Table~\ref{stdamotypes}.

\begin{DeprecateBlock}
\begin{C11synopsis}
void @\FuncDecl{shmem\_inc}@(TYPE *dest, int pe);
\end{C11synopsis}
where \TYPE{} is one of \{\CTYPE{int}, \CTYPE{long}, \CTYPE{long long}\}.

\begin{Csynopsis}
void @\FuncDecl{shmem\_\FuncParam{TYPENAME}\_inc}@(TYPE *dest, int pe);
\end{Csynopsis}
where \TYPE{} is one of \{\CTYPE{int}, \CTYPE{long}, \CTYPE{long long}\}
and has a corresponding \TYPENAME{} specified by Table~\ref{stdamotypes}.
\end{DeprecateBlock}

\begin{Fsynopsis}
INTEGER pe
CALL @\FuncDecl{SHMEM\_INT4\_INC}@(dest, pe)
CALL @\FuncDecl{SHMEM\_INT8\_INC}@(dest, pe)
\end{Fsynopsis}

\begin{apiarguments}

\apiargument{IN}{ctx}{\oldtext{The context on which to perform the operation.} \newtext{A context handle specifying the context on which to perform the operation.}
    When this argument is not provided, the operation is performed on
    \oldtext{\CONST{SHMEM\_CTX\_DEFAULT}} \newtext{the default context}.}
\apiargument{OUT}{dest}{The remotely accessible integer data object to be updated
    on the remote \ac{PE}. The type of \dest{} should match that implied in the
    SYNOPSIS section.}
\apiargument{IN}{pe}{An integer that indicates the \ac{PE} number on which
    \dest{} is to be  updated. When using \Fortran, it must be a default
    integer value.}

\end{apiarguments}

\apidescription{
    These  routines perform  an atomic increment operation on the \VAR{dest} data
    object on \ac{PE}.
    \newtext{
    If the context handle \VAR{ctx} does not correspond to a valid context,
    the behavior is undefined.
    }
}


\apidesctable{
    When using \Fortran, \VAR{dest} must be of the following type:
}{Routine}{Data type of \VAR{dest}}

\apitablerow{SHMEM\_INT4\_INC}{\CONST{4}-byte integer}
\apitablerow{SHMEM\_INT8\_INC}{\CONST{8}-byte integer}

\apireturnvalues{
    None.
}

\apinotes{
    None.
}

\begin{apiexamples}

\apicexample
    { The following \FUNC{shmem\_atomic\_inc} example is for
    \Cstd[11] programs: }
    {./example_code/shmem_atomic_inc_example.c}
    {}

\end{apiexamples}

\end{apidefinition}
