\apisummary{
    shmem\_alltoalls is a collective routine where each \ac{PE} exchanges a fixed amount of strided data with all other \acp{PE} \oldtext{in the active set} \newtext{participating in the collective}.
}

\begin{apidefinition}

%% C11
{\color{Green}
\begin{C11synopsis}
int @\FuncDecl{shmem\_alltoall32}@(void *dest, const void *source, size_t nelems, shmem_team_t team);
int @\FuncDecl{shmem\_alltoall64}@(void *dest, const void *source, size_t nelems, shmem_team_t team);
\end{C11synopsis}
}

\begin{Csynopsis}
\end{Csynopsis}
{\color{Green}
\begin{CsynopsisCol}
int @\FuncDecl{shmem\_team\_alltoall32}@(void *dest, const void *source, size_t nelems, shmem_team_t team);
int @\FuncDecl{shmem\_team\_alltoall64}@(void *dest, const void *source, size_t nelems, shmem_team_t team);
\end{CsynopsisCol}
}

\begin{DeprecateBlock}
\begin{CsynopsisCol}
void @\FuncDecl{shmem\_alltoalls32}@(void *dest, const void *source, ptrdiff_t dst, ptrdiff_t sst, size_t nelems, int PE_start, int logPE_stride, int PE_size, long *pSync);
void @\FuncDecl{shmem\_alltoalls64}@(void *dest, const void *source, ptrdiff_t dst, ptrdiff_t sst, size_t nelems, int PE_start, int logPE_stride, int PE_size, long *pSync);
\end{CsynopsisCol}
\end{DeprecateBlock}

\begin{Fsynopsis}
INTEGER pSync(SHMEM_ALLTOALLS_SYNC_SIZE)
INTEGER dst, sst, PE_start, logPE_stride, PE_size
INTEGER nelems 
CALL @\FuncDecl{SHMEM\_ALLTOALLS32}@(dest, source, dst, sst, nelems, PE_start, logPE_stride, PE_size, pSync)
CALL @\FuncDecl{SHMEM\_ALLTOALLS64}@(dest, source, dst, sst, nelems, PE_start, logPE_stride, PE_size, pSync)
\end{Fsynopsis}

\begin{apiarguments}

\apiargument{OUT}{dest}{A symmetric data object large enough to receive 
    the combined total of \VAR{nelems} elements from each \ac{PE} in the
    active set.}
\apiargument{IN}{source}{A symmetric data object that contains \VAR{nelems} 
    elements of data for each \ac{PE} in the active set, ordered according to 
    destination \ac{PE}.}
\apiargument{IN}{dst}{The stride between consecutive elements of the \dest{}
    data object.  The stride is scaled by the element size.  A
    value of \CONST{1} indicates contiguous data.  \VAR{dst} must be of type
    \CTYPE{ptrdiff\_t}.  When using \Fortran, it must be a default integer
    value.}
\apiargument{IN}{sst}{The  stride between consecutive elements of the
    \source{} data object.  The stride is scaled by the element size.
    A value of \CONST{1} indicates contiguous data.  \VAR{sst} must be
    of type \CTYPE{ptrdiff\_t}.  When using \Fortran, it must be a
    default integer value.}

\newtext{%
\apiargument{IN}{team}{A valid \openshmem team handle to a team which has been
    created without disabling support for collective operations.}
}

\begin{DeprecateBlock}
\apiargument{IN}{nelems}{The number of elements to exchange for each \ac{PE}.
    \VAR{nelems} must be of type size\_t for \CorCpp.  When using
    \Fortran, it must be a default integer value.}
\apiargument{IN}{PE\_start}{The lowest \ac{PE} number of the active set of
    \acp{PE}.  \VAR{PE\_start} must be of type integer.  When using \Fortran,
    it must be a default integer value.}
\apiargument{IN}{logPE\_stride}{The log (base 2) of the stride between
    consecutive \ac{PE} numbers in the active set.  \VAR{logPE\_stride} must be of
    type integer.  When using \Fortran, it must be a default integer value.}
\apiargument{IN}{PE\_size}{The number of \acp{PE} in the active set.
    \VAR{PE\_size} must be of type integer.  When using \Fortran, it must
    be a default integer value.}
\apiargument{IN}{pSync}{
    A symmetric work array of size \CONST{SHMEM\_ALLTOALLS\_SYNC\_SIZE}.
    In \CorCpp, \VAR{pSync} must be an array of elements of type \CTYPE{long}.
    In \Fortran, \VAR{pSync} must be an array of elements of default integer type.
    Every element of this array must be initialized with the value
    \CONST{SHMEM\_SYNC\_VALUE} before any of the \acp{PE} in the active set
    enter the routine.}
\end{DeprecateBlock}

\end{apiarguments}

\apidescription{
    The \FUNC{shmem\_alltoalls} routines are collective routines.
    \newtext{These routines are equivalent in functionality to the corresponding
    \FUNC{shmem\_alltoall} routines except that they add explicit stride values
    for accessing the source and destination data arrays, whereas the array
    access in \FUNC{shmem\_alltoall} is always with a stride of \CONST{1}.}

    Each \ac{PE} \oldtext{in the active set} \newtext{participating in the operation}
    exchanges \VAR{nelems} strided data elements of size
    32 bits (for \FUNC{shmem\_alltoalls32}) or 64 bits (for \FUNC{shmem\_alltoalls64})
    with all other \acp{PE} \oldtext{in the set} \newtext{participating in the operation}.
    Both strides, \VAR{dst} and \VAR{sst}, must be greater
    than or equal to \CONST{1}.

    \newtext{The same \dest{} and \source{} arrays and same values for values of
    arguments \VAR{dst}, \VAR{sst}, \VAR{nelems} must be passed by all \acp{PE}
    that participate in the collective.}
    
    Given a \ac{PE} \VAR{i} that is the \kth \ac{PE} \oldtext{in the active set}
    \newtext{participating in the operation} and a \ac{PE}
    \VAR{j} that is the \lth \ac{PE} \oldtext{in the active set}
    \newtext{participating in the operation}
    \ac{PE} \VAR{i} sends the \VAR{sst}*\lth block of the \VAR{source} data object to
    the \VAR{dst}*\kth block of the \VAR{dest} data object on
    \ac{PE} \VAR{j}.

{\color{Green}
    See the description of \FUNC{shmem\_alltoall} in section
    \ref{subsec:shmem_alltoall} for:
    \begin{itemize}
    \item Rules for \ac{PE} participation in the collective routine.
    \item The pre- and post-conditions for symmetric objects.
    \item Typing constraints for \dest{} and \source{} data objects.
    \end{itemize}
}
    
} 


\apireturnvalues{
   \newtext{Zero on successful local completion. Nonzero otherwise.}
}

\apinotes{
    \newtext{See notes for \FUNC{shmem\_alltoall} in section \ref{subsec:shmem_alltoall}}.
}

\begin{apiexamples}

\apicexample
    {This example shows a \FUNC{shmem\_alltoalls64} on two long elements among
    all \acp{PE}.}
    {./example_code/shmem_alltoalls_example.c}
    {}

\end{apiexamples}

\end{apidefinition}
