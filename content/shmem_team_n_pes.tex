\apisummary{
    Returns the total number of \acp{PE} in the provided team.
}

\begin{apidefinition}

\begin{Csynopsis}
int @\FuncDecl{shmem\_team\_n\_pes}@(shmem_team_t team);
\end{Csynopsis}

\begin{apiarguments}
\apiargument{IN}{team}{A valid SHMEM team handle.}
\end{apiarguments}

\apidescription{
The \FUNC{shmem\_team\_n\_pes} function returns the number of \acp{PE} in the
team. This will always be a value between 1 and N, where N is the total number of
\acp{PE} accessible to the \openshmem program. For the team \CONST{SHMEM\_TEAM\_WORLD},
this will return the same value as \FUNC{shmem\_n\_pes}.

Every team must have a least one member. All \acp{PE} in the team
will get back the same value for the team size.

Error checking will be done to ensure a valid team handle is provided.
All errors are considered fatal and will result in the job aborting
with an informative error message.
}

\apireturnvalues{
Total number of \acp{PE} in the provided team.
}

\apinotes{
By default, \openshmem creates two predefined teams that will be available
for use once the routine \FUNC{shmem\_init} has been called. These teams can be
referenced in the application by the constants \CONST{SHMEM\_TEAM\_WORLD} and
\CONST{SHMEM\_TEAM\_NODE}. Every \ac{PE}process is a member of the \CONST{SHMEM\_TEAM\_WORLD}
team, and its number in \CONST{SHMEM\_TEAM\_WORLD} corresponds to the value of its
global \ac{PE} number. The \CONST{SHMEM\_TEAM\_NODE} team contains only the set of \acp{PE}
that reside on the same node as the current PE.
}

\end{apidefinition}
