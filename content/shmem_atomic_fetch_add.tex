\apisummary{
    Performs an atomic fetch-and-add operation on a remote data object.
}

\begin{apidefinition}

\begin{C11synopsis}
TYPE @\FuncDecl{shmem\_atomic\_fetch\_add}@(TYPE *dest, TYPE value, int pe);
TYPE @\FuncDecl{shmem\_atomic\_fetch\_add}@(shmem_ctx_t ctx, TYPE *dest, TYPE value, int pe);
\end{C11synopsis}
where \TYPE{} is one of the standard \ac{AMO} types specified by
Table~\ref{stdamotypes}.

\begin{Csynopsis}
TYPE @\FuncDecl{shmem\_\FuncParam{TYPENAME}\_atomic\_fetch\_add}@(TYPE *dest, TYPE value, int pe);
TYPE @\FuncDecl{shmem\_ctx\_\FuncParam{TYPENAME}\_atomic\_fetch\_add}@(shmem_ctx_t ctx, TYPE *dest, TYPE value, int pe);
\end{Csynopsis}
where \TYPE{} is one of the standard \ac{AMO} types and has a corresponding
\TYPENAME{} specified by Table~\ref{stdamotypes}.

\begin{DeprecateBlock}
\begin{C11synopsis}
TYPE @\FuncDecl{shmem\_fadd}@(TYPE *dest, TYPE value, int pe);
\end{C11synopsis}
where \TYPE{} is one of \{\CTYPE{int}, \CTYPE{long}, \CTYPE{long long}\}.

\begin{Csynopsis}
TYPE @\FuncDecl{shmem\_\FuncParam{TYPENAME}\_fadd}@(TYPE *dest, TYPE value, int pe);
\end{Csynopsis}
where \TYPE{} is one of \{\CTYPE{int}, \CTYPE{long}, \CTYPE{long long}\}
and has a corresponding \TYPENAME{} specified by Table~\ref{stdamotypes}.
\end{DeprecateBlock}

\begin{Fsynopsis}
INTEGER pe
INTEGER*4 SHMEM_INT4_FADD, ires_i4, value_i4
ires\_i4 = @\FuncDecl{SHMEM\_INT4\_FADD}@(dest, value_i4, pe)
INTEGER*8 SHMEM_INT8_FADD, ires_i8, value_i8
ires\_i8 = @\FuncDecl{SHMEM\_INT8\_FADD}@(dest, value_i8, pe)
\end{Fsynopsis}

\begin{apiarguments}

\apiargument{IN}{ctx}{A context handle specifying the context on which to perform the operation.
    When this argument is not provided, the operation is performed on
    the default context.}
\apiargument{OUT}{dest}{The remotely accessible integer data object to be updated on
    the remote \ac{PE}. The type of \VAR{dest} should match that implied in the
    SYNOPSIS section.}
\apiargument{IN}{value}{The value to be atomically added to \VAR{dest}.  The
    type of \VAR{value} should match that implied in the SYNOPSIS section.}
\apiargument{IN}{pe}{An integer that indicates the \ac{PE} number on which
    \VAR{dest} is to be updated.  When using \Fortran, it must be a default
    integer value.}

\end{apiarguments}

\apidescription{
    \FUNC{shmem\_atomic\_fetch\_add} routines perform an atomic fetch-and-add operation.  An
    atomic fetch-and-add operation fetches the old \VAR{dest} and adds \VAR{value}
    to \VAR{dest} without the possibility of another atomic operation on the
    \VAR{dest} between the time of the fetch and the update.  These routines add
    \VAR{value} to \VAR{dest} on \VAR{pe} and return the previous contents of
    \VAR{dest} as an atomic operation.
    If the context handle \VAR{ctx} does not correspond to a valid context,
    the behavior is undefined.
}

\apidesctable{
    When using \Fortran, \VAR{dest} and \VAR{value} must be of the following type:
}{Routine}{Data type of \VAR{dest} and \VAR{value}}

\apitablerow{SHMEM\_INT4\_FADD}{\CONST{4}-byte integer}
\apitablerow{SHMEM\_INT8\_FADD}{\CONST{8}-byte integer}


\apireturnvalues{
    The contents that had been at the \VAR{dest} address on the remote \ac{PE}
    prior to the atomic addition operation.  The data type of the return value is
    the same as the \VAR{dest}.
}

\apinotes{
    None.
}

\begin{apiexamples}

\apicexample
        {The following \FUNC{shmem\_atomic\_fetch\_add} example is for
        \Cstd[11] programs:}
        {./example_code/shmem_atomic_fetch_add_example.c}
        {}

\end{apiexamples}

\end{apidefinition}
