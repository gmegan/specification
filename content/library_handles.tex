\TableIndex{Library Handles}
\TableIndex{Handles}

The \openshmem library provides a set of predefined named constant handles.
All named constants can be used in initialization expressions or assignments,
but not necessarily in array declarations or as labels in \Cstd switch statements.
This implies named constants to be link-time but not necessarily compile-time
constants.

\begin{longtable}{|p{0.45\textwidth}|p{0.5\textwidth}|}
\hline
\textbf{Handle} & \textbf{Description}
\tabularnewline \hline
\endhead
%%
\color{Green}
\LibHandleDecl{SHMEM\_TEAM\_WORLD} &
\color{Green}
Handle of type \CTYPE{shmem\_team\_t} that corresponds to the
default team of all \acp{PE} in the \openshmem program.  All point-to-point
communication operations and synchronizations that do not specify a team
are performed on the default team.
See Section~\ref{subsec:team} for more detail about its use.
\tabularnewline \hline
%%
\color{Green}
\LibHandleDecl{SHMEM\_TEAM\_NODE} &
\color{Green}
Handle of type \CTYPE{shmem\_team\_t} that corresponds a team of \acp{PE}
which share node level resources, such as shared memory, network
interfaces, etc. When this handle is used by some \ac{PE}, it will refer
to the node level team containing that \ac{PE}.
See Section~\ref{subsec:team} for more detail about its use.
\tabularnewline \hline
%%
\LibHandleDecl{SHMEM\_CTX\_DEFAULT} &
Handle of type \CTYPE{shmem\_ctx\_t} that corresponds to the
default communication context.  All point-to-point communication operations
and synchronizations that do not specify a context are performed on the
default context.
See Section~\ref{sec:ctx} for more detail about its use.
\tabularnewline \hline
%%
\end{longtable}
