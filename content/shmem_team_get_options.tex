\apisummary{
    Return the options flags describing the options applied to a given team
}

\begin{apidefinition}

\begin{Csynopsis}
long @\FuncDecl{shmem\_team\_get\_options}@(shmem_team_t team);
\end{Csynopsis}

\begin{apiarguments}
\apiargument{IN}{team}{A valid \openshmem team handle.}
\end{apiarguments}

\apidescription{
\FUNC{shmem\_team\_get\_options} returns a \CTYPE{long} value containing
all of the options which describe the given team. Options are requested when
new teams are created in the various \FUNC{shmem\_team\_split\_*} functions.
All of the requested options applied to the team by the library
implementation will be returned by \FUNC{shmem\_team\_get\_options}.

A library implementation will not apply any non-default options to a team,
other than those requested during team split functions.

\begin{FeedbackRequest}
A library implementation must apply all requested options to a team, even in
the event that the library does not make optimizations based on these options.
For example, suppose library implementation must always create teams with the same
overhead, no matter if the program disables collective support during team creation.
The library must still enable the \LibConstRef{SHMEM\_TEAM\_NOCOLLECTIVE} option
when it is requested, so that the \openshmem program will be portable across implementations.
\end{FeedbackRequest}

All \acp{PE} in the team will get back the same value for the team options.

Error checking will be done to ensure a valid team handle is provided.
Errors will result in a return value less than \CONST{0}.
}

\apireturnvalues{
The set of options applied to the given team. Multiple options are combined
with a bitwise OR and can be extracted with a bitwise AND. A return value of
\CONST{0} implies that the team uses only default options. A return value less than
\CONST{0} implies that the team handle is invalid.
}

\apinotes{
A use case for this function is to determine whether a given team can
support collective operations by testing for the \LibConstRef{SHMEM\_TEAM\_NOCOLLECTIVE}
option. When teams are created without support for collectives, they may still use
point to point operations to communicate and synchronize. So programmers may wish
to design frameworks with functions that provide alternative algorithms
for teams based on whether they do or do not support collectives.
}

\end{apidefinition}
