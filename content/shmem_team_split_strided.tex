\apisummary{
Create a new \openshmem team from a subset of the existing parent team \acp{PE},
where the subset is defined by the
\ac{PE} triplet (\VAR{PE\_start}, \VAR{PE\_stride}, and \VAR{PE\_size}) supplied to the function.}

\begin{apidefinition}

\begin{Csynopsis}
void @\FuncDecl{shmem\_team\_split\_strided}@(shmem_team_t parent_team, long options,
int PE_start, int PE_stride, int PE_size, shmem_team_t *new_team);
\end{Csynopsis}

\begin{apiarguments}
\apiargument{IN}{parent\_team}{A valid SHMEM team. The predefined teams
\LibHandleRef{SHMEM\_TEAM\_WORLD} or \LibHandleRef{SHMEM\_TEAM\_SHARED} may
be used, or any team created by the user.}

\apiargument{IN}{options}{The set of options requested for the new team.
Multiple options may be requested by combining them with a bitwise OR operation;
otherwise, \CONST{0} can be given if no options are requested.}

\apiargument{IN}{PE\_start}{The lowest \ac{PE} number of the subset of \acp{PE} from
the parent team that will form the new team}

\apiargument{IN}{PE\_stride}{The stride between team \ac{PE}
numbers in the parent team that comprise the subset of \acp{PE} that will form
the new team.}

\apiargument{IN}{PE\_size}{The number of \acp{PE} from the parent team in the subset
of \acp{PE} that will form the new team.}

\apiargument{OUT}{new\_team}{A new \openshmem team handle, representing a \ac{PE}
subset of all the \acp{PE} in the parent team that is created from
the \ac{PE} triplet provided.}

\end{apiarguments}

\apidescription{
The \FUNC{shmem\_team\_split\_strided} function is a collective routine.
It creates a new \openshmem team from a subset of the existing parent team,
where the subset is defined by the \ac{PE} triplet (\VAR{PE\_start},
\VAR{PE\_stride}, and \VAR{PE\_size}) supplied to the function.

It is important to note the use of the less restrictive
\VAR{PE\_stride} argument instead of \VAR{logPE\_stride}. This method of
creating a team with an arbitrary set of \acp{PE} is inherently restricted by
its parameters, but allows for many additional use-cases over using a
\VAR{logPE\_stride} parameter, and may provide an easier transition for
existing \openshmem programs to create and use \openshmem teams.

This function must be called by all processes contained in the \ac{PE} triplet
specification. It may be called by additional \acp{PE} not included in the
triplet specification, but for those processes a \VAR{new\_team} value of
\LibConstRef{SHMEM\_TEAM\_NULL} is returned. All calling processes must provide the
same values for the \ac{PE} triplet. This function will return a \VAR{new\_team}
containing the \ac{PE} subset specified by the triplet, and ordered by the
existing global \ac{PE} number. None of the parameters need to reside in
symmetric memory.

Error checking will be done to ensure a valid \ac{PE} triplet is provided,
and also to determine whether a valid team handle is provided for the
parent team.

\begin{FeedbackRequest}
All errors are considered fatal and will result in the job aborting with
an informative error message.
\end{FeedbackRequest}

The following options can be supplied during team split to restrict
team functions and enable performance optimizations.  When using a given
team, the application must comply with the requirements of all options
set on that team; otherwise, the behavior is undefined.
No options are enabled on \LibHandleRef{SHMEM\_TEAM\_WORLD} or \LibHandleRef{SHMEM\_TEAM\_SHARED}.

    \apitablerow{\LibConstRef{SHMEM\_TEAM\_NOCOLLECTIVE}}{
                 The new team will not be created with the necessary support
                 structures to enable team based collectives.
                 This will typically allow implementations to speed up team creation
                 and reduce \openshmem library footprint for teams with this option.
                 This option will not prevent the new team from using atomics or
                 other non-collective team based operations.}
}

\apireturnvalues{
None.
}

\apinotes{
See the description of team handles and predefined teams at the top of section \ref{subsec:team} for more information about semantics and usage.
}

\begin{apiexamples}

\end{apiexamples}

\end{apidefinition}
