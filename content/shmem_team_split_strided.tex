\apisummary{
Create a new \openshmem team from a subset of the existing parent team \acp{PE},
where the subset is defined by the
\ac{PE} triplet (\VAR{PE\_start}, \VAR{PE\_stride}, and \VAR{PE\_size}) supplied to the function.}

\begin{apidefinition}

\begin{Csynopsis}
void @\FuncDecl{shmem\_team\_split\_strided}@(shmem_team_t parent_team, int PE_start, int PE_stride,
     int PE_size, shmem_team_config_t *config, long config_mask, shmem_team_t *new_team);
\end{Csynopsis}

\begin{apiarguments}
\apiargument{IN}{parent\_team}{A valid SHMEM team. The predefined teams
\LibHandleRef{SHMEM\_TEAM\_WORLD} or \LibHandleRef{SHMEM\_TEAM\_NODE} may
be used, or any team created by the user.}

%% \apiargument{IN}{options}{The set of options requested for the new team.
%% Multiple options may be requested by combining them with a bitwise OR operation;
%% otherwise, \CONST{0} can be given if no options are requested.}

\apiargument{IN}{PE\_start}{The lowest \ac{PE} number of the subset of \acp{PE} from
the parent team that will form the new team}

\apiargument{IN}{PE\_stride}{The stride between team \ac{PE}
numbers in the parent team that comprise the subset of \acp{PE} that will form
the new team.}

\apiargument{IN}{PE\_size}{The number of \acp{PE} from the parent team in the subset
of \acp{PE} that will form the new team.}

\apiargument{INOUT}{config}{
  A pointer to the configuration parameters for the new team.}

\apiargument{IN}{config\_mask}{
  The bitwise mask representing the set of configuration parameters to use
  from \VAR{config}.}

\apiargument{OUT}{new\_team}{A new \openshmem team handle, representing a \ac{PE}
subset of all the \acp{PE} in the parent team that is created from
the \ac{PE} triplet provided.}

\end{apiarguments}

\apidescription{
The \FUNC{shmem\_team\_split\_strided} function is a collective routine.
It creates a new \openshmem team from a subset of the existing parent team,
where the subset is defined by the \ac{PE} triplet (\VAR{PE\_start},
\VAR{PE\_stride}, and \VAR{PE\_size}) supplied to the function.

This function must be called by all processes contained in the \ac{PE} triplet
specification. It may be called by additional \acp{PE} not included in the
triplet specification, but for those processes a \VAR{new\_team} value of
\LibConstRef{SHMEM\_TEAM\_NULL} is returned. All calling processes must provide the
same values for the \ac{PE} triplet. This function will return a \VAR{new\_team}
containing the \ac{PE} subset specified by the triplet, and ordered by the
existing global \ac{PE} number. None of the parameters need to reside in
symmetric memory.

The \VAR{config} argument specifies team configuration parameters, which are
described in Section~\ref{subsec:shmem_team_config_t}.

The \VAR{config\_mask} argument is a bitwise mask representing the set of
configuration parameters to use from \VAR{config}.
A \VAR{config\_mask} value of \CONST{0} indicates that all the field members
of \VAR{config} should be used.
Individual field masks can be combined through a bitwise OR operation
of the following library constants:

{
  \apitablerow{\LibConstRef{SHMEM\_TEAM\_NOCOLLECTIVE}}{
    The team should be created using the value of the
    \VAR{disable\_collectives} member of the configuration parameter
    \VAR{config}.
  }
  \apitablerow{\LibConstRef{SHMEM\_TEAM\_LOCAL\_LIMIT}}{
    The team should be created using the value of the
    \VAR{return\_local\_limit} member of the configuration parameter
    \VAR{config}.
  }
  \apitablerow{\LibConstRef{SHMEM\_TEAM\_NUM\_THREADS}}{
    The team should be created using the value of the
    \VAR{num\_threads} member of the configuration parameter \VAR{config}.
  }
}

Error checking will be done to ensure a valid \ac{PE} triplet is provided,
and also to determine whether a valid team handle is provided for the
parent team.

If \VAR{parent\_team} is equal to \LibConstRef{SHMEM\_TEAM\_NULL}, then
\VAR{new\_team} will be assigned the value \LibConstRef{SHMEM\_TEAM\_NULL}.
Otherwise, if \VAR{parent\_team} is an invalid team handle,
the behavior is undefined.
If \VAR{new\_team} cannot be created, it will be assigned the value
\LibConstRef{SHMEM\_TEAM\_NULL}.
}

\apireturnvalues{
  None.
}

\apinotes{
  It is important to note the use of the less restrictive
  \VAR{PE\_stride} argument instead of \VAR{logPE\_stride}. This method of
  creating a team with an arbitrary set of \acp{PE} is inherently restricted
  by its parameters, but allows for many additional use-cases over using a
  \VAR{logPE\_stride} parameter, and may provide an easier transition for
  existing \openshmem programs to create and use \openshmem teams.

  See the description of team handles and predefined teams at the top of
  Section~\ref{subsec:team} for more information about semantics and usage.
}

\begin{apiexamples}

\end{apiexamples}

\end{apidefinition}
