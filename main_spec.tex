\documentclass[10pt]{book}

\usepackage[letterpaper,top=2.5cm,bottom=2.5cm,left=2.5cm,right=2.5cm]{geometry}
\usepackage[T1]{fontenc}
\usepackage[utf8]{inputenc}
\usepackage{graphicx}
\usepackage{multicol}
\usepackage{multirow}
\usepackage[normalem]{ulem}
\usepackage{float}
\usepackage[usenames,dvipsnames]{color}
\usepackage{amsmath}
\usepackage{amsfonts}
\usepackage[table]{xcolor}
\usepackage{xspace}
\usepackage{xhfill}
\usepackage{fancyhdr}
\usepackage[nolist]{acronym}
\usepackage{listings}
% note sure after here
\usepackage{makeidx}
\usepackage[UKenglish]{isodate}
\usepackage{ifthen}
\usepackage{textcomp}
\usepackage{alltt}
\usepackage{ifpdf}
\ifpdf
\usepackage[pdftex,
            pagebackref=true,
            colorlinks=true,
            linkcolor=blue,
            unicode
           ]{hyperref}
\else
\usepackage[ps2pdf,
            pagebackref=true,
            colorlinks=true,
            linkcolor=blue,
            unicode
           ]{hyperref}
\usepackage{pspicture}
\fi
\usepackage{sectsty}
\usepackage{mathptmx}
\usepackage[scaled=.90]{helvet}
\usepackage{courier}
\usepackage[titles]{tocloft}
\usepackage{prettyref}
\usepackage{mdwlist}
\usepackage{enumitem}
\usepackage{framed}
\usepackage{pbox}
\usepackage{draftcopy}
\usepackage{draftwatermark}
\usepackage{wrapfig}
\usepackage{longtable}
\usepackage{caption}
\usepackage{subcaption}
\usepackage{csquotes}

%\acsetup{plural-ending=(s)}


\makeindex

\newcommand{\minitab}[2][l]{\begin{tabular}{@{}#1@{}}#2\end{tabular}}

\definecolor{ListingBG}{rgb}{0.91,0.91,0.91}
\definecolor{shadecolor}{rgb}{0.92,0.92,0.92}

\hyphenation{Open-SHMEM}

\renewcommand{\chaptername}{Chapter} 
\renewcommand{\appendixname}{Annex} 

% Place some penalty for doing the break
% The penalty for a ``\gb'' should be greater than a \hyphenpenalty.
% \hyphenpenalty is 50 in plain.tex.
\def\gb{\penalty10000\hskip 0pt plus 8em\penalty4800\hskip 0pt plus-8em%
\penalty10000}

% This macro enables that all "_" (underscore) characters in the pfd
% file are searchable, and that cut&paste will copy the "_" as underscore. 
% Without the following macro, the \_ is treated in searches and cut&paste
% as a " " (space character). 
% This macro does not modify the behavior of _ in math or in verbatim 
% environments. In verbatim environments, the "_" is always treated
% as a searchable character.
%
\DeclareRobustCommand{\_}{\texttt{\char`\_}} 
% 

\def\colorswapnt{\colorlet{saved}{.}\color{ForestGreen}}
\def\colorswapot{\colorlet{saved}{.}\color{red}}
\def\prevcolor{\color{saved}}

\newcommand{\newtext}[1]{\textcolor{ForestGreen}{#1}}
\newcommand{\oldtext}[1]{\textcolor{magenta}{\sout{#1}}}
\newcommand{\insertDocVersion}{1.5}
\newcommand{\openshmem}[1][]{%
  {Open\-SHMEM\ifthenelse{\equal{#1}{}}{}{~#1}}\xspace}
\newcommand{\HEADER}[1]{\textit{#1}}
\newcommand{\FUNC}[1]{\textit{#1}}
\newcommand{\CTYPE}[1]{\textit{#1}}
\newcommand{\VAR}[1]{\textit{#1}}
\newcommand{\ENVVAR}[1]{\textit{#1}}
\newcommand{\CONST}[1]{\textit{#1}}
\newcommand{\KEYWORD}[1]{\textit{#1}}
%%
\newcommand{\CorCpp}{\textit{C/C++}\xspace}
\newcommand{\CorCppFor}{\textit{C/C++/Fortran}\xspace}
\newcommand{\Fortran}[1][]{%
  \textit{Fortran\ifthenelse{\equal{#1}{}}{}{~#1}}\xspace}
\newcommand{\Cstd}[1][]{%
  \textit{C\ifthenelse{\equal{#1}{}}{}{#1}}\xspace}
\newcommand{\Cpp}[1][]{%
  \textit{C++\ifthenelse{\equal{#1}{}}{}{#1}}\xspace}
%%
\newcommand{\TYPE}{\emph{TYPE}}
\newcommand{\TYPENAME}{\emph{TYPENAME}}
\newcommand{\SIZE}{\emph{SIZE}}

\newcommand{\source}{\textit{source}}
\newcommand{\dest}{\textit{dest}}
\newcommand{\PUT}{\textit{Put}}
\newcommand{\GET}{\textit{Get}}
\newcommand{\OPR}[1]{\textit{#1}}
\newcommand{\shmemprefix}{\textit{SHMEM\_}}
\newcommand{\shmemprefixLC}{\textit{shmem\_}}
\newcommand{\shmemprefixC}{\textit{\_SHMEM\_}}
\newcommand{\ith}{${\textit{i}^{\text{\tiny th}}}$}
\newcommand{\jth}{${\textit{j}^{\text{\tiny th}}}$}
\newcommand{\kth}{${\textit{k}^{\text{\tiny th}}}$}
\newcommand{\lth}{${\textit{l}^{\text{\tiny th}}}$}

%% Generate indexed reference.
\newcommand{\EnvVarIndex}[1]{\index{#1}}
\newcommand{\FuncIndex}[1]{\index{#1}}
\newcommand{\LibConstIndex}[1]{\index{#1}}
\newcommand{\LibHandleIndex}[1]{\index{#1}}
\newcommand{\TableIndex}[1]{\index{#1}\index{Tables!#1}}
%% Write text and generate reference.
\newcommand{\EnvVarRef}[1]{\ENVVAR{#1}\EnvVarIndex{#1}}
\newcommand{\FuncRef}[1]{\FUNC{#1}\FuncIndex{#1}}
\newcommand{\LibConstRef}[1]{\CONST{#1}\LibConstIndex{#1}}
\newcommand{\LibHandleRef}[1]{\CONST{#1}\LibHandleIndex{#1}}
\newcommand{\TableCaptionRef}[1]{\caption{#1}\TableIndex{#1}}
%% Specialized declaration/creation and generate reference.
\newcommand{\EnvVarDecl}[1]{\EnvVarRef{#1}}
\newcommand{\FuncDecl}[1]{{\ListingsCurrentStyle{#1}}\FuncIndex{#1}}
\newcommand{\FuncParam}[1]{<{\ListingsKeywordStyle{#1}}>}
\newcommand{\LibConstDecl}[2][\CorCpp]{%
  \parbox[t]{5cm}{~\\[-4pt] #1: \\\hspace*{8mm} \LibConstRef{#2} \\~}}
\newcommand{\LibHandleDecl}[2][\CorCpp]{%
  \parbox[t]{0pt}{~\\[-4pt] #1: \\\hspace*{8mm} \LibHandleRef{#2} \\~}}

\begin{acronym}
\acro{RMA}{\emph{Remote Memory Access}}
\acro{RMO}{\emph{Remote Memory Operation}}
\acro{AMO}{\emph{Atomic Memory Operation}}
\acro{PE}{\emph{Processing Element}}
\acrodefplural{PE}[PEs]{\emph{Processing Elements}}
\acro{PGAS}{\emph{Partitioned Global Address Space}}
\acro{API}{\emph{Application Programming Interface}}
\acro{MPI}{\emph{Message Passing Interface}}
\acro{SPMD}{\emph{Single Program Multiple Data}}
\acro{ARL}{Army Research Laboratory}
\acro{AMD}{Advanced Micro Devices}
\acro{MPMD}{\emph{Multiple Program Multiple Data}}
\acro{TCP}{\emph{Transmission Control Protocol}}
\acro{UH}{University of Houston}
\acro{UO}{University of Oregon}
\acro{ORNL}{Oak Ridge National Laboratory}
\acro{LANL}{Los Alamos National Laboratory}
\acro{ESSC}{Extreme Scale Systems Center}
\acro{OSSS}{Open Source Software Solutions}
\acro{SGI}{Silicon Graphics International}
\acro{DoD}{U.S. Department of Defense}
\acro{SBU}{Stonybrook University}
\acro{UTK}{University of Tenneesee at Knoxville}
\acro{HPE}{Hewlett Packard Enterprise}
\end{acronym}


% Grab current listings style for use in environment escape to LaTeX.
% https://tex.stackexchange.com/a/209644
\makeatletter
\newcommand\ListingsCurrentStyle{}
\lst@AddToHook{Output}{\global\let\ListingsCurrentStyle\lst@thestyle}
\lst@AddToHook{OutputOther}{\global\let\ListingsCurrentStyle\lst@thestyle}
\newcommand\ListingsKeywordStyle{}
\lst@AddToHook{Output}{\global\let\ListingsKeywordStyle\lst@keywordstyle}
\lst@AddToHook{OutputOther}{\global\let\ListingsKeywordStyle\lst@keywordstyle}
\makeatother

%
% This is used to put line numbers on plain pages.  Used in draft.tex
%
\makeatletter

\def\withlinenumbers{\relax
  \def\@evenfoot{\hbox to 0pt{\hss\LineNumberRuler\hskip 1.5pc}\hfil}\relax
  \def\@oddfoot{\hfil\hbox to 0pt{\hskip 1.5pc\LineNumberRuler\hss}}}

\def\LineNumberRuler{\vbox to 0pt{\vss\normalsize \baselineskip13.6pt
    \lineskip 1pt \normallineskip 1pt \def\baselinestretch{1}\relax
    \LNR{1}\LNR{2}\LNR{3}\LNR{4}\LNR{5}\LNR{6}\LNR{7}\LNR{8}\LNR{9}
    \LNR{10}\LNR{11}\LNR{12}\LNR{13}\LNR{14}
        \LNR{15}\LNR{16}\LNR{17}\LNR{18}\LNR{19}
    \LNR{20}\LNR{21}\LNR{22}\LNR{23}\LNR{24}
        \LNR{25}\LNR{26}\LNR{27}\LNR{28}\LNR{29}
    \LNR{30}\LNR{31}\LNR{32}\LNR{33}\LNR{34}\LNR{35}
        \LNR{36}\LNR{37}\LNR{38}\LNR{39}
    \LNR{40}\LNR{41}\LNR{42}\LNR{43}\LNR{44}
        \LNR{45}\LNR{46}\LNR{47}\LNR{48}
    \vskip 31pt}}
\def\LNR#1{\hbox to 1pc{\hfil\tiny#1\hfil}}

\def\ps@plainwithlinenumbers{\let\@mkboth\@gobbletwo
     \def\@oddhead{}
     \def\@oddfoot{\hfil\rm\thepage\hfil
       \hbox to 0pt{\hskip 1.5pc\LineNumberRuler\hss}}
     \def\@evenhead{}
     \def\@evenfoot{\hbox to 0pt{\hss
     \LineNumberRuler\hskip 1.5pc}\rm\hfil\thepage\hfil}}

    % Contents is done with \chapter*{Contents}, so we need to turn off the
    % line numbers in this case.  Easiest to look at def

\newwrite\chappages
\immediate\openout\chappages=chappage.txt
\def\writespace{ }

\def\incontents{0}
\newif\ifcontents
\contentsfalse
\def\chapter{\clearpage \ifcontents\else\thispagestyle{plainwithlinenumbers}\fi
        \write\chappages{Chapter \thechapter\writespace - \the\count0}
        \global\@topnum\z@ \@afterindentfalse \secdef\@chapter\@schapter}

\makeatother

%
% End this is used to put line numbers on plain pages.  Used in draft.tex
%

%
% Use Sans Serif font for sections, etc.
%
%
\makeatletter
\def\section{\@startsection {section}{1}{\z@}{-3.5ex plus -1ex minus 
-.2ex}{2.3ex plus .2ex}{\Large\sf}}
\def\subsection{\@startsection{subsection}{2}{\z@}{-3.25ex plus -1ex minus 
-.2ex}{1.5ex plus .2ex}{\large\sf}}
\def\subsubsection{\@startsection{subsubsection}{3}{\z@}{-3.25ex plus 
-1ex minus -.2ex}{1.5ex plus .2ex}{\normalsize\sf\bf}}
\def\paragraph{\@startsection {paragraph}{4}{\z@}{3.25ex plus 1ex minus .2ex}
{-1em}{\normalsize\sf\bf}} % Indent after \paragraph
\makeatother
%
% End use Sans Serif font for sections, etc.  S. Otto
%


%
% This section is for example code listings
%
\definecolor{gray}{rgb}{0.92,0.92,0.92}

\lstset{ % set defaults for languages not otherwise defined
  breakatwhitespace=true,         % sets if automatic breaks should only happen at whitespace
  basicstyle=\ttfamily\footnotesize,
  breaklines=true,                 % sets automatic line breaking
  extendedchars=true,              % lets you use non-ASCII characters; for 8-bits 
                                   % encodings only, does not work with UTF-8
  keepspaces=true,                 % keeps spaces in text, useful for keeping indentation of code 
                                   % (possibly needs columns=flexible)
  morekeywords={*,...},            % if you want to add more keywords to the set
  showspaces=false,                % show spaces everywhere adding particular underscores; 
                                   % it overrides 'showstringspaces'
  showstringspaces=false,          % underline spaces within strings only
  showtabs=false,                  % show tabs within strings adding particular underscores
}

\def\StandardListing {
  \lstset {
    breakatwhitespace=false,         % sets if automatic breaks should only happen at whitespace
    basicstyle=\ttfamily\footnotesize,
    breaklines=true,                 % sets automatic line breaking
    escapeinside={\%*}{*)},          % if you want to add LaTeX within your code
    extendedchars=true,              % lets you use non-ASCII characters; for 8-bits 
                                     % encodings only, does not work with UTF-8
    keepspaces=true,                 % keeps spaces in text, useful for keeping
                                     % indentation of code (possibly needs columns=flexible)
    morekeywords={*,...},            % if you want to add more keywords to the set
    showspaces=false,                % show spaces everywhere adding particular underscores; 
                                     % it overrides 'showstringspaces'
    showstringspaces=false,          % underline spaces within strings only
    showtabs=false,                  % show tabs within strings adding particular underscores
    backgroundcolor=\color{gray}, 
  }
}

\def\ProgramNumberedListing {
  \StandardListing
  \lstset {
    numbers=left,
    numberstyle=\footnotesize
  }
}

\newcommand{\numberedlisting}[2] {
  \ProgramNumberedListing
  \lstinputlisting[#1]{#2}
  \StandardListing
}

\newcommand{\outputlisting}[2] {
\begin{minipage}{\linewidth}
\vspace{0.1in}
  \lstinputlisting[#1]{#2}
  \StandardListing
\vspace{0.1in}
\end{minipage}
}

\lstdefinelanguage{OSH+C}[]{C}{
  classoffset=1,
  morekeywords={
    size_t, ptrdiff_t,
    SHMEM_BCAST_SYNC_SIZE, SHMEM_SYNC_VALUE,
    start_pes,
    my_pe, _my_pe, shmem_my_pe,
    num_pes, _num_pes, shmem_n_pes,
    shmem_int_p, shmem_short_p, shmem_long_p,
    shmem_int_put, shmem_short_put, shmem_long_put,
    shmem_barrier_all, shmem_barrier,
    shmalloc,  shfree, shrealloc,
    shmem_broadcast32, shmem_broadcast64,
    shmem_short_inc, shmem_int_inc, shmem_long_inc,
    shmem_short_add, shmem_int_add, shmem_long_add,
    shmem_short_finc, shmem_int_finc, shmem_long_finc,
    shmem_short_fadd, shmem_int_fadd, shmem_long_fadd,
    shmem_set_lock, shmem_test_lock, shmem_clear_lock,
    shmem_long_sum_to_all,
    shmem_complexd_sum_to_all
  },
  keywordstyle=\color{black}\textbf,
  classoffset=0,
  sensitive=true
}

\lstdefinelanguage{OSH2+C}[]{OSH+C}{
  classoffset=1,
  morekeywords={
    shmem_init,
    shmem_finalize,
    shmem_malloc,
    shmem_my_pe,
    shmem_error,
    shmem_global_exit,
  },
  keywordstyle=\color{black}\textbf,
  classoffset=0,
  sensitive=true
}

\lstdefinelanguage{OSH+F}[]{Fortran}{
  classoffset=1,
  morekeywords={
    SHMEM_BCAST_SYNC_SIZE, SHMEM_SYNC_VALUE,
    start_pes,
    my_pe, shmem_my_pe,
    num_pes, shmem_n_pes,
    shmem_int_p, shmem_short_p, shmem_long_p,
    shmem_int_put, shmem_short_put, shmem_long_put,
    shmem_barrier_all, shmem_barrier,
    shpalloc,  shpdeallc, shpclmove,
    shmem_broadcast32, shmem_broadcast64,
    shmem_broadcast4, shmem_broadcast8,
    shmem_short_inc, shmem_int_inc, shmem_long_inc,
    shmem_short_add, shmem_int_add, shmem_long_add,
    shmem_short_finc, shmem_int_finc, shmem_long_finc,
    shmem_short_fadd, shmem_int_fadd, shmem_long_fadd,
    shmem_set_lock, shmem_test_lock, shmem_clear_lock,
    shmem_long_sum_to_all,
  },
  keywordstyle=\color{black}\textbf,
  classoffset=0,
  sensitive=false
}

\lstdefinelanguage{OSH2+F}[]{OSH+F}{
  classoffset=1,
  morekeywords={
    shmem_init,
    shmem_finalize,
    shmem_malloc,
    shmem_my_pe,
    shmem_error,
    shmem_global_exit,
  },
  keywordstyle=\color{black}\textbf,
  classoffset=0,
  sensitive=true
}

%
% End this section is for example code listings
%

%
% Deprecation Helpers
%

\newcommand{\strikeline}[1][red]{{\color{#1}\raisebox{.5ex}{\rule{1em}{.4pt}}}}
\newcommand{\stretchline}[1][red]{\xrfill[.5ex]{.4pt}[#1]}
\newcommand{\DeprecationStart}[1][red]{{\color{#1} deprecation start} \mbox{}}
\newcommand{\DeprecationEnd}[1][red]{{\color{#1} deprecation end} \mbox{}}

\newcommand{\StartDeprecateBlock}{
  {\strikeline\mbox{} \DeprecationStart \stretchline\mbox{}}}
\newcommand{\EndDeprecateBlock}{%
  \mbox{}\stretchline\mbox{} \DeprecationEnd \strikeline}

\newenvironment{DeprecateBlock}{%
  \par \StartDeprecateBlock \par}{\par \EndDeprecateBlock \par}

\newcommand{\StartInlineDeprecate}{%
  \strikeline\mbox{} \DeprecationStart \strikeline \mbox{}}
\newcommand{\EndInlineDeprecate}{%
  \strikeline\mbox{} \DeprecationEnd \strikeline}
\newenvironment{DeprecateInline}{\StartInlineDeprecate}{\EndInlineDeprecate}

%
% Library API description template commands
%

\newcommand{\deprecationstart}{\color{red} \raisebox{.5ex}{\rule{1em}{.4pt}}
  deprecation start \xrfill[.5ex]{.4pt}[red] \mbox{}}
\newcommand{\deprecationend}{\mbox{}\xrfill[.5ex]{.4pt}[red]\mbox{} \color{red}
  deprecation end \raisebox{.5ex}{\rule{1em}{.4pt}}}

\newenvironment{deprecate}{\deprecationstart \\}{\\ \deprecationend}

\newcommand{\apisummary}[1]{
    #1
\hfill
}

\newenvironment{apidefinition}{
\begin{description}
\item[SYNOPSIS] \hfill \\ \\ 
\vspace{-2em}
}
{
\end{description}
}

\lstnewenvironment{Cpp11synopsis}
{
  \textbf{C++11:}
  \lstset{language={C++}, backgroundcolor=\color{gray}, lineskip=2pt,
    escapechar=@,
  morekeywords={size_t, ptrdiff_t, TYPE, noreturn},
  aboveskip=0pt, belowskip=0pt}}{}

\lstnewenvironment{C11synopsis}
{ 
  \textbf{C11:} 
  \lstset{language={C}, backgroundcolor=\color{gray}, lineskip=2pt,
    escapechar=@,
    morekeywords={size_t, ptrdiff_t, TYPE, _Noreturn, shmem_ctx_t,
      shmem_team_t},
  aboveskip=0pt, belowskip=0pt}}{}

\lstnewenvironment{CsynopsisCol}
{ 
  \lstset{language={C}, backgroundcolor=\color{gray}, lineskip=2pt,
    escapechar=@,
    morekeywords={size_t, ptrdiff_t, TYPE, TYPENAME, SIZE, shmem_ctx_t,
      shmem_team_t},
  aboveskip=0pt, belowskip=0pt}}{}


\lstnewenvironment{Csynopsis}
{ 
  \textbf{C/C++:} 
  \lstset{language={C}, backgroundcolor=\color{gray}, lineskip=2pt,
    escapechar=@,
    morekeywords={size_t, ptrdiff_t, TYPE, TYPENAME, SIZE, shmem_ctx_t,
      shmem_team_t},
  aboveskip=0pt, belowskip=0pt}}{}

\lstnewenvironment{CsynopsisST}
{ 
  \textbf{C/C++:} 
  \color{red}  
  {\lstset{language={C}, backgroundcolor=\color{gray}, lineskip=2pt,
    escapechar=@,
    morekeywords={size_t, ptrdiff_t, TYPE, TYPENAME, SIZE, shmem_ctx_t,
      shmem_team_t},
    aboveskip=0pt, belowskip=0pt}}}{}

\lstnewenvironment{Fsynopsis}
{ \deprecationstart \\
  \textbf{FORTRAN:}
  \lstset{language={Fortran}, backgroundcolor=\color{gray}, lineskip=3pt,
    escapechar=@,
  deletekeywords=[2]{STATUS},
  deletekeywords=[3]{LOG}, aboveskip=0pt,
  belowskip=0pt}}
{ \deprecationend }

\newenvironment{apiarguments}{
\newcommand{\apiargument}[3]{
\begin{tabular}{p{2cm} p{2cm} p{10cm}}
\textbf{##1} & \textit{##2} & {##3} \\ 
\end{tabular}
}
\hfill
\item[DESCRIPTION] \hfill 

\begin{description}
\item[Arguments] \hfill \\
}
{
\hfill
\end{description}
}

\newcommand{\apidescription}[1]{
\begin{description}
\vspace{-1em}
\item[API description] \hfill \\ 
    \begin{sloppypar}
    #1
    \end{sloppypar}
\hfill
}

\newcommand{\apidesctable}[4] {\hfill \\ #1 \\ \\
    \begin{tabular}{p{5cm} p{9cm}}
       \hline
       #2 & #3 \\
       \hline \tabularnewline
       \end{tabular}\\
        #4
}  

\newcommand{\apireturnvalues}[1]{
\hfill 
\item[Return Values] \hfill \\
    #1
\\
\hfill
}

\newcommand{\apitablerow}[2]{
 \begin{tabular}{p{5cm} p{9cm}}
 #1 & #2 \tabularnewline
  \end{tabular}\\
}

\newcommand{\apinotes}[1]{
\item[Notes] \hfill \\
    #1
\hfill \\
\end{description}
}

\newcommand{\apiimpnotes}[1]{
\begin{description}
\item[Note to implementors] \hfill \\
    #1
\hfill \\
\end{description}
}

\newenvironment{apiexamples}{
\newcommand{\apicexample}[3]{
    ##1
    \lstinputlisting[language={C}, tabsize=2,
      basicstyle=\ttfamily\footnotesize,
      morekeywords={size_t, ptrdiff_t, shmem_ctx_t, shmem_team_t}]{##2}
    ##3 }
\newcommand{\apifexample}[3]{
    ##1
    \lstinputlisting[language={Fortran}, tabsize=2,
    basicstyle=\ttfamily\footnotesize, deletekeywords={TARGET}]{##2}
    ##3 }
\vspace{-2pt}
\item[EXAMPLES] \hfill \\
\vspace{-2pt}
}
{
}

%
% End library API description template commands
%


\begin{document}

\input{content/frontmatter}



\section{The OpenSHMEM Effort}\label{subsec:openshmem_effort}
\input{content/the_openshmem_effort}

\section{Programming Model Overview}\label{subsec:programming_model}
\openshmem implements \ac{PGAS} by defining remotely accessible data objects as
mechanisms to share information among \openshmem processes, or \acp{PE}, and
private data objects that are accessible by only the \ac{PE} itself. The \ac{API}
allows communication and synchronization operations on both private (local to
the PE initiating the operation) and remotely accessible data objects. The key
feature of \openshmem is that data transfer operations are
\emph{one-sided} in nature. This means that a local \ac{PE} executing
a data transfer routine does not require the participation of the remote \ac{PE}
to complete the routine. This allows for overlap between communication and
computation to hide data transfer latencies, which makes  \openshmem ideal for
unstructured, small/medium size data communication patterns. The \openshmem
library routines have the potential to provide a low-latency, high-bandwidth
communication \ac{API} for use in highly parallelized scalable programs.

The \openshmem interfaces can be used to implement \ac{SPMD} style programs.
It provides interfaces to start the \openshmem \acp{PE} in parallel and
communication and synchronization interfaces to access remotely accessible data
objects across \acp{PE}. These interfaces can be leveraged to divide a problem
into multiple sub-problems that can be solved independently or with coordination
using the communication and synchronization interfaces.  The \openshmem
specification defines library calls, constants, variables, and language bindings
for \Cstd.
The \Cpp interface is currently the same as that
for \Cstd. Unlike Unified Parallel C, \Fortran[2008], Titanium, X10, and Chapel, which are all
PGAS languages, \openshmem relies on the user to use the library calls  to
implement the correct semantics of its programming model.

An overview of the \openshmem routines is described below:

\begin{enumerate}

\item \textbf{Library Setup and Query}
\begin{enumerate}
  \item \OPR{Initialization}: The \openshmem library environment is initialized,
   where the \acp{PE} are either single or multithreaded.
  \item \OPR{Query}: The local \ac{PE} may get the number of \acp{PE} running
      the same program and its unique integer identifier.
  \item \OPR{Accessibility}: The local \ac{PE} can find out if a remote \ac{PE} is
      executing the same binary, or if a particular symmetric data object can be
      accessed by a remote \ac{PE}, or may obtain a pointer to a symmetric data
      object on the specified remote \ac{PE} on shared memory systems.
\end{enumerate}

\item \textbf{Symmetric Data Object Management}
\begin{enumerate}
  \item \OPR{Allocation}: All executing \acp{PE} must participate in the
      allocation of a symmetric data object with identical arguments.
  \item  \OPR{Deallocation}: All executing \acp{PE} must participate in the
      deallocation of the same symmetric data object with identical arguments.
  \item  \OPR{Reallocation}: All executing \acp{PE} must participate in the
      reallocation of the same symmetric data object with identical arguments.
\end{enumerate}

\item \textbf{Communication Management}
\begin{enumerate}
    \item \OPR{Contexts}: Contexts are containers for communication operations.
        Each context provides an environment where the operations performed on
        that context are ordered and completed independently of other operations
        performed by the application.
\end{enumerate}

\item \textbf{Remote Memory Access}
\begin{enumerate}
    \item \PUT: The local \ac{PE} specifies the \source{} data object, private
        or symmetric, that is copied to the symmetric data object on the remote
        \ac{PE}.
  \item \GET: The local \ac{PE} specifies the symmetric data object on the remote
      \ac{PE} that is copied to a data object, private or symmetric, on the local
      \ac{PE}.
\end{enumerate}

\item \textbf{Atomics}
\begin{enumerate}
    \item \OPR{Swap}: The \ac{PE} initiating the swap gets the old value of a
        symmetric data object from a remote \ac{PE} and copies a new value to
        that symmetric data object on the remote \ac{PE}.
  \item \OPR{Increment}: The \ac{PE} initiating the increment adds 1 to the
      symmetric data object on the remote \ac{PE}.
  \item \OPR{Add}: The \ac{PE} initiating the add specifies the value to be added
      to the symmetric data object on the remote \ac{PE}.
  \item \OPR{Bitwise Operations}: The \ac{PE} initiating the bitwise
      operation specifies the operand value to the bitwise operation to be
      performed on the symmetric data object on the remote \ac{PE}.
  \item \OPR{Compare and Swap}: The \ac{PE} initiating the swap gets the old value
      of the symmetric data object based on a value to be compared and copies a
      new value to the symmetric data object on the remote \ac{PE}.
  \item \OPR{Fetch and Increment}: The \ac{PE} initiating the increment adds 1 to
      the symmetric data object on the remote \ac{PE} and returns with the old
      value.
  \item \OPR{Fetch and Add}: The \ac{PE} initiating the add specifies the value to
      be added to the symmetric data object on the remote \ac{PE} and returns with
      the old value.
  \item \OPR{Fetch and Bitwise Operations}: The \ac{PE} initiating the bitwise
      operation specifies the operand value to the bitwise operation to be
      performed on the symmetric data object on the remote \ac{PE}
      and returns the old value.
\end{enumerate}

\item \textbf{Synchronization and Ordering}
\begin{enumerate}
  \item \OPR{Fence}: The \ac{PE} calling fence ensures ordering of
  \PUT, AMO, and memory store operations
  to symmetric data objects with respect to a specific
      destination \ac{PE}.
  \item \OPR{Quiet}: The \ac{PE} calling quiet ensures remote completion of remote access
      operations and stores to symmetric data objects.
  \item \OPR{Barrier}: All or some \acp{PE} collectively synchronize and ensure
      completion of all remote and local updates prior to any \ac{PE} returning
      from the call.
\end{enumerate}

\item \textbf{Collective Communication}
\begin{enumerate}
  \item \OPR{Broadcast}: The \VAR{root} \ac{PE} specifies a symmetric data
      object to be copied to a symmetric data object on one or more remote
      \acp{PE} (not including itself).
  \item \OPR{Collection}: All \acp{PE} participating in the routine get the result
      of concatenated symmetric objects contributed by each of the \acp{PE} in
      another symmetric data object.
  \item \OPR{Reduction}: All \acp{PE} participating in the routine get the result
      of an associative binary routine over elements of the specified symmetric
      data object on another symmetric data object.
  \item \OPR{All-to-All}: All \acp{PE} participating in the routine exchange
      a fixed amount of contiguous or strided data with all other \acp{PE}
      in the active set.
\end{enumerate}

\item \textbf{Mutual Exclusion}
\begin{enumerate}
  \item \OPR{Set Lock}: The \ac{PE} acquires exclusive access to the region
      bounded by the symmetric \VAR{lock} variable.
  \item \OPR{Test Lock}: The \ac{PE} tests the symmetric \VAR{lock} variable
      for availability.
  \item \OPR{Clear Lock}: The \ac{PE} which has previously acquired the
      \VAR{lock} releases it.
\end{enumerate}

\begin{DeprecateBlock}
\item \textbf{Data Cache Control}
\begin{enumerate}
  \item Implementation of mechanisms to exploit the capabilities of hardware cache
      if available.
\end{enumerate}
\end{DeprecateBlock}

\end{enumerate}


\section{Memory Model}\label{subsec:memory_model}
\begin{figure}[h]
\includegraphics[width=0.95\textwidth]{figures/mem_model}      
\caption{\openshmem Memory Model}
\label{fig:mem_model}                                               
\end{figure}      
%
An \openshmem program consists of data objects that are private to each \ac{PE}
and data  objects that are remotely accessible by all \acp{PE}. Private data
objects are stored in the local memory of each \ac{PE} and can only be accessed
by the \ac{PE} itself; these data objects cannot be accessed by other \acp{PE}
via \openshmem routines. Private data objects follow the memory model of
\Cstd or \Fortran. Remotely accessible objects, however, can be accessed by
remote \acp{PE} using \openshmem routines.  Remotely accessible data objects are
called \emph{Symmetric Data Objects}.  Each symmetric data object has a
corresponding object with the same name, type, and size on all \acp{PE} where that object is
accessible via the \openshmem \ac{API}\footnote{For efficiency reasons,
the same offset (from an arbitrary memory address) for symmetric data
objects might be used on all \acp{PE}. Further discussion about symmetric heap
layout and implementation efficiency can be found in section
\ref{subsec:shfree}}.  (For the definition of what is accessible, see the
descriptions for \FUNC{shmem\_pe\_accessible} and \FUNC{shmem\_addr\_accessible}
in sections \ref{subsec:shmem_pe_accessible} and
\ref{subsec:shmem_addr_accessible}.) Symmetric data objects accessed via typed and
type-generic \openshmem interfaces are required to be naturally aligned based on their type
requirements and underlying architecture.  In \openshmem the following kinds of
data objects are symmetric:
%
\begin{itemize}
\item
  \begin{deprecate}
    \Fortran data objects in common blocks or with the \CTYPE{SAVE} attribute.
    These data objects must not be defined in a dynamic shared object (DSO).
  \end{deprecate}
\item Global and static \Cstd and \Cpp variables. These data objects must
  not  be defined in a DSO.
\item
  \begin{deprecate}
    \Fortran arrays allocated with \FUNC{shpalloc}
  \end{deprecate}
\item \Cstd and \Cpp data allocated by \openshmem memory management routines
  (Section~\ref{sec:memory_management})
\end{itemize}       

\openshmem dynamic memory allocation routines (\FUNC{shpalloc} and
\FUNC{shmem\_malloc}) allow collective allocation of \emph{Symmetric Data
Objects} on a special memory region called the \emph{Symmetric Heap}. The
Symmetric Heap is created during the execution of a program at a memory location
determined by the implementation. The Symmetric Heap may reside in different
memory regions on different \acp{PE}. Figure~\ref{fig:mem_model} shows how
\openshmem implements a \ac{PGAS} model using remotely accessible symmetric
objects and private data objects when executing an \openshmem program.
Symmetric data objects are stored on the symmetric heap or in the global/static
memory section of each \ac{PE}. 

\subsection{Atomicity Guarantees}\label{subsec:amo_guarantees}

\openshmem contains a number of routines that perform atomic operations on
symmetric data objects, which are defined in Section \ref{sec:amo}.
The atomic routines
guarantee that concurrent accesses by any of these routines to the same
location and using the same datatype (specified in Tables~\ref{stdamotypes} and
\ref{extamotypes}) will be exclusive.
\openshmem atomic operations do not guarantee exclusivity in the following
scenarios, all of which result in undefined behavior.
\begin{enumerate}
    \item When concurrent accesses to the same location are performed using
        \openshmem atomic operations using different datatypes.
    \item When atomic and non-atomic \openshmem operations are used to access
        the same location concurrently.
    \item When \openshmem atomic operations and non-\openshmem operations (e.g.
        load and store operations) are used to access the same location
        concurrently.
\end{enumerate}
For example, during the execution of an atomic remote integer increment, i.e. \FUNC{shmem\_atomic\_inc},
operation on a symmetric variable \VAR{X}, no other \openshmem atomic operation
may access \VAR{X}.  After the increment, \VAR{X} will have increased its value
by \CONST{1} on the destination \ac{PE}, at which point other atomic operations
may then modify that \VAR{X}.  However, access to the symmetric object \VAR{X}
with non-atomic operations, such as one-sided \OPR{put} or \OPR{get} operations,
will invalidate the atomicity guarantees.

\cexample
    {The following \CorCpp example illustrates scenario 1.  In this example,
    different datatypes are used to access the same location concurrently,
    resulting in undefined behavior.  The undefined behavior can be resolved by
    using the same datatype in all concurrent operations.  For example, the
    32-bit value can be left-shifted and a 64-bit atomic OR operation can be
    used.}
    {./example_code/amo_scenario_1.c}

\cexample
    {The following \CorCpp example illustrates scenario 2.  In this example,
    atomic increment operations are concurrent with a non-atomic reduction
    operation, resulting in undefined behavior.  The undefined behavior can be
    resolved by inserting a barrier operation before the reduction.  The
    barrier ensures that all local and remote AMOs have completed before the
    reduction operation accesses $x$.}
    {./example_code/amo_scenario_2.c}

\cexample
    {The following \CorCpp example illustrates scenario 3.  In this example, an
    \openshmem atomic increment operation is concurrent with a local increment
    operation, resulting in undefined behavior.  The undefined behavior can be
    resolved by replacing the local increment operation with an \openshmem
    atomic increment.}
    {./example_code/amo_scenario_3.c}


\section{Execution Model}\label{subsec:execution_model}
An \openshmem program consists of a set of \openshmem processes called \acp{PE}
that execute in an \ac{SPMD}-like model where each \ac{PE} can take a different
execution path. For example, a \ac{PE} can be implemented using an OS
process. The \acp{PE} may be either single or multithreaded.
The \acp{PE} progress asynchronously, and can communicate/synchronize
via the \openshmem interfaces.  All \acp{PE} in an \openshmem program should
start by calling the initialization routine \FUNC{shmem\_init}%
\footnote{\FUNC{start\_pes} has been deprecated as of \openshmem[1.2]}
or \FUNC{shmem\_init\_thread} before using any of the other \openshmem library routines.
An \openshmem program concludes its use of the \openshmem library when all \acp{PE} call
\FUNC{shmem\_finalize} or any \ac{PE} calls \FUNC{shmem\_global\_exit}.
During a call to \FUNC{shmem\_finalize}, the \openshmem library must
complete all pending communication and release all the resources associated to
the library using an implicit collective synchronization across \acp{PE}.
Calling any \openshmem routine after \FUNC{shmem\_finalize} leads to undefined
behavior.

The \acp{PE} of the \openshmem program are identified by unique integers.  The
identifiers are integers assigned in a monotonically increasing manner from zero
to one less than the total number of \acp{PE}. \ac{PE} identifiers are used for
\openshmem calls (e.g. to specify \OPR{put} or \OPR{get} routines on symmetric
data objects, collective synchronization calls) or to dictate a control flow for
\acp{PE} using constructs of \Cstd. The identifiers are fixed for
the life of the \openshmem program.

\subsection{Progress of OpenSHMEM Operations}\label{subsec:progress}

The \openshmem model assumes that computation and communication are naturally
overlapped. \openshmem programs are expected to exhibit progression of
communication both with and without \openshmem calls. Consider a \ac{PE} that is
engaged in a computation with no \openshmem calls. Other \acp{PE} should be able
to communicate (\OPR{put}, \OPR{get}, \OPR{atomic}, etc) and
complete communication operations with that computationally-bound \ac{PE}
without that \ac{PE} issuing any explicit \openshmem calls. One-sided \openshmem
communication calls involving that \ac{PE} should progress regardless of when
that \ac{PE} next engages in an \openshmem call.

\textbf{Note to implementors:}
\begin{itemize}
  \item An \openshmem implementation for hardware that does not provide
      asynchronous communication capabilities may require a software progress
      thread in order to process remotely-issued communication requests without
      explicit program calls to the \openshmem library.
  \item High performance implementations of \openshmem are expected to leverage
      hardware offload capabilities and provide asynchronous one-sided
      communication without software assistance.
  \item Implementations should avoid deferring the execution of one-sided
      operations until a synchronization point where data is known to be
      available. High-quality implementations should attempt asynchronous delivery
      whenever possible, for performance reasons. Additionally, the \openshmem
      community discourages releasing \openshmem implementations that do not
      provide asynchronous one-sided operations, as these have very limited
      performance value for \openshmem programs.
\end{itemize}


\section{Language Bindings and Conformance}\label{subsec:bindings}
\openshmem provides ISO \Cstd language bindings. Any implementation that
provides \Cstd bindings can claim conformance to the specification. The
\openshmem header file \HEADER{shmem.h} for \Cstd must contain only the
interfaces and constant names defined in this specification.

\openshmem \acp{API} can be implemented as either routines or macros. However,
implementing the interfaces using macros is strongly discouraged as this could
severely limit the use of external profiling tools and high-level compiler
optimizations. An \openshmem program should avoid defining routine names,
variables, or identifiers with the prefix \shmemprefix, \shmemprefixC, or with
\openshmem \ac{API} names.

All \openshmem extension \acp{API} that are not part of this specification must
be defined in the \HEADER{shmemx.h} include file for language bindings. This
header file must exist, even if no extensions are provided. Any extensions
shall use the \FUNC{shmemx\_} prefix for all routine, variable, and constant
names.


\section{Library Constants}\label{subsec:library_constants}
\TableIndex{Library Constants}
\TableIndex{Constants}

The \openshmem library provides a set of compile-time constants that may
be used to specify options to API routines, provide implementation-specific
parameters, or return information about the implementation.
All constants that start with \CONST{\_SHMEM\_*} are deprecated,
but provided for backwards compatibility.

\begin{longtable}{|p{0.45\textwidth}|p{0.5\textwidth}|}
\hline
\textbf{Constant} & \textbf{Description}
\tabularnewline \hline
\endhead
%%
\LibConstDecl{SHMEM\_THREAD\_SINGLE} &
The \openshmem thread support level which specifies that the program
must not be multithreaded.
See Section~\ref{subsec:thread_support} for more detail about its use.
\tabularnewline \hline
%%
\LibConstDecl{SHMEM\_THREAD\_FUNNELED} &
The \openshmem thread support level which specifies that the program
may be multithreaded but must ensure that only the main thread invokes
the \openshmem interfaces.
See Section~\ref{subsec:thread_support} for more detail about its use.
\tabularnewline \hline
%%
\LibConstDecl{SHMEM\_THREAD\_SERIALIZED} &
The \openshmem thread support level which specifies that the program
may be multithreaded but must ensure that the \openshmem interfaces
are not invoked concurrently by multiple threads.
See Section~\ref{subsec:thread_support} for more detail about its use.
\tabularnewline \hline
%%
\LibConstDecl{SHMEM\_THREAD\_MULTIPLE} &
The \openshmem thread support level which specifies that the program
may be multithreaded and any thread may invoke the \openshmem interfaces.
See Section~\ref{subsec:thread_support} for more detail about its use.
\tabularnewline \hline
%%
\LibConstDecl{SHMEM\_CTX\_SERIALIZED} &
The context creation option which specifies that the given context
is shareable but will not be used by multiple threads concurrently.
See Section~\ref{subsec:shmem_ctx_create} for more detail about its use.
\tabularnewline \hline
%%
\LibConstDecl{SHMEM\_CTX\_PRIVATE} &
The context creation option which specifies that the given context
will be used only by the thread that created it.
See Section~\ref{subsec:shmem_ctx_create} for more detail about its use.
\tabularnewline \hline
%%
\LibConstDecl{SHMEM\_CTX\_NOSTORE} &
The context creation option which specifies that quiet and fence operations
performed on the given context are not required to enforce completion and
ordering of memory store operations.
See Section~\ref{subsec:shmem_ctx_create} for more detail about its use.
\tabularnewline \hline
%%
\LibConstDecl{SHMEM\_SYNC\_VALUE}
\begin{DeprecateBlock}
  \LibConstDecl{\_SHMEM\_SYNC\_VALUE}
  \LibConstDecl[\Fortran]{SHMEM\_SYNC\_VALUE}
\end{DeprecateBlock}
&
The value used to initialize the elements of \VAR{pSync} arrays.
The value of this constant is implementation specific.
See Section~\ref{subsec:coll} for more detail about its use.
\tabularnewline \hline
%%
\LibConstDecl{SHMEM\_SYNC\_SIZE}
\begin{DeprecateBlock}
  \LibConstDecl[\Fortran]{SHMEM\_SYNC\_SIZE}
\end{DeprecateBlock}
&
Length of a work array that can be used with any SHMEM collective
communication operation.
Work arrays sized for specific operations may consume less memory.
The value of this constant is implementation specific.
See Section~\ref{subsec:coll} for more detail about its use.
\tabularnewline \hline
%%
\LibConstDecl{SHMEM\_BCAST\_SYNC\_SIZE}
\begin{DeprecateBlock}
  \LibConstDecl{\_SHMEM\_BCAST\_SYNC\_SIZE}
  \LibConstDecl[\Fortran]{SHMEM\_BCAST\_SYNC\_SIZE}
\end{DeprecateBlock}
&
Length of the \VAR{pSync} arrays needed for broadcast routines. The value
of this constant is implementation specific.
See Section~\ref{subsec:shmem_broadcast} for more detail about its use.
\tabularnewline \hline
%%
\LibConstDecl{SHMEM\_REDUCE\_SYNC\_SIZE}
\begin{DeprecateBlock}
  \LibConstDecl{\_SHMEM\_REDUCE\_SYNC\_SIZE}
  \LibConstDecl[\Fortran]{SHMEM\_REDUCE\_SYNC\_SIZE}
\end{DeprecateBlock}
&
Length of the work arrays needed for reduction routines.
The value of this constant is implementation specific.
See Section~\ref{subsec:shmem_reductions} for more detail about its use.
\tabularnewline \hline
%%
\LibConstDecl{SHMEM\_BARRIER\_SYNC\_SIZE}
\begin{DeprecateBlock}
  \LibConstDecl{\_SHMEM\_BARRIER\_SYNC\_SIZE}
  \LibConstDecl[\Fortran]{SHMEM\_BARRIER\_SYNC\_SIZE}
\end{DeprecateBlock}
&
Length of the work array needed for barrier routines.
The value of this constant is implementation specific.
See Section~\ref{subsec:shmem_barrier} for more detail about its use.

\tabularnewline \hline
%%
\LibConstDecl{SHMEM\_COLLECT\_SYNC\_SIZE}
\begin{DeprecateBlock}
  \LibConstDecl{\_SHMEM\_COLLECT\_SYNC\_SIZE}
  \LibConstDecl[\Fortran]{SHMEM\_COLLECT\_SYNC\_SIZE}
\end{DeprecateBlock}
&
Length of the work array needed for collect routines.
The value of this constant is implementation specific.
See Section~\ref{subsec:shmem_collect} for more detail about its use.
\tabularnewline \hline
%%
\LibConstDecl{SHMEM\_ALLTOALL\_SYNC\_SIZE}
\begin{DeprecateBlock}
  \LibConstDecl[\Fortran]{SHMEM\_ALLTOALL\_SYNC\_SIZE}
\end{DeprecateBlock}
&
Length of the work array needed for \FUNC{shmem\_alltoall} routines.
The value of this constant is implementation specific.
See Section~\ref{subsec:shmem_alltoall} for more detail about its use.
\tabularnewline \hline
%%
\LibConstDecl{SHMEM\_ALLTOALLS\_SYNC\_SIZE}
\begin{DeprecateBlock}
  \LibConstDecl[\Fortran]{SHMEM\_ALLTOALLS\_SYNC\_SIZE}
\end{DeprecateBlock}
&
Length of the work array needed for \FUNC{shmem\_alltoalls} routines.
The value of this constant is implementation specific.
See Section~\ref{subsec:shmem_alltoalls} for more detail about its use.
\tabularnewline \hline
%%
\LibConstDecl{SHMEM\_REDUCE\_MIN\_WRKDATA\_SIZE}
\begin{DeprecateBlock}
  \LibConstDecl{\_SHMEM\_REDUCE\_MIN\_WRKDATA\_SIZE}
  \LibConstDecl[\Fortran]{SHMEM\_REDUCE\_MIN\_WRKDATA\_SIZE}
\end{DeprecateBlock}
&
Minimum length of work arrays used in various collective routines.
\tabularnewline \hline
%%
\LibConstDecl{SHMEM\_MAJOR\_VERSION}
\begin{DeprecateBlock}
  \LibConstDecl{\_SHMEM\_MAJOR\_VERSION}
  \LibConstDecl[\Fortran]{SHMEM\_MAJOR\_VERSION}
\end{DeprecateBlock}
&
Integer representing the major version of \openshmem Specification in use.
\tabularnewline \hline
%%
\LibConstDecl{SHMEM\_MINOR\_VERSION}
\begin{DeprecateBlock}
  \LibConstDecl{\_SHMEM\_MINOR\_VERSION}
  \LibConstDecl[\Fortran]{SHMEM\_MINOR\_VERSION}
\end{DeprecateBlock}
&
Integer representing the minor version of \openshmem Specification in use.
\tabularnewline \hline
%%
\LibConstDecl{SHMEM\_MAX\_NAME\_LEN}
\begin{DeprecateBlock}
  \LibConstDecl{\_SHMEM\_MAX\_NAME\_LEN}
  \LibConstDecl[\Fortran]{SHMEM\_MAX\_NAME\_LEN}
\end{DeprecateBlock}
&
Integer representing the maximum length of \CONST{SHMEM\_VENDOR\_STRING}.
\tabularnewline \hline
%%
\LibConstDecl{SHMEM\_VENDOR\_STRING}
\begin{DeprecateBlock}
  \LibConstDecl{\_SHMEM\_VENDOR\_STRING}
  \LibConstDecl[\Fortran]{SHMEM\_VENDOR\_STRING}
\end{DeprecateBlock}
&
String representing vendor defined information of size at most
\CONST{SHMEM\_MAX\_NAME\_LEN}.
In \CorCpp{}, the string is terminated by a null character.  In \Fortran, the
string of size less than \CONST{SHMEM\_MAX\_NAME\_LEN} is padded with blank
characters up to size \CONST{SHMEM\_MAX\_NAME\_LEN}.
\tabularnewline \hline
%%
\LibConstDecl{SHMEM\_CMP\_EQ}
\begin{DeprecateBlock}
  \LibConstDecl{\_SHMEM\_CMP\_EQ}
  \LibConstDecl[\Fortran]{SHMEM\_CMP\_EQ}
\end{DeprecateBlock}
&
An integer constant expression corresponding to the
``equal to'' comparison operation.
See Section~\ref{subsec:p2p_intro} for more detail about its use.
\tabularnewline \hline
%%
\LibConstDecl{SHMEM\_CMP\_NE}
\begin{DeprecateBlock}
  \LibConstDecl{\_SHMEM\_CMP\_NE}
  \LibConstDecl[\Fortran]{SHMEM\_CMP\_NE}
\end{DeprecateBlock}
&
An integer constant expression corresponding to the
``not equal to'' comparison operation.
See Section~\ref{subsec:p2p_intro} for more detail about its use.
\tabularnewline \hline
%%
\LibConstDecl{SHMEM\_CMP\_LT}
\begin{DeprecateBlock}
  \LibConstDecl{\_SHMEM\_CMP\_LT}
  \LibConstDecl[\Fortran]{SHMEM\_CMP\_LT}
\end{DeprecateBlock}
&
An integer constant expression corresponding to the
``less than'' comparison operation.
See Section~\ref{subsec:p2p_intro} for more detail about its use.
\tabularnewline \hline
%%
\LibConstDecl{SHMEM\_CMP\_LE}
\begin{DeprecateBlock}
  \LibConstDecl{\_SHMEM\_CMP\_LE}
  \LibConstDecl[\Fortran]{SHMEM\_CMP\_LE}
\end{DeprecateBlock}
&
An integer constant expression corresponding to the
``less than or equal to'' comparison operation.
See Section~\ref{subsec:p2p_intro} for more detail about its use.
\tabularnewline \hline
%%
\LibConstDecl{SHMEM\_CMP\_GT}
\begin{DeprecateBlock}
  \LibConstDecl{\_SHMEM\_CMP\_GT}
  \LibConstDecl[\Fortran]{SHMEM\_CMP\_GT}
\end{DeprecateBlock}
&
An integer constant expression corresponding to the
``greater than'' comparison operation.
See Section~\ref{subsec:p2p_intro} for more detail about its use.
\tabularnewline \hline
%%
\LibConstDecl{SHMEM\_CMP\_GE}
\begin{DeprecateBlock}
  \LibConstDecl{\_SHMEM\_CMP\_GE}
  \LibConstDecl[\Fortran]{SHMEM\_CMP\_GE}
\end{DeprecateBlock}
&
An integer constant expression corresponding to the
``greater than or equal to'' comparison operation.
See Section~\ref{subsec:p2p_intro} for more detail about its use.
\tabularnewline \hline
%%
\end{longtable}


\section{Library Handles}\label{subsec:library_handles}
\TableIndex{Library Handles}
\TableIndex{Handles}

The \openshmem library provides a set of predefined named constant handles.
All named constants can be used in initialization expressions or assignments,
but not necessarily in array declarations or as labels in \Cstd switch statements.
This implies named constants to be link-time but not necessarily compile-time
constants.

\begin{longtable}{|p{0.45\textwidth}|p{0.5\textwidth}|}
\hline
\textbf{Handle} & \textbf{Description}
\tabularnewline \hline
\endhead
%%
\color{Green}
\LibHandleDecl{SHMEM\_TEAM\_WORLD} &
\color{Green}
Handle of type \CTYPE{shmem\_team\_t} that corresponds to the
default team of all \acp{PE} in the \openshmem program.  All point-to-point
communication operations and synchronizations that do not specify a team
are performed on the default team.
See Section~\ref{subsec:team} for more detail about its use.
\tabularnewline \hline
%%
\color{Green}
\LibHandleDecl{SHMEM\_TEAM\_SHARED} &
\color{Green}
Handle of type \CTYPE{shmem\_team\_t} that corresponds a team of \acp{PE}
that share a memory domain. When this handle is used by some \ac{PE},
it will refer to the team of all \acp{PE} that would return a non-null
pointer from \FUNC{shmem\_ptr} for symmetric objects on that \ac{PE},
and vice versa. This means that symmetric objects on each \ac{PE} are
directly load/store accessible by all \acp{PE} in the team.
See Section~\ref{subsec:team} for more detail about its use.
\tabularnewline \hline
%%
\LibHandleDecl{SHMEM\_CTX\_DEFAULT} &
Handle of type \CTYPE{shmem\_ctx\_t} that corresponds to the
default communication context.  All point-to-point communication operations
and synchronizations that do not specify a context are performed on the
default context.
See Section~\ref{sec:ctx} for more detail about its use.
\tabularnewline \hline
%%
\end{longtable}


\section{Environment Variables }\label{subsec:environment_variables}
\input{content/environment_variables}




\clearpage



\section{OpenSHMEM Library API}\label{sec:openshmem_library_api}

\subsection{Library Setup, Exit, and Query Routines}
The library setup and query interfaces that initialize and monitor the parallel
environment of the \acp{PE}.

\subsubsection{\textbf{SHMEM\_INIT}}\label{subsec:shmem_init}
\apisummary{
    A collective operation that allocates and initializes the resources used by
    the \openshmem library.
}

\begin{apidefinition}

\begin{Csynopsis}
void @\FuncDecl{shmem\_init}@(void);
\end{Csynopsis}

\begin{apiarguments}
    \apiargument{None.}{}{}
\end{apiarguments}

\apidescription{
    \FUNC{shmem\_init} allocates and initializes resources used by the \openshmem
    library. It is a collective operation that all \acp{PE} must call before any
    other \openshmem routine may be called. At the end of the \openshmem program
    which it initialized, the call to \FUNC{shmem\_init} must be matched with a
    call to \FUNC{shmem\_finalize}. After the first call to \FUNC{shmem\_init}, a
    subsequent call to \FUNC{shmem\_init} or \FUNC{shmem\_init\_thread} in the
    same program results in undefined behavior.
}

\apireturnvalues{
    None.
}

\apinotes{
    As of \openshmem[1.2], the use of \FUNC{start\_pes} has been
    deprecated and calls to it should be replaced with calls to \FUNC{shmem\_init}.
    While support for \FUNC{start\_pes} is still required in \openshmem libraries,
    users are encouraged to use \FUNC{shmem\_init}. An important difference between
    \FUNC{shmem\_init} and \FUNC{start\_pes} is that multiple calls to
    \FUNC{shmem\_init} within a program results in undefined behavior, while in the
    case of \FUNC{start\_pes}, any subsequent calls to \FUNC{start\_pes} after the
    first one results in a no-op.
}

\begin{apiexamples}

\apifexample
{ The following \FUNC{shmem\_init} example is for \Cstd[11] programs: }
    { example_code/shmem_init_example.c }
    {}

\end{apiexamples}

\end{apidefinition}


\subsubsection{\textbf{SHMEM\_MY\_PE}}\label{subsec:shmem_my_pe}
\apisummary{
    Returns the number of the calling \ac{PE}.
}

\begin{apidefinition}

\begin{Csynopsis}
int @\FuncDecl{shmem\_my\_pe}@(void);
\end{Csynopsis}

\begin{apiarguments}
    \apiargument{None.}{}{}
\end{apiarguments}

\apidescription{
    This routine returns the \ac{PE} number of the calling \ac{PE}.  It accepts no
    arguments.  The result is an integer between \CONST{0} and \VAR{npes} -
    \CONST{1}, where \VAR{npes} is the total number of \acp{PE} executing the
    current program.
}

\apireturnvalues{
    Integer - Between \CONST{0} and \VAR{npes} - \CONST{1}
}

\apinotes{
    Each \ac{PE} has a unique number or identifier. As of \openshmem[1.2]
    the use of \FUNC{\_my\_pe} has been deprecated. Although \openshmem
    libraries are required to support the call, users are encouraged to use
    \FUNC{shmem\_my\_pe} instead.  The behavior and signature  of the routine
    \FUNC{shmem\_my\_pe} remains unchanged from the deprecated \FUNC{\_my\_pe}
    version.
}

\end{apidefinition}


\subsubsection{\textbf{SHMEM\_N\_PES}}\label{subsec:shmem_n_pes}
\apisummary{
    Returns the number of \acp{PE} running in a program.
}

\begin{apidefinition}

\begin{Csynopsis}
int @\FuncDecl{shmem\_n\_pes}@(void);
\end{Csynopsis}

\begin{apiarguments}
    \apiargument{None.}{}{}
\end{apiarguments}

\apidescription{
    The routine returns the number of \acp{PE} running in the program.
}

\apireturnvalues{
    Integer -  Number of \acp{PE} running in the \openshmem program.
}

\apinotes{
    As of \openshmem[1.2] the use of \FUNC{\_num\_pes} has been
    deprecated. Although \openshmem libraries are required to support the call,
    users are encouraged to use \FUNC{shmem\_n\_pes} instead.  The behavior and
    signature  of the routine \FUNC{shmem\_n\_pes} remains unchanged from the
    deprecated \FUNC{\_num\_pes} version.
}

\begin{apiexamples}

\apicexample
	 {The following \FUNC{shmem\_my\_pe} and \FUNC{shmem\_n\_pes} example is for
	  \CorCpp{} programs:}
    {./example_code/shmem_npes_example.c}
    {}

\end{apiexamples}

\end{apidefinition}


\subsubsection{\textbf{SHMEM\_FINALIZE}}\label{subsec:shmem_finalize}
\apisummary{
    A collective operation that releases all resources used by the \openshmem
    library.  This only terminates the \openshmem portion of a program, not the
    entire program.
}

\begin{apidefinition}

\begin{Csynopsis}
void @\FuncDecl{shmem\_finalize}@(void);
\end{Csynopsis}

\begin{apiarguments}
    \apiargument{None.}{}{}
\end{apiarguments}

\apidescription{
    \FUNC{shmem\_finalize} is a collective operation that ends the \openshmem
    portion of a program previously initialized by \FUNC{shmem\_init} or \FUNC{shmem\_init\_thread} and
    releases all resources used by the \openshmem library. This collective
    operation requires all \acp{PE} to participate in the call. There is an
    implicit global barrier in \FUNC{shmem\_finalize} to ensure that pending
    communications are completed and that no resources are released until all
    \acp{PE} have entered \FUNC{shmem\_finalize}.
    \oldtext{This routine destroys all shareable contexts.}
    \newtext{
    This routine destroys all teams created by the \openshmem program.
    As a result, all shareable contexts are destroyed.
    }  The user is
    responsible for destroying all contexts with the
    \CONST{SHMEM\_CTX\_PRIVATE} option enabled prior to calling this routine;
    otherwise, the behavior is undefined.
    \FUNC{shmem\_finalize} must be
    the last \openshmem library call encountered in the \openshmem portion of a
    program. A call to \FUNC{shmem\_finalize} will release all resources
    initialized by a corresponding call to \FUNC{shmem\_init} or \FUNC{shmem\_init\_thread}. All processes
    that represent the \acp{PE} will still exist after the
    call to \FUNC{shmem\_finalize} returns, but they will no longer have access
    to resources that have been released.
}

\apireturnvalues{
    None.
}

\apinotes{
    \FUNC{shmem\_finalize} releases all resources used by the \openshmem library
    including the symmetric memory heap and pointers initiated by
    \FUNC{shmem\_ptr}. This collective operation requires all \acp{PE} to
    participate in the call, not just a subset of the \acp{PE}. The
    non-\openshmem portion of a program may continue after a call to
    \FUNC{shmem\_finalize} by all \acp{PE}.
}

\begin{apiexamples}

\apicexample
    {The following finalize example is for \Cstd[11] programs:}
    {./example_code/shmem_finalize_example.c}
    {}

\end{apiexamples}

\end{apidefinition}


\subsubsection{\textbf{SHMEM\_GLOBAL\_EXIT}}\label{subsec:shmem_global_exit}
\apisummary{
    A routine that allows any \ac{PE} to force termination of an entire program.
}

\begin{apidefinition}

\begin{C11synopsis}
_Noreturn void @\FuncDecl{shmem\_global\_exit}@(int status);
\end{C11synopsis}

\begin{Csynopsis}
void @\FuncDecl{shmem\_global\_exit}@(int status);
\end{Csynopsis}

\begin{apiarguments}
    \apiargument{IN}{status}{The exit status from the main program.}
\end{apiarguments}

\apidescription{
    \FUNC{shmem\_global\_exit} is a non-collective routine that allows any one
    \ac{PE} to force termination of an \openshmem program for all \acp{PE},
    passing an exit status to the execution environment. This routine terminates
    the entire program, not just the \openshmem portion.  When any \ac{PE} calls
    \FUNC{shmem\_global\_exit}, it results in the immediate notification to all
    \acp{PE} to terminate.  \FUNC{shmem\_global\_exit} flushes I/O and releases
    resources in accordance with \CorCpp language requirements for normal
    program termination. If more than one \ac{PE} calls
    \FUNC{shmem\_global\_exit}, then the exit status returned to the environment
    shall be one of the values passed to \FUNC{shmem\_global\_exit} as the
    status argument.  There is no return to the caller of
    \FUNC{shmem\_global\_exit}; control is returned from the \openshmem program
    to the execution environment for all \acp{PE}.
}

\apireturnvalues{
    None.
}


\apinotes{
    \FUNC{shmem\_global\_exit} may be used in situations where one or more
    \acp{PE} have determined that the program has completed and/or should
    terminate early.  Accordingly, the integer status argument can be used to
    pass any information about the nature of the exit; e.g., that the program
    encountered an error or found a solution.
    Since \FUNC{shmem\_global\_exit} is a non-collective
    routine, there is no implied synchronization, and all \acp{PE} must
    terminate regardless of their current execution state. While I/O must be
    flushed for standard language I/O calls from \CorCpp, it is
    implementation dependent as to how I/O done by other means (e.g., third
    party I/O libraries) is handled. Similarly, resources are released
    according to \CorCpp standard language requirements, but this may not
    include all resources allocated for the \openshmem program. However, a
    quality implementation will make a best effort to flush all I/O and clean
    up all resources.
}

\begin{apiexamples}

\apicexample
    {}
    {./example_code/shmem_global_exit_example.c}
    {}

\end{apiexamples}

\end{apidefinition}


\subsubsection{\textbf{SHMEM\_PE\_ACCESSIBLE}}\label{subsec:shmem_pe_accessible}
\apisummary{
    Determines whether a \ac{PE} is accessible via \openshmem's data transfer
    routines.
}

\begin{apidefinition}

\begin{Csynopsis}
int @\FuncDecl{shmem\_pe\_accessible}@(int pe);
\end{Csynopsis}

\begin{apiarguments}
    \apiargument{IN}{pe}{Specific \ac{PE} to be checked for accessibility from
    the local \ac{PE}.}
\end{apiarguments}

\apidescription{
    \FUNC{shmem\_pe\_accessible} is a query routine that indicates whether a
    specified \ac{PE} is accessible via \openshmem from the local \ac{PE}. The
    \FUNC{shmem\_pe\_accessible} routine returns a value indicating whether the remote
    \ac{PE} is a process running from the same executable file as the local
    \ac{PE}, thereby indicating whether full support for symmetric data objects,
    which may reside in either static memory or the symmetric heap, is available.
}

\apireturnvalues{
    The return value is 1 if the specified \ac{PE} is a valid remote \ac{PE}
    for \openshmem routines; otherwise, it is 0.
}

\apinotes{
    This routine may be particularly useful for hybrid programming with other
    communication libraries (such as \ac{MPI}) or parallel languages.  For
    example, when an \ac{MPI} job uses \ac{MPMD} mode, multiple executable
    \ac{MPI} programs are executed as part of the same MPI job.  In such cases,
    \openshmem support may only be available between processes running from the
    same executable file.  In addition, some environments may allow a hybrid
    job to span multiple network partitions.  In such scenarios, \openshmem
    support may only be available between \acp{PE} within the same partition.
}

\end{apidefinition}


\subsubsection{\textbf{SHMEM\_ADDR\_ACCESSIBLE}}\label{subsec:shmem_addr_accessible}
\apisummary{
    Determines whether an address is accessible via \openshmem data transfer
    routines from the specified remote \ac{PE}.
}

\begin{apidefinition}

\begin{Csynopsis}
int @\FuncDecl{shmem\_addr\_accessible}@(const void *addr, int pe);
\end{Csynopsis}

\begin{apiarguments}
    \apiargument{IN}{addr}{Data object on the local \ac{PE}.}
    \apiargument{IN}{pe}{Integer id of a remote \ac{PE}.}
\end{apiarguments}

\apidescription{
    \FUNC{shmem\_addr\_accessible} is a query routine that indicates whether a local
    address is accessible via \openshmem routines from the specified remote \ac{PE}.

    This routine verifies that the data object is symmetric and accessible with
    respect to a remote \ac{PE} via \openshmem data transfer routines.  The
    specified address \VAR{addr} is a data object on the local \ac{PE}.
}

\apireturnvalues{
    The return value is \CONST{1} if \VAR{addr} is a symmetric data object
    and accessible via \openshmem routines from the specified remote \ac{PE};
    otherwise, it is \CONST{0}.
}

\apinotes{
    This routine may be particularly useful for hybrid programming with other
    communication libraries (such as \ac{MPI}) or parallel languages.  For
    example, when an \ac{MPI} job uses \ac{MPMD} mode, multiple executable
    \ac{MPI} programs may use \openshmem routines.  In such cases, static
    memory, such as a \Cstd global variable, is
    symmetric between processes running from the same executable file, but is
    not symmetric between processes running from different executable files.
    Data allocated from the symmetric heap (e.g., using \FUNC{shmem\_malloc})
    is symmetric across the same or different executable files.
}

\end{apidefinition}


\subsubsection{\textbf{SHMEM\_PTR}}\label{subsec:shmem_ptr}
\apisummary{
    Returns a local pointer to a symmetric data object on the specified \ac{PE}.
}

\begin{apidefinition}

\begin{Csynopsis}
void *@\FuncDecl{shmem\_ptr}@(const void *dest, int pe);
\end{Csynopsis}

\begin{apiarguments}
\apiargument{IN}{dest}{The symmetric data object to be referenced.}
\apiargument{IN}{pe}{An integer that indicates the \ac{PE} number on which \dest{} is to
		 be accessed.}
\end{apiarguments}

\apidescription{
    \FUNC{shmem\_ptr} returns an address that may be used to directly reference
    \dest{} on the specified \ac{PE}.  This address can be assigned to a pointer.
    After that, ordinary loads and stores to this remote address may be performed.

    The \FUNC{shmem\_ptr} routine can provide an efficient means to accomplish
    communication, for example when a sequence of reads and writes to a data
    object on a remote \ac{PE} does not match the access pattern provided in an
    \openshmem data transfer routine like \FUNC{shmem\_put} or
    \FUNC{shmem\_iget}.
}

\apireturnvalues{
    The address of the \dest{} data object is returned when it is accessible
    using memory loads and stores.  Otherwise, a null pointer is returned.
}

\apinotes{
    When calling \FUNC{shmem\_ptr}, \dest{} is the address of the referenced
    symmetric data object on the calling \ac{PE}.
}

\begin{apiexamples}

\apicexample
    {This is the equivalent program written in \Cstd[11]:}
    {./example_code/shmem_ptr_example.c}
    {}

\end{apiexamples}

\end{apidefinition}


\subsubsection{\textbf{SHMEM\_INFO\_GET\_VERSION}}\label{subsec:shmem_info_get_version}
\apisummary{
    Returns the major and minor version of the library implementation.
}

\begin{apidefinition}

\begin{Csynopsis}
void @\FuncDecl{shmem\_info\_get\_version}@(int *major, int *minor);
\end{Csynopsis}

\begin{apiarguments}
    \apiargument{OUT}{major}{The major version of the \openshmem Specification in use.}
    \apiargument{OUT}{minor}{The minor version of the \openshmem Specification in use.}
\end{apiarguments}

\apidescription{
    This routine returns the major and minor version of the \openshmem Specification
    in use.  For a given library implementation, the major and minor version
    returned by these calls are consistent with the library constants
    \CONST{SHMEM\_MAJOR\_VERSION} and \CONST{SHMEM\_MINOR\_VERSION}.
}

\apireturnvalues{
    None.
}

\apinotes{
    None.
}

\end{apidefinition}


\subsubsection{\textbf{SHMEM\_INFO\_GET\_NAME}}\label{subsec:shmem_info_get_name}
\apisummary{
    This routine returns the vendor defined name string that is consistent
    with the library constant \CONST{SHMEM\_VENDOR\_STRING}.
}

\begin{apidefinition}

\begin{Csynopsis}
void @\FuncDecl{shmem\_info\_get\_name}@(char *name);
\end{Csynopsis}

\begin{apiarguments}
    \apiargument{OUT}{name}{The vendor defined string.}
\end{apiarguments}

\apidescription{
    This routine returns the vendor defined name string of size defined by
    the library constant \CONST{SHMEM\_MAX\_NAME\_LEN}. The program calling
    this function provides the \VAR{name} memory buffer of at least size
    \CONST{SHMEM\_MAX\_NAME\_LEN}. The implementation copies the vendor defined
    string of size at most \CONST{SHMEM\_MAX\_NAME\_LEN} to \VAR{name}. In
    \CorCpp, the string is terminated by a null character. If the
    \VAR{name} memory buffer is provided with size less than
    \CONST{SHMEM\_MAX\_NAME\_LEN}, behavior is undefined. For a given library
    implementation, the vendor string returned is consistent with the library
    constant \CONST{SHMEM\_VENDOR\_STRING}.
}

\apireturnvalues{
    None.
}

\apinotes{
    None.
}

\end{apidefinition}


\subsubsection{\textbf{START\_PES}}\label{subsec:start_pes}
\apisummary{
    Called at the beginning of an \openshmem program to initialize the execution
    environment. This routine is deprecated and is provided for backwards
    compatibility. Implementations must include it, and the routine should
    function properly and may notify the user about deprecation of its use.
}

\begin{apidefinition}

\begin{DeprecateBlock}
\begin{Csynopsis}
void @\FuncDecl{start\_pes}@(int npes);
\end{Csynopsis}
\end{DeprecateBlock}

\begin{apiarguments}
       \apiargument{npes}{Unused}{ Should be set to \CONST{0}.}
\end{apiarguments}

\apidescription{
     The \FUNC{start\_pes} routine initializes the \openshmem execution
     environment.  An \openshmem program must call \FUNC{start\_pes},
     \FUNC{shmem\_init}, or \FUNC{shmem\_init\_thread} before calling any other \openshmem routine.  Unlike
     \FUNC{shmem\_init} and \FUNC{shmem\_init\_thread}, \FUNC{start\_pes} does not require a call to
     \FUNC{shmem\_finalize}.  Instead, the \openshmem library is implicitly
     finalized when the program exits.  Implicit finalization is collective and
     includes a global synchronization to ensure that all pending communication
     is completed before resources are released.
}

\apireturnvalues{
    None.
}

\apinotes{
    If any other \openshmem call occurs before \FUNC{start\_pes}, the
    behavior is undefined.  Although it is recommended to set \VAR{npes} to
    \CONST{0} for \FUNC{start\_pes}, this is not mandated.  The value is ignored.
    Calling \FUNC{start\_pes} more than once has no subsequent
    effect.

    As of \openshmem[1.2] the use of \FUNC{start\_pes} has
    been deprecated. Although \openshmem libraries are required to support the
    call, users are encouraged to use \FUNC{shmem\_init} or
    \FUNC{shmem\_init\_thread} instead.
}

\end{apidefinition}


\subsection{Thread Support}
\label{subsec:thread_support}
\input{content/threads_intro.tex}

\subsubsection{\textbf{SHMEM\_INIT\_THREAD}}
\label{subsec:shmem_init_thread}
\input{content/shmem_init_thread}

\subsubsection{\textbf{SHMEM\_QUERY\_THREAD}}
\label{subsec:shmem_query_thread}
\input{content/shmem_query_thread}


\subsection{Memory Management Routines}
\label{sec:memory_management}

\openshmem provides a set of \acp{API} for managing the symmetric heap. The
\acp{API} allow one to dynamically allocate, deallocate, reallocate and align
symmetric data objects in the symmetric heap.

\subsubsection{\textbf{SHMEM\_MALLOC, SHMEM\_FREE, SHMEM\_REALLOC, SHMEM\_ALIGN}}\label{subsec:shfree}
\apisummary{
    Collective symmetric heap memory management routines.
}

\begin{apidefinition}

\begin{Csynopsis}
void *@\FuncDecl{shmem\_malloc}@(size_t size);
void @\FuncDecl{shmem\_free}@(void *ptr);
void *@\FuncDecl{shmem\_realloc}@(void *ptr, size_t size);
void *@\FuncDecl{shmem\_align}@(size_t alignment, size_t size);
\end{Csynopsis}

\begin{apiarguments}
    \apiargument{IN}{size}{The size, in bytes, of a block to be
        allocated from the symmetric heap. This argument is of type \CTYPE{size\_t}}
    \apiargument{IN}{ptr}{Pointer to a block within the symmetric heap.}
    \apiargument{IN}{alignment}{Byte alignment of the block allocated from the
        symmetric heap.}
\end{apiarguments}


\apidescription{
    The \FUNC{shmem\_malloc}, \FUNC{shmem\_free}, \FUNC{shmem\_realloc}, and
    \FUNC{shmem\_align} routines are collective operations that require
    participation by all \acp{PE}.

    The \FUNC{shmem\_malloc} routine returns a pointer to a block of at least
    \VAR{size} bytes, which shall be suitably aligned so that it may be
    assigned to a pointer to any type of object.  This space is allocated from
    the symmetric heap (in contrast to \FUNC{malloc}, which allocates from the
    private heap).  When \VAR{size} is zero, the \FUNC{shmem\_malloc} routine
    performs no action and returns a null pointer.
    
    The \FUNC{shmem\_align} routine allocates a block in the symmetric heap that has
    a byte alignment specified by the \VAR{alignment} argument.  The value of
    \VAR{alignment} shall be a multiple of \CONST{sizeof(void *)} that is also
    a power of two.  Otherwise, the behavior is undefined.  When \VAR{size} is
    zero, the \FUNC{shmem\_align} routine performs no action and returns a null
    pointer.
    
    The \FUNC{shmem\_free} routine causes the block to which \VAR{ptr} points to be
    deallocated, that is, made available for further allocation.  If \VAR{ptr} is a
    null pointer, no action is performed.
           
    The \FUNC{shmem\_realloc} routine changes the size of the block to which
    \VAR{ptr} points to the size (in bytes) specified by \VAR{size}.  The contents
    of the block are unchanged up to the lesser of the new and old sizes. If the new
    size is larger, the newly allocated portion of the block is
    uninitialized.  If \VAR{ptr} is a null pointer, the
    \FUNC{shmem\_realloc} routine behaves like the \FUNC{shmem\_malloc} routine for
    the specified size.  If \VAR{size} is \CONST{0} and \VAR{ptr} is not a
    null pointer, the block to which it points is freed. If the space cannot
    be allocated, the block to which \VAR{ptr} points is unchanged.
    
    The \FUNC{shmem\_malloc}, \FUNC{shmem\_align}, \FUNC{shmem\_free}, and \FUNC{shmem\_realloc} routines
    are provided  so that multiple \acp{PE} in a program can allocate symmetric,
    remotely accessible memory blocks.  These memory blocks can then be used with
    \openshmem communication routines.  When no action is performed, these
    routines return without performing a barrier.
    Otherwise, each of these routines includes at least one
    call to a procedure that is semantically equivalent to \FUNC{shmem\_barrier\_all}:
    \FUNC{shmem\_malloc} and \FUNC{shmem\_align} call a
    barrier on exit; \FUNC{shmem\_free} calls a barrier on entry; and
    \FUNC{shmem\_realloc} may call barriers on both entry and exit, depending on
    whether an existing allocation is modified and whether new memory is allocated, respectively.
    This ensures that all
    \acp{PE} participate in the memory allocation, and that the memory on other
    \acp{PE} can be used as soon as the local \ac{PE} returns.
    The implicit barriers performed by these routines quiet the
    default context.  It is the user's responsibility to ensure that no
    communication operations involving the given memory block are pending on
    other contexts prior to calling
    the \FUNC{shmem\_free} and \FUNC{shmem\_realloc} routines.
    The user is also
    responsible for calling these routines with identical argument(s) on all
    \acp{PE}; if differing \VAR{ptr}, \VAR{size}, or \VAR{alignment} arguments are used, the behavior of the call
    and any subsequent \openshmem calls is undefined.
}

\apireturnvalues{
    The \FUNC{shmem\_malloc} routine returns a pointer to the allocated space;
    otherwise, it returns a null pointer.
    
    The \FUNC{shmem\_free} routine returns no value.
    
    The \FUNC{shmem\_realloc} routine returns a pointer to the allocated space
    (which may have moved); otherwise, all \acp{PE} return a null pointer.
    
    The \FUNC{shmem\_align} routine returns an aligned pointer whose value is a
    multiple of \VAR{alignment}; otherwise, it returns a null pointer.
}

\apinotes{ 
    As of \openshmem[1.2] the use of \FUNC{shmalloc}, \FUNC{shmemalign},
    \FUNC{shfree},  and \FUNC{shrealloc} has been deprecated. Although \openshmem
    libraries are required to support the calls, users are encouraged to use
    \FUNC{shmem\_malloc}, \FUNC{shmem\_align}, \FUNC{shmem\_free}, and
    \FUNC{shmem\_realloc} instead.  The behavior and signature  of the routines
    remains unchanged from the deprecated versions.
    					 
    The total size of the symmetric heap is determined at job startup.  One can
    specify the size of the heap using the \ENVVAR{SHMEM\_SYMMETRIC\_SIZE} environment
    variable (where available).	
    
    The \FUNC{shmem\_malloc}, \FUNC{shmem\_free}, and \FUNC{shmem\_realloc} routines
    differ from the private heap allocation routines in that all \acp{PE} in a
    program must call them (a barrier is used to ensure this).

    When the \VAR{ptr} argument in a call to \FUNC{shmem\_realloc} corresponds
    to a buffer allocated using \FUNC{shmem\_align}, the buffer returned by
    \FUNC{shmem\_realloc} is not guaranteed to maintain the alignment requested
    in the original call to \FUNC{shmem\_align}.
}

\apiimpnotes{
    The symmetric heap allocation routines always return a pointer to corresponding
    symmetric objects across all \acp{PE}. The \openshmem specification does not
    require that the virtual addresses are equal across all \acp{PE}. Nevertheless,
    the implementation must avoid costly address translation operations in the
    communication path, including $O(N)$ memory translation tables,
    where $N$ is the number of \acp{PE}.  In order to avoid address translations, the
    implementation may re-map the allocated block of memory based on agreed virtual
    address.  Additionally, some operating systems provide an option to disable
    virtual address randomization, which enables predictable allocation of virtual
    memory addresses.
}

\end{apidefinition}


\subsubsection{\textbf{SHMEM\_CALLOC}}\label{subsec:shmem_calloc}
\apisummary{
  Allocate a zeroed block of symmetric memory.
}

\begin{apidefinition}

\begin{Csynopsis}
void *@\FuncDecl{shmem\_calloc}@(size_t count, size_t size);
\end{Csynopsis}

\begin{apiarguments}
  \apiargument{IN}{count}{The number of elements to allocate.}
  \apiargument{IN}{size}{The size in bytes of each element to allocate.}
\end{apiarguments}


\apidescription{
  The \FUNC{shmem\_calloc} routine is a collective operation that allocates a
  region of remotely-accessible
  memory for an array of \VAR{count} objects of \VAR{size} bytes each and
  returns a pointer to the lowest byte address of the allocated symmetric
  memory. The space is initialized to all bits zero.

  If the allocation succeeds, the pointer returned shall be suitably
  aligned so that it may be assigned to a pointer to any type of object.
  If the allocation does not succeed, or either \VAR{count} or \VAR{size} is
  \CONST{0}, the return value is a null pointer.

  The values for \VAR{count} and \VAR{size} shall each be equal across
  all \acp{PE} calling \FUNC{shmem\_calloc}; otherwise, the behavior is
  undefined.

  When \VAR{count} or \VAR{size} is \CONST{0}, the \FUNC{shmem\_calloc} routine
  returns without performing a barrier.  Otherwise, this
  routine calls a procedure that is semantically equivalent to
  \FUNC{shmem\_barrier\_all} on exit.
}

\apireturnvalues{
  The \FUNC{shmem\_calloc} routine returns a pointer to the lowest byte
  address of the allocated space; otherwise, it returns a null pointer.
}

\apinotes{
  None.
}

\end{apidefinition}


\subsubsection{\textbf{SHPALLOC}}\label{subsec:shpalloc}
\apisummary{
    Allocates a block of memory from the symmetric heap.
}

\begin{apidefinition}

\begin{Fsynopsis}
POINTER (addr, A(1))
INTEGER length, errcode, abort
CALL @\FuncDecl{SHPALLOC}@(addr, length, errcode, abort)
\end{Fsynopsis}

\begin{apiarguments}
    \apiargument{OUT}{addr}{First word address of the allocated block.}
    \apiargument{IN}{length}{Number of words of memory requested. One word is 32 bits.}
    \apiargument{OUT}{errcode}{Error code is \CONST{0} if no error was detected;
    otherwise, it is a negative integer code for the type of error.}
    \apiargument{IN}{abort}{Abort code; nonzero requests abort on error;
    \CONST{0}  requests an error code.}
\end{apiarguments}

\apidescription{   
    \FUNC{SHPALLOC} allocates a block of memory from the program's symmetric heap
    that is greater than or equal to the size requested. To maintain symmetric heap
    consistency, all \acp{PE} in an program must call \FUNC{SHPALLOC} with the same
    value of length; if any  \acp{PE} are missing, the program will hang.
    
    By using the \Fortran \CONST{POINTER} mechanism in the following manner, 
    array \VAR{A} can be used to refer to the block allocated by \FUNC{SHPALLOC}:
    \CONST{POINTER} (\VAR{addr}, \VAR{A}())
}

\apireturnvalues{}
    \apitablerow{Error Code}{Condition}
    \apitablerow{ \CONST{-1} }{Length is not an integer greater than \CONST{0}}
    \apitablerow{\CONST{-2}}{ No more memory is available from the system (checked if the
    request cannot be satisfied from the available blocks on the symmetric heap).}

\apinotes{  
    The total size of the symmetric heap is determined at job startup.  One may
    adjust the size of the heap using the \ENVVAR{SHMEM\_SYMMETRIC\_SIZE} environment
    variable (if available).	
}

\apiimpnotes{
    The symmetric heap allocation routines always return a pointer to corresponding
    symmetric objects across all \acp{PE}. The \openshmem specification does not
    require that the virtual addresses are equal across all \acp{PE}. Nevertheless,
    the implementation must avoid costly address translation operations in the
    communication path, including order $N$ (where $N$ is the number of \acp{PE})
    memory translation tables.  In order to avoid address translations, the
    implementation may re-map the allocated block of memory based on agreed virtual
    address.  Additionally, some operating systems provide an option to disable
    virtual address randomization, which enables predictable allocation of virtual
    memory addresses.
}

\end{apidefinition}


\subsubsection{\textbf{SHPCLMOVE}}\label{subsec:shpclmove}
\input{content/shpclmove.tex}

\subsubsection{\textbf{SHPDEALLC}}\label{subsec:shpdeallc}
\input{content/shpdeallc.tex}




\color{Green}
\subsection{Team Management Routines}\label{subsec:team}

The \acp{PE} in an \openshmem program can communicate either using
point-to-point routines that specify the \ac{PE} number of the target
\ac{PE} or using collective routines which operate over some predefined
set of \acp{PE}. Teams in \openshmem allow programs to group subsets
of \acp{PE} for collective communications and provide a contiguous reindexing
of the \acp{PE} within that subset that can be used in point-to-point communication.

An \openshmem team is a set of \acp{PE} defined by calling a specific team
split routine with a parent team argument and other arguments to further
specify how the parent team is to be split into one or more new teams.
A team created by a \FUNC{shmem\_team\_split\_*} routine can be used as the parent team
for a subsequent call to a team split routine.  A team persists and can
be used for multiple collective routine calls until it is destroyed by
\FUNC{shmem\_team\_destroy}.

Every team must have a least one member. Any attempt to create a team over an empty set of \acp{PE} will result in no new team being created.

A ``team handle'' is an opaque object with type \CTYPE{shmem\_team\_t} that is used
to reference a defined team.  Team handles are created by one of the team split
routines and destroyed by the team destroy routine. Team handles have local
semantics only. That is, team handles should not be stored in shared variables
and used across other \acp{PE}. Doing so will result in undefined behavior.

By default, \openshmem creates predefined teams that will be available
for use once the routine \FUNC{shmem\_init} has been called. See
Section~\ref{subsec:library_handles} for a description of all predefined team handles
provided by \openshmem. Predefined \CTYPE{shmem\_team\_t} handles can be used as
the parent team when creating new \openshmem teams.

Every \ac{PE} is a member of the default team, which may be referenced
through the team handle \LibHandleRef{SHMEM\_TEAM\_WORLD},
and its number in the default team is equal to the
value of its \ac{PE} number as returned by \FUNC{shmem\_my\_pe}.

A special team handle value, \LibConstRef{SHMEM\_TEAM\_NULL}, may be used to
indicate that a returned team handle is not valid. This value can be tested
against to check for successful split operations and can be assigned to user
declared team handles as a sentinel value.

Teams that are created by a \FUNC{shmem\_team\_split\_*} routine may be
provided a configuration argument that specifies options that may affect
a team's capabilities and may allow for optimized performance.
This configuration argument is of type \CTYPE{shmem\_team\_config\_t}, which
is detailed further in Section~\ref{subsec:shmem_team_config_t}.

%% Teams may be created with options that change team behavior and may allow for
%% more optimized performance. These options are described in
%% Section~\ref{subsec:library_constants} and in the various descriptions of the
%% team split routines below. In particular, teams may be created with an option
%% to disable support for collective communications, which allows implementations
%% to reduce team creation overheads for those teams. In that case, the team is
%% just a local reindexing of some set of \acp{PE} that can be used for
%% point to point communications or as parent teams in subsequent split operations.

\subsubsection{\textbf{SHMEM\_TEAM\_MY\_PE}}\label{subsec:shmem_team_my_pe}
\apisummary{
    Returns the number of calling \ac{PE} within the provided team.
}

\begin{apidefinition}

\begin{Csynopsis}
int @\FuncDecl{shmem\_team\_my\_pe}@(shmem_team_t team);
\end{Csynopsis}

\begin{apiarguments}
\apiargument{IN}{team}{A valid SHMEM team handle.}
\end{apiarguments}

\apidescription{
The \FUNC{shmem\_team\_my\_pe} function returns the number of calling \ac{PE} within the
provided team. The number will be a value between 0 and N-1,
for a team of size N. Each member of the team has a unique number.
For the team \CONST{SHMEM\_TEAM\_WORLD}, this will return the same value
as \FUNC{shmem\_my\_pe}.

Error checking will be done to ensure a valid team handle is provided.
All errors are considered fatal, and will result in the job aborting
with an informative error message.
}

\apireturnvalues{
The number of the calling \ac{PE} within the provided team.
}

\apinotes{
By default, \openshmem creates two predefined teams that will be available
for use once the routine \FUNC{shmem\_init} has been called. These teams can be
referenced in the application by the handles \CONST{SHMEM\_TEAM\_WORLD} and
\CONST{SHMEM\_TEAM\_NODE}. Every PE process is a member of the \CONST{SHMEM\_TEAM\_WORLD}
team, and its number in \CONST{SHMEM\_TEAM\_WORLD} corresponds to the value of its
global \ac{PE} number. The \CONST{SHMEM\_TEAM\_NODE} team contains only the set of \acp{PE}
that reside on the same node as the current \ac{PE}.
}

\end{apidefinition}


\subsubsection{\textbf{SHMEM\_TEAM\_N\_PES}}\label{subsec:shmem_team_n_pes}
\apisummary{
    Returns the total number of \acp{PE} in the provided team.
}

\begin{apidefinition}

\begin{Csynopsis}
int @\FuncDecl{shmem\_team\_n\_pes}@(shmem_team_t team);
\end{Csynopsis}

\begin{apiarguments}
\apiargument{IN}{team}{A valid \openshmem team handle.}
\end{apiarguments}

\apidescription{
The \FUNC{shmem\_team\_n\_pes} function returns the number of \acp{PE} in the
team. This will always be a value between 1 and N, where N is the total number of
\acp{PE} accessible to the \openshmem program. For the team
\LibHandleRef{SHMEM\_TEAM\_WORLD}, this will return the same value as
\FUNC{shmem\_n\_pes}.

Every team must have a least one member. All \acp{PE} in the team
will get back the same value for the team size.

Error checking will be done to ensure a valid team handle is provided.
Errors will result in a return value less than \CONST{0}.
}

\begin{FeedbackRequest}
\apireturnvalues{
Total number of \acp{PE} in the provided team, or a value less than
\CONST{0} if the team handle is invalid.
}
\end{FeedbackRequest}

\apinotes{
By default, \openshmem creates two predefined teams that will be available
for use once the routine \FUNC{shmem\_init} has been called. These teams can be
referenced in the application by the constants \LibHandleRef{SHMEM\_TEAM\_WORLD} and
\LibHandleRef{SHMEM\_TEAM\_NODE}. Every \ac{PE} is a member of the \LibHandleRef{SHMEM\_TEAM\_WORLD}
team, and its number in \LibHandleRef{SHMEM\_TEAM\_WORLD} corresponds to the value of its
global \ac{PE} number. The \LibHandleRef{SHMEM\_TEAM\_NODE} team contains the set of only those
\acp{PE} that reside on the same node as the current \ac{PE}.
}

\end{apidefinition}


\subsubsection{\textbf{SHMEM\_TEAM\_CONFIG\_T}}
\label{subsec:shmem_team_config_t}
\apisummary{
  A structure type representing team configuration arguments
}

\begin{apidefinition}

  \begin{Csynopsis}
typedef struct {
  int disable_collectives;
  int return_local_limit;
  int num_threads;
} shmem_team_config_t;
\end{Csynopsis}

  \vspace{1.0em}

  \apidescription{
    A team configuration argument acts as both input and output to the
    \FUNC{shmem\_team\_split\_*} routines.
    As an input, it specifies the requested capabilities of the team to be
    created.
    As an output, the configuration argument is conditionally updated on
    whether team creation is successful.
    If successful, the configuration argument is not modified;
    if unsuccessful, it is updated to specify the limiting configuration
    parameter(s).

    The \VAR{disable\_collectives} member allows for teams to be created
    without support for collective communications, which allows implementations
    to reduce team creation overheads for those teams.
    When its value is zero, it specifies that the team should have collectives
    enabled.
    When nonzero, the team will not support collective operations, which
    allows implementations to reduce team creation overheads.

    The \VAR{return\_local\_limit} member controls whether, after a failed
    team creation, the team configuration argument is updated with the
    locally restrictive parameter(s) or the most restrictive parameter(s)
    across the \acp{PE} of the new team.
    When its value is zero, the most restrictive parameters are returned;
    otherwise, the locally restrictive parameters are returned.

    The \VAR{num\_threads} member specifies the number of threads that will
    create contexts from the new team.
    It must have a nonnegative value.
    See Section~\ref{sec:ctx} for more on communication contexts and
    Section~\ref{subsec:shmem_team_create_ctx} for team-based context creation.
  }

  \apinotes{
    None.
  }

\end{apidefinition}


\subsubsection{\textbf{SHMEM\_TEAM\_GET\_CONFIG}}\label{subsec:shmem_team_get_config}
\apisummary{
  Return the configuration parameters of a given team
}

\begin{apidefinition}

\begin{Csynopsis}
void @\FuncDecl{shmem\_team\_get\_config}@(shmem_team_t team, shmem_team_config_t *team);
\end{Csynopsis}

\begin{apiarguments}
  \apiargument{IN}{team}{A valid \openshmem team handle.}
  \apiargument{OUT}{config}{
    A pointer to the configuration parameters for the new team.}
\end{apiarguments}

\apidescription{
\FUNC{shmem\_team\_get\_config} returns through the \VAR{config} argument
the configuration parameters of the given team, which were specified when the
team was created.

\begin{FeedbackRequest}
A library implementation must apply all requested options to a team, even in
the event that the library does not make optimizations based on these options.
For example, suppose library implementation must always create teams with the same
overhead, no matter if the program disables collective support during team creation.
The library must still enable the \LibConstRef{SHMEM\_TEAM\_NOCOLLECTIVE} option
when it is requested, so that the \openshmem program will be portable across implementations.
\end{FeedbackRequest}

All \acp{PE} in the team will get back the same parameter values for the team options.

If the \VAR{team} argument does not specify a valid team, the behavior is
undefined.
}

\apireturnvalues{
  None.
}

\apinotes{
A use case for this function is to determine whether a given team can
support collective operations by testing for the \LibConstRef{SHMEM\_TEAM\_NOCOLLECTIVE}
option. When teams are created without support for collectives, they may still use
point to point operations to communicate and synchronize. So programmers may wish
to design frameworks with functions that provide alternative algorithms
for teams based on whether they do or do not support collectives.
}

\end{apidefinition}


\subsubsection{\textbf{SHMEM\_TEAM\_TRANSLATE}}\label{subsec:shmem_team_translate}
\apisummary{
    Translates a given \ac{PE} number to the corresponding \ac{PE} number in another team.
}

\begin{apidefinition}

\begin{Csynopsis}
int @\FuncDecl{shmem\_team\_translate\_pe}@(shmem_team_t src_team, int src_pe,
    shmem_team_t dest_team);
\end{Csynopsis}

\begin{apiarguments}
\apiargument{IN}{src\_team}{A valid SHMEM team handle.}
\apiargument{IN}{src\_pe}{A \ac{PE} number in src\_team.}
\apiargument{IN}{dest\_team}{A valid SHMEM team handle.}
\end{apiarguments}

\apidescription{
The \FUNC{shmem\_team\_translate\_pe} function will translate a given \ac{PE} number
to the corresponding \ac{PE} number in another team.
Specifically, given the \VAR{src\_pe} in \VAR{src\_team}, this function returns that
\ac{PE}'s number in \VAR{dest\_team}. If \VAR{src\_pe} is not a member of both the
\VAR{src\_team} and \VAR{dest\_team}, a value less than \CONST{0} is returned.

If \LibHandleRef{SHMEM\_TEAM\_WORLD} is provided as the \VAR{dest\_team} parameter, this function
acts as a global \ac{PE} number translator and will return the corresponding
\LibHandleRef{SHMEM\_TEAM\_WORLD} number. This may be useful when performing point-to-
point operations between \acp{PE} in a subset, as point-to-point operations
that do not take a context argument require the global \LibHandleRef{SHMEM\_TEAM\_WORLD}
\ac{PE} number.

Error checking will be done to ensure valid team handles are provided.
Errors will result in a return value less than \CONST{0}.
}

\begin{FeedbackRequest}
\apireturnvalues{
The specified \ac{PE}'s number in the \VAR{dest\_team}, or a value less than \CONST{0} if any
team handle arguments are invalid or the \VAR{src\_pe} is not in both the source and destination teams.
}
\end{FeedbackRequest}

\apinotes{
By default, \openshmem creates two predefined teams that will be available
for use once the routine \FUNC{shmem\_init} has been called. These teams can be
referenced in the application by the constants \LibHandleRef{SHMEM\_TEAM\_WORLD} and
\LibHandleRef{SHMEM\_TEAM\_NODE}. Every \ac{PE} is a member of the \LibHandleRef{SHMEM\_TEAM\_WORLD}
team, and its number in \LibHandleRef{SHMEM\_TEAM\_WORLD} corresponds to the value of its
global \ac{PE} number. The \LibHandleRef{SHMEM\_TEAM\_NODE} team contains the set of only those
\acp{PE} that reside on the same node as the current \ac{PE}.
}

\end{apidefinition}


\subsubsection{\textbf{SHMEM\_TEAM\_SPLIT\_STRIDED}}\label{subsec:shmem_team_split_strided}
\apisummary{
Create a new \openshmem team from a subset of the existing parent team \acp{PE},
where the subset is defined by the
\ac{PE} triplet (\VAR{PE\_start}, \VAR{PE\_stride}, and \VAR{PE\_size}) supplied to the routine.}

\begin{apidefinition}

\begin{Csynopsis}
int @\FuncDecl{shmem\_team\_split\_strided}@(shmem_team_t parent_team, int PE_start, int PE_stride,
     int PE_size, shmem_team_config_t *config, long config_mask, shmem_team_t *new_team);
\end{Csynopsis}

\begin{apiarguments}
\apiargument{IN}{parent\_team}{A valid \openshmem team. The predefined teams, such as
\LibHandleRef{SHMEM\_TEAM\_WORLD}, may be used, or any team created by the user.}

\apiargument{IN}{PE\_start}{The lowest \ac{PE} number of the subset of \acp{PE} from
the parent team that will form the new team}

\apiargument{IN}{PE\_stride}{The stride between team \ac{PE}
numbers in the parent team that comprise the subset of \acp{PE} that will form
the new team.}

\apiargument{IN}{PE\_size}{The number of \acp{PE} from the parent team in the subset
of \acp{PE} that will form the new team.}

\apiargument{INOUT}{config}{
  A pointer to the configuration parameters for the new team.}

\apiargument{IN}{config\_mask}{
  The bitwise mask representing the set of configuration parameters to use
  from \VAR{config}.}

\apiargument{OUT}{new\_team}{A new \openshmem team handle, representing a \ac{PE}
subset of all the \acp{PE} in the parent team that is created from
the \ac{PE} triplet provided.}

\end{apiarguments}

\apidescription{
The \FUNC{shmem\_team\_split\_strided} routine is a collective routine.
It creates a new \openshmem team from a subset of the existing parent team,
where the subset is defined by the \ac{PE} triplet (\VAR{PE\_start},
\VAR{PE\_stride}, and \VAR{PE\_size}) supplied to the routine.

This routine must be called by all processes contained in the \ac{PE} triplet
specification. It may be called by additional \acp{PE} not included in the
triplet specification, but for those processes a \VAR{new\_team} value of
\LibConstRef{SHMEM\_TEAM\_NULL} is returned. All calling processes must provide the
same values for the \ac{PE} triplet. This routine will return a \VAR{new\_team}
containing the \ac{PE} subset specified by the triplet, and ordered by the
existing global \ac{PE} number. None of the parameters need to reside in
symmetric memory.

The \VAR{config} argument specifies team configuration parameters, which are
described in Section~\ref{subsec:shmem_team_config_t}.

The \VAR{config\_mask} argument is a bitwise mask representing the set of
configuration parameters to use from \VAR{config}.
A \VAR{config\_mask} value of \CONST{0} indicates that the team
should be created with the default values for all configuration parameters.
See Section~\ref{subsec:shmem_team_config_t} for field mask names and
default configuration parameters.

If \VAR{parent\_team} is an invalid team handle, the behavior is undefined.

If \VAR{parent\_team} compares equal to \LibConstRef{SHMEM\_TEAM\_NULL}, then no
new team will be created, and  \VAR{new\_team} will be assigned the value
\LibConstRef{SHMEM\_TEAM\_NULL}.

If an invalid \ac{PE} triplet is provided, then the \VAR{new\_team} will not be created.

If \VAR{new\_team} cannot be created, then it will be assigned the value
\LibConstRef{SHMEM\_TEAM\_NULL}.
}

\apireturnvalues{
  Zero on successful creation of \VAR{new\_team}, nonzero otherwise.
}

\apinotes{
  It is important to note the use of the less restrictive
  \VAR{PE\_stride} argument instead of \VAR{logPE\_stride}. This method of
  creating a team with an arbitrary set of \acp{PE} is inherently restricted
  by its parameters, but allows for many additional use-cases over using a
  \VAR{logPE\_stride} parameter, and may provide an easier transition for
  existing \openshmem programs to create and use \openshmem teams.

  See the description of team handles and predefined teams at the top of
  Section~\ref{subsec:team} for more information about semantics and usage.
}

\begin{apiexamples}

\end{apiexamples}

\end{apidefinition}


\subsubsection{\textbf{SHMEM\_TEAM\_SPLIT\_2D}}\label{subsec:shmem_team_split_2d}
\apisummary{
Create two new teams by splitting an existing parent team into two subsets based on a
2D Cartesian space defined by xrange argument and the yrange derived from xrange
and the parent team size. These ranges describe the Cartesian space in X and Y
dimensions.}

\begin{apidefinition}

\begin{Csynopsis}
void @\FuncDecl{shmem\_team\_split\_2d}@(shmem_team_t parent_team, int xrange,
long xaxis_options, shmem_team_t *xaxis_team, long yaxis_options, shmem_team_t *yaxis_team);
\end{Csynopsis}

\begin{apiarguments}
\apiargument{IN}{parent\_team}{A valid \openshmem team. Any predefined teams, such as
\LibHandleRef{SHMEM\_TEAM\_WORLD}, may be used, or any team created by the user.}

\apiargument{IN}{xrange}{A non-negative integer representing the number of
elements in the first dimension.}

\apiargument{IN}{xaxis\_options}{The set of options requested for the new X-axis team.
Multiple options may be requested by combining them with a bitwise OR operation;
otherwise, \CONST{0} can be given if no options are requested.}

\apiargument{OUT}{xaxis\_team}{A new \ac{PE} team handle representing a \ac{PE}
subset consisting of all the \acp{PE} that have the same coordinate along the X-axis 
as the calling \ac{PE}.}

\apiargument{IN}{yaxis\_options}{The set of options requested for the new Y-axis team.
These options do not have to be the same as the options for the new X-axis team.}

\apiargument{OUT}{yaxis\_team}{A new \ac{PE} team handle representing a \ac{PE}
subset consisting of all the \acp{PE} that have the same coordinate along the Y-axis 
as the calling \ac{PE}.}
\end{apiarguments}

\apidescription{
The \FUNC{shmem\_team\_split\_2d} routine is a collective routine. It creates two
new teams by splitting an existing parent team into up to two subsets based on a
2D Cartesian space. The user provides the size of the X dimension, which is then
used to derive the size of the Y dimension based on the size of the parent team.
The size of the Y dimension will be equal to $ceiling ( N \div xrange )$, where
\VAR{N} is the size of the parent team. In other words,
$xrange \times yrange \geq N$, so that every \ac{PE} in the parent team has a
unique \VAR{(x,y)} location the 2D Cartesian space.

After the split operation, each of the new teams will contain all \acp{PE} that
have the same coordinate along the X-axis and Y-axis, respectively, as the calling
\ac{PE}. The \acp{PE} are numbered in the new teams based on the position of the
\ac{PE} along the given axis.

Any valid \openshmem team can be used as the parent team. This routine must be
called by all \acp{PE} in the parent team. The value of \VAR{xrange} must be
non-negative and all \acp{PE} in the parent team must pass the same value for
\VAR{xrange}. None of the parameters need to reside in symmetric memory.

Error checking will be done to detect if the value \LibConstRef{SHMEM\_TEAM\_NULL}
is passed as the parent team handle. In that case, the program will abort with
an informative error message. If an invalid team handle is passed to the
routine, the behavior is undefined.

The options in the table below can be supplied during team split to restrict
team functions and enable performance optimizations. All \acp{PE} that will
be in the same resultant new team must specify the same options. The \acp{PE}
in the parent team \emph{do not} have to all provide the same options for new
teams.

When using a given team, the application must comply with the requirements
of all options set on that team; otherwise, the behavior is undefined.
No options are enabled on \LibHandleRef{SHMEM\_TEAM\_WORLD} or on other
predefined \openshmem teams.

    \apitablerow{\LibConstRef{SHMEM\_TEAM\_NOCOLLECTIVE}}{
                 The new team will not be created with the necessary support
                 structures to enable team based collectives.
                 This will typically allow implementations to speed up team creation
                 and reduce \openshmem library footprint for teams with this option.
                 This option will not prevent the new team from using atomics or
                 other non-collective team based operations.}
}

\apireturnvalues{
None.
}

\apinotes{
Since the split may result in a 2D space with more points than there are members of
the parent team, there may be a final, incomplete row of the 2D mapping of the parent
team. This means that the resultant X-axis teams may vary in size by up to 1 \ac{PE},
and that there may be one resultant Y-axis team of smaller size than all of the other
Y-axis teams.

The following grid shows the 12 teams that would result from splitting a parent team
of size 10 with \VAR{xrange} of 3. The numbers in the grid cells are the \ac{PE} numbers
in the parent team. The rows are the Y-axis teams. The columns are the X-axis teams.

\begin{center}
\begin{tabular}{|l|l|l|l|}
 \hline
      & x=0 & x=1 & x=2  \\ \hline
 y=0  & 0   & 1   & 2  \\ \hline
 y=1  & 3   & 4   & 5  \\ \hline
 y=2  & 6   & 7   & 8  \\ \hline
 y=3  & 9     \\
 \cline{0-1} 
\end{tabular}
\end{center}

It would be legal, for example, if \acp{PE} 0, 3, 6, 9 specified a different value
for \VAR{xaxis\_options} than all of the other \acp{PE}, as long as options match
for all \acp{PE} in each of the new teams.

See the description of team handles and predefined teams at the top of section
\ref{subsec:team} for more information about team handle semantics and usage.
}

\begin{apiexamples}

\end{apiexamples}

\end{apidefinition}


\subsubsection{\textbf{SHMEM\_TEAM\_DESTROY}}\label{subsec:shmem_team_destroy}
\apisummary{
    Destroys existing team.
}

\begin{apidefinition}

\begin{Csynopsis}
int @\FuncDecl{shmem\_team\_destroy}@(shmem_team_t team);
\end{Csynopsis}

\begin{apiarguments}
\apiargument{IN}{team}{A valid SHMEM team handle.}
\end{apiarguments}

\apidescription{
The shmem\_team\_destroy function destroys an existing team. This is a
collective call, in which every member of the team being destroyed needs
to participate. This will free all internal memory structures associated
with the team and invalidate the team handle. Upon return, the team
handle is set to \CONST{SHMEM\_TEAM\_NULL}, after which it can no longer be
used for team API calls.

It is considered erroneous to free \CONST{SHMEM\_TEAM\_WORLD} or
\CONST{SHMEM\_TEAM\_NODE}. Error checking will be done to ensure a valid
team handle is provided. All errors are considered fatal, and will
result in the job aborting with an informative error message.
}

\apireturnvalues{
None.
}

\apinotes{
Note that SHMEM team handles have local semantics only. That is, team
handles should not be stored in shared variables and used across other
processes. Doing so will result in unpredictable behavior.
}

\end{apidefinition}

\color{Black}




\subsection{Communication Management Routines}
\label{sec:ctx}
All \openshmem RMA, AMO, and memory ordering routines are
performed on a communication context.  The communication context defines an
independent ordering and completion environment, allowing users to manage the
overlap of communication with computation and also to manage communication
operations performed by separate threads within a multithreaded \ac{PE}.  For
example, in single-threaded environments, contexts may be used to pipeline
communication and computation.  In multithreaded environments, contexts may
additionally provide thread isolation, eliminating overheads resulting from
thread interference.

Context handles are of type \CTYPE{shmem\_ctx\_t} and are valid for
language-level assignment and equality comparison.  A handle to the desired context is
passed as an argument in the \Cstd \CTYPE{shmem\_ctx\_*} and type-generic API
routines.  API routines that do not accept a context argument operate on the
default context.  The default context can be used explicitly through the
\LibHandleRef{SHMEM\_CTX\_DEFAULT} handle.

\newtext{
Every communication context is associated with a team.
This association is established at context creation.
Communication contexts created by \FUNC{shmem\_ctx\_create} are
associated with the default team, while contexts created by
\FUNC{shmem\_team\_create\_ctx} are associated with a team specified at
context creation.
The default context is associated with the default team.
A context's associated team specifies the set of \acp{PE} over which
\ac{PE}-specific routines that operate on a communication context,
explicitly or implicitly, are performed.
All point-to-point routines that operate on this context will do so with
respect to the team-relative \ac{PE} numbering of the associated team.
All collective routines that operate on this context will do so across
the associated team.
}

\subsubsection{\textbf{SHMEM\_CTX\_CREATE}}
\label{subsec:shmem_ctx_create}
\apisummary{
    Create a communication context \newtext{locally}.
}

\begin{apidefinition}

\begin{Csynopsis}
int @\FuncDecl{shmem\_ctx\_create}@(long options, shmem_ctx_t *ctx);
\end{Csynopsis}

\begin{apiarguments}
    \apiargument{IN}{options}{The set of options requested for the given context.
        Multiple options may be requested by combining them with a bitwise
        OR operation; otherwise, \CONST{0} can be given if no options are
        requested.}
    \apiargument{OUT}{ctx}{A handle to the newly created context.}
\end{apiarguments}

\apidescription{
    The \FUNC{shmem\_ctx\_create} routine creates a new communication context
    and returns its handle through the \VAR{ctx} argument.  If the context was
    created successfully, a value of zero is returned
    \newtext{and the context handle pointed to by \VAR{ctx} specifies a valid context};
    otherwise, a nonzero value is returned
    \newtext{and the context handle pointed to by \VAR{ctx} is not modified}.
    An unsuccessful context
    creation call is not treated as an error and the \openshmem library remains
    in a correct state.  The creation call can be reattempted with different
    options or after additional resources become available.

    \newtext{
    A newly created communication context has an initial association with the
    default team.
    All \openshmem routines that operate on this context will do so with
    respect to the associated \ac{PE} team.
    That is, all point-to-point routines operating on this context will use
    team-relative \ac{PE} numbering.
    }

    By default, contexts are {\em shareable} and, when it is allowed by the
    threading model provided by the \openshmem library, they can be used concurrently by
    multiple threads within the PE where they were created.
    %
    The following options can be supplied during context creation to restrict
    this usage model and enable performance optimizations.  When using a given
    context, the application must comply with the requirements of all options
    set on that context; otherwise, the behavior is undefined.
    No options are enabled on the default context.

        \apitablerow{\LibConstRef{SHMEM\_CTX\_SERIALIZED}}{
            The given context is shareable; however, it will not be used by multiple threads
            concurrently.  When the \CONST{SHMEM\_CTX\_SERIALIZED} option is
            set, the user must ensure that operations involving the given
            context are serialized by the application.}

        \apitablerow{\LibConstRef{SHMEM\_CTX\_PRIVATE}}{
            The given context will be used only by the thread that created it.}

        \apitablerow{\LibConstRef{SHMEM\_CTX\_NOSTORE}}{
            Quiet and fence operations performed on the given context are not
            required to enforce completion and ordering of memory store
            operations.
            When ordering of store operations is needed, the application must
            perform a synchronization operation on a context without the
            \CONST{SHMEM\_CTX\_NOSTORE} option enabled.}

}

\apireturnvalues{
    Zero on success and nonzero otherwise.
}

\apinotes{
    None.
}

\end{apidefinition}



\subsubsection{\textbf{SHMEM\_CTX\_DESTROY}}
\label{subsec:shmem_ctx_destroy}
\apisummary{
    Destroy a \newtext{locally created} communication context.
}

\begin{apidefinition}

\begin{Csynopsis}
void @\FuncDecl{shmem\_ctx\_destroy}@(shmem_ctx_t ctx);
\end{Csynopsis}

\begin{apiarguments}
    \apiargument{IN}{ctx}{Handle to the context that will be destroyed.}
\end{apiarguments}

\apidescription{
    \FUNC{shmem\_ctx\_destroy} destroys a context that was created by a call to
    \FUNC{shmem\_ctx\_create} or \FUNC{shmem\_team\_create\_ctx}.
    It is the user's responsibility to ensure that
    the context is not used after it has been destroyed, for example when the
    destroyed context is used by multiple threads.  This function
    performs an implicit quiet operation on the given context before it is freed.

    \newtext{
    If \VAR{ctx} is a handle to the default context, the behavior is undefined.
    }
}

\apireturnvalues{
    None.
}

\apinotes{
    \oldtext{
    It is invalid to pass \CONST{SHMEM\_CTX\_DEFAULT} to this routine.
    }

    Destroying a context makes it impossible for the user to complete
    communication operations that are pending on that context.  This includes
    nonblocking communication operations, whose local buffers are only returned
    to the user after the operations have been completed.  An implicit quiet is
    performed when freeing a context to avoid this ambiguity.

    A context with the \CONST{SHMEM\_CTX\_PRIVATE} option enabled must be
    destroyed by the thread that created it.
}

\begin{apiexamples}

    \apicexample
    {The following example demonstrates the use of contexts in a multithreaded
    \Cstd[11] program that uses OpenMP for threading.  This example shows the
    shared counter load balancing method and illustrates the use of contexts
    for thread isolation.}
    {./example_code/shmem_ctx.c}
    {}

    \apicexample
    {The following example demonstrates the use of contexts in a
    single-threaded \Cstd[11] program that performs a summation reduction where
    the data contained in the \VAR{in\_buf} arrays on all \acp{PE} is reduced into
    the \VAR{out\_buf} arrays on all \acp{PE}.  The buffers are divided into
    segments and processing of the segments is pipelined.  Contexts are used
    to overlap an all-to-all exchange of data for segment \VAR{p} with the
    local reduction of segment \VAR{p-1}.}
    {./example_code/shmem_ctx_pipelined_reduce.c}
    {}

\end{apiexamples}

\end{apidefinition}



\newtext{
\subsubsection{\textbf{SHMEM\_CTX\_GET\_TEAM}}
\label{subsec:shmem_ctx_get_team}
\apisummary{
  Retrieve the team associated with the communication context.
}

\begin{apidefinition}

  \begin{Csynopsis}
int @\FuncDecl{shmem\_ctx\_get\_team}@(shmem_ctx_t ctx, shmem_team_t *team);
  \end{Csynopsis}

  \begin{apiarguments}

    \apiargument{IN}{ctx}{
      A handle to a communication context.
    }

    \apiargument{OUT}{team}{
      A pointer to a handle to the associated \ac{PE} team.
    }

  \end{apiarguments}

  \apidescription{
    The \FUNC{shmem\_ctx\_get\_team} routine returns a handle to the \ac{PE}
    team associated with the specified communication context \VAR{ctx}.
    The team handle is returned through the pointer argument \VAR{team}.

    If \VAR{ctx} is the default context, the returned team is guaranteed
    to be \CONST{SHMEM\_TEAM\_WORLD}.

    If \VAR{ctx} is an invalid context, the argument \VAR{team} is not
    modified and a value of \CONST{-1} is returned.

    If \VAR{team} is a null pointer, a value of \CONST{-1} is returned.
  }

  \apireturnvalues{
    Zero on success; otherwise, \CONST{-1}.
  }

  \apinotes{
    None.
  }

\end{apidefinition}

}

\newtext{
\subsubsection{\textbf{SHMEM\_TEAM\_CREATE\_CTX}}
\label{subsec:shmem_team_create_ctx}
\apisummary{
  Create a communication context collectively.
}

\begin{apidefinition}

\begin{Csynopsis}
int @\FuncDecl{shmem\_team\_create\_ctx}@(shmem_team_t team, long options, shmem_ctx_t *ctx);
\end{Csynopsis}

\begin{apiarguments}
  \apiargument{IN}{team}{A handle to the specified \ac{PE} team.}
  \apiargument{IN}{options}{
    The set of options requested for the given context.
    Multiple options may be requested by combining them with a bitwise OR
    operation; otherwise, \CONST{0} can be given if no options are requested.}
  \apiargument{OUT}{ctx}{A handle to the newly created context.}
\end{apiarguments}

\apidescription{
  The \FUNC{shmem\_team\_create\_ctx} routine creates a new communication
  context and returns its handle through the \VAR{ctx} argument.
  This context is created collectively by all \acp{PE} in the team
  specified by the \VAR{team} argument.
  The specified team may not have the \LibConstRef{SHMEM\_TEAM\_NOCOLLECTIVE}
  option enabled; otherwise, the behavior is undefined.

  %% All \openshmem routines that operate on this context will do so with
  %% respect to the associated \ac{PE} team.
  %% That is, all point-to-point routines operating on this context will use
  %% team-relative \ac{PE} numbering.

  In addition to the team, the \FUNC{shmem\_team\_create\_ctx} routine accepts
  the same arguments and provides all the same return conditions as the
  \FUNC{shmem\_ctx\_create} routine.
  The call is either collectively successful or collectively fails across
  all \acp{PE} in the team.

  As \FUNC{shmem\_team\_create\_ctx} is collective, it includes a call to a
  procedure semantically equivalent to \FUNC{shmem\_team\_sync} on both entry
  and exit.
}

\apireturnvalues{
  Zero on success and nonzero otherwise.
}

\apinotes{
  Depending on the \openshmem implementation, system configuration, and
  application communication pattern, some applications may observe higher
  performance with collectively created contexts than with locally created
  contexts.
}

\end{apidefinition}

}

\newtext{
\subsubsection{\textbf{SHMEM\_TEAM\_DESTROY\_CTX}}
\label{subsec:shmem_team_destroy_ctx}
\apisummary{
  Destroy a collectively created communication context.
}

\begin{apidefinition}

\begin{Csynopsis}
void @\FuncDecl{shmem\_team\_destroy\_ctx}@(shmem_ctx_t ctx);
\end{Csynopsis}

\begin{apiarguments}
  \apiargument{IN}{ctx}{Handle to the context that will be destroyed.}
\end{apiarguments}

\apidescription{
  \FUNC{shmem\_team\_destroy\_ctx} collectively destroys a context that was
  created by a call to \FUNC{shmem\_team\_create\_ctx}.
  It is the user's responsibility to ensure that the context is not used
  after it has been destroyed.

  As \FUNC{shmem\_team\_create\_ctx} is collective, it includes calls to
  procedures semantically equivalent to \FUNC{shmem\_team\_barrier} on entry
  and \FUNC{shmem\_team\_sync} on exit.

  It is invalid to pass \CONST{SHMEM\_CTX\_DEFAULT} or a context handle
  returned by a call to \FUNC{shmem\_ctx\_create} to this routine.
}

\apireturnvalues{
  None.
}

\apinotes{
  None.
}

\end{apidefinition}

}


\subsection{Remote Memory Access Routines}\label{sec:rma}
The \ac{RMA} routines described in this section can be used to perform
reads from and writes to symmetric data objects. These operations
are one-sided, meaning that the \ac{PE} invoking an operation provides all
communication parameters and the targeted \ac{PE} is passive. A characteristic
of one-sided communication is that it decouples communication from
synchronization. One-sided communication mechanisms transfer data; however,
they do not synchronize the sender of the data with the receiver of the data.

\openshmem \ac{RMA} routines are performed on symmetric data objects.  The
initiator \ac{PE} of a call is designated as the \emph{origin} \ac{PE} and the
\ac{PE} targeted by an operation is designated as the \emph{destination} \ac{PE}.  The
\source{} and \dest{} designators refer to the data objects that an operation
reads from and writes to.  In the case of the remote update routine, \PUT{},
the origin \ac{PE} provides the \source{} data object and the destination
\ac{PE} provides the \dest{} data object. In the case of the remote read
routine, \GET{}, the origin \ac{PE} provides the \dest{} data object and the
destination \ac{PE} provides the \source{} data object.

Where appropriate compiler support is available, \openshmem provides type-generic 
one-sided communication interfaces via \Cstd[11] generic selection
(\Cstd[11]~\S6.5.1.1\footnote{Formally, the \Cstd[11] specification is ISO/IEC 9899:2011(E).})
for block, scalar, and block-strided put and get communication. 
Such type-generic routines are supported for the ``standard \ac{RMA} types''
listed in Table \ref{stdrmatypes}.

The standard \ac{RMA} types include the exact-width integer types defined in
\HEADER{stdint.h} by \Cstd[99]%
\footnote{Formally, the \Cstd[99] specification is ISO/IEC~9899:1999(E).}%
~\S7.18.1.1 and \Cstd[11]~\S7.20.1.1. When the \Cstd translation environment
does not provide exact-width integer types with \HEADER{stdint.h}, an
\openshmem implemementation is not required to provide support for these types.

\begin{table}[h]
  \begin{center}
    \begin{tabular}{|l|l|}
      \hline
      \TYPE              & \TYPENAME  \\ \hline
      float              & float      \\ \hline
      double             & double     \\ \hline
      long double        & longdouble \\ \hline
      char               & char       \\ \hline
      signed char        & schar      \\ \hline
      short              & short      \\ \hline
      int                & int        \\ \hline
      long               & long       \\ \hline
      long long          & longlong   \\ \hline
      unsigned char      & uchar      \\ \hline
      unsigned short     & ushort     \\ \hline
      unsigned int       & uint       \\ \hline
      unsigned long      & ulong      \\ \hline
      unsigned long long & ulonglong  \\ \hline
      int8\_t            & int8       \\ \hline
      int16\_t           & int16      \\ \hline
      int32\_t           & int32      \\ \hline
      int64\_t           & int64      \\ \hline
      uint8\_t           & uint8      \\ \hline
      uint16\_t          & uint16     \\ \hline
      uint32\_t          & uint32     \\ \hline
      uint64\_t          & uint64     \\ \hline
      size\_t            & size       \\ \hline
      ptrdiff\_t         & ptrdiff    \\ \hline
    \end{tabular}
    \TableCaptionRef{Standard \ac{RMA} Types and Names}
    \label{stdrmatypes}
  \end{center} 
\end{table}


\subsubsection{\textbf{SHMEM\_PUT}}\label{subsec:shmem_put}
\apisummary{
    The  put routines  provide  a method for copying data from a contiguous local
    data object to a data object on a specified \ac{PE}.
}

\begin{apidefinition}

\begin{C11synopsis}
void @\FuncDecl{shmem\_put}@(TYPE *dest, const TYPE *source, size_t nelems, int pe);
void @\FuncDecl{shmem\_put}@(shmem_ctx_t ctx, TYPE *dest, const TYPE *source, size_t nelems, int pe);
\end{C11synopsis}
where \TYPE{} is one of the standard \ac{RMA} types specified by Table \ref{stdrmatypes}.

\begin{Csynopsis}
void @\FuncDecl{shmem\_\FuncParam{TYPENAME}\_put}@(TYPE *dest, const TYPE *source, size_t nelems, int pe);
void @\FuncDecl{shmem\_ctx\_\FuncParam{TYPENAME}\_put}@(shmem_ctx_t ctx, TYPE *dest, const TYPE *source, size_t nelems, int pe);
\end{Csynopsis}
where \TYPE{} is one of the standard \ac{RMA} types and has a corresponding \TYPENAME{} specified by Table \ref{stdrmatypes}.

\begin{CsynopsisCol}
void @\FuncDecl{shmem\_put\FuncParam{SIZE}}@(void *dest, const void *source, size_t nelems, int pe);
void @\FuncDecl{shmem\_ctx\_put\FuncParam{SIZE}}@(shmem_ctx_t ctx, void *dest, const void *source, size_t nelems, int pe);
\end{CsynopsisCol}
where \SIZE{} is one of \CONST{8, 16, 32, 64, 128}.

\begin{CsynopsisCol}
void @\FuncDecl{shmem\_putmem}@(void *dest, const void *source, size_t nelems, int pe);
void @\FuncDecl{shmem\_ctx\_putmem}@(shmem_ctx_t ctx, void *dest, const void *source, size_t nelems, int pe);
\end{CsynopsisCol}

\begin{Fsynopsis}
CALL @\FuncDecl{SHMEM\_CHARACTER\_PUT}@(dest, source, nelems, pe)
CALL @\FuncDecl{SHMEM\_COMPLEX\_PUT}@(dest, source, nelems, pe)
CALL @\FuncDecl{SHMEM\_DOUBLE\_PUT}@(dest, source, nelems, pe)
CALL @\FuncDecl{SHMEM\_INTEGER\_PUT}@(dest, source, nelems, pe)
CALL @\FuncDecl{SHMEM\_LOGICAL\_PUT}@(dest, source, nelems, pe)
CALL @\FuncDecl{SHMEM\_PUT4}@(dest, source, nelems, pe)
CALL @\FuncDecl{SHMEM\_PUT8}@(dest, source, nelems, pe)
CALL @\FuncDecl{SHMEM\_PUT32}@(dest, source, nelems, pe)
CALL @\FuncDecl{SHMEM\_PUT64}@(dest, source, nelems, pe)
CALL @\FuncDecl{SHMEM\_PUT128}@(dest, source, nelems, pe)
CALL @\FuncDecl{SHMEM\_PUTMEM}@(dest, source, nelems, pe)
CALL @\FuncDecl{SHMEM\_REAL\_PUT}@(dest, source, nelems, pe)
\end{Fsynopsis}

\begin{apiarguments}
    \apiargument{IN}{ctx}{A context handle specifying the context on which to perform the operation.
      When this argument is not provided, the operation is performed on
      the default context.}
    \apiargument{OUT}{dest}{Data object to be updated on the remote \ac{PE}. This
    data object must be remotely accessible.}
    \apiargument{IN}{source}{Data object containing the data to be copied.}
    \apiargument{IN}{nelems}{Number of elements in the \VAR{dest} and \VAR{source}
    arrays. \VAR{nelems} must be of type \VAR{size\_t} for \Cstd. When using
    \Fortran, it must be a constant, variable, or array element of default
    integer type.}
    \apiargument{IN}{pe}{\ac{PE} number of the remote \ac{PE}. \VAR{pe} must be
    of type integer. When using \Fortran, it must be a constant, variable,
    or array element of default integer type.}
\end{apiarguments}

\apidescription{
    The routines return after the data has been copied out of the \source{} array
    on the local \ac{PE}.  The delivery of data words into the data object on the
    destination \ac{PE} may occur in any order.  Furthermore, two successive put
    routines may deliver data out of order unless a call to \FUNC{shmem\_fence} is
    introduced between the two calls.
    If the context handle \VAR{ctx} does not correspond to a valid context,
    the behavior is undefined.
 }

\apidesctable{
    The \dest{} and \source{} data objects must conform to certain typing
    constraints, which are as follows:}
    {Routine}{Data type of \VAR{dest} and \VAR{source}}
    \apitablerow{shmem\_putmem}{\Fortran: Any noncharacter type. \Cstd: Any
        data  type.  nelems is scaled in bytes.}
    \apitablerow{shmem\_put4, shmem\_put32}{Any noncharacter type
        that has a storage size equal to \CONST{32} bits.}
    \apitablerow{shmem\_put8}{\Cstd: Any noncharacter type that
        has a storage size equal to \CONST{8} bits.}
    \apitablerow{}{\Fortran: Any noncharacter type that
        has a storage size equal to \CONST{64} bits.}
    \apitablerow{shmem\_put64}{Any noncharacter type that
        has a storage size equal to \CONST{64} bits.}
    \apitablerow{shmem\_put128}{Any noncharacter type that has a
        storage size equal to \CONST{128} bits.}
    \apitablerow{SHMEM\_CHARACTER\_PUT}{Elements of type character.  \VAR{nelems}
    is  the number  of	 characters to transfer. The actual character lengths of
    the \source{} and \dest{} variables are ignored. }
    \apitablerow{SHMEM\_COMPLEX\_PUT}{Elements of type complex of default size.}
    \apitablerow{SHMEM\_DOUBLE\_PUT}{Elements of type double precision. }
    \apitablerow{SHMEM\_INTEGER\_PUT}{Elements of type integer.}
    \apitablerow{SHMEM\_LOGICAL\_PUT}{Elements of type logical.}
    \apitablerow{SHMEM\_REAL\_PUT}{Elements of type real.}

\apireturnvalues{
    None.
}
\apinotes{
    When using \Fortran, data types must be of default size.  For example,
    a real variable must be declared as \CONST{REAL},  \CONST{REAL*4},  or
    \CONST{REAL(KIND=KIND(1.0))}.
    As of \openshmem[1.2], the \Fortran API routine \FUNC{SHMEM\_PUT} has
    been deprecated, and either \FUNC{SHMEM\_PUT8} or \FUNC{SHMEM\_PUT64} should
    be used in its place.
}

\begin{apiexamples}

\apicexample
    { The following \FUNC{shmem\_put} example is for \Cstd[11] programs:}
    {./example_code/shmem_put_example.c}
    {} 
\end{apiexamples}

\end{apidefinition}


\subsubsection{\textbf{SHMEM\_P}}\label{subsec:shmem_p}
\apisummary{
    Copies one data item to a remote \ac{PE}.
}

\begin{apidefinition}

\begin{C11synopsis}
void @\FuncDecl{shmem\_p}@(TYPE *dest, TYPE value, int pe);
void @\FuncDecl{shmem\_p}@(shmem_ctx_t ctx, TYPE *dest, TYPE value, int pe);
\end{C11synopsis}
where \TYPE{} is one of the standard \ac{RMA} types specified by Table \ref{stdrmatypes}.

\begin{Csynopsis}
void @\FuncDecl{shmem\_\FuncParam{TYPENAME}\_p}@(TYPE *dest, TYPE value, int pe);
void @\FuncDecl{shmem\_ctx\_\FuncParam{TYPENAME}\_p}@(shmem_ctx_t ctx, TYPE *dest, TYPE value, int pe);
\end{Csynopsis}
where \TYPE{} is one of the standard \ac{RMA} types and has a corresponding \TYPENAME{} specified by Table \ref{stdrmatypes}.

\begin{apiarguments}
  \apiargument{IN}{ctx}{\oldtext{The context on which to perform the operation.} \newtext{A context handle specifying the context on which to perform the operation.}
    When this argument is not provided, the operation is performed on
    \oldtext{\CONST{SHMEM\_CTX\_DEFAULT}} \newtext{the default context}.}
  \apiargument{OUT}{dest}{The remotely accessible array element or scalar data object
    which will receive the data on the remote \ac{PE}.}
  \apiargument{IN}{value}{The value to be transferred to \VAR{dest} on the
    remote \ac{PE}.}
  \apiargument{IN}{pe}{The number of the remote \ac{PE}.}
\end{apiarguments}

\apidescription{
    These routines provide a very low latency put capability for single elements of
    most basic types.
    
    As with \FUNC{shmem\_put}, these routines start the remote transfer and may
    return before the data is delivered to the remote \ac{PE}.  Use
    \FUNC{shmem\_quiet} to force completion of all remote \PUT{} transfers.

    \newtext{
    If the context handle \VAR{ctx} does not correspond to a valid context,
    the behavior is undefined.
    }
}

\apireturnvalues{
    None.
}

\apinotes{
    None.
}

\begin{apiexamples}

    \apicexample
    {The following example uses \FUNC{shmem\_p} in a \Cstd[11] program.}
    {./example_code/shmem_p_example.c}
    {}

\end{apiexamples}

\end{apidefinition}


\subsubsection{\textbf{SHMEM\_IPUT}}\label{subsec:shmem_iput}
\apisummary{
    Copies strided data to a specified \ac{PE}.
}

\begin{apidefinition}

\begin{C11synopsis}
void @\FuncDecl{shmem\_iput}@(TYPE *dest, const TYPE *source, ptrdiff_t dst, ptrdiff_t sst, size_t nelems, int pe);
void @\FuncDecl{shmem\_iput}@(shmem_ctx_t ctx, TYPE *dest, const TYPE *source, ptrdiff_t dst, ptrdiff_t sst, size_t nelems, int pe);
\end{C11synopsis}
where \TYPE{} is one of the standard \ac{RMA} types specified by Table \ref{stdrmatypes}.

\begin{Csynopsis}
void @\FuncDecl{shmem\_\FuncParam{TYPENAME}\_iput}@(TYPE *dest, const TYPE *source, ptrdiff_t dst, ptrdiff_t sst, size_t nelems, int pe);
void @\FuncDecl{shmem\_ctx\_\FuncParam{TYPENAME}\_iput}@(shmem_ctx_t ctx, TYPE *dest, const TYPE *source, ptrdiff_t dst, ptrdiff_t sst, size_t nelems, int pe);
\end{Csynopsis}
where \TYPE{} is one of the standard \ac{RMA} types and has a corresponding \TYPENAME{} specified by Table \ref{stdrmatypes}.

\begin{CsynopsisCol}
void @\FuncDecl{shmem\_iput\FuncParam{SIZE}}@(void *dest, const void *source, ptrdiff_t dst, ptrdiff_t sst, size_t nelems, int pe);
void @\FuncDecl{shmem\_ctx\_iput\FuncParam{SIZE}}@(shmem_ctx_t ctx, void *dest, const void *source, ptrdiff_t dst, ptrdiff_t sst, size_t nelems, int pe);
\end{CsynopsisCol}
where \SIZE{} is one of \CONST{8, 16, 32, 64, 128}.

\begin{apiarguments}
    \apiargument{IN}{ctx}{A context handle specifying the context on which to perform the operation.
        When this argument is not provided, the operation is performed on
        the default context.}
    \apiargument{OUT}{dest}{Array to be updated on the remote \ac{PE}. This data
        object  must be remotely accessible.}
    \apiargument{IN}{source}{Array containing the data to be copied.}
    \apiargument{IN}{dst}{The stride between consecutive elements of the \dest{}
        array.  The stride is scaled by the element size of the \dest{} array.  A
        value of \CONST{1} indicates contiguous data.  \VAR{dst} must be of type
        \CTYPE{ptrdiff\_t}.}
    \apiargument{IN}{sst}{The  stride between consecutive elements of the
        \source{} array.  The stride is scaled by the element size of the \source{}
        array.  A  value of \CONST{1} indicates contiguous data.  \VAR{sst} must be
        of type \CTYPE{ptrdiff\_t}.}
    \apiargument{IN}{nelems}{Number of elements in the \dest{} and \source{}
        arrays.  \VAR{nelems} must be of type \VAR{size\_t} for \Cstd.}
    \apiargument{IN}{pe}{\ac{PE} number of the remote \ac{PE}.  \VAR{pe} must be
        of type integer.}
\end{apiarguments}


\apidescription{
    The \FUNC{iput} routines provide a method  for  copying strided data
    elements (specified by \VAR{sst}) of an array from a \source{} array on the
    local \ac{PE} to locations specified by stride \VAR{dst} on a \dest{} array
    on specified remote \ac{PE}. Both strides, \VAR{dst} and \VAR{sst}, must be
    greater than or equal to \CONST{1}. The routines return when the data has
    been copied out of the \VAR{source} array on the local \ac{PE} but not
    necessarily before the data has been delivered to the remote data object.
    If the context handle \VAR{ctx} does not correspond to a valid context,
    the behavior is undefined.
}

\apireturnvalues{
    None.
}

\apinotes{
    See Section \ref{subsec:memory_model} for a definition of the term
    remotely accessible.
}

\begin{apiexamples}

\apicexample
    {Consider the following \FUNC{shmem\_iput} example for \Cstd[11] programs.}
    {./example_code/shmem_iput_example.c}
    {}
\end{apiexamples}

\end{apidefinition}


\subsubsection{\textbf{SHMEM\_GET}}\label{subsec:shmem_get}
\apisummary{
    Copies data from a specified \ac{PE}.
}

\begin{apidefinition}

\begin{C11synopsis}
void @\FuncDecl{shmem\_get}@(TYPE *dest, const TYPE *source, size_t nelems, int pe);
void @\FuncDecl{shmem\_get}@(shmem_ctx_t ctx, TYPE *dest, const TYPE *source, size_t nelems, int pe);
\end{C11synopsis}
where \TYPE{} is one of the standard \ac{RMA} types specified by Table \ref{stdrmatypes}.

\begin{Csynopsis}
void @\FuncDecl{shmem\_\FuncParam{TYPENAME}\_get}@(TYPE *dest, const TYPE *source, size_t nelems, int pe);
void @\FuncDecl{shmem\_ctx\_\FuncParam{TYPENAME}\_get}@(shmem_ctx_t ctx, TYPE *dest, const TYPE *source, size_t nelems, int pe);
\end{Csynopsis}
where \TYPE{} is one of the standard \ac{RMA} types and has a corresponding \TYPENAME{} specified by Table \ref{stdrmatypes}.

\begin{CsynopsisCol}
void @\FuncDecl{shmem\_get\FuncParam{SIZE}}@(void *dest, const void *source, size_t  nelems, int pe);
void @\FuncDecl{shmem\_ctx\_get\FuncParam{SIZE}}@(shmem_ctx_t ctx, void *dest, const void *source, size_t  nelems, int pe);
\end{CsynopsisCol}
where \SIZE{} is one of \CONST{8, 16, 32, 64, 128}.

\begin{CsynopsisCol}
void @\FuncDecl{shmem\_getmem}@(void *dest, const void *source, size_t nelems, int pe);
void @\FuncDecl{shmem\_ctx\_getmem}@(shmem_ctx_t ctx, void *dest, const void *source, size_t nelems, int pe);
\end{CsynopsisCol}

\begin{Fsynopsis}
INTEGER nelems, pe
CALL @\FuncDecl{SHMEM\_CHARACTER\_GET}@(dest, source, nelems, pe)
CALL @\FuncDecl{SHMEM\_COMPLEX\_GET}@(dest, source, nelems, pe)
CALL @\FuncDecl{SHMEM\_DOUBLE\_GET}@(dest, source, nelems, pe)
CALL @\FuncDecl{SHMEM\_GET4}@(dest, source, nelems, pe)
CALL @\FuncDecl{SHMEM\_GET8}@(dest, source, nelems, pe)
CALL @\FuncDecl{SHMEM\_GET32}@(dest, source, nelems, pe)
CALL @\FuncDecl{SHMEM\_GET64}@(dest, source, nelems, pe)
CALL @\FuncDecl{SHMEM\_GET128}@(dest, source, nelems, pe)
CALL @\FuncDecl{SHMEM\_GETMEM}@(dest, source, nelems, pe)
CALL @\FuncDecl{SHMEM\_INTEGER\_GET}@(dest, source, nelems, pe)
CALL @\FuncDecl{SHMEM\_LOGICAL\_GET}@(dest, source, nelems, pe)
CALL @\FuncDecl{SHMEM\_REAL\_GET}@(dest, source, nelems, pe)
\end{Fsynopsis}

\begin{apiarguments}
    \apiargument{IN}{ctx}{\oldtext{The context on which to perform the operation.} \newtext{A context handle specifying the context on which to perform the operation.}
        When this argument is not provided, the operation is performed on
        \oldtext{\CONST{SHMEM\_CTX\_DEFAULT}} \newtext{the default context}.}
    \apiargument{OUT}{dest}{Local data object to be updated.}
    \apiargument{IN}{source}{Data object on the \ac{PE} identified by \VAR{pe}
        that contains the data to be copied.  This data object must be remotely
        accessible.}
    \apiargument{IN}{nelems}{Number of elements in the \dest{} and \source{}
        arrays. \VAR{nelems} must be of type \VAR{size\_t} for \Cstd. When
        using \Fortran, it must be a constant, variable, or array element of default
        integer type.}
    \apiargument{IN}{pe}{\ac{PE}  number of the remote \ac{PE}.  \VAR{pe} must
        be of type integer. When using \Fortran, it must be a constant,
        variable, or array element of default integer type.}
\end{apiarguments}

\apidescription{
   The get routines provide a method for copying a contiguous symmetric data
   object from a different \ac{PE} to a contiguous data object on the local
   \ac{PE}.  The routines return after the data has been delivered to the
   \dest{} array on the local \ac{PE}. 
   \newtext{
   If the context handle \VAR{ctx} does not correspond to a valid context,
   the behavior is undefined.
   }
}

\apidesctable{
    The  \dest{} and \source{} data objects must conform to typing constraints,
    which are as follows:
}{Routine}{Data type of \VAR{dest} and \VAR{source}}

    \apitablerow{shmem\_getmem}{\Fortran: Any noncharacter type. \Cstd: Any
        data  type.  nelems is scaled in bytes.}
    \apitablerow{shmem\_get4, shmem\_get32}{Any noncharacter type
        that has a storage size equal to \CONST{32} bits.}
    \apitablerow{shmem\_get8}{\Cstd: Any noncharacter type that
        has a storage size equal to \CONST{8} bits.}
    \apitablerow{}{\Fortran: Any noncharacter type that
        has a storage size equal to \CONST{64} bits.}
    \apitablerow{shmem\_get64}{Any noncharacter type that
        has a storage size equal to \CONST{64} bits.}
    \apitablerow{shmem\_get128}{Any  noncharacter type that has a
        storage size equal to \CONST{128} bits.}
    \apitablerow{SHMEM\_CHARACTER\_GET}{Elements of type character. \VAR{nelems} is
    the number  of characters  to transfer. The actual character
    lengths of the \source{} and \dest{} variables are ignored.}
    \apitablerow{SHMEM\_COMPLEX\_GET}{Elements of type complex of default
       size.}
    \apitablerow{SHMEM\_DOUBLE\_GET}{\Fortran: Elements of type double precision.}
    \apitablerow{SHMEM\_INTEGER\_GET}{Elements of type integer.}
    \apitablerow{SHMEM\_LOGICAL\_GET}{Elements of type logical.}
    \apitablerow{SHMEM\_REAL\_GET}{Elements of type real.}

\apireturnvalues{
    None.
}

\apinotes{
    See Section \ref{subsec:memory_model} for a definition of the term
    remotely accessible.
    When using \Fortran, data types must be of default size.  For example, a real
    variable must be declared as \CONST{REAL}, \CONST{REAL*4},  or
    \CONST{REAL(KIND=KIND(1.0))}.
}

\begin{apiexamples}

\apifexample
    {Consider this example for \Fortran.}
    {./example_code/shmem_get_example.f90}
    {}

\end{apiexamples}

\end{apidefinition}


\subsubsection{\textbf{SHMEM\_G}}\label{subsec:shmem_g}
\apisummary{
    Copies one data item from a remote \ac{PE}
}

\begin{apidefinition}

\begin{C11synopsis}
TYPE @\FuncDecl{shmem\_g}@(const TYPE *source, int pe);
TYPE @\FuncDecl{shmem\_g}@(shmem_ctx_t ctx, const TYPE *source, int pe);
\end{C11synopsis}
where \TYPE{} is one of the standard \ac{RMA} types specified by Table \ref{stdrmatypes}.

\begin{Csynopsis}
TYPE @\FuncDecl{shmem\_\FuncParam{TYPENAME}\_g}@(const TYPE *source, int pe);
TYPE @\FuncDecl{shmem\_ctx\_\FuncParam{TYPENAME}\_g}@(shmem_ctx_t ctx, const TYPE *source, int pe);
\end{Csynopsis}
where \TYPE{} is one of the standard \ac{RMA} types and has a corresponding \TYPENAME{} specified by Table \ref{stdrmatypes}.

\begin{apiarguments}
  \apiargument{IN}{ctx}{\oldtext{The context on which to perform the operation.} \newtext{A context handle specifying the context on which to perform the operation.}
    When this argument is not provided, the operation is performed on
    \oldtext{\CONST{SHMEM\_CTX\_DEFAULT}} \newtext{the default context}.}
  \apiargument{IN}{source}{The remotely accessible array element or scalar data object.}
  \apiargument{IN}{pe}{The number of the remote \ac{PE} on which \VAR{source} resides.}
\end{apiarguments}

\apidescription{
  These routines provide a very low latency get capability for single elements
  of most basic types. 
  \newtext{
  If the context handle \VAR{ctx} does not correspond to a valid context,
  the behavior is undefined.
  }
}

\apireturnvalues{
    Returns a single element of type specified in the synopsis.
}

\apinotes{
    None.
}

\begin{apiexamples}

\apicexample
    {The following \FUNC{shmem\_g} example is for \Cstd[11] programs:}
    {./example_code/shmem_g_example.c}
    {}
\end{apiexamples}

\end{apidefinition}


\subsubsection{\textbf{SHMEM\_IGET}}\label{subsec:shmem_iget}
\apisummary{
    Copies strided data from a specified \ac{PE}.
}

\begin{apidefinition}

\begin{C11synopsis}
void @\FuncDecl{shmem\_iget}@(TYPE *dest, const TYPE *source, ptrdiff_t dst, ptrdiff_t sst, size_t nelems, int pe);
void @\FuncDecl{shmem\_iget}@(shmem_ctx_t ctx, TYPE *dest, const TYPE *source, ptrdiff_t dst, ptrdiff_t sst, size_t nelems, int pe);
\end{C11synopsis}
where \TYPE{} is one of the standard \ac{RMA} types specified by Table \ref{stdrmatypes}.

\begin{Csynopsis}
void @\FuncDecl{shmem\_\FuncParam{TYPENAME}\_iget}@(TYPE *dest, const TYPE *source, ptrdiff_t dst, ptrdiff_t sst, size_t nelems, int pe);
void @\FuncDecl{shmem\_ctx\_\FuncParam{TYPENAME}\_iget}@(shmem_ctx_t ctx, TYPE *dest, const TYPE *source, ptrdiff_t dst, ptrdiff_t sst, size_t nelems, int pe);
\end{Csynopsis}
where \TYPE{} is one of the standard \ac{RMA} types and has a corresponding \TYPENAME{} specified by Table \ref{stdrmatypes}.

\begin{CsynopsisCol}
void @\FuncDecl{shmem\_iget\FuncParam{SIZE}}@(void *dest, const void *source, ptrdiff_t dst, ptrdiff_t sst, size_t  nelems, int pe);
void @\FuncDecl{shmem\_ctx\_iget\FuncParam{SIZE}}@(shmem_ctx_t ctx, void *dest, const void *source, ptrdiff_t dst, ptrdiff_t sst, size_t nelems, int pe);
\end{CsynopsisCol}
where \SIZE{} is one of \CONST{8, 16, 32, 64, 128}.

\begin{Fsynopsis}
INTEGER dst, sst, nelems, pe
CALL @\FuncDecl{SHMEM\_COMPLEX\_IGET}@(dest, source, dst, sst, nelems, pe)
CALL @\FuncDecl{SHMEM\_DOUBLE\_IGET}@(dest, source, dst, sst, nelems, pe)
CALL @\FuncDecl{SHMEM\_IGET4}@(dest, source, dst, sst, nelems, pe)
CALL @\FuncDecl{SHMEM\_IGET8}@(dest, source, dst, sst, nelems, pe)
CALL @\FuncDecl{SHMEM\_IGET32}@(dest, source, dst, sst, nelems, pe)
CALL @\FuncDecl{SHMEM\_IGET64}@(dest, source, dst, sst, nelems, pe)
CALL @\FuncDecl{SHMEM\_IGET128}@(dest, source, dst, sst, nelems, pe)
CALL @\FuncDecl{SHMEM\_INTEGER\_IGET}@(dest, source, dst, sst, nelems, pe)
CALL @\FuncDecl{SHMEM\_LOGICAL\_IGET}@(dest, source, dst, sst, nelems, pe)
CALL @\FuncDecl{SHMEM\_REAL\_IGET}@(dest, source, dst, sst, nelems, pe)
\end{Fsynopsis}

\begin{apiarguments}
    \apiargument{IN}{ctx}{\oldtext{The context on which to perform the operation.} \newtext{A context handle specifying the context on which to perform the operation.}
        When this argument is not provided, the operation is performed on
        \oldtext{\CONST{SHMEM\_CTX\_DEFAULT}} \newtext{the default context}.}
    \apiargument{OUT}{dest}{Array to be updated on the local \ac{PE}. }
    \apiargument{IN}{source}{Array containing the data to be copied on the remote \ac{PE}.}
    \apiargument{IN}{dst}{The stride between consecutive elements of the \dest{}
        array.  The stride is scaled by the element size of the \dest{} array.
        A  value of \CONST{1} indicates contiguous data. \VAR{dst} must be of
        type \CTYPE{ptrdiff\_t}.  When using  \Fortran,  it  must
        be a default integer value.}
    \apiargument{IN}{sst}{The stride between consecutive elements of the
        \source{} array.  The stride is scaled by the element size of the \source{}
        array.  A  value of \CONST{1} indicates contiguous data.  \VAR{sst} must be
        of type \CTYPE{ptrdiff\_t}.  When using  \Fortran,  it  must
        be a default integer value.}
    \apiargument{IN}{nelems}{Number of elements in the \dest{} and \source{}
        arrays.  \VAR{nelems} must be of type \VAR{size\_t} for \Cstd. When
        using \Fortran, it must be  a constant, variable, or array element of
        default integer type.}
    \apiargument{IN}{pe}{\ac{PE} number of the remote \ac{PE}.  \VAR{pe} must be
        of type integer. When using  \Fortran, it must be a constant,
        variable, or array element of default integer type.}
\end{apiarguments}

\apidescription{
    The \FUNC{iget} routines provide a method for copying strided data elements from
    a symmetric array from a specified remote \ac{PE} to strided locations on a
    local array.  The routines return when the data has been copied into the local
    \VAR{dest} array.
    \newtext{
    If the context handle \VAR{ctx} does not correspond to a valid context,
    the behavior is undefined.
    }
}

\apidesctable{
    The \VAR{dest} and \VAR{source} data objects must conform to typing
    constraints, which are as follows:}
    {Routine}{Data type of \VAR{dest} and \VAR{source}}
    \apitablerow{shmem\_iget4, shmem\_iget32}{Any noncharacter type
        that has a storage size equal to \CONST{32} bits.}
    \apitablerow{shmem\_iget8}{\Cstd: Any noncharacter type that
        has a storage size equal to \CONST{8} bits.}
    \apitablerow{}{\Fortran: Any noncharacter type that
        has a storage size equal to \CONST{64} bits.}
    \apitablerow{shmem\_iget64}{Any noncharacter type that
        has a storage size equal to \CONST{64} bits.}
    \apitablerow{shmem\_iget128}{Any noncharacter type that has a
        storage size equal to \CONST{128} bits.}
    \apitablerow{SHMEM\_COMPLEX\_IGET}{Elements of type complex of default size.}
    \apitablerow{SHMEM\_DOUBLE\_IGET}{\Fortran: Elements of type double precision.}
    \apitablerow{SHMEM\_INTEGER\_IGET}{Elements of type integer.}
    \apitablerow{SHMEM\_LOGICAL\_IGET}{Elements of type logical.}
    \apitablerow{SHMEM\_REAL\_IGET}{Elements of type real.}

\apireturnvalues{
    None.
}

\apinotes{
    When using \Fortran, data types must be of default size. For example, a
    real variable must be declared as \CONST{REAL}, \CONST{REAL*4}, or
    \CONST{REAL(KIND=KIND(1.0))}. 
}

\begin{apiexamples}

\apifexample
    {The following example uses \FUNC{shmem\_logical\_iget}  in a \Fortran
    program.} 
    {./example_code/shmem_iget_example.f90}
    {}

\end{apiexamples}

\end{apidefinition}



\subsection{Non-blocking Remote Memory Access Routines}\label{sec:rma_nbi}

\subsubsection{\textbf{SHMEM\_PUT\_NBI}}\label{subsec:shmem_put_nbi}
\apisummary{
    The nonblocking put routines provide a method for copying data
    from a contiguous local data object to a data object on a specified \ac{PE}.
}

\begin{apidefinition}

\begin{C11synopsis}
void @\FuncDecl{shmem\_put\_nbi}@(TYPE *dest, const TYPE *source, size_t nelems, int pe);
void @\FuncDecl{shmem\_put\_nbi}@(shmem_ctx_t ctx, TYPE *dest, const TYPE *source, size_t nelems, int pe);
\end{C11synopsis}
where \TYPE{} is one of the standard \ac{RMA} types specified by Table \ref{stdrmatypes}.

\begin{Csynopsis}
void @\FuncDecl{shmem\_\FuncParam{TYPENAME}\_put\_nbi}@(TYPE *dest, const TYPE *source, size_t nelems, int pe);
void @\FuncDecl{shmem\_ctx\_\FuncParam{TYPENAME}\_put\_nbi}@(shmem_ctx_t ctx, TYPE *dest, const TYPE *source, size_t nelems, int pe);
\end{Csynopsis}
where \TYPE{} is one of the standard \ac{RMA} types and has a corresponding \TYPENAME{} specified by Table \ref{stdrmatypes}.

\begin{CsynopsisCol}
void @\FuncDecl{shmem\_put\FuncParam{SIZE}\_nbi}@(void *dest, const void *source, size_t nelems, int pe);
void @\FuncDecl{shmem\_ctx\_put\FuncParam{SIZE}\_nbi}@(shmem_ctx_t ctx, void *dest, const void *source, size_t nelems, int pe);
\end{CsynopsisCol}
where \SIZE{} is one of \CONST{8, 16, 32, 64, 128}.

\begin{CsynopsisCol}
void @\FuncDecl{shmem\_putmem\_nbi}@(void *dest, const void *source, size_t nelems, int pe);
void @\FuncDecl{shmem\_ctx\_putmem\_nbi}@(shmem_ctx_t ctx, void *dest, const void *source, size_t nelems, int pe);
\end{CsynopsisCol}

\begin{apiarguments}
  \apiargument{IN}{ctx}{A context handle specifying the context on which to perform the operation.
    When this argument is not provided, the operation is performed on
    the default context.}
  \apiargument{OUT}{dest}{Data object to be updated on the remote \ac{PE}. This
    data object must be remotely accessible.}
  \apiargument{IN}{source}{Data object containing the data to be copied.}
  \apiargument{IN}{nelems}{Number of elements in the \VAR{dest} and \VAR{source}
    arrays. \VAR{nelems} must be of type \VAR{size\_t} for \Cstd.}
  \apiargument{IN}{pe}{\ac{PE} number of the remote \ac{PE}. \VAR{pe} must be
    of type integer.}
\end{apiarguments}

\apidescription{
    The routines return after posting the operation.  The operation is considered
    complete after a subsequent call to \FUNC{shmem\_quiet}.
    At the completion of \FUNC{shmem\_quiet}, the data has been copied into the \dest{} array
    on the destination \ac{PE}.
    The delivery of data words into the data object on the
    destination \ac{PE} may occur in any order.
    Furthermore, two successive put
    routines may deliver data out of order unless a call to \FUNC{shmem\_fence} is
    introduced between the two calls.
    If the context handle \VAR{ctx} does not correspond to a valid context,
    the behavior is undefined.
 }

\apireturnvalues{
    None.
}
\apinotes{ None.}

\end{apidefinition}


\subsubsection{\textbf{SHMEM\_GET\_NBI}}\label{subsec:shmem_get_nbi}
\apisummary{
    The nonblocking get routines provide a method for copying data from a
    contiguous remote data object on the specified \ac{PE} to the local data object. 
}

\begin{apidefinition}

\begin{C11synopsis}
void @\FuncDecl{shmem\_get\_nbi}@(TYPE *dest, const TYPE *source, size_t nelems, int pe);
void @\FuncDecl{shmem\_get\_nbi}@(shmem_ctx_t ctx, TYPE *dest, const TYPE *source, size_t nelems, int pe);
\end{C11synopsis}
where \TYPE{} is one of the standard \ac{RMA} types specified by Table \ref{stdrmatypes}.

\begin{Csynopsis}
void @\FuncDecl{shmem\_\FuncParam{TYPENAME}\_get\_nbi}@(TYPE *dest, const TYPE *source, size_t nelems, int pe);
void @\FuncDecl{shmem\_ctx\_\FuncParam{TYPENAME}\_get\_nbi}@(shmem_ctx_t ctx, TYPE *dest, const TYPE *source, size_t nelems, int pe);
\end{Csynopsis}
where \TYPE{} is one of the standard \ac{RMA} types and has a corresponding \TYPENAME{} specified by Table \ref{stdrmatypes}.

\begin{CsynopsisCol}
void @\FuncDecl{shmem\_get\FuncParam{SIZE}\_nbi}@(void *dest, const void *source, size_t  nelems, int pe);
void @\FuncDecl{shmem\_ctx\_get\FuncParam{SIZE}\_nbi}@(shmem_ctx_t ctx, void *dest, const void *source, size_t  nelems, int pe);
\end{CsynopsisCol}
where \SIZE{} is one of \CONST{8, 16, 32, 64, 128}.

\begin{CsynopsisCol}
void @\FuncDecl{shmem\_getmem\_nbi}@(void *dest, const void *source, size_t nelems, int pe);
void @\FuncDecl{shmem\_ctx\_getmem\_nbi}@(shmem_ctx_t ctx, void *dest, const void *source, size_t nelems, int pe);
\end{CsynopsisCol}

\begin{Fsynopsis}
INTEGER nelems, pe
CALL @\FuncDecl{SHMEM\_CHARACTER\_GET\_NBI}@(dest, source, nelems, pe)
CALL @\FuncDecl{SHMEM\_COMPLEX\_GET\_NBI}@(dest, source, nelems, pe)
CALL @\FuncDecl{SHMEM\_DOUBLE\_GET\_NBI}@(dest, source, nelems, pe)
CALL @\FuncDecl{SHMEM\_GET4\_NBI}@(dest, source, nelems, pe)
CALL @\FuncDecl{SHMEM\_GET8\_NBI}@(dest, source, nelems, pe)
CALL @\FuncDecl{SHMEM\_GET32\_NBI}@(dest, source, nelems, pe)
CALL @\FuncDecl{SHMEM\_GET64\_NBI}@(dest, source, nelems, pe)
CALL @\FuncDecl{SHMEM\_GET128\_NBI}@(dest, source, nelems, pe)
CALL @\FuncDecl{SHMEM\_GETMEM\_NBI}@(dest, source, nelems, pe)
CALL @\FuncDecl{SHMEM\_INTEGER\_GET\_NBI}@(dest, source, nelems, pe)
CALL @\FuncDecl{SHMEM\_LOGICAL\_GET\_NBI}@(dest, source, nelems, pe)
CALL @\FuncDecl{SHMEM\_REAL\_GET\_NBI}@(dest, source, nelems, pe)
\end{Fsynopsis}

\begin{apiarguments}
    \apiargument{IN}{ctx}{A context handle specifying the context on which to perform the operation.
        When this argument is not provided, the operation is performed on
        the default context.}
    \apiargument{OUT}{dest}{Local data object to be updated.}
    \apiargument{IN}{source}{Data object on the \ac{PE} identified by \VAR{pe}
        that contains the data to be copied.  This data object must be remotely
        accessible.}
    \apiargument{IN}{nelems}{Number of elements in the \dest{} and \source{}
        arrays. \VAR{nelems} must be of type \VAR{size\_t} for \Cstd. When
        using \Fortran, it must be a constant, variable, or array element of default
        integer type.}
    \apiargument{IN}{pe}{\ac{PE}  number of the remote \ac{PE}.  \VAR{pe} must
        be of type integer. When using \Fortran, it must be a constant,
        variable, or array element of default integer type.}
\end{apiarguments}

\apidescription{
    The get routines provide a method for copying a contiguous symmetric data
   object from a different \ac{PE} to a contiguous data object on the local
   \ac{PE}.   The routines return after posting the operation.  The operation is considered 
    complete after a subsequent call to \FUNC{shmem\_quiet}. 
    At the completion of \FUNC{shmem\_quiet}, the 
    data has been delivered to the \dest{} array on the local \ac{PE}. 
    If the context handle \VAR{ctx} does not correspond to a valid context,
    the behavior is undefined.
}

\apidesctable{
    The  \dest{} and \source{} data objects must conform to typing constraints,
    which are as follows:
}{Routine}{Data type of \VAR{dest} and \VAR{source}}

    \apitablerow{shmem\_getmem\_nbi}{\Fortran: Any noncharacter type. \Cstd:
        Any  data  type.  nelems is scaled in bytes.}
    \apitablerow{shmem\_get4\_nbi, shmem\_get32\_nbi}{Any noncharacter type
        that has a storage size equal to \CONST{32} bits.}
    \apitablerow{shmem\_get8\_nbi}{\Cstd: Any noncharacter type that
        has a storage size equal to \CONST{8} bits.}
    \apitablerow{}{\Fortran: Any noncharacter type that
        has a storage size equal to \CONST{64} bits.}
    \apitablerow{shmem\_get64\_nbi}{Any noncharacter type that
        has a storage size equal to \CONST{64} bits.}
    \apitablerow{shmem\_get128\_nbi}{Any noncharacter type that has a
        storage size equal to \CONST{128} bits.}
    \apitablerow{SHMEM\_CHARACTER\_GET\_NBI}{Elements of type character. \VAR{nelems} is
    the number  of characters  to transfer. The actual character
    lengths of the \source{} and \dest{} variables are ignored.}
    \apitablerow{SHMEM\_COMPLEX\_GET\_NBI}{Elements of type complex of default
       size.}
    \apitablerow{SHMEM\_DOUBLE\_GET\_NBI}{\Fortran: Elements of type double precision.}
    \apitablerow{SHMEM\_INTEGER\_GET\_NBI}{Elements of type integer.}
    \apitablerow{SHMEM\_LOGICAL\_GET\_NBI}{Elements of type logical.}
    \apitablerow{SHMEM\_REAL\_GET\_NBI}{Elements of type real.}

\apireturnvalues{
    None.
}

\apinotes{
    See Section \ref{subsec:memory_model} for a definition of the term
    remotely accessible.
    When using \Fortran, data types must be of default size.  For example, a real
    variable must be declared as \CONST{REAL}, \CONST{REAL*4},  or
    \CONST{REAL(KIND=KIND(1.0))}.
}

\end{apidefinition}



\subsection{Atomic Memory Operations}\label{sec:amo}
\input{content/atomics_intro}

\subsubsection{\textbf{SHMEM\_ATOMIC\_FETCH}}
\label{subsec:shmem_atomic_fetch}
\apisummary{
    Atomically fetches the value of a remote data object.
}

\begin{apidefinition}

\begin{C11synopsis}
TYPE @\FuncDecl{shmem\_atomic\_fetch}@(const TYPE *source, int pe);
TYPE @\FuncDecl{shmem\_atomic\_fetch}@(shmem_ctx_t ctx, const TYPE *source, int pe);
\end{C11synopsis}
where \TYPE{} is one of the extended \ac{AMO} types specified by
Table~\ref{extamotypes}.

\begin{Csynopsis}
TYPE @\FuncDecl{shmem\_\FuncParam{TYPENAME}\_atomic\_fetch}@(const TYPE *source, int pe);
TYPE @\FuncDecl{shmem\_ctx\_\FuncParam{TYPENAME}\_atomic\_fetch}@(shmem_ctx_t ctx, const TYPE *source, int pe);
\end{Csynopsis}
where \TYPE{} is one of the extended \ac{AMO} types and has a corresponding
\TYPENAME{} specified by Table~\ref{extamotypes}.

\begin{DeprecateBlock}
\begin{C11synopsis}
TYPE @\FuncDecl{shmem\_fetch}@(const TYPE *source, int pe);
\end{C11synopsis}
where \TYPE{} is one of \{\CTYPE{float}, \CTYPE{double}, \CTYPE{int},
\CTYPE{long}, \CTYPE{long long}\}.

\begin{Csynopsis}
TYPE @\FuncDecl{shmem\_\FuncParam{TYPENAME}\_fetch}@(const TYPE *source, int pe);
\end{Csynopsis}
where \TYPE{} is one of \{\CTYPE{float}, \CTYPE{double}, \CTYPE{int},
\CTYPE{long}, \CTYPE{long long}\} and has a corresponding
\TYPENAME{} specified by Table~\ref{extamotypes}.
\end{DeprecateBlock}

\begin{apiarguments}

  \apiargument{IN}{ctx}{A context handle specifying the context on which to perform the operation.
    When this argument is not provided, the operation is performed on
    the default context.}
  \apiargument{IN}{source}{The remotely accessible data object to be fetched from
    the remote \ac{PE}.}
  \apiargument{IN}{pe}{An integer that indicates the \ac{PE} number from which
    \VAR{source} is to be fetched.}

\end{apiarguments}

\apidescription{
    \FUNC{shmem\_atomic\_fetch} performs an atomic fetch operation.
    It returns the contents of the \VAR{source} as an atomic operation.
    If the context handle \VAR{ctx} does not correspond to a valid context,
    the behavior is undefined.
}

\apireturnvalues{
    The contents at the \VAR{source} address on the remote \ac{PE}.
    The data type of the return value is the same as the type of
    the remote data object.
}

\apinotes{
    None.
}

\end{apidefinition}


\subsubsection{\textbf{SHMEM\_ATOMIC\_SET}}
\label{subsec:shmem_atomic_set}
\apisummary{
    Atomically sets the value of a remote data object.
}

\begin{apidefinition}

\begin{C11synopsis}
void @\FuncDecl{shmem\_atomic\_set}@(TYPE *dest, TYPE value, int pe);
void @\FuncDecl{shmem\_atomic\_set}@(shmem_ctx_t ctx, TYPE *dest, TYPE value, int pe);
\end{C11synopsis}
where \TYPE{} is one of the extended \ac{AMO} types specified by
Table~\ref{extamotypes}.

\begin{Csynopsis}
void @\FuncDecl{shmem\_\FuncParam{TYPENAME}\_atomic\_set}@(TYPE *dest, TYPE value, int pe);
void @\FuncDecl{shmem\_ctx\_\FuncParam{TYPENAME}\_atomic\_set}@(shmem_ctx_t ctx, TYPE *dest, TYPE value, int pe);
\end{Csynopsis}
where \TYPE{} is one of the extended \ac{AMO} types and has a corresponding
\TYPENAME{} specified by Table~\ref{extamotypes}.

\begin{DeprecateBlock}
\begin{C11synopsis}
void @\FuncDecl{shmem\_set}@(TYPE *dest, TYPE value, int pe);
\end{C11synopsis}
where \TYPE{} is one of \{\CTYPE{float}, \CTYPE{double}, \CTYPE{int},
\CTYPE{long}, \CTYPE{long long}\}.

\begin{Csynopsis}
void @\FuncDecl{shmem\_\FuncParam{TYPENAME}\_set}@(TYPE *dest, TYPE value, int pe);
\end{Csynopsis}
where \TYPE{} is one of \{\CTYPE{float}, \CTYPE{double}, \CTYPE{int},
\CTYPE{long}, \CTYPE{long long}\} and has a corresponding
\TYPENAME{} specified by Table~\ref{extamotypes}.
\end{DeprecateBlock}

\begin{apiarguments}

\apiargument{IN}{ctx}{A context handle specifying the context on which to perform the operation.
    When this argument is not provided, the operation is performed on
    the default context.}
\apiargument{OUT}{dest}{The remotely accessible data object to be set on
    the remote \ac{PE}.}
\apiargument{IN}{value}{The value to be atomically written to the remote \ac{PE}.}
\apiargument{IN}{pe}{An integer that indicates the \ac{PE} number on which
    \VAR{dest} is to be updated.}

\end{apiarguments}

\apidescription{
    \FUNC{shmem\_atomic\_set} performs an atomic set operation. It writes the
    \VAR{value} into \VAR{dest} on \VAR{pe} as an atomic operation.
    If the context handle \VAR{ctx} does not correspond to a valid context,
    the behavior is undefined.
}

\apireturnvalues{
    None.
}

\apinotes{
    None.
}

\end{apidefinition}


\subsubsection{\textbf{SHMEM\_ATOMIC\_COMPARE\_SWAP}}
\label{subsec:shmem_atomic_compare_swap}
\apisummary{
    Performs an atomic conditional swap on a remote data object.
}

\begin{apidefinition}

\begin{C11synopsis}
TYPE @\FuncDecl{shmem\_atomic\_compare\_swap}@(TYPE *dest, TYPE cond, TYPE value, int pe);
TYPE @\FuncDecl{shmem\_atomic\_compare\_swap}@(shmem_ctx_t ctx, TYPE *dest, TYPE cond, TYPE value, int pe);
\end{C11synopsis}
where \TYPE{} is one of the standard \ac{AMO} types specified by
Table~\ref{stdamotypes}.

\begin{Csynopsis}
TYPE @\FuncDecl{shmem\_\FuncParam{TYPENAME}\_atomic\_compare\_swap}@(TYPE *dest, TYPE cond, TYPE value, int pe);
TYPE @\FuncDecl{shmem\_ctx\_\FuncParam{TYPENAME}\_atomic\_compare\_swap}@(shmem_ctx_t ctx, TYPE *dest, TYPE cond, TYPE value, int pe);
\end{Csynopsis}
where \TYPE{} is one of the standard \ac{AMO} types and has a corresponding
\TYPENAME{} specified by Table~\ref{stdamotypes}.

\begin{DeprecateBlock}
\begin{C11synopsis}
TYPE @\FuncDecl{shmem\_cswap}@(TYPE *dest, TYPE cond, TYPE value, int pe);
\end{C11synopsis}
where \TYPE{} is one of \{\CTYPE{int}, \CTYPE{long}, \CTYPE{long long}\}.

\begin{Csynopsis}
TYPE @\FuncDecl{shmem\_\FuncParam{TYPENAME}\_cswap}@(TYPE *dest, TYPE cond, TYPE value, int pe);
\end{Csynopsis}
where \TYPE{} is one of \{\CTYPE{int}, \CTYPE{long}, \CTYPE{long long}\}
and has a corresponding \TYPENAME{} specified by Table~\ref{stdamotypes}.
\end{DeprecateBlock}

\begin{apiarguments}
    \apiargument{IN}{ctx}{A context handle specifying the context on which to perform the operation.
        When this argument is not provided, the operation is performed on
        the default context.}
    \apiargument{OUT}{dest}{The remotely accessible integer data object to be
        updated on the remote \ac{PE}. }
    \apiargument{IN}{cond}{\VAR{cond} is compared to the remote \VAR{dest}
        value. If \VAR{cond} and the remote \VAR{dest} are equal, then \VAR{value}
        is swapped into the remote \VAR{dest}; otherwise, the remote \VAR{dest} is
        unchanged.  In either case, the old value of the remote \VAR{dest} is
        returned as the routine return value. \VAR{cond} must be of the same data
        type as \VAR{dest}.}
    \apiargument{IN}{value}{The value to be atomically written to the remote
        \ac{PE}. \VAR{value} must be the same data type as \VAR{dest}.}
    \apiargument{IN}{pe}{An integer that indicates the \ac{PE} number upon which
        \VAR{dest} is to be updated.}
\end{apiarguments}

\apidescription{
    The conditional swap routines conditionally update a \VAR{dest} data object on
    the specified \ac{PE} and return the prior contents of the data object in one
    atomic operation.
    If the context handle \VAR{ctx} does not correspond to a valid context,
    the behavior is undefined.
}

\apireturnvalues{
    The contents that had been in the \VAR{dest} data object on the remote
    \ac{PE} prior to the conditional swap. Data type is the same as the
    \VAR{dest} data type.
}

\apinotes{
    None.
}

\begin{apiexamples}

\apicexample
    {The following call ensures that the first \ac{PE} to execute the
    conditional swap will successfully write its \ac{PE} number to
    \VAR{race\_winner} on \ac{PE} \CONST{0}.}
    {./example_code/shmem_atomic_compare_swap_example.c}
    {}

\end{apiexamples}

\end{apidefinition}


\subsubsection{\textbf{SHMEM\_ATOMIC\_SWAP}}
\label{subsec:shmem_atomic_swap}
\apisummary{
    Performs an atomic swap to a remote data object.
}

\begin{apidefinition}

\begin{C11synopsis}
TYPE @\FuncDecl{shmem\_atomic\_swap}@(TYPE *dest, TYPE value, int pe);
TYPE @\FuncDecl{shmem\_atomic\_swap}@(shmem_ctx_t ctx, TYPE *dest, TYPE value, int pe);
\end{C11synopsis}
where \TYPE{} is one of the extended \ac{AMO} types specified by Table \ref{extamotypes}.

\begin{Csynopsis}
TYPE @\FuncDecl{shmem\_\FuncParam{TYPENAME}\_atomic\_swap}@(TYPE *dest, TYPE value, int pe);
TYPE @\FuncDecl{shmem\_ctx\_\FuncParam{TYPENAME}\_atomic\_swap}@(shmem_ctx_t ctx, TYPE *dest, TYPE value, int pe);
\end{Csynopsis}
where \TYPE{} is one of the extended \ac{AMO} types and has a corresponding \TYPENAME{} specified by Table \ref{extamotypes}.

\begin{DeprecateBlock}
\begin{C11synopsis}
TYPE @\FuncDecl{shmem\_swap}@(TYPE *dest, TYPE value, int pe);
\end{C11synopsis}
where \TYPE{} is one of \{\CTYPE{float}, \CTYPE{double}, \CTYPE{int},
\CTYPE{long}, \CTYPE{long long}\}.

\begin{Csynopsis}
TYPE @\FuncDecl{shmem\_\FuncParam{TYPENAME}\_swap}@(TYPE *dest, TYPE value, int pe);
\end{Csynopsis}
where \TYPE{} is one of \{\CTYPE{float}, \CTYPE{double}, \CTYPE{int},
\CTYPE{long}, \CTYPE{long long}\} and has a corresponding
\TYPENAME{} specified by Table~\ref{extamotypes}.
\end{DeprecateBlock}

\begin{apiarguments}
  \apiargument{IN}{ctx}{A context handle specifying the context on which to perform the operation.
    When this argument is not provided, the operation is performed on
    the default context.}
  \apiargument{OUT}{dest}{The  remotely accessible integer data object to be
    updated on the remote \ac{PE}.	 When using \CorCpp, the type of
    \dest{} should match that  implied in the SYNOPSIS section.}
  \apiargument{IN}{value}{The value to be atomically written to the remote
    \ac{PE}. \VAR{value}  is the same type as \dest.}
  \apiargument{IN}{pe}{ An integer that indicates the \ac{PE} number on which
    \dest{} is to be updated.}
\end{apiarguments}

\apidescription{
    \FUNC{shmem\_atomic\_swap} performs an atomic swap operation.
    It writes \VAR{value} into \dest{} on \ac{PE} and returns the previous
    contents of \dest{} as an atomic operation.
    If the context handle \VAR{ctx} does not correspond to a valid context,
    the behavior is undefined.
}

\apireturnvalues{
       The content that had been at the \dest{} address on the remote \ac{PE}
       prior to the swap is returned.
}

\apinotes{
    None.
}

\begin{apiexamples}

\apicexample
    {The example below swaps values between odd numbered \acp{PE} and
    their right (modulo) neighbor and outputs the result of swap.}
    {./example_code/shmem_atomic_swap_example.c}
    {}

\end{apiexamples}

\end{apidefinition}


\subsubsection{\textbf{SHMEM\_ATOMIC\_FETCH\_INC}}
\label{subsec:shmem_atomic_fetch_inc}
\apisummary{
    Performs an atomic fetch-and-increment  operation on a remote data object.
}

\begin{apidefinition}

\begin{C11synopsis}
TYPE @\FuncDecl{shmem\_atomic\_fetch\_inc}@(TYPE *dest, int pe);
TYPE @\FuncDecl{shmem\_atomic\_fetch\_inc}@(shmem_ctx_t ctx, TYPE *dest, int pe);
\end{C11synopsis}
where \TYPE{} is one of the standard \ac{AMO} types specified by
Table~\ref{stdamotypes}.

\begin{Csynopsis}
TYPE @\FuncDecl{shmem\_\FuncParam{TYPENAME}\_atomic\_fetch\_inc}@(TYPE *dest, int pe);
TYPE @\FuncDecl{shmem\_ctx\_\FuncParam{TYPENAME}\_atomic\_fetch\_inc}@(shmem_ctx_t ctx, TYPE *dest, int pe);
\end{Csynopsis}
where \TYPE{} is one of the standard \ac{AMO} types and has a corresponding
\TYPENAME{} specified by Table~\ref{stdamotypes}.

\begin{DeprecateBlock}
\begin{C11synopsis}
TYPE @\FuncDecl{shmem\_finc}@(TYPE *dest, int pe);
\end{C11synopsis}
where \TYPE{} is one of \{\CTYPE{int}, \CTYPE{long}, \CTYPE{long long}\}.

\begin{Csynopsis}
TYPE @\FuncDecl{shmem\_\FuncParam{TYPENAME}\_finc}@(TYPE *dest, int pe);
\end{Csynopsis}
where \TYPE{} is one of \{\CTYPE{int}, \CTYPE{long}, \CTYPE{long long}\}
and has a corresponding \TYPENAME{} specified by Table~\ref{stdamotypes}.
\end{DeprecateBlock}

\begin{Fsynopsis}
INTEGER pe
INTEGER*4 SHMEM_INT4_FINC, ires_i4
ires\_i4 = @\FuncDecl{SHMEM\_INT4\_FINC}@(dest, pe)
INTEGER*8 SHMEM_INT8_FINC, ires_i8
ires\_i8 = @\FuncDecl{SHMEM\_INT8\_FINC}@(dest, pe)
\end{Fsynopsis}


\begin{apiarguments}

\apiargument{IN}{ctx}{A context handle specifying the context on which to perform the operation.
    When this argument is not provided, the operation is performed on
    the default context.}
\apiargument{OUT}{dest}{The remotely accessible integer data object to be updated
    on the remote \ac{PE}. The type of \dest{} should match that implied in the
    SYNOPSIS section.}
\apiargument{IN}{pe}{An integer that indicates the \ac{PE} number on which
    \dest{} is to be updated.  When using \Fortran, it must be a default
    integer value.}

\end{apiarguments}


\apidescription{
   These routines perform a fetch-and-increment operation.  The \dest{} on
   \ac{PE} \VAR{pe} is increased by one and the routine returns the previous
   contents of \dest{} as an atomic operation.
   If the context handle \VAR{ctx} does not correspond to a valid context,
   the behavior is undefined.
}

\apidesctable{
    When using \Fortran, \VAR{dest} must be of the following type:
}{Routine}{Data type of \VAR{dest}}

\apitablerow{SHMEM\_INT4\_FINC}{\CONST{4}-byte integer}
\apitablerow{SHMEM\_INT8\_FINC}{\CONST{8}-byte integer}

\apireturnvalues{
    The contents that had been at the \dest{} address on the remote \ac{PE} prior to
    the increment.  The data type of the return value is the same as the \dest.
}

\apinotes{
    None.
}

\begin{apiexamples}

\apicexample
    {The following \FUNC{shmem\_atomic\_fetch\_inc} example is for
    \Cstd[11] programs:}
    {./example_code/shmem_atomic_fetch_inc_example.c}
    {}

\end{apiexamples}

\end{apidefinition}


\subsubsection{\textbf{SHMEM\_ATOMIC\_INC}}
\label{subsec:shmem_atomic_inc}
\apisummary{
    Performs an atomic increment operation on a remote data object.
}

\begin{apidefinition}

\begin{C11synopsis}
void @\FuncDecl{shmem\_atomic\_inc}@(TYPE *dest, int pe);
void @\FuncDecl{shmem\_atomic\_inc}@(shmem_ctx_t ctx, TYPE *dest, int pe);
\end{C11synopsis}
where \TYPE{} is one of the standard \ac{AMO} types specified by
Table~\ref{stdamotypes}.

\begin{Csynopsis}
void @\FuncDecl{shmem\_\FuncParam{TYPENAME}\_atomic\_inc}@(TYPE *dest, int pe);
void @\FuncDecl{shmem\_ctx\_\FuncParam{TYPENAME}\_atomic\_inc}@(shmem_ctx_t ctx, TYPE *dest, int pe);
\end{Csynopsis}
where \TYPE{} is one of the standard \ac{AMO} types and has a corresponding
\TYPENAME{} specified by Table~\ref{stdamotypes}.

\begin{DeprecateBlock}
\begin{C11synopsis}
void @\FuncDecl{shmem\_inc}@(TYPE *dest, int pe);
\end{C11synopsis}
where \TYPE{} is one of \{\CTYPE{int}, \CTYPE{long}, \CTYPE{long long}\}.

\begin{Csynopsis}
void @\FuncDecl{shmem\_\FuncParam{TYPENAME}\_inc}@(TYPE *dest, int pe);
\end{Csynopsis}
where \TYPE{} is one of \{\CTYPE{int}, \CTYPE{long}, \CTYPE{long long}\}
and has a corresponding \TYPENAME{} specified by Table~\ref{stdamotypes}.
\end{DeprecateBlock}

\begin{Fsynopsis}
INTEGER pe
CALL @\FuncDecl{SHMEM\_INT4\_INC}@(dest, pe)
CALL @\FuncDecl{SHMEM\_INT8\_INC}@(dest, pe)
\end{Fsynopsis}

\begin{apiarguments}

\apiargument{IN}{ctx}{A context handle specifying the context on which to perform the operation.
    When this argument is not provided, the operation is performed on
    the default context.}
\apiargument{OUT}{dest}{The remotely accessible integer data object to be updated
    on the remote \ac{PE}. The type of \dest{} should match that implied in the
    SYNOPSIS section.}
\apiargument{IN}{pe}{An integer that indicates the \ac{PE} number on which
    \dest{} is to be  updated. When using \Fortran, it must be a default
    integer value.}

\end{apiarguments}

\apidescription{
    These  routines perform  an atomic increment operation on the \VAR{dest} data
    object on \ac{PE}.
    If the context handle \VAR{ctx} does not correspond to a valid context,
    the behavior is undefined.
}


\apidesctable{
    When using \Fortran, \VAR{dest} must be of the following type:
}{Routine}{Data type of \VAR{dest}}

\apitablerow{SHMEM\_INT4\_INC}{\CONST{4}-byte integer}
\apitablerow{SHMEM\_INT8\_INC}{\CONST{8}-byte integer}

\apireturnvalues{
    None.
}

\apinotes{
    None.
}

\begin{apiexamples}

\apicexample
    { The following \FUNC{shmem\_atomic\_inc} example is for
    \Cstd[11] programs: }
    {./example_code/shmem_atomic_inc_example.c}
    {}

\end{apiexamples}

\end{apidefinition}


\subsubsection{\textbf{SHMEM\_ATOMIC\_FETCH\_ADD}}
\label{subsec:shmem_atomic_fetch_add}
\apisummary{
    Performs an atomic fetch-and-add operation on a remote data object.
}

\begin{apidefinition}

\begin{C11synopsis}
TYPE @\FuncDecl{shmem\_atomic\_fetch\_add}@(TYPE *dest, TYPE value, int pe);
TYPE @\FuncDecl{shmem\_atomic\_fetch\_add}@(shmem_ctx_t ctx, TYPE *dest, TYPE value, int pe);
\end{C11synopsis}
where \TYPE{} is one of the standard \ac{AMO} types specified by
Table~\ref{stdamotypes}.

\begin{Csynopsis}
TYPE @\FuncDecl{shmem\_\FuncParam{TYPENAME}\_atomic\_fetch\_add}@(TYPE *dest, TYPE value, int pe);
TYPE @\FuncDecl{shmem\_ctx\_\FuncParam{TYPENAME}\_atomic\_fetch\_add}@(shmem_ctx_t ctx, TYPE *dest, TYPE value, int pe);
\end{Csynopsis}
where \TYPE{} is one of the standard \ac{AMO} types and has a corresponding
\TYPENAME{} specified by Table~\ref{stdamotypes}.

\begin{DeprecateBlock}
\begin{C11synopsis}
TYPE @\FuncDecl{shmem\_fadd}@(TYPE *dest, TYPE value, int pe);
\end{C11synopsis}
where \TYPE{} is one of \{\CTYPE{int}, \CTYPE{long}, \CTYPE{long long}\}.

\begin{Csynopsis}
TYPE @\FuncDecl{shmem\_\FuncParam{TYPENAME}\_fadd}@(TYPE *dest, TYPE value, int pe);
\end{Csynopsis}
where \TYPE{} is one of \{\CTYPE{int}, \CTYPE{long}, \CTYPE{long long}\}
and has a corresponding \TYPENAME{} specified by Table~\ref{stdamotypes}.
\end{DeprecateBlock}

\begin{Fsynopsis}
INTEGER pe
INTEGER*4 SHMEM_INT4_FADD, ires_i4, value_i4
ires\_i4 = @\FuncDecl{SHMEM\_INT4\_FADD}@(dest, value_i4, pe)
INTEGER*8 SHMEM_INT8_FADD, ires_i8, value_i8
ires\_i8 = @\FuncDecl{SHMEM\_INT8\_FADD}@(dest, value_i8, pe)
\end{Fsynopsis}

\begin{apiarguments}

\apiargument{IN}{ctx}{\oldtext{The context on which to perform the operation.} \newtext{A context handle specifying the context on which to perform the operation.}
    When this argument is not provided, the operation is performed on
    \oldtext{\CONST{SHMEM\_CTX\_DEFAULT}} \newtext{the default context}.}
\apiargument{OUT}{dest}{The remotely accessible integer data object to be updated on
    the remote \ac{PE}. The type of \VAR{dest} should match that implied in the
    SYNOPSIS section.}
\apiargument{IN}{value}{The value to be atomically added to \VAR{dest}.  The
    type of \VAR{value} should match that implied in the SYNOPSIS section.}
\apiargument{IN}{pe}{An integer that indicates the \ac{PE} number on which
    \VAR{dest} is to be updated.  When using \Fortran, it must be a default
    integer value.}

\end{apiarguments}

\apidescription{
    \FUNC{shmem\_atomic\_fetch\_add} routines perform an atomic fetch-and-add operation.  An
    atomic fetch-and-add operation fetches the old \VAR{dest} and adds \VAR{value}
    to \VAR{dest} without the possibility of another atomic operation on the
    \VAR{dest} between the time of the fetch and the update.  These routines add
    \VAR{value} to \VAR{dest} on \VAR{pe} and return the previous contents of
    \VAR{dest} as an atomic operation.
    \newtext{
    If the context handle \VAR{ctx} does not correspond to a valid context,
    the behavior is undefined.
    }
}

\apidesctable{
    When using \Fortran, \VAR{dest} and \VAR{value} must be of the following type:
}{Routine}{Data type of \VAR{dest} and \VAR{value}}

\apitablerow{SHMEM\_INT4\_FADD}{\CONST{4}-byte integer}
\apitablerow{SHMEM\_INT8\_FADD}{\CONST{8}-byte integer}


\apireturnvalues{
    The contents that had been at the \VAR{dest} address on the remote \ac{PE}
    prior to the atomic addition operation.  The data type of the return value is
    the same as the \VAR{dest}.
}

\apinotes{
    None.
}

\begin{apiexamples}

\apicexample
        {The following \FUNC{shmem\_atomic\_fetch\_add} example is for
        \Cstd[11] programs:}
        {./example_code/shmem_atomic_fetch_add_example.c}
        {}

\end{apiexamples}

\end{apidefinition}


\subsubsection{\textbf{SHMEM\_ATOMIC\_ADD}}
\label{subsec:shmem_atomic_add}
\apisummary{
    Performs an atomic add operation on a remote symmetric data object.
}

\begin{apidefinition}

\begin{C11synopsis}
void @\FuncDecl{shmem\_atomic\_add}@(TYPE *dest, TYPE value, int pe);
void @\FuncDecl{shmem\_atomic\_add}@(shmem_ctx_t ctx, TYPE *dest, TYPE value, int pe);
\end{C11synopsis}
where \TYPE{} is one of the standard \ac{AMO} types specified by
Table~\ref{stdamotypes}.

\begin{Csynopsis}
void @\FuncDecl{shmem\_\FuncParam{TYPENAME}\_atomic\_add}@(TYPE *dest, TYPE value, int pe);
void @\FuncDecl{shmem\_ctx\_\FuncParam{TYPENAME}\_atomic\_add}@(shmem_ctx_t ctx, TYPE *dest, TYPE value, int pe);
\end{Csynopsis}
where \TYPE{} is one of the standard \ac{AMO} types and has a corresponding
\TYPENAME{} specified by Table~\ref{stdamotypes}.

\begin{DeprecateBlock}
\begin{C11synopsis}
void @\FuncDecl{shmem\_add}@(TYPE *dest, TYPE value, int pe);
\end{C11synopsis}
where \TYPE{} is one of \{\CTYPE{int}, \CTYPE{long}, \CTYPE{long long}\}.

\begin{Csynopsis}
void @\FuncDecl{shmem\_\FuncParam{TYPENAME}\_add}@(TYPE *dest, TYPE value, int pe);
\end{Csynopsis}
where \TYPE{} is one of \{\CTYPE{int}, \CTYPE{long}, \CTYPE{long long}\}
and has a corresponding \TYPENAME{} specified by Table~\ref{stdamotypes}.
\end{DeprecateBlock}

\begin{Fsynopsis}
INTEGER pe
INTEGER*4  value_i4
CALL @\FuncDecl{SHMEM\_INT4\_ADD}@(dest, value_i4, pe)
INTEGER*8 value_i8
CALL @\FuncDecl{SHMEM\_INT8\_ADD}@(dest, value_i8, pe)
\end{Fsynopsis}

\begin{apiarguments}
    \apiargument{IN}{ctx}{\oldtext{The context on which to perform the operation.} \newtext{A context handle specifying the context on which to perform the operation.}
        When this argument is not provided, the operation is performed on
        \oldtext{\CONST{SHMEM\_CTX\_DEFAULT}} \newtext{the default context}.}
    \apiargument{OUT}{dest}{The remotely accessible integer data object to be
        updated  on the remote \ac{PE}.  When using \CorCpp, the type of
        \dest{} should match that implied in the SYNOPSIS section.}
    \apiargument{IN}{value}{The value to be atomically added to \dest. When using \CorCpp, the type of \VAR{value} should match that  implied  in
        the SYNOPSIS  section.  When using \Fortran, it must be of type
        integer with an element size of \dest.}
    \apiargument{IN}{pe}{An integer that indicates the \ac{PE} number upon which
        \dest{} is to be updated.  When using \Fortran, it must be a default
        integer value.}
\end{apiarguments}

\apidescription{
    The \FUNC{shmem\_atomic\_add} routine performs an atomic add operation. It adds
    \VAR{value} to \dest{} on \ac{PE} \VAR{pe} and atomically updates the \dest{}
    without returning the value.
    \newtext{
    If the context handle \VAR{ctx} does not correspond to a valid context,
    the behavior is undefined.
    }
 }

\apidesctable{
    When using \Fortran, \VAR{dest} and \VAR{value} must be of the following type:
}{Routine}{Data type of \VAR{dest} and \VAR{value}}

\apitablerow{SHMEM\_INT4\_ADD}{\CONST{4}-byte integer}
\apitablerow{SHMEM\_INT8\_ADD}{\CONST{8}-byte integer}

\apireturnvalues{
    None.
}

\apinotes{
    None.
}

\begin{apiexamples}

\apicexample
    {}
    {./example_code/shmem_atomic_add_example.c}
    {}

\end{apiexamples}

\end{apidefinition}


\subsubsection{\textbf{SHMEM\_ATOMIC\_FETCH\_AND}}
\label{subsec:shmem_atomic_fetch_and}
\apisummary{
  Atomically perform a fetching bitwise AND operation on a remote data object.
}

\begin{apidefinition}

\begin{C11synopsis}
TYPE @\FuncDecl{shmem\_atomic\_fetch\_and}@(TYPE *dest, TYPE value, int pe);
TYPE @\FuncDecl{shmem\_atomic\_fetch\_and}@(shmem_ctx_t ctx, TYPE *dest, TYPE value, int pe);
\end{C11synopsis}
where \TYPE{} is one of the bitwise \ac{AMO} types specified by
Table~\ref{bitamotypes}.

\begin{Csynopsis}
TYPE @\FuncDecl{shmem\_\FuncParam{TYPENAME}\_atomic\_fetch\_and}@(TYPE *dest, TYPE value, int pe);
TYPE @\FuncDecl{shmem\_ctx\_\FuncParam{TYPENAME}\_atomic\_fetch\_and}@(shmem_ctx_t ctx, TYPE *dest, TYPE value, int pe);
\end{Csynopsis}
where \TYPE{} is one of the bitwise \ac{AMO} types and has a corresponding
\TYPENAME{} specified by Table~\ref{bitamotypes}.

\begin{apiarguments}

  \apiargument{IN}{ctx}{A context handle specifying the context on which to perform the operation.
    When this argument is not provided, the operation is performed on
    the default context.}
  \apiargument{OUT}{dest}{A pointer to the remotely accessible data object to
    be updated.}
  \apiargument{IN}{value}{The operand to the bitwise AND operation.}
  \apiargument{IN}{pe}{An integer value for the \ac{PE} on which \VAR{dest}
    is to be updated.}

\end{apiarguments}

\apidescription{
  \FUNC{shmem\_atomic\_fetch\_and} atomically performs a fetching bitwise AND
  on the remotely accessible data object pointed to by \VAR{dest} at PE
  \VAR{pe} with the operand \VAR{value}.
  If the context handle \VAR{ctx} does not correspond to a valid context,
  the behavior is undefined.
}

\apireturnvalues{
  The value pointed to by \VAR{dest} on PE \VAR{pe} immediately before the
  operation is performed.
}

\apinotes{
  None.
}

\end{apidefinition}


\subsubsection{\textbf{SHMEM\_ATOMIC\_AND}}
\label{subsec:shmem_atomic_and}
\apisummary{
  Atomically perform a non-fetching bitwise AND operation on a
  remote data object.
}

\begin{apidefinition}

\begin{C11synopsis}
void @\FuncDecl{shmem\_atomic\_and}@(TYPE *dest, TYPE value, int pe);
void @\FuncDecl{shmem\_atomic\_and}@(shmem_ctx_t ctx, TYPE *dest, TYPE value, int pe);
\end{C11synopsis}
where \TYPE{} is one of the bitwise \ac{AMO} types specified by
Table~\ref{bitamotypes}.

\begin{Csynopsis}
void @\FuncDecl{shmem\_\FuncParam{TYPENAME}\_atomic\_and}@(TYPE *dest, TYPE value, int pe);
void @\FuncDecl{shmem\_ctx\_\FuncParam{TYPENAME}\_atomic\_and}@(shmem_ctx_t ctx, TYPE *dest, TYPE value, int pe);
\end{Csynopsis}
where \TYPE{} is one of the bitwise \ac{AMO} types and has a corresponding
\TYPENAME{} specified by Table~\ref{bitamotypes}.

\begin{apiarguments}

  \apiargument{IN}{ctx}{A context handle specifying the context on which to perform the operation.
    When this argument is not provided, the operation is performed on
    the default context.}
  \apiargument{OUT}{dest}{A pointer to the remotely accessible data object to
    be updated.}
  \apiargument{IN}{value}{The operand to the bitwise AND operation.}
  \apiargument{IN}{pe}{An integer value for the \ac{PE} on which \VAR{dest}
    is to be updated.}

\end{apiarguments}

\apidescription{
  \FUNC{shmem\_atomic\_and} atomically performs a non-fetching bitwise AND
  on the remotely accessible data object pointed to by \VAR{dest} at PE
  \VAR{pe} with the operand \VAR{value}.
  If the context handle \VAR{ctx} does not correspond to a valid context,
  the behavior is undefined.
}

\apireturnvalues{
  None.
}

\apinotes{
  None.
}

\end{apidefinition}


\subsubsection{\textbf{SHMEM\_ATOMIC\_FETCH\_OR}}
\label{subsec:shmem_atomic_fetch_or}
\apisummary{
  Atomically perform a fetching bitwise OR operation on a remote data object.
}

\begin{apidefinition}

\begin{C11synopsis}
TYPE @\FuncDecl{shmem\_atomic\_fetch\_or}@(TYPE *dest, TYPE value, int pe);
TYPE @\FuncDecl{shmem\_atomic\_fetch\_or}@(shmem_ctx_t ctx, TYPE *dest, TYPE value, int pe);
\end{C11synopsis}
where \TYPE{} is one of the bitwise \ac{AMO} types specified by
Table~\ref{bitamotypes}.

\begin{Csynopsis}
TYPE @\FuncDecl{shmem\_\FuncParam{TYPENAME}\_atomic\_fetch\_or}@(TYPE *dest, TYPE value, int pe);
TYPE @\FuncDecl{shmem\_ctx\_\FuncParam{TYPENAME}\_atomic\_fetch\_or}@(shmem_ctx_t ctx, TYPE *dest, TYPE value, int pe);
\end{Csynopsis}
where \TYPE{} is one of the bitwise \ac{AMO} types and has a corresponding
\TYPENAME{} specified by Table~\ref{bitamotypes}.

\begin{apiarguments}

  \apiargument{IN}{ctx}{A context handle specifying the context on which to perform the operation.
    When this argument is not provided, the operation is performed on
    the default context.}
  \apiargument{OUT}{dest}{A pointer to the remotely accessible data object to
    be updated.}
  \apiargument{IN}{value}{The operand to the bitwise OR operation.}
  \apiargument{IN}{pe}{An integer value for the \ac{PE} on which \VAR{dest}
    is to be updated.}

\end{apiarguments}

\apidescription{
  \FUNC{shmem\_atomic\_fetch\_or} atomically performs a fetching bitwise OR
  on the remotely accessible data object pointed to by \VAR{dest} at PE
  \VAR{pe} with the operand \VAR{value}.
  If the context handle \VAR{ctx} does not correspond to a valid context,
  the behavior is undefined.
}

\apireturnvalues{
  The value pointed to by \VAR{dest} on PE \VAR{pe} immediately before the
  operation is performed.
}

\apinotes{
  None.
}

\end{apidefinition}


\subsubsection{\textbf{SHMEM\_ATOMIC\_OR}}
\label{subsec:shmem_atomic_or}
\apisummary{
  Atomically perform a non-fetching bitwise OR operation on a
  remote data object.
}

\begin{apidefinition}

\begin{C11synopsis}
void @\FuncDecl{shmem\_atomic\_or}@(TYPE *dest, TYPE value, int pe);
void @\FuncDecl{shmem\_atomic\_or}@(shmem_ctx_t ctx, TYPE *dest, TYPE value, int pe);
\end{C11synopsis}
where \TYPE{} is one of the bitwise \ac{AMO} types specified by
Table~\ref{bitamotypes}.

\begin{Csynopsis}
void @\FuncDecl{shmem\_\FuncParam{TYPENAME}\_atomic\_or}@(TYPE *dest, TYPE value, int pe);
void @\FuncDecl{shmem\_ctx\_\FuncParam{TYPENAME}\_atomic\_or}@(shmem_ctx_t ctx, TYPE *dest, TYPE value, int pe);
\end{Csynopsis}
where \TYPE{} is one of the bitwise \ac{AMO} types and has a corresponding
\TYPENAME{} specified by Table~\ref{bitamotypes}.

\begin{apiarguments}

  \apiargument{IN}{ctx}{\oldtext{The context on which to perform the operation.} \newtext{A context handle specifying the context on which to perform the operation.}
    When this argument is not provided, the operation is performed on
    \oldtext{\CONST{SHMEM\_CTX\_DEFAULT}} \newtext{the default context}.}
  \apiargument{OUT}{dest}{A pointer to the remotely accessible data object to
    be updated.}
  \apiargument{IN}{value}{The operand to the bitwise OR operation.}
  \apiargument{IN}{pe}{An integer value for the \ac{PE} on which \VAR{dest}
    is to be updated.}

\end{apiarguments}

\apidescription{
  \FUNC{shmem\_atomic\_or} atomically performs a non-fetching bitwise OR
  on the remotely accessible data object pointed to by \VAR{dest} at PE
  \VAR{pe} with the operand \VAR{value}.
  \newtext{
  If the context handle \VAR{ctx} does not correspond to a valid context,
  the behavior is undefined.
  }
}

\apireturnvalues{
  None.
}

\apinotes{
  None.
}

\end{apidefinition}


\subsubsection{\textbf{SHMEM\_ATOMIC\_FETCH\_XOR}}
\label{subsec:shmem_atomic_fetch_xor}
\apisummary{
  Atomically perform a fetching bitwise exclusive OR (XOR) operation on a
  remote data object.
}

\begin{apidefinition}

\begin{C11synopsis}
TYPE @\FuncDecl{shmem\_atomic\_fetch\_xor}@(TYPE *dest, TYPE value, int pe);
TYPE @\FuncDecl{shmem\_atomic\_fetch\_xor}@(shmem_ctx_t ctx, TYPE *dest, TYPE value, int pe);
\end{C11synopsis}
where \TYPE{} is one of the bitwise \ac{AMO} types specified by
Table~\ref{bitamotypes}.

\begin{Csynopsis}
TYPE @\FuncDecl{shmem\_\FuncParam{TYPENAME}\_atomic\_fetch\_xor}@(TYPE *dest, TYPE value, int pe);
TYPE @\FuncDecl{shmem\_ctx\_\FuncParam{TYPENAME}\_atomic\_fetch\_xor}@(shmem_ctx_t ctx, TYPE *dest, TYPE value, int pe);
\end{Csynopsis}
where \TYPE{} is one of the bitwise \ac{AMO} types and has a corresponding
\TYPENAME{} specified by Table~\ref{bitamotypes}.

\begin{apiarguments}

  \apiargument{IN}{ctx}{\oldtext{The context on which to perform the operation.} \newtext{A context handle specifying the context on which to perform the operation.}
    When this argument is not provided, the operation is performed on
    \oldtext{\CONST{SHMEM\_CTX\_DEFAULT}} \newtext{the default context}.}
  \apiargument{OUT}{dest}{A pointer to the remotely accessible data object to
    be updated.}
  \apiargument{IN}{value}{The operand to the bitwise XOR operation.}
  \apiargument{IN}{pe}{An integer value for the \ac{PE} on which \VAR{dest}
    is to be updated.}

\end{apiarguments}

\apidescription{
  \FUNC{shmem\_atomic\_fetch\_xor} atomically performs a fetching bitwise XOR
  on the remotely accessible data object pointed to by \VAR{dest} at PE
  \VAR{pe} with the operand \VAR{value}.
  \newtext{
  If the context handle \VAR{ctx} does not correspond to a valid context,
  the behavior is undefined.
  }
}

\apireturnvalues{
  The value pointed to by \VAR{dest} on PE \VAR{pe} immediately before the
  operation is performed.
}

\apinotes{
  None.
}

\end{apidefinition}


\subsubsection{\textbf{SHMEM\_ATOMIC\_XOR}}
\label{subsec:shmem_atomic_xor}
\apisummary{
  Atomically perform a non-fetching bitwise exclusive OR (XOR) operation on a
  remote data object.
}

\begin{apidefinition}

\begin{C11synopsis}
void @\FuncDecl{shmem\_atomic\_xor}@(TYPE *dest, TYPE value, int pe);
void @\FuncDecl{shmem\_atomic\_xor}@(shmem_ctx_t ctx, TYPE *dest, TYPE value, int pe);
\end{C11synopsis}
where \TYPE{} is one of the bitwise \ac{AMO} types specified by
Table~\ref{bitamotypes}.

\begin{Csynopsis}
void @\FuncDecl{shmem\_\FuncParam{TYPENAME}\_atomic\_xor}@(TYPE *dest, TYPE value, int pe);
void @\FuncDecl{shmem\_ctx\_\FuncParam{TYPENAME}\_atomic\_xor}@(shmem_ctx_t ctx, TYPE *dest, TYPE value, int pe);
\end{Csynopsis}
where \TYPE{} is one of the bitwise \ac{AMO} types and has a corresponding
\TYPENAME{} specified by Table~\ref{bitamotypes}.

\begin{apiarguments}

  \apiargument{IN}{ctx}{\oldtext{The context on which to perform the operation.} \newtext{A context handle specifying the context on which to perform the operation.}
    When this argument is not provided, the operation is performed on
    \oldtext{\CONST{SHMEM\_CTX\_DEFAULT}} \newtext{the default context}.}
  \apiargument{OUT}{dest}{A pointer to the remotely accessible data object to
    be updated.}
  \apiargument{IN}{value}{The operand to the bitwise XOR operation.}
  \apiargument{IN}{pe}{An integer value for the \ac{PE} on which \VAR{dest}
    is to be updated.}

\end{apiarguments}

\apidescription{
  \FUNC{shmem\_atomic\_xor} atomically performs a non-fetching bitwise XOR
  on the remotely accessible data object pointed to by \VAR{dest} at PE
  \VAR{pe} with the operand \VAR{value}.
  \newtext{
  If the context handle \VAR{ctx} does not correspond to a valid context,
  the behavior is undefined.
  }
}

\apireturnvalues{
  None.
}

\apinotes{
  None.
}

\end{apidefinition}






\subsection{Collective Routines}\label{subsec:coll}
\emph{Collective routines} are defined as communication or synchronization
operations on a group of \acp{PE} called an active set. The collective
routines require all \acp{PE} in the active set to simultaneously call the
routine.  A \ac{PE} that is not in the active set calling the collective
routine results in undefined behavior.  All collective routines have an
active set as an input parameter except \FUNC{shmem\_barrier\_all} and
\FUNC{shmem\_sync\_all}. Both \FUNC{shmem\_barrier\_all} and
\FUNC{shmem\_sync\_all} must be called by all \acp{PE} of the \openshmem program.

The active set is defined by the arguments \VAR{PE\_start}, \VAR{logPE\_stride},
and \VAR{PE\_size}.  \VAR{PE\_start} specifies the starting \ac{PE} number and
is the lowest numbered PE in the active set.  The stride between successive
\acp{PE} in the active set is $2^{logPE\_stride}$ and \VAR{logPE\_stride} must
be greater than or equal to zero.  \VAR{PE\_size} specifies the number of
\acp{PE} in the active set and must be greater than zero.  The active set must
satisfy the requirement that its last member corresponds to a valid \ac{PE}
number, that is
$0 \le PE\_start + (PE\_size - 1) * 2^{logPE\_stride} < npes$.
All \acp{PE} participating in the collective routine must provide the same
values for these arguments.  If any of these requirements are not met, the
behavior is undefined.

Another argument important to collective routines is \VAR{pSync}, which is a
symmetric work array.  All \acp{PE} participating in a collective must pass the
same \VAR{pSync} array.  On completion of a collective call, the \VAR{pSync} is
restored to its original contents.  The user is permitted to reuse a \VAR{pSync}
array if all previous collective routines using the \VAR{pSync} array have been
completed by all participating \acp{PE}.  One can use a synchronization
collective routine such as \FUNC{shmem\_barrier} to ensure completion of previous collective
routines. The \FUNC{shmem\_barrier} and \FUNC{shmem\_sync} routines allow the same
\VAR{pSync} array to be used on consecutive calls as long as the \acp{PE}
in the active set do not change.

All collective routines defined in the Specification are blocking.  The
collective routines return on completion.  The collective routines defined in
the \openshmem Specification are:

\begin{itemize}
\item \FUNC{shmem\_barrier\_all}
\item \FUNC{shmem\_barrier}
\item \FUNC{shmem\_sync\_all}
\item \FUNC{shmem\_sync}
\item \FUNC{shmem\_broadcast\{32, 64\}}
\item \FUNC{shmem\_collect\{32, 64\}}
\item \FUNC{shmem\_fcollect\{32, 64\}}
\item Reductions for the following operations: AND, MAX, MIN, SUM, PROD, OR, XOR
\item \FUNC{shmem\_alltoall\{32, 64\}}
\item \FUNC{shmem\_alltoalls\{32, 64\}}
\end{itemize}


\subsubsection{\textbf{SHMEM\_BARRIER\_ALL}}\label{subsec:shmem_barrier_all}
\apisummary{
    Registers the arrival of a \ac{PE} at a barrier and blocks the \ac{PE}
    until all other \acp{PE} arrive at the barrier and all local
    updates and remote memory updates on the default context are completed.
}

\begin{apidefinition}

\begin{Csynopsis}
void @\FuncDecl{shmem\_barrier\_all}@(void);
\end{Csynopsis}

\begin{apiarguments}

    \apiargument{None.}{}{}

\end{apiarguments}

\apidescription{   
    The \FUNC{shmem\_barrier\_all} routine \oldtext{registers the arrival of a \ac{PE} at
    a barrier. Barriers are} \newtext{is} a mechanism for synchronizing all \acp{PE} \newtext{in the default team} at
    once. This routine blocks the \newtext{calling} \ac{PE} until all \acp{PE} have called

    \FUNC{shmem\_barrier\_all}. In a multithreaded \openshmem
    program, only the calling thread is blocked\newtext{, however,
    it may not be called concurrently by multiple threads in the same \ac{PE}}.

    Prior to synchronizing with other \acp{PE}, \FUNC{shmem\_barrier\_all}
    ensures completion of all previously issued memory stores and remote memory
    updates issued on the default context via \openshmem \acp{AMO} and
    \ac{RMA} routine calls such
    as \FUNC{shmem\_int\_add}, \FUNC{shmem\_put32},
    \FUNC{shmem\_put\_nbi}, and \FUNC{shmem\_get\_nbi}.
}

\apireturnvalues{
    None.
}

\apinotes{
    \newtext{%
    The \FUNC{shmem\_barrier\_all} routine is equivalent to calling
    \FUNC{shmem\_ctx\_quiet} on the default context followed by
    calling \FUNC{shmem\_team\_sync} on the default team.
    }

    \oldtext{%
    The \FUNC{shmem\_barrier\_all} routine can be used to
    portably ensure that memory access operations observe remote updates in the order
    enforced by initiator \acp{PE}.
    }

    Calls to \FUNC{shmem\_ctx\_quiet} can be performed prior
    to calling the barrier routine to ensure completion of operations issued on
    additional contexts.
}

\begin{apiexamples}

\apicexample
    { The following \FUNC{shmem\_barrier\_all} example is for \Cstd[11] programs:}
    {./example_code/shmem_barrierall_example.c}
    {}

\end{apiexamples}

\end{apidefinition}


\subsubsection{\textbf{SHMEM\_BARRIER}}\label{subsec:shmem_barrier}
\begin{DeprecateBlock}
\apisummary{
    Performs all operations described in the \FUNC{shmem\_barrier\_all} interface
    but with respect to a subset of \acp{PE} defined by the active set.
}

\begin{apidefinition}

\begin{Csynopsis}
void @\FuncDecl{shmem\_barrier}@(int PE_start, int logPE_stride, int PE_size, long *pSync);
\end{Csynopsis}

\begin{apiarguments}

\apiargument{IN}{PE\_start}{The lowest \ac{PE} number of the active set of \acp{PE}.
    \VAR{PE\_start} must be of type integer.}
\apiargument{IN}{logPE\_stride}{The log (base 2) of the stride between consecutive
    \ac{PE} numbers in the active set.  \VAR{logPE\_stride} must be of type integer.}
\apiargument{IN}{PE\_size}{The number of  \acp{PE} in the active set.  \VAR{PE\_size}
    must be of type integer.}
\apiargument{IN}{pSync}{
    A symmetric work array of size \CONST{SHMEM\_BARRIER\_SYNC\_SIZE}.
    In \CorCpp, \VAR{pSync} must be an array of elements of type \CTYPE{long}.
    Every element
    of this array must be initialized to \CONST{SHMEM\_SYNC\_VALUE} before any of
    the \acp{PE} in the active set enter \FUNC{shmem\_barrier} the first time.}

\end{apiarguments}

\apidescription{
    \FUNC{shmem\_barrier} is a collective synchronization routine over an
    active set. Control returns from \FUNC{shmem\_barrier} after all \acp{PE} in
    the active set (specified by \VAR{PE\_start}, \VAR{logPE\_stride}, and
    \VAR{PE\_size}) have called \FUNC{shmem\_barrier}.

    As with all \openshmem collective routines, each of these routines assumes that
    only \acp{PE} in the active set call the routine.  If a \ac{PE} not  in  the
    active set calls an \openshmem collective routine, the behavior is undefined.

    The values of arguments \VAR{PE\_start}, \VAR{logPE\_stride}, and \VAR{PE\_size}
    must be the same value on all \acp{PE} in the active set.  The same work array must be
    passed in \VAR{pSync} to all \acp{PE} in the active set.

    \FUNC{shmem\_barrier} ensures that all previously issued stores and remote
    memory updates, including \acp{AMO} and \ac{RMA} operations, done by any of the
    \acp{PE} in the active set on the default context are complete before returning.

    The  same  \VAR{pSync} array may be reused on consecutive calls   to
    \FUNC{shmem\_barrier} if the same active set is used.

{\color{Green}
    \FUNC{shmem\_barrier} has been deprecated in favor of the equivalent
    call to \FUNC{shmem\_quiet} followed by a call to
    \FUNC{shmem\_sync} on a team or active set with the desired
    set of \acp{PE}.
}
}

\apireturnvalues{
    None.
}

\apinotes{
    If the \VAR{pSync} array is initialized at the run time, all
    \acp{PE} must be synchronized before the first call to \FUNC{shmem\_barrier}
    (e.g., by \FUNC{shmem\_barrier\_all}) to ensure the array has been initialized
    by all \acp{PE} before it is used.

    If  the active set does not change, \FUNC{shmem\_barrier} can  be called
    repeatedly with the same \VAR{pSync} array.  No additional synchronization
    beyond that implied by \FUNC{shmem\_barrier} itself is necessary in this case.

    The \FUNC{shmem\_barrier} routine can be used to
    portably ensure that memory access operations observe remote updates in the order
    enforced by initiator \acp{PE}.

    Calls to \FUNC{shmem\_ctx\_quiet} can be performed prior
    to calling the barrier routine to ensure completion of operations issued on
    additional contexts.

    \newtext{
    No team-based barrier is provided by \openshmem, as a team may have any
    number of communication contexts associated with the team.
    Applications seeking such an idiom should call \FUNC{shmem\_ctx\_quiet}
    on the desired context, followed by a call to \FUNC{shmem\_team\_sync}
    on the desired team.
    }
}

\begin{apiexamples}

\apicexample
	{The following barrier example is for \Cstd[11] programs:}
	{./example_code/shmem_barrier_example.c}
	{}

\end{apiexamples}

\end{apidefinition}
\end{DeprecateBlock}


\subsubsection{\textbf{SHMEM\_SYNC\_ALL}}\label{subsec:shmem_sync_all}
\begin{DeprecateBlock}
\apisummary{
    \newtext{Performs all operations described in the \FUNC{shmem\_sync} interface
    but implicitly operates on \LibConstRef{SHMEM\_TEAM\_WORLD}.}
}

\begin{apidefinition}

\begin{Csynopsis}
void @\FuncDecl{shmem\_sync\_all}@(void);
\end{Csynopsis}

\begin{apiarguments}

    \apiargument{None.}{}{}

\end{apiarguments}

\apidescription{
{\color{Green}
    This routine blocks the \ac{PE} until all \acp{PE} in the \openshmem
    program have called \FUNC{shmem\_sync\_all}. In a multithreaded \openshmem
    program, only the calling thread is blocked.

    In contrast with the \FUNC{shmem\_barrier\_all} routine,
    \FUNC{shmem\_sync\_all} only ensures completion and visibility of previously issued memory
    stores and does not ensure completion of remote memory updates issued via
    \openshmem routines.

    The \FUNC{shmem\_sync\_all} routine is deprecated in favor of the equivalent call to
    \FUNC{shmem\_sync(SHMEM\_TEAM\_WORLD)}.
}
}

\apireturnvalues{
    None.
}

\apinotes{
    None.
}

\end{apidefinition}
\end{DeprecateBlock}


\subsubsection{\textbf{SHMEM\_SYNC}}\label{subsec:shmem_sync}
\apisummary{
    \newtext{Registers the arrival of a \ac{PE} at a synchronization point and suspends \ac{PE}
    execution until all other \acp{PE} in a given \openshmem team or active set
    arrive at the same synchronization point.}
}

\begin{apidefinition}

{\color{ForestGreen}
\begin{C11synopsis}
int @\FuncDecl{shmem\_sync}@(shmem_team_t team);
\end{C11synopsis}

\begin{Csynopsis}
int @\FuncDecl{shmem\_team\_sync}@(shmem_team_t team);
\end{Csynopsis}
}

\begin{DeprecateBlock}
\begin{CsynopsisCol}
void @\FuncDecl{shmem\_sync}@(int PE_start, int logPE_stride, int PE_size, long *pSync);
\end{CsynopsisCol}
\end{DeprecateBlock}

\begin{apiarguments}

\newtext{%
\apiargument{IN}{team}{The team over which to perform the operation.}%
}

\begin{DeprecateBlock}
\apiargument{IN}{PE\_start}{The lowest \ac{PE} number of the active set of
    \acp{PE}.  \VAR{PE\_start} must be of type integer.}
\apiargument{IN}{logPE\_stride}{The log (base 2) of the stride between
    consecutive \ac{PE} numbers in the active set.  \VAR{logPE\_stride} must be
    of type integer.}
\apiargument{IN}{PE\_size}{The number of \acp{PE} in the active set.
    \VAR{PE\_size} must be of type integer.}
\apiargument{IN}{pSync}{A symmetric work array. In \CorCpp, \VAR{pSync} must be
    of type \CTYPE{long} and size \CONST{SHMEM\_BARRIER\_SYNC\_SIZE}.  Every element of
    this array must be initialized to \CONST{SHMEM\_SYNC\_VALUE} before any of the
    \acp{PE} in the active set enter \FUNC{shmem\_sync} the first time.}
\end{DeprecateBlock}

\end{apiarguments}

\apidescription{
    \FUNC{shmem\_sync} is a collective synchronization routine over
    \newtext{an existing \openshmem team or} an active set

{\color{Green}
    The routine registers the arrival of a \ac{PE} at a synchronization point in the program.
    This is a fast mechanism for synchronizing all \acp{PE} that participate in this
    collective call. The routine blocks the calling \ac{PE} until all \ac{PE} in the
    specified team or active set have called \FUNC{shmem\_sync}. In a multithreaded \openshmem
    program, only the calling thread is blocked.

    Team-based sync routines operate over all \acp{PE} in the provided team argument. All
    \acp{PE} in the provided team must participate in the sync operation. If a team created without
    support for collectives is passed to this or any other team collective routine, the
    behavior is undefined. If an invalid team handle or \LibConstRef{SHMEM\_TEAM\_NULL}
    is passed to this routine, the behavior is undefined.

    Active-set-based sync routines operate over all \acp{PE} in the active set
    defined by the \VAR{PE\_start}, \VAR{logPE\_stride}, \VAR{PE\_size} triplet.
}

    As with all \oldtext{\openshmem} \newtext{active set-based} collective routines,
    each of these routines assumes
    that only \acp{PE} in the active set call the routine.  If a \ac{PE} not in
    the active set calls an \oldtext{\openshmem} \newtext{active set-based} collective routine,
    the behavior is undefined.

    The values of arguments \VAR{PE\_start}, \VAR{logPE\_stride}, and
    \VAR{PE\_size} must be equal on all \acp{PE} in the active set.  The same
    work array must be passed in \VAR{pSync} to all \acp{PE} in the active set.

    In contrast with the \FUNC{shmem\_barrier} routine, \FUNC{shmem\_sync} only
    ensures completion and visibility of previously issued memory stores and does not ensure
    completion of remote memory updates issued via \openshmem routines.

    The same \VAR{pSync} array may be reused on consecutive calls to
    \FUNC{shmem\_sync} if the same active set is used.
}

\apireturnvalues{
    \newtext{Zero on successful local completion. Nonzero otherwise.}
}

\apinotes{

\newtext{%
    There are no specifically defined error codes for sync operations.
    See section \ref{subsec:error_handling} for expected error checking and
    return code behavior specific to implementations. For portable
    error checking and debugging behavior, programs should do their own checks
    for invalid team handles or \LibConstRef{SHMEM\_TEAM\_NULL}
    }

    If the \VAR{pSync} array is initialized at run time, another method of
    synchronization (e.g., \FUNC{shmem\_sync\_all}) must be used before
    the initial use of that \VAR{pSync} array by \FUNC{shmem\_sync}.

    If the active set does not change, \FUNC{shmem\_sync} can be called
    repeatedly with the same \VAR{pSync} array.  No additional synchronization
    beyond that implied by \FUNC{shmem\_sync} itself is necessary in this case.

    The \FUNC{shmem\_sync} routine can be used to portably ensure that
    memory access operations observe remote updates in the order enforced by the
    initiator \acp{PE}, provided that the initiator PE ensures completion of remote
    updates with a call to \FUNC{shmem\_quiet} prior to the call to the
    \FUNC{shmem\_sync} routine.
}

\begin{apiexamples}

\apicexample
    {The following \FUNC{shmem\_sync\_all} and \FUNC{shmem\_sync} example is
    for \Cstd[11] programs:}
    {./example_code/shmem_sync_example.c}
    {}

\end{apiexamples}

\end{apidefinition}


\subsubsection{\textbf{SHMEM\_TEAM\_BROADCAST}}\label{subsec:shmem_team_broadcast}
\apisummary{
    Broadcasts a block of data from one \ac{PE} to one or more destination
    \acp{PE}.
}

\begin{apidefinition}

\begin{Csynopsis}
void @\FuncDecl{shmem\_team\_broadcast32}@(shmem_team_t team, void *dest, const void *source, size_t nelems, int PE_root);
void @\FuncDecl{shmem\_team\_broadcast64}@(shmem_team_t team, void *dest, const void *source, size_t nelems, int PE_root);
\end{Csynopsis}

\begin{apiarguments}

\apiargument{IN}{team}{A valid SHMEM team handle to a team which has been created with support for collective operations.}
\apiargument{OUT}{dest}{A symmetric data object.} 
\apiargument{IN}{source}{A symmetric data object that can be of any data type
    that is permissible for the \dest{} argument.}
\apiargument{IN}{nelems}{The number of elements in \source.  For
    \FUNC{shmem\_team\_broadcast32}, this is the number of
    32-bit halfwords.  nelems must be of type \VAR{size\_t} in \Cstd.}
\apiargument{IN}{PE\_root}{Zero-based ordinal of the \ac{PE}, with respect to
    the team, from which the data is copied. Must be greater than or equal to
    0 and less than the result of calling \FUNC{shmem\_team\_n\_pes(team)}.
    \VAR{PE\_root} must be of type integer.} 

\end{apiarguments}

\apidescription{   
    \openshmem broadcast routines are collective routines over an existing team.
    They copy data object \source{} on the processor specified by \VAR{PE\_root}
    and store the values at \dest{} on the other \acp{PE} that are members of the
    team. The data is not copied to the \dest{} area on the root \ac{PE}.
    
    As with all \openshmem team collective routines, each of these routines assumes that
    only \acp{PE} in the given team call the routine.  If a \ac{PE} not in the
    team calls an \openshmem team collective routine, the behavior is undefined.

    If the team has been created with the \LibConstRef{SHMEM\_TEAM\_NOCOLLECTIVE} option,
    it will not have the required support structures to complete this routine. If
    such a team is passed to this or any other team collective routine, the behavior
    is undefined.

    As with all \openshmem routines where the operation occurs for a given team -
    either when the team is an argument to the routine, or when the team is an attribute
    of the context argument to a routine - the \ac{PE} numbers are relative to the team,
    and must be in the range of 0 to the result of \FUNC{shmem\_team\_n\_pes(team)}.
    
    The values of the argument \VAR{PE\_root} must be the same value on all \acp{PE} in
    the team. The same \dest{} and \source{} data objects must be passed by all \acp{PE}
    in the team.

    Upon return from a broadcast routine, the following are true for the local
    \ac{PE}:
    \begin{itemize}
    \item If the current \ac{PE} is not the root \ac{PE},
      the \dest{} data object is updated.
    \item The \source{} data object may be safely reused.
    \end{itemize}
}

\apidesctable{
The  \dest{}  and \source{} data  objects must conform to certain typing
constraints, which are as follows:
}{Routine}{Data type of \VAR{dest} and \VAR{source}}

\apitablerow{shmem\_broadcast64}{Any noncharacter
    type that has an element size of \CONST{64} bits. No
    \CorCpp{} structures are allowed.}
\apitablerow{shmem\_broadcast32}{Any noncharacter
    type that has an element size of \CONST{32} bits. No
    \CorCpp{} structures are allowed.}

\apireturnvalues{
    None.
}

\apinotes{
    All \openshmem team collective routines use symmetric data structures associated
    with the team to synchronize and share data. By default, new teams that result from
    split operations will have these structures.

    Multiple calls to the same collective routine for the same team by different threads
    must avoid any simultaneous updates to these structures. In general, this will mean
    that threads will need to serialize access to teams.
}

\end{apidefinition}


\subsubsection{\textbf{SHMEM\_BROADCAST}}\label{subsec:shmem_broadcast}
\apisummary{
    Broadcasts a block of data from one \ac{PE} to one or more destination
    \acp{PE}.
}

\begin{apidefinition}

%% C11
{\color{Green}
\begin{C11synopsis}
int @\FuncDecl{shmem\_broadcast\FuncParam{SIZE}}@(shmem_team_t team, void *dest, const void *source, size_t nelems, int PE_root);
\end{C11synopsis}
where \SIZE{} is one of \CONST{32, 64}.

\begin{CsynopsisCol}
int @\FuncDecl{shmem\_broadcast}@(shmem_team_t team, TYPE *dest, const TYPE *source, size_t nelems, int PE_root);
\end{CsynopsisCol}
where \TYPE{} is one of the standard \ac{RMA} types specified by Table \ref{stdrmatypes}.

\begin{CsynopsisCol}
int @\FuncDecl{shmem\_broadcastmem}@(shmem_team_t team, TYPE *dest, const TYPE *source, size_t nelems, int PE_root);
\end{CsynopsisCol}
}

%% C/C++
\begin{Csynopsis}
\end{Csynopsis}
{\color{Green}

\begin{CsynopsisCol}
int @\FuncDecl{shmem\_team\_broadcast\FuncParam{SIZE}}@(shmem_team_t team, void *dest, const void *source, size_t nelems, int PE_root);
\end{CsynopsisCol}
where \SIZE{} is one of \CONST{32, 64}.

\begin{CsynopsisCol}
int @\FuncDecl{shmem\_team\_\FuncParam{TYPENAME}\_broadcast}@(shmem_team_t team, TYPE *dest, const TYPE *source, size_t nelems, int PE_root);
\end{CsynopsisCol}
where \TYPE{} is one of the standard \ac{RMA} types and has a corresponding \TYPENAME{} specified by Table \ref{stdrmatypes}.

\begin{CsynopsisCol}
int @\FuncDecl{shmem\_team\_broadcastmem}@(shmem_team_t team, TYPE *dest, const TYPE *source, size_t nelems, int PE_root);
\end{CsynopsisCol}
}

\begin{DeprecateBlock}
\begin{CsynopsisCol}
void @\FuncDecl{shmem\_broadcast32}@(void *dest, const void *source, size_t nelems, int PE_root, int PE_start, int logPE_stride, int PE_size, long *pSync);
void @\FuncDecl{shmem\_broadcast64}@(void *dest, const void *source, size_t nelems, int PE_root, int PE_start, int logPE_stride, int PE_size, long *pSync);
\end{CsynopsisCol}
\end{DeprecateBlock}

\begin{Fsynopsis}
INTEGER nelems, PE_root, PE_start, logPE_stride, PE_size
INTEGER pSync(SHMEM_BCAST_SYNC_SIZE)
CALL @\FuncDecl{SHMEM\_BROADCAST4}@(dest, source, nelems, PE_root, PE_start, logPE_stride, PE_size, pSync)
CALL @\FuncDecl{SHMEM\_BROADCAST8}@(dest, source, nelems, PE_root, PE_start, logPE_stride, PE_size, pSync)
CALL @\FuncDecl{SHMEM\_BROADCAST32}@(dest, source, nelems, PE_root, PE_start, logPE_stride, PE_size,pSync)
CALL @\FuncDecl{SHMEM\_BROADCAST64}@(dest, source, nelems, PE_root, PE_start, logPE_stride, PE_size,pSync)
\end{Fsynopsis}
 
\begin{apiarguments}

\apiargument{OUT}{dest}{A symmetric data object. \newtext{See the table below in this description
    for allowable types.}} 
\apiargument{IN}{source}{A symmetric data object that can be of any data type
    that is permissible for the \dest{} argument.}
\apiargument{IN}{nelems}{The number of elements in \source.
    nelems must be of type \VAR{size\_t} in \Cstd.  When
    using \Fortran, it must be a default integer value.}
\apiargument{IN}{PE\_root}{Zero-based ordinal of the \ac{PE}, with respect to
    the \newtext{team or} active set, from which the data is copied.
    \VAR{PE\_root} must be of type \CTYPE{int}.
    When using \Fortran, it must be a default integer value.}

\newtext{%
\apiargument{IN}{team}{The team over which to perform the operation.}%
}

\begin{DeprecateBlock}
\apiargument{IN}{PE\_start}{The lowest \ac{PE} number of the active set of
    \acp{PE}.  \VAR{PE\_start} must be of type integer.  When using \Fortran,
    it must be a default integer value.}
\apiargument{IN}{logPE\_stride}{ The log (base 2) of the stride between
    consecutive \ac{PE} numbers in the active set. \VAR{log\_PE\_stride} must be of
    type integer.  When using \Fortran, it must be a default integer value.}
\apiargument{IN}{PE\_size}{ The number of \acp{PE} in the active set.
    \VAR{PE\_size} must be of type integer.  When using \Fortran, it must be a
    default integer value.}
\apiargument{IN}{pSync}{
    A symmetric work array of size \CONST{SHMEM\_BCAST\_SYNC\_SIZE}.
    In \CorCpp, \VAR{pSync} must be an array of elements of type \CTYPE{long}.
    In \Fortran, \VAR{pSync} must be an array of elements of default integer type.
    Every element of this array must be initialized with the value
    \CONST{SHMEM\_SYNC\_VALUE} before any of the \acp{PE} in the active set
    enters \FUNC{shmem\_broadcast}.}
\end{DeprecateBlock}

\end{apiarguments}

\apidescription{   
    \openshmem broadcast routines are collective routines \newtext{over an active set or
    existing \openshmem team}. They copy data object
    \source{} on the processor specified by \VAR{PE\_root} and store the values at
    \dest{} on the other \acp{PE} \newtext{particpating in the collective operation.}
    \oldtext{specified by the triplet \VAR{PE\_start}, \VAR{logPE\_stride}, \VAR{PE\_size}.} %%
    The data is not copied to the \dest{} area on the root \ac{PE}.

    {\color{Green}
    The same \dest{} and \source{} data objects and the same value of \VAR{PE\_root} must be
    passed by all \acp{PE} particpating in the collective operation.

    Team-based broadcast routines operate over all \acp{PE} in the provided team argument. All
    \acp{PE} in the provided team must participate in the operation.
    If an invalid team handle or \LibConstRef{SHMEM\_TEAM\_NULL} is passed to this routine,
    the behavior is undefined.

    As with all team-based \openshmem routines, \ac{PE}
    numbering is relative to the team. The specified root \ac{PE} must be a valid \ac{PE}
    number for the team, between \CONST{0} and \VAR{N-1}, where \VAR{N} is
    the size of the team.

    Active-set-based broadcast routines operate over all \acp{PE} in the active set
    defined by the \VAR{PE\_start}, \VAR{logPE\_stride}, \VAR{PE\_size} triplet.
    }

    As with all \newtext{active-set-based} \oldtext{\openshmem} collective routines,
    each of these routines assumes that
    only \acp{PE} in the active set call the routine.  If a \ac{PE} not in the
    active set calls an \newtext{active-set-based} \oldtext{\openshmem}
    collective routine, the behavior is undefined.
    
    The values of arguments \VAR{PE\_root}, \VAR{PE\_start}, \VAR{logPE\_stride},
    and \VAR{PE\_size} must be the same value on all \acp{PE} in the active set.
    \newtext{The value of \VAR{PE\_root} must be between \CONST{0} and \VAR{PE\_size}.}
    The same \VAR{pSync} work array must be passed by all \acp{PE} in the active set.

    Before any \ac{PE} calls a broadcast routine, the following conditions must be ensured:
    \begin{itemize}
    \item The \dest{} array on all \acp{PE} \newtext{participating in the broadcast}
      \oldtext{in the active set} %%
      is ready to accept the broadcast data.
    \item \newtext{If using active-set-based routines,} the
      \VAR{pSync} array on all \acp{PE} in the
      active set is not still in use from a prior call to a collective
      \openshmem routine.
    \end{itemize}
    Otherwise, the behavior is undefined.
    
    Upon return from a broadcast routine, the following are true for the local
    \ac{PE}:
    \begin{itemize}
    \item If the current \ac{PE} is not the root \ac{PE},
      the \dest{} data object is updated.
    \item The \source{} data object may be safely reused.
    \item \newtext{If using active-set-based routines,}
    the values in the \VAR{pSync} array are restored to the original values.
    \end{itemize}
}

\apidesctable{
The  \dest{}  and \source{} data  objects must conform to certain typing
constraints, which are as follows:
}{Routine}{Data type of \VAR{dest} and \VAR{source}}

\apitablerow{shmem\_broadcastmem}{\Cstd: Any data  type.  nelems is scaled in bytes.}
\apitablerow{shmem\_broadcast8, shmem\_broadcast64}{Any noncharacter
    type that has an element size of \CONST{64} bits. No \Fortran derived types \newtext{nor} \oldtext{or} 
    \CorCpp{} structures are allowed.}
\apitablerow{shmem\_broadcast4, shmem\_broadcast32}{Any noncharacter
    type that has an element size of \CONST{32} bits. No \Fortran derived types \newtext{nor} \oldtext{or} 
    \CorCpp{} structures are allowed.}

\apireturnvalues{
   \newtext{Zero on successful local completion. Nonzero otherwise.}
}

\apinotes{
\newtext{%
    There are no specifically defined error codes for these routines.
    See section \ref{subsec:error_handling} for expected error checking and
    return code behavior specific to implementations. For portable
    error checking and debugging behavior, programs should do their own checks
    for invalid team handles or \LibConstRef{SHMEM\_TEAM\_NULL}
    }

    All \openshmem broadcast routines restore \VAR{pSync} to its original contents.
    Multiple calls to \openshmem routines that use the same \VAR{pSync} array do not
    require that \VAR{pSync} be reinitialized after the first call.
    
    The user must ensure that the \VAR{pSync} array is not being updated by any
    \ac{PE} in the active set while any of the \acp{PE} participates in processing
    of an \openshmem broadcast routine. Be careful to avoid these situations: If the
    \VAR{pSync} array is initialized at run time, before its first use, some type of synchronization is
    needed to ensure that all \acp{PE} in the active set have initialized
    \VAR{pSync} before any of them enter an \openshmem routine called with the
    \VAR{pSync} synchronization array.  A \VAR{pSync} array may be reused on a
    subsequent \openshmem broadcast routine only if none of the \acp{PE} in the
    active set are still processing a prior \openshmem broadcast routine call that
    used the same \VAR{pSync} array. In general, this can be ensured only by doing
    some type of synchronization.        

    Team handle error checking and integer return codes are currently undefined.
    Implementations may define these behaviors as needed, but programs should
    ensure portability by doing their own checks for invalid team handles and for
    \LibConstRef{SHMEM\_TEAM\_NULL}.
}

\begin{apiexamples}

\apicexample
    {In the following example, the call to \FUNC{shmem\_broadcast64} copies \source{}
    on \ac{PE} $0$ to \dest{} on \acp{PE} $1\dots npes-1$.
    
    \CorCpp{} example:}
    {./example_code/shmem_broadcast_example.c}
    {}

\end{apiexamples}

\end{apidefinition}


\subsubsection{\textbf{SHMEM\_COLLECT, SHMEM\_FCOLLECT}}\label{subsec:shmem_collect}
\apisummary{
    Concatenates blocks of data from multiple \acp{PE} to an array in every
    \ac{PE}.
}

\begin{apidefinition}

%% C11
{\color{Green}
\begin{C11synopsis}
int @\FuncDecl{shmem\_collect32}@(void *dest, const void *source, size_t nelems, shmem_team_t team);
int @\FuncDecl{shmem\_collect64}@(void *dest, const void *source, size_t nelems, shmem_team_t team);
int @\FuncDecl{shmem\_fcollect32}@(void *dest, const void *source, size_t nelems, shmem_team_t team);
int @\FuncDecl{shmem\_fcollect64}@(void *dest, const void *source, size_t nelems, shmem_team_t team);
\end{C11synopsis}
}

\begin{Csynopsis}
\end{Csynopsis}
{\color{Green}
\begin{CsynopsisCol}
int @\FuncDecl{shmem\_team\_collect32}@(void *dest, const void *source, size_t nelems, shmem_team_t team);
int @\FuncDecl{shmem\_team\_collect64}@(void *dest, const void *source, size_t nelems, shmem_team_t team);
int @\FuncDecl{shmem\_team\_fcollect32}@(void *dest, const void *source, size_t nelems, shmem_team_t team);
int @\FuncDecl{shmem\_team\_fcollect64}@(void *dest, const void *source, size_t nelems, shmem_team_t team);
\end{CsynopsisCol}
}
\begin{DeprecateBlock}
\begin{CsynopsisCol}
void @\FuncDecl{shmem\_collect32}@(void *dest, const void *source, size_t nelems, int PE_start, int logPE_stride, int PE_size, long *pSync);
void @\FuncDecl{shmem\_collect64}@(void *dest, const void *source, size_t nelems, int PE_start, int logPE_stride, int PE_size, long *pSync);
void @\FuncDecl{shmem\_fcollect32}@(void *dest, const void *source, size_t nelems, int PE_start, int logPE_stride, int PE_size, long *pSync);
void @\FuncDecl{shmem\_fcollect64}@(void *dest, const void *source, size_t nelems, int PE_start, int logPE_stride, int PE_size, long *pSync);
\end{CsynopsisCol}
\end{DeprecateBlock}

\begin{Fsynopsis}
INTEGER nelems
INTEGER PE_start, logPE_stride, PE_size
INTEGER pSync(SHMEM_COLLECT_SYNC_SIZE)
CALL @\FuncDecl{SHMEM\_COLLECT4}@(dest, source, nelems, PE_start, logPE_stride, PE_size, pSync)
CALL @\FuncDecl{SHMEM\_COLLECT8}@(dest, source, nelems, PE_start, logPE_stride, PE_size, pSync)
CALL @\FuncDecl{SHMEM\_COLLECT32}@(dest, source, nelems, PE_start, logPE_stride, PE_size, pSync)
CALL @\FuncDecl{SHMEM\_COLLECT64}@(dest, source, nelems, PE_start, logPE_stride, PE_size, pSync)
CALL @\FuncDecl{SHMEM\_FCOLLECT4}@(dest, source, nelems, PE_start, logPE_stride, PE_size, pSync)
CALL @\FuncDecl{SHMEM\_FCOLLECT8}@(dest, source, nelems, PE_start, logPE_stride, PE_size, pSync)
CALL @\FuncDecl{SHMEM\_FCOLLECT32}@(dest, source, nelems, PE_start, logPE_stride, PE_size, pSync)
CALL @\FuncDecl{SHMEM\_FCOLLECT64}@(dest, source, nelems, PE_start, logPE_stride, PE_size, pSync)
\end{Fsynopsis}

\begin{apiarguments}

\apiargument{OUT}{dest}{A symmetric array large enough
    to accept the concatenation of the \source{} arrays on all participating \acp{PE}.
    \newtext{See table below in this description for allowable data types.}}
\apiargument{IN}{source}{A symmetric data object that can be of any type permissible
    for the \dest{} argument.}
\apiargument{IN}{nelems}{The number of elements in the \source{} array. \VAR{nelems}
    must be of type \VAR{size\_t} for \Cstd. When using \Fortran, it must be
    a default integer value.}

\newtext{%
\apiargument{IN}{team}{A valid \openshmem team handle to a team which has been
    created without disabling support for collective operations.}
}

\apiargument{IN}{PE\_start}{The lowest \ac{PE} number of the active set of
    \acp{PE}.  \VAR{PE\_start} must be of type integer.  When using \Fortran,
    it must be a default integer value.}
\apiargument{IN}{logPE\_stride}{The log (base \CONST{2}) of the stride between
    consecutive \ac{PE} numbers in the active set. \VAR{logPE\_stride} must be of
    type integer.  When using \Fortran, it must be a default integer value.}
\apiargument{IN}{PE\_size}{The number of \acp{PE} in the active set. \VAR{PE\_size}
    must be of type integer.  When using  \Fortran, it must be a default
    integer value.}
\apiargument{IN}{pSync}{
    A symmetric work array of size \CONST{SHMEM\_COLLECT\_SYNC\_SIZE}.
    In \CorCpp, \VAR{pSync} must be an array of elements of type \CTYPE{long}.
    In \Fortran, \VAR{pSync} must be an array of elements of default integer type.
    Every element of this array must be initialized with the value
    \CONST{SHMEM\_SYNC\_VALUE} before any of the \acp{PE} in the active set
    enter \FUNC{shmem\_collect} or \FUNC{shmem\_fcollect}.}

\end{apiarguments}

\apidescription{
{\color{Green}
    \openshmem \FUNC{collect} and \FUNC{fcollect} routines concatenate \VAR{nelems}
    \CONST{64}-bit or \CONST{32}-bit data items from the \source{} array into the
    \dest{} array, over an \openshmem team or active set
    in processor number order. The resultant \dest{} array contains the contribution from
    \acp{PE} as follows:
    
    \begin{itemize}
    \item For an active set, the data from \ac{PE} \VAR{PE\_start} is first, then the
    contribution from \ac{PE} \VAR{PE\_start} + \VAR{PE\_stride} second, and so on.
    \item For a team, the data from \ac{PE} number \CONST{0} in the team is first, then the
    contribution from \ac{PE} \CONST{1} in the team, and so on.
    \end{itemize}
    
    The collected result is written to the \dest{} array for all \acp{PE}
    that participate in the collective. The same \dest{} and \source{}
    arrays must be passed by all \acp{PE} that participate in the collective.
}
    
    The \FUNC{fcollect} routines require that \VAR{nelems} be the same value in all
    participating \acp{PE}, while the \FUNC{collect} routines allow \VAR{nelems} to
    vary from \ac{PE} to \ac{PE}.

{\color{Green}
    Team-based collect routines operate over all \acp{PE} in the provided team argument. All
    \acp{PE} in the provided team must participate in the collective. If a team created without
    support for collectives is passed to this or any other team collective routine, the
    behavior is undefined.

    Active-set-based broadcast routines operate over all \acp{PE} in the active set
    defined by the \VAR{PE\_start}, \VAR{logPE\_stride}, \VAR{PE\_size} triplet.
    As with all active-set-based collective routines,
    each of these routines assumes that
    only \acp{PE} in the active set call the routine. If a \ac{PE} not in the
    active set and calls this collective routine, the behavior is undefined.
}
    
    The values of arguments \VAR{PE\_start}, \VAR{logPE\_stride}, and \VAR{PE\_size}
    must be the same value on all \acp{PE} in the active set. The same
    \oldtext{\dest{} and \source{} arrays and the same} %%
    \VAR{pSync} work array must be passed by all \acp{PE} in the active set.
    
    Upon return from a collective routine, the following are true for the local
    \ac{PE}:
    \begin{itemize}
    \item The \dest{} array is updated and the \source{} array may be safely reused. 
    \item \newtext{For active-set-based collectives,} the values in the \VAR{pSync} array are
    restored to the original values.
    \end{itemize}
}

{\color{Green}
\apidesctable{
The  \dest{}  and \source{} data  objects must conform to certain typing
constraints, which are as follows:
}{Routine}{Data type of \VAR{dest} and \VAR{source}}
\apitablerow{\FUNC{shmem\_collect8}, \FUNC{shmem\_collect64}, \FUNC{shmem\_fcollect8}, \FUNC{shmem\_fcollect64}}%
    {Any noncharacter type that has an element size of \CONST{64} bits. No \Fortran derived types nor
    \CorCpp{} structures are allowed.}
\apitablerow{\FUNC{shmem\_collect4}, \FUNC{shmem\_collect32}, \FUNC{shmem\_fcollect4}, \FUNC{shmem\_fcollect32}}%
    {Any noncharacter type that has an element size of \CONST{32} bits. No \Fortran derived types nor
    \CorCpp{} structures are allowed.}
}

\apireturnvalues{
    \newtext{Zero on successful local completion. Nonzero otherwise.}
}

\apinotes{
\newtext{%
    There are no specifically defined error codes for sync operations.
    See section \ref{subsec:error_handling} for expected error checking and
    return code behavior specific to implementations. For portable
    error checking and debugging behavior, programs should do their own checks
    for invalid team handles or \LibConstRef{SHMEM\_TEAM\_NULL}.
}

    All \openshmem collective routines reset the values in \VAR{pSync} before they
    return, so a particular \VAR{pSync} buffer need only be initialized the first
    time it is used.
    
    The user must ensure that the \VAR{pSync} array is not being updated on any \ac{PE}
    in the active set while any of the \acp{PE} participate in processing of an
    \openshmem collective routine.  Be careful to avoid these situations: If the
    \VAR{pSync} array is initialized at run time, some type of synchronization is
    needed to ensure that all \acp{PE} in the working set have initialized
    \VAR{pSync} before any of them  enter an \openshmem routine called with the
    \VAR{pSync} synchronization array.  A \VAR{pSync} array can be reused on a
    subsequent \openshmem collective routine only if none of the \acp{PE} in the
    active set  are still processing a  prior \openshmem collective routine call
    that used the same \VAR{pSync} array.  In general, this may be ensured only by
    doing some type of synchronization.  
    
    The collective routines operate on active \ac{PE} sets that have a
    non-power-of-two \VAR{PE\_size} with some performance degradation.  They operate
    with no performance degradation when \VAR{nelems} is a non-power-of-two value.
}

\begin{apiexamples}

\apicexample
    {The following \FUNC{shmem\_collect} example is for \CorCpp{} programs:}
    {./example_code/shmem_collect_example.c}
    {}

\apifexample
    {The following \FUNC{SHMEM\_COLLECT} example is for \Fortran programs:}
    {./example_code/shmem_collect_example.f90}
    {}

\end{apiexamples}

\end{apidefinition}


\subsubsection{\textbf{SHMEM\_TEAM\_COLLECT, SHMEM\_TEAM\_FCOLLECT}}\label{subsec:shmem_team_collect}
\apisummary{
    Concatenates blocks of data from multiple \acp{PE} int a team to an array in every
    \ac{PE} in the team.
}

\begin{apidefinition}

\begin{Csynopsis}
void @\FuncDecl{shmem\_team\_collect32}@(shmem_team_t team, void *dest, const void *source, size_t nelems);
void @\FuncDecl{shmem\_team\_collect64}@(shmem_team_t team, void *dest, const void *source, size_t nelems);
\end{Csynopsis}

\begin{apiarguments}

\apiargument{IN}{team}{A valid \openshmem team handle to a team which has been
    created without disabling support for collective operations.}
\apiargument{OUT}{dest}{A symmetric array large enough
    to accept the concatenation of the \source{} arrays on all \acp{PE} in the team.
    See table below in this description for allowable data types.}
\apiargument{IN}{source}{A symmetric data object that can be of any type permissible
    for the \dest{} argument.}
\apiargument{IN}{nelems}{The number of elements in the \source{} array. \VAR{nelems}
    must be of type \VAR{size\_t}.}

\end{apiarguments}

\apidescription{   
    \openshmem \FUNC{team\_collect} and \FUNC{team\_fcollect} are collective routines
    over an existing team. These routines concatenate \VAR{nelems}
    \CONST{64}-bit or \CONST{32}-bit data items from the \source{} array into the
    \dest{} array, over all \acp{PE} in the specified \VAR{team} in processor number order.
    The resultant \dest{} array contains the contribution from \ac{PE} with \VAR{team} number 0
    first, then the contribution from \ac{PE} with \VAR{team} number 1, and so on.
    The collected result is written to the \dest{} array for all \acp{PE} in the team.
    
    The \FUNC{fcollect} routines require that \VAR{nelems} be the same value in all
    participating \acp{PE}, while the \FUNC{collect} routines allow \VAR{nelems} to
    vary from \ac{PE} to \ac{PE}.

    As with all \openshmem team collective routines, each of these routines assumes that
    only \acp{PE} in the given team call the routine.  If a \ac{PE} not in the
    team calls an \openshmem team collective routine, the behavior is undefined.

    If the team has been created with the \LibConstRef{SHMEM\_TEAM\_NOCOLLECTIVE} option,
    it will not have the required support structures to complete this routine. If
    such a team is passed to this or any other team collective routine, the behavior
    is undefined.

    As with all \openshmem routines where the operation occurs for a given team -
    either when the team is an argument to the routine, or when the team is an attribute
    of the context argument to a routine - the \ac{PE} numbers are relative to the team,
    and must be in the range of 0 to N-1, where N is the result of \FUNC{shmem\_team\_n\_pes(team)}.

    The same \dest{} and \source{} data objects must be passed by all \acp{PE}
    in the team.

    Upon return from a collective routine, the following are true for the local
    \ac{PE}:
    \begin{itemize}
    \item The \dest{} array is updated.
    \item The \source{} array may be safely reused.
    \end{itemize}

    Error checking will be done to ensure a valid team handle is provided.
    All errors are considered fatal and will result in the job aborting
    with an informative error message.
}

\apidesctable{
The  \dest{}  and \source{} data  objects must conform to certain typing
constraints, which are as follows:
}{Routine}{Data type of \VAR{dest} and \VAR{source}}

\apitablerow{shmem\_team\_collect64, shmem\_team\_fcollect64}{Any noncharacter
    type that has an element size of \CONST{64} bits.
    \CorCpp{} structures are NOT allowed.}
\apitablerow{shmem\_team\_collect32, shmem\_team\_fcollect32}{Any noncharacter
    type that has an element size of \CONST{32} bits.
    \CorCpp{} structures are NOT allowed.}

\apireturnvalues{
    None.
}

\apinotes{
    All \openshmem team collective routines use symmetric data structures associated
    with the team to synchronize and share data. By default, new teams that result from
    split operations will have these structures.

    Multiple calls to the same collective routine for the same team by different threads
    must avoid any simultaneous updates to these structures. In general, this will mean
    that threads will need to serialize access to teams.
}

\begin{apiexamples}

\end{apiexamples}

\end{apidefinition}


\subsubsection{\textbf{SHMEM\_REDUCTIONS}}\label{subsec:shmem_reductions}
\apisummary{
    The following functions perform reduction operations across all
    \acp{PE} in a set of \acp{PE}.
}

\begin{apidefinition}

\begin{table}[h]
  \begin{center}
    \begin{tabular}{|l|l|l|l|l|}
      \hline
      \TYPE              & \TYPENAME  & \multicolumn{3}{c|}{Operations Supporting \TYPE}\\ \hline
      short              & short      & AND, OR, XOR & MAX, MIN & SUM, PROD \\ \hline
      int                & int        & AND, OR, XOR & MAX, MIN & SUM, PROD \\ \hline
      long               & long       & AND, OR, XOR & MAX, MIN & SUM, PROD \\ \hline
      long long          & longlong   & AND, OR, XOR & MAX, MIN & SUM, PROD \\ \hline
      float              & float      & & MAX, MIN & SUM, PROD \\ \hline
      double             & double     & & MAX, MIN & SUM, PROD \\ \hline
      long double        & longdouble & & MAX, MIN & SUM, PROD \\ \hline
      double \_Complex   & complexd   & & & SUM, PROD \\ \hline
      float  \_Complex   & complexf   & & & SUM, PROD \\ \hline
    \end{tabular}
    \TableCaptionRef{Reduction Types, Names and Supporting Operations}
    \label{reducetypes}
  \end{center} 
\end{table}

\paragraph{AND}
Performs a bitwise AND reduction across a set of \acp{PE}.\newline

%% C11
{\color{Green}
\begin{C11synopsis}
void @\FuncDecl{shmem\_and\_to\_all}@(TYPE *dest, const TYPE *source, int nreduce, shmem_team_t team);
\end{C11synopsis}
where \TYPE{} is one of the integer types supported for the AND operation as specified by Table \ref{reducetypes}.
}

%% C/C++
\begin{Csynopsis}
\end{Csynopsis}
{\color{Green}
\begin{CsynopsisCol}
void @\FuncDecl{shmem\_team\_\FuncParam{TYPENAME}\_and\_to\_all}@(TYPE *dest, const TYPE *source, int nreduce, shmem_team_t team);
\end{CsynopsisCol}
}
\begin{DeprecateBlock}
\begin{CsynopsisCol}
void @\FuncDecl{shmem\_\FuncParam{TYPENAME}\_and\_to\_all}@(TYPE *dest, const TYPE *source, int nreduce, int PE_start, int logPE_stride, int PE_size, short *pWrk, long *pSync);
\end{CsynopsisCol}
\end{DeprecateBlock}
\newtext{where \TYPE{} is one of the integer types supported for the AND operation and has a corresponding \TYPENAME{} as specified by Table \ref{reducetypes}.}

%% Fortran
\begin{Fsynopsis}
CALL @\FuncDecl{SHMEM\_INT4\_AND\_TO\_ALL}@(dest, source, nreduce, PE_start, logPE_stride, PE_size, pWrk, pSync)
CALL @\FuncDecl{SHMEM\_INT8\_AND\_TO\_ALL}@(dest, source, nreduce, PE_start, logPE_stride, PE_size, pWrk, pSync)
\end{Fsynopsis}
%%

\paragraph{OR}
Performs a bitwise OR reduction across a set of \acp{PE}.\newline

%% C11
{\color{Green}
\begin{C11synopsis}
void @\FuncDecl{shmem\_or\_to\_all}@(TYPE *dest, const TYPE *source, int nreduce, shmem_team_t team);
\end{C11synopsis}
where \TYPE{} is one of the integer types supported for the OR operation as specified by Table \ref{reducetypes}.
}

%% C/C++
\begin{Csynopsis}
\end{Csynopsis}
{\color{Green}
\begin{CsynopsisCol}
void @\FuncDecl{shmem\_team\_\FuncParam{TYPENAME}\_or\_to\_all}@(TYPE *dest, const TYPE *source, int nreduce, shmem_team_t team);
\end{CsynopsisCol}
}
\begin{DeprecateBlock}
\begin{CsynopsisCol}
void @\FuncDecl{shmem\_\FuncParam{TYPENAME}\_or\_to\_all}@(TYPE *dest, const TYPE *source, int nreduce, int PE_start, int logPE_stride, int PE_size, short *pWrk, long *pSync);
\end{CsynopsisCol}
\end{DeprecateBlock}
\newtext{where \TYPE{} is one of the integer types supported for the OR operation and has a corresponding \TYPENAME{} as specified by Table \ref{reducetypes}.}

%% Fortran
\begin{Fsynopsis}
CALL @\FuncDecl{SHMEM\_INT4\_OR\_TO\_ALL}@(dest, source, nreduce, PE_start, logPE_stride, PE_size, pWrk, pSync)
CALL @\FuncDecl{SHMEM\_INT8\_OR\_TO\_ALL}@(dest, source, nreduce, PE_start, logPE_stride, PE_size, pWrk, pSync)
\end{Fsynopsis}
%%

\paragraph{XOR}
Performs a bitwise exclusive OR (XOR) reduction across a set of \acp{PE}.\newline

%% C11
{\color{Green}
\begin{C11synopsis}
void @\FuncDecl{shmem\_xor\_to\_all}@(TYPE *dest, const TYPE *source, int nreduce, shmem_team_t team);
\end{C11synopsis}
where \TYPE{} is one of the integer types supported for the XOR operation as specified by Table \ref{reducetypes}.
}

%% C/C++
\begin{Csynopsis}
\end{Csynopsis}
{\color{Green}
\begin{CsynopsisCol}
void @\FuncDecl{shmem\_team\_\FuncParam{TYPENAME}\_xor\_to\_all}@(TYPE *dest, const TYPE *source, int nreduce, shmem_team_t team);
\end{CsynopsisCol}
}
\begin{DeprecateBlock}
\begin{CsynopsisCol}
void @\FuncDecl{shmem\_\FuncParam{TYPENAME}\_xor\_to\_all}@(TYPE *dest, const TYPE *source, int nreduce, int PE_start, int logPE_stride, int PE_size, short *pWrk, long *pSync);
\end{CsynopsisCol}
\end{DeprecateBlock}
\newtext{where \TYPE{} is one of the integer types supported for the XOR operation and has a corresponding \TYPENAME{} as specified by Table \ref{reducetypes}.}

%% Fortran
\begin{Fsynopsis}
CALL @\FuncDecl{SHMEM\_INT4\_XOR\_TO\_ALL}@(dest, source, nreduce, PE_start, logPE_stride, PE_size, pWrk, pSync)
CALL @\FuncDecl{SHMEM\_INT8\_XOR\_TO\_ALL}@(dest, source, nreduce, PE_start, logPE_stride, PE_size, pWrk, pSync)
\end{Fsynopsis}
%%

\paragraph{MAX}
Performs a maximum-value reduction across a set of \acp{PE}.\newline

%% C11
{\color{Green}
\begin{C11synopsis}
void @\FuncDecl{shmem\_max\_to\_all}@(TYPE *dest, const TYPE *source, int nreduce, shmem_team_t team);
\end{C11synopsis}
where \TYPE{} is one of the integer or real types supported for the MAX operation as specified by Table \ref{reducetypes}.
}

%% C/C++
\begin{Csynopsis}
\end{Csynopsis}
{\color{Green}
\begin{CsynopsisCol}
void @\FuncDecl{shmem\_team\_\FuncParam{TYPENAME}\_max\_to\_all}@(TYPE *dest, const TYPE *source, int nreduce, shmem_team_t team);
\end{CsynopsisCol}
}
\begin{DeprecateBlock}
\begin{CsynopsisCol}
void @\FuncDecl{shmem\_\FuncParam{TYPENAME}\_max\_to\_all}@(TYPE *dest, const TYPE *source, int nreduce, int PE_start, int logPE_stride, int PE_size, short *pWrk, long *pSync);
\end{CsynopsisCol}
\end{DeprecateBlock}
\newtext{where \TYPE{} is one of the integer or real types supported for the MAX operation and has a corresponding \TYPENAME{} as specified by Table \ref{reducetypes}.}

%% Fortran
\begin{Fsynopsis}
CALL @\FuncDecl{SHMEM\_INT4\_MAX\_TO\_ALL}@(dest, source, nreduce, PE_start, logPE_stride, PE_size, pWrk, pSync)
CALL @\FuncDecl{SHMEM\_INT8\_MAX\_TO\_ALL}@(dest, source, nreduce, PE_start, logPE_stride, PE_size, pWrk, pSync)
CALL @\FuncDecl{SHMEM\_REAL4\_MAX\_TO\_ALL}@(dest, source, nreduce, PE_start, logPE_stride, PE_size, pWrk, pSync)
CALL @\FuncDecl{SHMEM\_REAL8\_MAX\_TO\_ALL}@(dest, source, nreduce, PE_start, logPE_stride, PE_size, pWrk, pSync)
CALL @\FuncDecl{SHMEM\_REAL16\_MAX\_TO\_ALL}@(dest, source, nreduce, PE_start, logPE_stride, PE_size, pWrk, pSync)
\end{Fsynopsis}

\paragraph{MIN}
Performs a minimum-value reduction across a set of \acp{PE}.\newline

%% C11
{\color{Green}
\begin{C11synopsis}
void @\FuncDecl{shmem\_min\_to\_all}@(TYPE *dest, const TYPE *source, int nreduce, shmem_team_t team);
\end{C11synopsis}
where \TYPE{} is one of the integer or real types supported for the MIN operation as specified by Table \ref{reducetypes}.
}

%% C/C++
\begin{Csynopsis}
\end{Csynopsis}
{\color{Green}
\begin{CsynopsisCol}
void @\FuncDecl{shmem\_team\_\FuncParam{TYPENAME}\_min\_to\_all}@(TYPE *dest, const TYPE *source, int nreduce, shmem_team_t team);
\end{CsynopsisCol}
}
\begin{DeprecateBlock}
\begin{CsynopsisCol}
void @\FuncDecl{shmem\_\FuncParam{TYPENAME}\_min\_to\_all}@(TYPE *dest, const TYPE *source, int nreduce, int PE_start, int logPE_stride, int PE_size, short *pWrk, long *pSync);
\end{CsynopsisCol}
\end{DeprecateBlock}
\newtext{where \TYPE{} is one of the integer or real types supported for the MIN operation and has a corresponding \TYPENAME{} as specified by Table \ref{reducetypes}.}

%% Fortran
\begin{Fsynopsis}
CALL @\FuncDecl{SHMEM\_INT4\_MIN\_TO\_ALL}@(dest, source, nreduce, PE_start, logPE_stride, PE_size, pWrk, pSync)
CALL @\FuncDecl{SHMEM\_INT8\_MIN\_TO\_ALL}@(dest, source, nreduce, PE_start, logPE_stride, PE_size, pWrk, pSync)
CALL @\FuncDecl{SHMEM\_REAL4\_MIN\_TO\_ALL}@(dest, source, nreduce, PE_start, logPE_stride, PE_size, pWrk, pSync)
CALL @\FuncDecl{SHMEM\_REAL8\_MIN\_TO\_ALL}@(dest, source, nreduce, PE_start, logPE_stride, PE_size, pWrk, pSync)
CALL @\FuncDecl{SHMEM\_REAL16\_MIN\_TO\_ALL}@(dest, source, nreduce, PE_start, logPE_stride, PE_size, pWrk, pSync)
\end{Fsynopsis}

\paragraph{SUM}
Performs a sum reduction across a set of \acp{PE}.\newline

%% C11
{\color{Green}
\begin{C11synopsis}
void @\FuncDecl{shmem\_sum\_to\_all}@(TYPE *dest, const TYPE *source, int nreduce, shmem_team_t team);
\end{C11synopsis}
where \TYPE{} is one of the integer, real, or complex types supported for the SUM operation as specified by Table \ref{reducetypes}.
}

%% C/C++
\begin{Csynopsis}
\end{Csynopsis}
{\color{Green}
\begin{CsynopsisCol}
void @\FuncDecl{shmem\_team\_\FuncParam{TYPENAME}\_sum\_to\_all}@(TYPE *dest, const TYPE *source, int nreduce, shmem_team_t team);
\end{CsynopsisCol}
}
\begin{DeprecateBlock}
\begin{CsynopsisCol}
void @\FuncDecl{shmem\_\FuncParam{TYPENAME}\_sum\_to\_all}@(TYPE *dest, const TYPE *source, int nreduce, int PE_start, int logPE_stride, int PE_size, short *pWrk, long *pSync);
\end{CsynopsisCol}
\end{DeprecateBlock}
\newtext{where \TYPE{} is one of the integer, real, or complex types supported for the SUM operation and has a corresponding \TYPENAME{} as specified by Table \ref{reducetypes}.}

%% Fortran
\begin{Fsynopsis}
CALL @\FuncDecl{SHMEM\_COMP4\_SUM\_TO\_ALL}@(dest, source, nreduce, PE_start, logPE_stride, PE_size, pWrk, pSync)
CALL @\FuncDecl{SHMEM\_COMP8\_SUM\_TO\_ALL}@(dest, source, nreduce, PE_start, logPE_stride, PE_size, pWrk, pSync)
CALL @\FuncDecl{SHMEM\_INT4\_SUM\_TO\_ALL}@(dest, source, nreduce, PE_start, logPE_stride, PE_size, pWrk, pSync)
CALL @\FuncDecl{SHMEM\_INT8\_SUM\_TO\_ALL}@(dest, source, nreduce, PE_start, logPE_stride, PE_size, pWrk, pSync)
CALL @\FuncDecl{SHMEM\_REAL4\_SUM\_TO\_ALL}@(dest, source, nreduce, PE_start, logPE_stride, PE_size, pWrk, pSync)
CALL @\FuncDecl{SHMEM\_REAL8\_SUM\_TO\_ALL}@(dest, source, nreduce, PE_start, logPE_stride, PE_size, pWrk, pSync)
CALL @\FuncDecl{SHMEM\_REAL16\_SUM\_TO\_ALL}@(dest, source, nreduce, PE_start, logPE_stride, PE_size, pWrk, pSync)
\end{Fsynopsis}

\paragraph{PROD}
Performs a product reduction across a set of \acp{PE}.\newline

%% C11
{\color{Green}
\begin{C11synopsis}
void @\FuncDecl{shmem\_prod\_to\_all}@(TYPE *dest, const TYPE *source, int nreduce, shmem_team_t team);
\end{C11synopsis}
where \TYPE{} is one of the integer, real, or complex types supported for the PROD operation as specified by Table \ref{reducetypes}.
}

%% C/C++
\begin{Csynopsis}
\end{Csynopsis}
{\color{Green}
\begin{CsynopsisCol}
void @\FuncDecl{shmem\_team\_\FuncParam{TYPENAME}\_prod\_to\_all}@(TYPE *dest, const TYPE *source, int nreduce, shmem_team_t team);
\end{CsynopsisCol}
}
\begin{DeprecateBlock}
\begin{CsynopsisCol}
void @\FuncDecl{shmem\_\FuncParam{TYPENAME}\_prod\_to\_all}@(TYPE *dest, const TYPE *source, int nreduce, int PE_start, int logPE_stride, int PE_size, short *pWrk, long *pSync);
\end{CsynopsisCol}
\end{DeprecateBlock}
\newtext{where \TYPE{} is one of the integer, real, or complex types supported for the PROD operation and has a corresponding \TYPENAME{} as specified by Table \ref{reducetypes}.}

%% Fortran
\begin{Fsynopsis}
CALL @\FuncDecl{SHMEM\_COMP4\_PROD\_TO\_ALL}@(dest, source, nreduce, PE_start, logPE_stride, PE_size, pWrk, pSync)
CALL @\FuncDecl{SHMEM\_COMP8\_PROD\_TO\_ALL}@(dest, source, nreduce, PE_start, logPE_stride, PE_size, pWrk, pSync)
CALL @\FuncDecl{SHMEM\_INT4\_PROD\_TO\_ALL}@(dest, source, nreduce, PE_start, logPE_stride, PE_size, pWrk, pSync)
CALL @\FuncDecl{SHMEM\_INT8\_PROD\_TO\_ALL}@(dest, source, nreduce, PE_start, logPE_stride, PE_size, pWrk, pSync)
CALL @\FuncDecl{SHMEM\_REAL4\_PROD\_TO\_ALL}@(dest, source, nreduce, PE_start, logPE_stride, PE_size, pWrk, pSync)
CALL @\FuncDecl{SHMEM\_REAL8\_PROD\_TO\_ALL}@(dest, source, nreduce, PE_start, logPE_stride, PE_size, pWrk, pSync)
CALL @\FuncDecl{SHMEM\_REAL16\_PROD\_TO\_ALL}@(dest, source, nreduce, PE_start, logPE_stride, PE_size, pWrk, pSync)
\end{Fsynopsis}



\begin{apiarguments}

\apiargument{OUT}{dest}{A symmetric array, of length \VAR{nreduce} elements, to
    receive the result of the reduction routines.  The data type of \dest{} varies
    with the version of the reduction routine being called.  When calling from
    \CorCpp, refer to the SYNOPSIS section for data type information.}
\apiargument{IN}{source}{ A symmetric array, of length \VAR{nreduce} elements, that
    contains one element for each separate reduction routine.  The \source{}
    argument must have the same data type as \dest.}
\apiargument{IN}{nreduce}{The number of elements in the \dest{} and \source{}
    arrays.  \VAR{nreduce} must be of type integer.  When using \Fortran, it
    must be a default integer value.}

\newtext{%
\apiargument{IN}{team}{The team over which to perform the operation.}%
}

\begin{DeprecateBlock}
\apiargument{IN}{PE\_start}{The lowest \ac{PE} number of the active set of
    \acp{PE}.  \VAR{PE\_start} must be of type integer.  When using \Fortran,
    it must be a default integer value.}
\apiargument{IN}{logPE\_stride}{The log (base 2) of the stride between consecutive
    \ac{PE} numbers in the active set.  \VAR{logPE\_stride} must be of type integer.
    When using \Fortran, it must be a default integer value.}
\apiargument{IN}{PE\_size}{The number of \acp{PE} in the active set.
    \VAR{PE\_size} must be of type integer.  When using \Fortran, it must be a
    default integer value.}
\apiargument{IN}{pWrk}{
    A symmetric work array of size at least
    max(\VAR{nreduce}/2 + 1, \CONST{SHMEM\_REDUCE\_MIN\_WRKDATA\_SIZE})
    elements.}
\apiargument{IN}{pSync}{
    A symmetric work array of size \CONST{SHMEM\_REDUCE\_SYNC\_SIZE}.
    In \CorCpp, \VAR{pSync} must be an array of elements of type \CTYPE{long}.
    In \Fortran, \VAR{pSync} must be an array of elements of default integer type.
    Every element of this array must be initialized with the value
    \CONST{SHMEM\_SYNC\_VALUE} before any of the \acp{PE} in the active set
    enter the reduction routine.}
\end{DeprecateBlock}

\end{apiarguments}

\apidescription{
    \openshmem reduction routines \newtext{are collective routines over an active set or
    existing \openshmem team that} compute one or more reductions across symmetric
    arrays on multiple \acp{PE}.  A reduction performs an associative binary routine
    across a set of values.

    The \VAR{nreduce} argument determines the number of separate reductions to
    perform.  The \source{} array on all \acp{PE} \newtext{participating in the reduction}
    \oldtext{in the active set} %%
    provides one element for each reduction.  The results of the reductions are placed in the
    \dest{} array on all \acp{PE} \newtext{participating in the reduction.}
    \oldtext{in the active set.} %%
    
    The \source{} and \dest{} arrays may be the same array, but they may not be
    overlapping arrays. The same \dest{} and \source{} arrays
    must be passed to all \acp{PE} \newtext{participating in the reduction.}
    \oldtext{in the active set.} %%

    {\color{Green}
    Team-based reduction routines operate over all \acp{PE} in the provided team argument. All
    \acp{PE} in the provided team must participate in the reduction.

    If the team has been created with the \LibConstRef{SHMEM\_TEAM\_NOCOLLECTIVE} option,
    it will not have the required support structures to complete this routine. If
    such a team is passed to this or any other team collective routine, the behavior
    is undefined.
    }

    \newtext{Active-set-based reduction routines operate over all \acp{PE} in} the active set
    \oldtext{is} %%
    defined by the \VAR{PE\_start}, \VAR{logPE\_stride}, \VAR{PE\_size} triplet.
    
    As with all \newtext{active-set-based}
    \oldtext{\openshmem} %%
    collective routines, each of these routines assumes
    that only \acp{PE} in the active set call the routine.  If a \ac{PE} not in
    the active set calls an \newtext{active-set-based}
    \oldtext{\openshmem} %%
    collective routine, the behavior is undefined.

    The values of arguments \VAR{nreduce}, \VAR{PE\_start}, \VAR{logPE\_stride},
    and \VAR{PE\_size} must be equal on all \acp{PE} in the active set.
    The same \VAR{pWrk} and \VAR{pSync} work arrays must be passed to all
    \acp{PE} in the active set.

    Before any \ac{PE} calls a reduction routine, the following conditions must be ensured:
    \begin{itemize}
    \item The \dest{} array on all \acp{PE} \newtext{participating in the reduction}
      \oldtext{in the active set} %%
      is ready to accept the results of the \OPR{reduction}.
    \item \newtext{If using active-set-based routines,} the
      \VAR{pWrk} and \VAR{pSync} arrays on all \acp{PE} in the
      active set are not still in use from a prior call to a collective
      \openshmem routine.
    \end{itemize}
    Otherwise, the behavior is undefined.
    
    Upon return from a reduction routine, the following are true for the local
    \ac{PE}:
    \begin{itemize}
    \item The \dest{} array is updated and the \source{} array may be safely reused.
    \item \newtext{If using active-set-based routines,}
    the values in the \VAR{pSync} array are restored to the original values.
    \end{itemize}

    The complex-typed interfaces are only provided for sum and product reductions.
    When the \Cstd translation environment does not support complex types
    \footnote{That is, under \Cstd language standards prior to \Cstd[99] or under \Cstd[11]
    when \CONST{\_\_STDC\_NO\_COMPLEX\_\_} is defined to 1}, an \openshmem
    implementation is not required to provide support for these
    complex-typed interfaces.
}



%\deprecationstart
\apidesctable{
    When calling from \Fortran, the \dest{} date types are as follows:
}{Routine}{Data type}
    \apitablerow{shmem\_int8\_and\_to\_all}{Integer, with an element size of 8 bytes.}
    \apitablerow{shmem\_int4\_and\_to\_all}{Integer, with an element size of 4 bytes.}
    \apitablerow{shmem\_comp8\_max\_to\_all}{Complex, with an element size equal to two 8-byte real values.}
    \apitablerow{shmem\_int4\_max\_to\_all}{Integer, with an element size of 4 bytes.}
    \apitablerow{shmem\_int8\_max\_to\_all}{Integer, with an element size of 8 bytes.}
    \apitablerow{shmem\_real4\_max\_to\_all}{Real, with an element size of 4 bytes.}
    \apitablerow{shmem\_real16\_max\_to\_all}{Real, with an element size of 16 bytes.}
    \apitablerow{shmem\_int4\_min\_to\_all}{Integer, with an element size of 4 bytes.}
    \apitablerow{shmem\_int8\_min\_to\_all}{Integer, with an element size of 8 bytes.}
    \apitablerow{shmem\_real4\_min\_to\_all}{Real, with an element size of 4 bytes.}
    \apitablerow{shmem\_real8\_min\_to\_all}{Real, with an element size of 8 bytes.}
    \apitablerow{shmem\_real16\_min\_to\_all}{Real,with an element size of 16 bytes.}
    \apitablerow{shmem\_comp4\_sum\_to\_all}{Complex, with an element size equal to two 4-byte real values.}
    \apitablerow{shmem\_comp8\_sum\_to\_all}{Complex, with an element size equal to two 8-byte real values.}
    \apitablerow{shmem\_int4\_sum\_to\_all}{Integer, with an element size of 4 bytes.}
    \apitablerow{shmem\_int8\_sum\_to\_all}{Integer, with an element size of 8 bytes..}
    \apitablerow{shmem\_real4\_sum\_to\_all}{Real, with an element size of 4 bytes.}
    \apitablerow{shmem\_real8\_sum\_to\_all}{Real, with an element size of 8 bytes.}
    \apitablerow{shmem\_real16\_sum\_to\_all}{Real, with an element size of 16 bytes.}
    \apitablerow{shmem\_comp4\_prod\_to\_all}{Complex, with an element size equal to two 4-byte real values.}
    \apitablerow{shmem\_comp8\_prod\_to\_all}{Complex, with an element size equal to two 8-byte real values.}
    \apitablerow{shmem\_int4\_prod\_to\_all}{Integer, with an element size of 4 bytes.}
    \apitablerow{shmem\_int8\_prod\_to\_all}{Integer, with an element size of 8 bytes.}
    \apitablerow{shmem\_real4\_prod\_to\_all}{Real, with an element size of 4 bytes.}
    \apitablerow{shmem\_real8\_prod\_to\_all}{Real, with an element size of 8 bytes.}
    \apitablerow{shmem\_real16\_prod\_to\_all}{Real, with an element size of 16 bytes.}
    \apitablerow{shmem\_int8\_or\_to\_all}{Integer, with an element size of 8 bytes.}
    \apitablerow{shmem\_int4\_or\_to\_all}{Integer, with an element size of 4 bytes.}
    \apitablerow{shmem\_int8\_xor\_to\_all}{Integer, with an element size of 8 bytes.}
    \apitablerow{shmem\_int4\_xor\_to\_all}{Integer, with an element size of 4 bytes.}

%\deprecationend


\apireturnvalues{
    None.
}

\apinotes{
    All \openshmem reduction routines reset the values in \VAR{pSync} before they
    return, so a particular \VAR{pSync} buffer need only be initialized the first
    time it is used. The user must ensure that the \VAR{pSync} array is not being updated on any \ac{PE}
    in the active set while any of the \acp{PE} participate in processing of an
    \openshmem reduction routine. Be careful to avoid the following situations: If
    the \VAR{pSync} array is initialized at run time, some type of synchronization
    is needed to ensure that all \acp{PE} in the working set have initialized
    \VAR{pSync} before any of them enter an \openshmem routine called with the
    \VAR{pSync} synchronization array. A \VAR{pSync} or \VAR{pWrk} array can be
    reused in a subsequent reduction routine call only if none of the \acp{PE} in
    the active set are still processing a prior reduction routine call that used
    the same \VAR{pSync} or \VAR{pWrk} arrays. In general, this can be assured only
    by doing some type of synchronization.
}

\begin{apiexamples}

\apifexample
    {This \Fortran reduction example statically initializes the \VAR{pSync} array
    and finds the logical \OPR{AND} of the integer variable \VAR{FOO} across all
    even \acp{PE}.}
    {./example_code/shmem_and_example.f90}
    {}

\apifexample
    {This \Fortran example statically initializes the \VAR{pSync} array and finds
    the \OPR{maximum} value of real variable \VAR{FOO} across all even \acp{PE}.}
    {./example_code/shmem_max_example.f90}
    {}

\apifexample
    { This \Fortran example statically initializes the \VAR{pSync} array and finds
    the \OPR{minimum} value of real variable \VAR{FOO} across all the even
    \acp{PE}.}
    {./example_code/shmem_min_example.f90}
    {}

\apifexample
    {This \Fortran example statically initializes the \VAR{pSync} array and finds
    the \OPR{sum} of the real variable \VAR{FOO} across all even \acp{PE}.}
    {./example_code/shmem_sum_example.f90}
    {}

\apifexample
    {This \Fortran example statically initializes the \VAR{pSync} array and finds
    the \OPR{product} of the real variable \VAR{FOO} across all the even \acp{PE}.}
    {./example_code/shmem_prod_example.f90}
    {}

\apifexample
    {This \Fortran example statically initializes the \VAR{pSync} array and finds
    the logical \OPR{OR} of the integer variable \VAR{FOO} across all even
    \acp{PE}.}
    {./example_code/shmem_or_example.f90}
    {}

\apifexample
    {This \Fortran example statically initializes the \VAR{pSync} array and
    computes the exclusive \OPR{XOR} of variable \VAR{FOO} across all even
    \acp{PE}.}
    {./example_code/shmem_xor_example.f90}
    {}

\end{apiexamples}

\end{apidefinition}


\subsubsection{\textbf{SHMEM\_ALLTOALL}}\label{subsec:shmem_alltoall}
\apisummary{
    shmem\_alltoall is a collective routine where each \ac{PE} exchanges a fixed amount of data with all other \acp{PE} \oldtext{in the active set} \newtext{participating in the collective}.
}

\begin{apidefinition}

%% C11
{\color{Green}
\begin{C11synopsis}
int @\FuncDecl{shmem\_alltoall}@(shmem_team_t team, TYPE *dest, const TYPE *source, size_t nelems);
\end{C11synopsis}
where \TYPE{} is one of the standard \ac{RMA} types specified by Table \ref{stdrmatypes}.
}

\begin{Csynopsis}
\end{Csynopsis}
{\color{Green}
\begin{CsynopsisCol}
int @\FuncDecl{shmem\_\FuncParam{TYPENAME}\_alltoall}@(shmem_team_t team, TYPE *dest, const TYPE *source, size_t nelems);
\end{CsynopsisCol}
where \TYPE{} is one of the standard \ac{RMA} types and has a corresponding \TYPENAME{} specified by Table \ref{stdrmatypes}.

\begin{CsynopsisCol}
int @\FuncDecl{shmem\_alltoallmem}@(shmem_team_t team, void *dest, const void *source, size_t nelems);
\end{CsynopsisCol}
}

\begin{DeprecateBlock}
\begin{CsynopsisCol}
void @\FuncDecl{shmem\_alltoall32}@(void *dest, const void *source, size_t nelems, int PE_start, int logPE_stride, int PE_size, long *pSync);
void @\FuncDecl{shmem\_alltoall64}@(void *dest, const void *source, size_t nelems, int PE_start, int logPE_stride, int PE_size, long *pSync);
\end{CsynopsisCol}
\end{DeprecateBlock}

\begin{Fsynopsis}
INTEGER pSync(SHMEM_ALLTOALL_SYNC_SIZE)
INTEGER PE_start, logPE_stride, PE_size, nelems
CALL @\FuncDecl{SHMEM\_ALLTOALL32}@(dest, source, nelems, PE_start, logPE_stride, PE_size, pSync)
CALL @\FuncDecl{SHMEM\_ALLTOALL64}@(dest, source, nelems, PE_start, logPE_stride, PE_size, pSync)
\end{Fsynopsis}

\begin{apiarguments}

\newtext{%
\apiargument{IN}{team}{A valid \openshmem team handle to a team.}
}

\apiargument{OUT}{dest}{A symmetric data object large enough to receive
    the combined total of \VAR{nelems} elements from each \ac{PE} in the
    active set.}
\apiargument{IN}{source}{A symmetric data object that contains \VAR{nelems}
    elements of data for each \ac{PE} in the active set, ordered according to
    destination \ac{PE}.}
\apiargument{IN}{nelems}{The number of elements to exchange for each \ac{PE}.
    \VAR{nelems} must be of type size\_t for \CorCpp.  When using
    \Fortran, it must be a default integer value.}

\begin{DeprecateBlock}
\apiargument{IN}{PE\_start}{The lowest \ac{PE} number of the active set of
    \acp{PE}.  \VAR{PE\_start} must be of type integer.  When using \Fortran,
    it must be a default integer value.}
\apiargument{IN}{logPE\_stride}{The log (base 2) of the stride between
    consecutive \ac{PE} numbers in the active set.  \VAR{logPE\_stride} must be of
    type integer.  When using \Fortran, it must be a default integer value.}
\apiargument{IN}{PE\_size}{The number of \acp{PE} in the active set.
    \VAR{PE\_size} must be of type integer.  When using \Fortran, it must
    be a default integer value.}
\apiargument{IN}{pSync}{
    A symmetric work array of size \CONST{SHMEM\_ALLTOALL\_SYNC\_SIZE}.
    In \CorCpp, \VAR{pSync} must be an array of elements of type \CTYPE{long}.
    In \Fortran, \VAR{pSync} must be an array of elements of default integer type.
    Every element of this array must be initialized with the value
    \CONST{SHMEM\_SYNC\_VALUE} before any of the \acp{PE} in the active set
    enter the routine.}
\end{DeprecateBlock}

\end{apiarguments}

\apidescription{
{\color{Green}
    The \FUNC{shmem\_alltoall} routines are collective routines. Each \ac{PE}
    participating in the operation exchanges \VAR{nelems} data elements
    with all other \acp{PE} participating in the operation.
    The size of a data element is:
    \begin{itemize}
    \item 32 bits for \FUNC{shmem\_alltoall32}
    \item 64 bits for \FUNC{shmem\_alltoall64}
    \item 8 bits for \FUNC{shmem\_alltoallmem}
    \item \FUNC{sizeof}(\TYPE{}) for alltoall routines taking typed \VAR{source} and \VAR{dest}
    \end{itemize}
}

    The data being sent and received are
    stored in a contiguous symmetric data object. The total size of each \acp{PE}
    \VAR{source} object and \VAR{dest} object is \VAR{nelems} times the size of
    an element \oldtext{(32 bits or 64 bits) times \VAR{PE\_size}}
    \newtext{times \VAR{N}, where \VAR{N} equals the number of \acp{PE} participating
    in the operation}.
    The \VAR{source} object contains oldtext{\VAR{PE\_size}} \VAR{N} blocks of data
    (where the size of each block is defined by \VAR{nelems}) and each block of data
    is sent to a different \ac{PE}.

    \newtext{The same \dest{} and \source{}
    arrays, and same value for \newtext{nelems}
    must be passed by all \acp{PE} that participate in the collective.}

    Given a \ac{PE} \VAR{i} that is the \kth \ac{PE} \oldtext{in the active set}
    \newtext{participating in the operation} and a \ac{PE}
    \VAR{j} that is the \lth \ac{PE} \oldtext{in the active set}
    \newtext{participating in the operation},
    \ac{PE} \VAR{i} sends the \lth block of its \VAR{source} object to
    the \kth block of
    the \VAR{dest} object of \ac{PE} \VAR{j}.

{\color{Green}
    Team-based collect routines operate over all \acp{PE} in the provided team
    argument. All \acp{PE} in the provided team must participate in the collective.

    Active-set-based collective routines operate over all \acp{PE} in the active set
    defined by the \VAR{PE\_start}, \VAR{logPE\_stride}, \VAR{PE\_size} triplet.
}
    As with all \oldtext{\openshmem} \newtext{active-set-based} collective routines,
    this routine assumes that only \acp{PE} in the active set call the routine.
    If a \ac{PE} not in the active set calls an  \oldtext{\openshmem}
    \newtext{active-set-based} collective routine,
    the behavior is undefined.

    The values of arguments \oldtext{\VAR{nelems},} \VAR{PE\_start}, \VAR{logPE\_stride},
    and \VAR{PE\_size} must be equal on all \acp{PE} in the active set. The same
    \oldtext{\VAR{dest} and \VAR{source} data objects, and the same} \VAR{pSync} work
    array must be passed to all \acp{PE} in the active set.

    Before any \ac{PE} calls a \FUNC{shmem\_alltoall} routine,
    the following conditions must be ensured:
    \begin{itemize}
    \item The \VAR{dest} data object on all \acp{PE} in the active set is
      ready to accept the \FUNC{shmem\_alltoall} data.
    \item \newtext{For active-set-based routines}, the \VAR{pSync} array
    on all \acp{PE} in the active set is not still in use from a prior call
    to a \FUNC{shmem\_alltoall} routine.
    \end{itemize}
    Otherwise, the behavior is undefined.

    Upon return from a \FUNC{shmem\_alltoall} routine, the following is true for
    the local PE:
    \begin{itemize}
    \item Its \VAR{dest} symmetric data object is completely updated and
    the data has been copied out of the \VAR{source} data object.
    \item \newtext{For active-set-based routines, }
    the values in the \VAR{pSync} array are restored to the original values.
    \end{itemize}
}

\apidesctable{
The  \dest{}  and \source{} data  objects must conform to certain typing
constraints, which are as follows:
}{Routine}{Data type of \VAR{dest} and \VAR{source}}

\apitablerow{shmem\_alltoall64}{\CONST{64} bits aligned.}
\apitablerow{shmem\_alltoall32}{\CONST{32} bits aligned.}

\apireturnvalues{
    \newtext{Zero on successful local completion. Nonzero otherwise.}
}

\apinotes{
    This routine restores \VAR{pSync} to its original contents.  Multiple calls
    to \openshmem\ routines that use the same \VAR{pSync} array do not require
    that \VAR{pSync} be reinitialized after the first call.
    The user must ensure that the \VAR{pSync} array is not being updated by any
    \ac{PE} in the active set while any of the \acp{PE} participates in
    processing of an \openshmem\ \FUNC{shmem\_alltoall} routine. Be careful to
    avoid these situations: If the \VAR{pSync} array is initialized at run time,
    some type of synchronization is needed to ensure that all \acp{PE} in the
    active set have initialized \VAR{pSync} before any of them enter an
    \openshmem\ routine called with the \VAR{pSync} synchronization array.  A
    \VAR{pSync} array may be reused on a subsequent \openshmem\
    \FUNC{shmem\_alltoall} routine only if none of the \acp{PE} in the
    active set are still processing a prior \openshmem\ \FUNC{shmem\_alltoall}
    routine call that used the same \VAR{pSync} array.  In general, this can be
    ensured only by doing some type of synchronization.
}

\begin{apiexamples}

\apicexample
    {This \CorCpp{} example shows a \FUNC{shmem\_int64\_alltoall} on two 64-bit integers among all
    \acp{PE}.}
    {./example_code/shmem_alltoall_example.c}
    {}

\end{apiexamples}

\end{apidefinition}



\subsubsection{\textbf{SHMEM\_ALLTOALLS}}\label{subsec:shmem_alltoalls}
\apisummary{
    shmem\_alltoalls is a collective routine where each \ac{PE} exchanges a fixed amount of strided data with all other \acp{PE} \oldtext{in the active set} \newtext{participating in the collective}.
}

\begin{apidefinition}

%% C11
{\color{Green}
\begin{C11synopsis}
int @\FuncDecl{shmem\_alltoalls\FuncParam{SIZE}}@(shmem_team_t team, void *dest, const void *source, ptrdiff_t dst, ptrdiff_t sst, size_t nelems);
\end{C11synopsis}
where \SIZE{} is one of \CONST{32, 64}.

\begin{CsynopsisCol}
int @\FuncDecl{shmem\_alltoalls}@(shmem_team_t team, TYPE *dest, const TYPE *source, ,ptrdiff_t dst, ptrdiff_t sst, size_t nelems);
\end{CsynopsisCol}
where \TYPE{} is one of the standard \ac{RMA} types specified by Table \ref{stdrmatypes}.

\begin{CsynopsisCol}
int @\FuncDecl{shmem\_alltoallsmem}@(shmem_team_t team, void *dest, const void *source, ptrdiff_t dst, ptrdiff_t sst, size_t nelems);
\end{CsynopsisCol}
}

\begin{Csynopsis}
\end{Csynopsis}
{\color{Green}
\begin{CsynopsisCol}
int @\FuncDecl{shmem\_team\_alltoalls\FuncParam{SIZE}}@(shmem_team_t team, void *dest, const void *source, ptrdiff_t dst, ptrdiff_t sst, size_t nelems);
\end{CsynopsisCol}
where \SIZE{} is one of \CONST{32, 64}.

\begin{CsynopsisCol}
int @\FuncDecl{shmem\_team\_\FuncParam{TYPENAME}\_alltoalls}@(shmem_team_t team, TYPE *dest, const TYPE *source, ptrdiff_t dst, ptrdiff_t sst, size_t nelems);
\end{CsynopsisCol}
where \TYPE{} is one of the standard \ac{RMA} types and has a corresponding \TYPENAME{} specified by Table \ref{stdrmatypes}.

\begin{CsynopsisCol}
int @\FuncDecl{shmem\_team\_alltoallsmem}@(shmem_team_t team, void *dest, const void *source, ptrdiff_t dst, ptrdiff_t sst, size_t nelems);
\end{CsynopsisCol}
}


\begin{DeprecateBlock}
\begin{CsynopsisCol}
void @\FuncDecl{shmem\_alltoalls32}@(void *dest, const void *source, ptrdiff_t dst, ptrdiff_t sst, size_t nelems, int PE_start, int logPE_stride, int PE_size, long *pSync);
void @\FuncDecl{shmem\_alltoalls64}@(void *dest, const void *source, ptrdiff_t dst, ptrdiff_t sst, size_t nelems, int PE_start, int logPE_stride, int PE_size, long *pSync);
\end{CsynopsisCol}
\end{DeprecateBlock}

\begin{Fsynopsis}
INTEGER pSync(SHMEM_ALLTOALLS_SYNC_SIZE)
INTEGER dst, sst, PE_start, logPE_stride, PE_size
INTEGER nelems 
CALL @\FuncDecl{SHMEM\_ALLTOALLS32}@(dest, source, dst, sst, nelems, PE_start, logPE_stride, PE_size, pSync)
CALL @\FuncDecl{SHMEM\_ALLTOALLS64}@(dest, source, dst, sst, nelems, PE_start, logPE_stride, PE_size, pSync)
\end{Fsynopsis}

\begin{apiarguments}

\newtext{%
\apiargument{IN}{team}{A valid \openshmem team handle.}
}

\apiargument{OUT}{dest}{A symmetric data object large enough to receive 
    the combined total of \VAR{nelems} elements from each \ac{PE} in the
    active set.}
\apiargument{IN}{source}{A symmetric data object that contains \VAR{nelems} 
    elements of data for each \ac{PE} in the active set, ordered according to 
    destination \ac{PE}.}
\apiargument{IN}{dst}{The stride between consecutive elements of the \dest{}
    data object.  The stride is scaled by the element size.  A
    value of \CONST{1} indicates contiguous data.  \VAR{dst} must be of type
    \CTYPE{ptrdiff\_t}.  When using \Fortran, it must be a default integer
    value.}
\apiargument{IN}{sst}{The  stride between consecutive elements of the
    \source{} data object.  The stride is scaled by the element size.
    A value of \CONST{1} indicates contiguous data.  \VAR{sst} must be
    of type \CTYPE{ptrdiff\_t}.  When using \Fortran, it must be a
    default integer value.}



\begin{DeprecateBlock}
\apiargument{IN}{nelems}{The number of elements to exchange for each \ac{PE}.
    \VAR{nelems} must be of type size\_t for \CorCpp.  When using
    \Fortran, it must be a default integer value.}
\apiargument{IN}{PE\_start}{The lowest \ac{PE} number of the active set of
    \acp{PE}.  \VAR{PE\_start} must be of type integer.  When using \Fortran,
    it must be a default integer value.}
\apiargument{IN}{logPE\_stride}{The log (base 2) of the stride between
    consecutive \ac{PE} numbers in the active set.  \VAR{logPE\_stride} must be of
    type integer.  When using \Fortran, it must be a default integer value.}
\apiargument{IN}{PE\_size}{The number of \acp{PE} in the active set.
    \VAR{PE\_size} must be of type integer.  When using \Fortran, it must
    be a default integer value.}
\apiargument{IN}{pSync}{
    A symmetric work array of size \CONST{SHMEM\_ALLTOALLS\_SYNC\_SIZE}.
    In \CorCpp, \VAR{pSync} must be an array of elements of type \CTYPE{long}.
    In \Fortran, \VAR{pSync} must be an array of elements of default integer type.
    Every element of this array must be initialized with the value
    \CONST{SHMEM\_SYNC\_VALUE} before any of the \acp{PE} in the active set
    enter the routine.}
\end{DeprecateBlock}

\end{apiarguments}

\apidescription{
    The \FUNC{shmem\_alltoalls} routines are collective routines.
    \newtext{These routines are equivalent in functionality to the corresponding
    \FUNC{shmem\_alltoall} routines except that they add explicit stride values
    for accessing the source and destination data arrays, whereas the array
    access in \FUNC{shmem\_alltoall} is always with a stride of \CONST{1}.}

    Each \ac{PE} \oldtext{in the active set} \newtext{participating in the operation}
    exchanges \VAR{nelems} strided data elements \oldtext{of size
    32 bits (for \FUNC{shmem\_alltoalls32}) or 64 bits (for \FUNC{shmem\_alltoalls64})}
    with all other \acp{PE} \oldtext{in the set} \newtext{participating in the operation}.
    Both strides, \VAR{dst} and \VAR{sst}, must be greater
    than or equal to \CONST{1}.

    \newtext{The same \dest{} and \source{} arrays and same values for values of
    arguments \VAR{dst}, \VAR{sst}, \VAR{nelems} must be passed by all \acp{PE}
    that participate in the collective.}
    
    Given a \ac{PE} \VAR{i} that is the \kth \ac{PE} \oldtext{in the active set}
    \newtext{participating in the operation} and a \ac{PE}
    \VAR{j} that is the \lth \ac{PE} \oldtext{in the active set}
    \newtext{participating in the operation}
    \ac{PE} \VAR{i} sends the \VAR{sst}*\lth block of the \VAR{source} data object to
    the \VAR{dst}*\kth block of the \VAR{dest} data object on
    \ac{PE} \VAR{j}.

{\color{Green}
    See the description of \FUNC{shmem\_alltoall} in section
    \ref{subsec:shmem_alltoall} for:
    \begin{itemize}
    \item Data element sizes for the different sized and typed \FUNC{shmem\_alltoalls} variants.
    \item Rules for \ac{PE} participation in the collective routine.
    \item The pre- and post-conditions for symmetric objects.
    \item Typing constraints for \dest{} and \source{} data objects.
    \end{itemize}
}
    
} 


\apireturnvalues{
   \newtext{Zero on successful local completion. Nonzero otherwise.}
}

\apinotes{
    \newtext{See notes for \FUNC{shmem\_alltoall} in section \ref{subsec:shmem_alltoall}}.
}

\begin{apiexamples}

\apicexample
    {This example shows a \FUNC{shmem\_alltoalls64} on two long elements among
    all \acp{PE}.}
    {./example_code/shmem_alltoalls_example.c}
    {}

\end{apiexamples}

\end{apidefinition}





\subsection{Point-To-Point Synchronization Routines}\label{subsec:p2p_intro}
The following section discusses \openshmem \acp{API} that provide a mechanism
for synchronization between two \acp{PE} based on the value of a symmetric data
object.
The point-to-point synchronization routines can be used to portably ensure
that memory access operations observe remote updates in the order enforced by
the initiator \ac{PE} using the \FUNC{shmem\_fence} and \FUNC{shmem\_quiet}
routines.

Where appropriate compiler support is available, \openshmem provides
type-generic point-to-point synchronization interfaces via \Cstd[11] generic
selection. Such type-generic routines are supported for the
``point-to-point synchronization types'' identified in
Table~\ref{p2psynctypes}.

The point-to-point synchronization types include some of the exact-width
integer types defined in \HEADER{stdint.h} by \Cstd[99]~\S7.18.1.1 and
\Cstd[11]~\S7.20.1.1. When the \Cstd translation environment
does not provide exact-width integer types with \HEADER{stdint.h}, an
\openshmem implemementation is not required to provide support for these types.
The \FUNC{shmem\_test\_any} and \FUNC{shmem\_wait\_until\_any} routines
require the \CONST{SIZE\_MAX} macro defined in \HEADER{stdint.h} by
\Cstd[99]~\S7.18.3 and \Cstd[11]~\S7.20.3.

\begin{table}[h]
  \begin{center}
    \begin{tabular}{|l|l|}
      \hline
      \TYPE              & \TYPENAME  \\ \hline
      short              & short      \\ \hline
      int                & int        \\ \hline
      long               & long       \\ \hline
      long long          & longlong   \\ \hline
      unsigned short     & ushort     \\ \hline
      unsigned int       & uint       \\ \hline
      unsigned long      & ulong      \\ \hline
      unsigned long long & ulonglong  \\ \hline
      int32\_t           & int32      \\ \hline
      int64\_t           & int64      \\ \hline
      uint32\_t          & uint32     \\ \hline
      uint64\_t          & uint64     \\ \hline
      size\_t            & size       \\ \hline
      ptrdiff\_t         & ptrdiff    \\ \hline
    \end{tabular}
    \TableCaptionRef{Point-to-Point Synchronization Types and Names}
    \label{p2psynctypes}
  \end{center}
\end{table}

The point-to-point synchronization interface provides named constants whose
values are integer constant expressions that specify the comparison operators
used by \openshmem synchronization routines.
The constant names and associated operations are
presented in Table~\ref{p2p-consts}.

\begin{table}[h]
  \begin{center}
    \begin{tabular}{ll}
      \hline
      Constant Name                 & Comparison               \\ \hline
      \LibConstRef{SHMEM\_CMP\_EQ}  & Equal                    \\
      \LibConstRef{SHMEM\_CMP\_NE}  & Not equal                \\
      \LibConstRef{SHMEM\_CMP\_GT}  & Greater than             \\
      \LibConstRef{SHMEM\_CMP\_GE}  & Greater than or equal to \\
      \LibConstRef{SHMEM\_CMP\_LT}  & Less than                \\
      \LibConstRef{SHMEM\_CMP\_LE}  & Less than or equal to    \\ \hline
    \end{tabular}
    \TableCaptionRef{Point-to-Point Comparison Constants}
    \label{p2p-consts}
  \end{center}
\end{table}


\subsubsection{\textbf{SHMEM\_WAIT\_UNTIL}}\label{subsec:shmem_wait_until}
\apisummary{
    Wait for a variable on the local \ac{PE} to change.
}

\begin{apidefinition}

\begin{C11synopsis}
void @\FuncDecl{shmem\_wait\_until}@(TYPE *ivar, int cmp, TYPE cmp_value);
\end{C11synopsis}
where \TYPE{} is one of the point-to-point synchronization types specified by
Table \ref{p2psynctypes}.

\begin{Csynopsis}
void @\FuncDecl{shmem\_\FuncParam{TYPENAME}\_wait\_until}@(TYPE *ivar, int cmp, TYPE cmp_value);
\end{Csynopsis}
where \TYPE{} is one of the point-to-point synchronization types and has a
corresponding \TYPENAME{} specified by Table~\ref{p2psynctypes}.

\begin{DeprecateBlock}
\begin{CsynopsisCol}
void @\FuncDecl{shmem\_wait\_until}@(long *ivar, int cmp, long cmp_value);
void @\FuncDecl{shmem\_wait}@(long *ivar, long cmp_value);
void @\FuncDecl{shmem\_\FuncParam{TYPENAME}\_wait}@(TYPE *ivar, TYPE cmp_value);
\end{CsynopsisCol}
where \TYPE{} is one of \{\CTYPE{short}, \CTYPE{int}, \CTYPE{long},
\CTYPE{long long}\} and has a corresponding \TYPENAME{} specified by
Table~\ref{p2psynctypes}.
\end{DeprecateBlock}

\begin{apiarguments}

\apiargument{IN}{ivar}{A remotely accessible integer variable. When using \CorCpp,
    the type of \VAR{ivar} should match that implied in the SYNOPSIS section.} 
\apiargument{IN}{cmp}{The compare operator that compares \VAR{ivar} with
  \VAR{cmp\_value}.
  When using \CorCpp, it must be of type \CTYPE{int}.}
\apiargument{IN}{cmp\_value}{\VAR{cmp\_value} must be of type integer.  When
    using \CorCpp, the type of \VAR{cmp\_value} should match that implied in the
    SYNOPSIS section.}

\end{apiarguments}

\apidescription{
    The \FUNC{shmem\_wait} and \FUNC{shmem\_wait\_until} operations block until
    the value contained in the symmetric data object, \VAR{ivar}, at the
    calling \ac{PE} satisfies the wait condition.  In an \openshmem program
    with single-threaded \acp{PE}, the \VAR{ivar} object at the calling \ac{PE}
    may be updated by an \ac{RMA}, \ac{AMO}, or store operation performed by another
    \ac{PE}.  In an \openshmem program with multithreaded \acp{PE}, the
    \VAR{ivar} object at the calling \ac{PE} may be updated by an \ac{RMA}, \ac{AMO}, or
    store operation performed by a thread located within the calling \ac{PE} or
    within another \ac{PE}.

    These routines can be used to implement point-to-point synchronization
    between \acp{PE} or between threads within the same \ac{PE}.  A call to
    \FUNC{shmem\_wait} blocks until the value of
    \VAR{ivar} at the calling \ac{PE} is not equal to \VAR{cmp\_value}.  A call
    to \FUNC{shmem\_wait\_until} blocks until the value of \VAR{ivar} at the
    calling \ac{PE} satisfies the wait condition specified by the comparison
    operator, \VAR{cmp}, and comparison value, \VAR{cmp\_value}.
}

\apireturnvalues{
    None.
}

\apinotes{
  As of \openshmem[1.4], the \FUNC{shmem\_wait} routine is deprecated;
  however, \FUNC{shmem\_wait} is equivalent to \FUNC{shmem\_wait\_until}
  where \VAR{cmp} is \CONST{SHMEM\_CMP\_NE}.
}

\apiimpnotes{
    Implementations must ensure that \FUNC{shmem\_wait} and
    \FUNC{shmem\_wait\_until} do not return before the update of the memory
    indicated by \VAR{ivar} is fully complete.  Partial updates to the memory
    must not cause \FUNC{shmem\_wait} or \FUNC{shmem\_wait\_until} to return.
}

\end{apidefinition}


\subsubsection{\textbf{SHMEM\_TEST}}\label{subsec:shmem_test}
\apisummary{
  Test whether a variable on the local \ac{PE} has changed.
}

\begin{apidefinition}

\begin{C11synopsis}
int @\FuncDecl{shmem\_test}@(TYPE *ivar, int cmp, TYPE cmp_value);
\end{C11synopsis}
where \TYPE{} is one of the point-to-point synchronization types specified by
Table \ref{p2psynctypes}.

\begin{Csynopsis}
int @\FuncDecl{shmem\_\FuncParam{TYPENAME}\_test}@(TYPE *ivar, int cmp, TYPE cmp_value);
\end{Csynopsis}
where \TYPE{} is one of the point-to-point synchronization types and has a
corresponding \TYPENAME{} specified by Table \ref{p2psynctypes}.

\begin{apiarguments}

  \apiargument{OUT}{ivar}{A pointer to a remotely accessible data object.}
  \apiargument{IN}{cmp}{The comparison operator that compares \VAR{ivar} with
    \VAR{cmp\_value}.}
  \apiargument{IN}{cmp\_value}{The value against which the object pointed to
    by \VAR{ivar} will be compared.}

\end{apiarguments}

\apidescription{
  \FUNC{shmem\_test} tests the numeric comparison of the symmetric object
  pointed to by \VAR{ivar} with the value \VAR{cmp\_value} according to the
  comparison operator \VAR{cmp}.
}

\apireturnvalues{
  \FUNC{shmem\_test} returns 1 if the comparison of the symmetric object
  pointed to by \VAR{ivar} with the value \VAR{cmp\_value} according to the
  comparison operator \VAR{cmp} evaluates to true; otherwise, it returns 0.
}

\apinotes{
  None.
}

\begin{apiexamples}
  \apicexample
      {The following example demonstrates the use of \FUNC{shmem\_test} to
        wait on an array of symmetric objects and return the index of an
        element that satisfies the specified condition.}
      {./example_code/shmem_test_example1.c}
      {}
\end{apiexamples}

\end{apidefinition}






\subsection{Memory Ordering Routines}\label{subsec:memory_order}
The following section discusses \openshmem \acp{API} that provide mechanisms to
ensure ordering and/or delivery of \OPR{Put}, \ac{AMO}, memory store,
and non-blocking \PUT{} and \GET{} routines to symmetric data objects.

\subsubsection{\textbf{SHMEM\_FENCE}}\label{subsec:shmem_fence}
\apisummary{
    Assures ordering of delivery of \PUT{}, \ac{AMO}, memory store, and nonblocking \PUT{} routines
    to symmetric data objects.
}

\begin{apidefinition}

\begin{Csynopsis}
void @\FuncDecl{shmem\_fence}@(void);
void @\FuncDecl{shmem\_ctx\_fence}@(shmem_ctx_t ctx);
\end{Csynopsis}

\begin{apiarguments}
    \apiargument{IN}{ctx}{A context handle specifying the context on which to perform the operation.
        When this argument is not provided, the operation is performed on
        the default context.}
\end{apiarguments}

\apidescription{
    This routine assures ordering of delivery of \PUT{}, \ac{AMO}, memory store, and nonblocking \PUT{}
    routines to symmetric data objects.  All \PUT{}, \ac{AMO}, memory store, and nonblocking \PUT{}
    routines to symmetric data objects issued to a particular remote \ac{PE}
    on the given context prior
    to the call to \FUNC{shmem\_fence} are guaranteed to be delivered before any
    subsequent \PUT{}, \ac{AMO}, memory store, and nonblocking \PUT{} routines to symmetric data
    objects to the same \ac{PE}. \FUNC{shmem\_fence} guarantees order of delivery,
    not completion. It does not guarantee order of delivery of nonblocking \GET{} routines.
    If \VAR{ctx} has the value \CONST{SHMEM\_CTX\_INVALID}, no operation is
    performed.
}

\apireturnvalues{
    None.
}

\apinotes{
    \FUNC{shmem\_fence} only provides per-\ac{PE} ordering guarantees and does not
    guarantee completion of delivery.
    \FUNC{shmem\_fence} also does not have an effect on the ordering between memory
    accesses issued by the target PE. \FUNC{shmem\_wait\_until}, \FUNC{shmem\_test},
    \FUNC{shmem\_barrier}, \FUNC{shmem\_barrier\_all} routines can be called by the target PE to guarantee
    ordering of its memory accesses.
    There is a subtle difference between
    \FUNC{shmem\_fence} and \FUNC{shmem\_quiet}, in that, \FUNC{shmem\_quiet}
    guarantees completion of \PUT{}, \ac{AMO}, memory store, and nonblocking \PUT{} routines to
    symmetric data objects which makes the updates visible to all other
    \acp{PE}.

    The \FUNC{shmem\_quiet} routine should be called if completion of \PUT{},
    \ac{AMO}, memory store, and nonblocking \PUT{} routines to symmetric data objects is desired
    when multiple remote \acp{PE} are involved.

    In an \openshmem program with multithreaded \acp{PE}, it is the
    user's responsibility to ensure ordering between operations issued by the threads
    in a \ac{PE} that target symmetric memory (e.g. \PUT{}, \ac{AMO}, memory stores,
    and nonblocking routines) and calls by threads in that \ac{PE} to
    \FUNC{shmem\_fence}. The \FUNC{shmem\_fence} routine can enforce memory store ordering only for the
    calling thread. Thus, to ensure ordering for memory stores performed by a thread that is
    not the thread calling \FUNC{shmem\_fence}, the update must be made visible to the
    calling thread according to the rules of the memory model associated with
    the threading environment.
}

\begin{apiexamples}

\apicexample
    {The following example uses \FUNC{shmem\_fence} in a \Cstd[11] program: }
    {./example_code/shmem_fence_example.c}
    {\VAR{Put1} will be ordered to be delivered before \VAR{put3} and \VAR{put2}
    will be ordered to be delivered before \VAR{put4}.}

\end{apiexamples}

\end{apidefinition}


\subsubsection{\textbf{SHMEM\_QUIET}}\label{subsec:shmem_quiet}
\apisummary{
    Waits for completion of all outstanding \PUT{}, \ac{AMO}, memory store,
    and nonblocking \PUT{} and \GET{} routines to symmetric data
    objects issued by a \ac{PE}.
}

\begin{apidefinition}

\begin{Csynopsis}
void @\FuncDecl{shmem\_quiet}@(void);
void @\FuncDecl{shmem\_ctx\_quiet}@(shmem_ctx_t ctx);
\end{Csynopsis}

\begin{Fsynopsis}
CALL @\FuncDecl{SHMEM\_QUIET}@
\end{Fsynopsis}

\begin{apiarguments}
    \apiargument{IN}{ctx}{\oldtext{The context on which to perform the operation.} \newtext{A context handle specifying the context on which to perform the operation.}
        When this argument is not provided, the operation is performed on
        \oldtext{\CONST{SHMEM\_CTX\_DEFAULT}} \newtext{the default context}.}
\end{apiarguments}

\apidescription{ 
    The \FUNC{shmem\_quiet} routine ensures completion of \PUT{}, \ac{AMO},
    memory store, and nonblocking \PUT{} and \GET{} routines on
    symmetric data objects issued by the calling \ac{PE} on the given context. All \PUT{}, \ac{AMO},
    memory store, and nonblocking \PUT{} and \GET{} routines to
    symmetric data objects are guaranteed to be completed and visible to all
    \acp{PE} when \FUNC{shmem\_quiet} returns. 
    \newtext{
    If \VAR{ctx} has the value \CONST{SHMEM\_CTX\_INVALID}, no operation is
    performed.
    }
}


\apireturnvalues{
    None.
}

\apinotes{ 
    \FUNC{shmem\_quiet} is most useful as a way of ensuring completion of
    several \PUT{}, \ac{AMO}, memory store, and nonblocking \PUT{}
    and \GET{} routines to symmetric data objects initiated by the calling
    \ac{PE}.  For example, one might use \FUNC{shmem\_quiet} to await delivery
    of a block of data before issuing another \PUT{} or nonblocking
    \PUT{} routine, which sets a completion flag on another \ac{PE}.
     \FUNC{shmem\_quiet} is not usually needed if
    \FUNC{shmem\_barrier\_all} or \FUNC{shmem\_barrier} are called.  The barrier
    routines wait for the completion of outstanding writes (\PUT{}, \ac{AMO},
    memory stores, and nonblocking \PUT{} and \GET{} routines) to
    symmetric data objects on all \acp{PE}.

    In an \openshmem program with multithreaded \acp{PE}, it is the
    user's responsibility to ensure ordering between operations issued by the threads
    in a \ac{PE} that target symmetric memory (e.g. \PUT{}, \ac{AMO}, memory stores,
    and nonblocking routines) and calls by threads in that \ac{PE} to
    \FUNC{shmem\_quiet}. The \FUNC{shmem\_quiet} routine can enforce memory store ordering only for the
    calling thread. Thus, to ensure ordering for memory stores performed by a thread that is
    not the thread calling \FUNC{shmem\_quiet}, the update must be made visible to the
    calling thread according to the rules of the memory model associated with
    the threading environment.

     A call to \FUNC{shmem\_quiet} by a thread completes the operations posted prior
     to calling \FUNC{shmem\_quiet}. If the user intends to also complete operations
     issued by a thread that is not the thread calling \FUNC{shmem\_quiet}, the
     user must ensure that the operations are performed prior to the call to
     \FUNC{shmem\_quiet}. This may require the use of a synchronization
     operation provided by the threading package. For example, when using POSIX
     Threads, the user may call the \FUNC{pthread\_barrier\_wait} routine to
     ensure that all threads have issued operations before a thread calls
     \FUNC{shmem\_quiet}.

    \FUNC{shmem\_quiet} does not have an effect on the ordering between memory
    accesses issued by the target PE. \FUNC{shmem\_wait\_until},
    \FUNC{shmem\_test}, \FUNC{shmem\_barrier}, \FUNC{shmem\_barrier\_all} routines
    can be called by the target PE to guarantee ordering of its memory accesses.
}

\begin{apiexamples}

\apicexample
    {The following example uses \FUNC{shmem\_quiet} in a \Cstd[11] program: }
    {./example_code/shmem_quiet_example.c}
    {\VAR{Put1} and \VAR{put2} will be completed and visible before \VAR{put3}
    and \VAR{put4}.}
\end{apiexamples}

\end{apidefinition}


\subsubsection{Synchronization and Communication Ordering in OpenSHMEM}
\input{content/synchronization_model.tex}






\subsection{Distributed Locking Routines}
The following section discusses \openshmem locks as a mechanism to provide
mutual exclusion. Three routines are available for distributed locking,
\textit{set, test} and \textit{clear}.

\subsubsection{\textbf{SHMEM\_LOCK}}\label{subsec:shmem_lock}
\apisummary{
    Releases, locks, and tests a mutual exclusion memory lock.
}
\begin{apidefinition}

\begin{Csynopsis}
void @\FuncDecl{shmem\_clear\_lock}@(long *lock);
void @\FuncDecl{shmem\_set\_lock}@(long *lock);
int @\FuncDecl{shmem\_test\_lock}@(long *lock);
\end{Csynopsis}

\begin{Fsynopsis}
INTEGER lock, SHMEM_TEST_LOCK
CALL @\FuncDecl{SHMEM\_CLEAR\_LOCK}@(lock)
CALL @\FuncDecl{SHMEM\_SET\_LOCK}@(lock)
I = @\FuncDecl{SHMEM\_TEST\_LOCK}@(lock)
\end{Fsynopsis}

\begin{apiarguments}
\apiargument{IN}{lock}{A symmetric data object that is a scalar variable or an array
    of length \CONST{1}.  This data object must be set to \CONST{0} on all
    \acp{PE} prior to the first use.  \VAR{lock} must be of type \CONST{long}.
    When using \Fortran, it must be of default kind.}
\end{apiarguments}

\apidescription{
    The \FUNC{shmem\_set\_lock} routine sets a mutual exclusion lock after
    waiting for the lock to be freed by any other \ac{PE} currently holding
    the lock.  Waiting \acp{PE} are assured of getting the lock in a
    first-come, first-served manner.  The \FUNC{shmem\_test\_lock} routine sets
    a mutual exclusion lock only if it is currently cleared.  By using this
    routine, a \ac{PE} can avoid blocking on a set lock.  If the lock is
    currently set, the routine returns without waiting.  The
    \FUNC{shmem\_clear\_lock} routine releases a lock previously set by
    \FUNC{shmem\_set\_lock} or \FUNC{shmem\_test\_lock} after performing a
    quiet operation on the default context to ensure that all symmetric memory
    accesses that occurred during the critical region are complete.  These
    routines are appropriate for protecting a critical region from simultaneous
    update by multiple \acp{PE}.

    The \openshmem lock API provides a non-reentrant mutex.  Thus, a call to
    \FUNC{shmem\_set\_lock} or \FUNC{shmem\_test\_lock} when the calling PE
    already holds the given lock will result in undefined behavior.  In a
    multithreaded \openshmem program, the user must ensure that such calls do
    not occur.
}

\apireturnvalues{
    The \FUNC{shmem\_test\_lock} routine returns \CONST{0} if the lock was
    originally cleared and this call was able to set the lock. A value of
    \CONST{1} is returned if the lock had been set and the call returned without
    waiting to set the lock.
}

\apinotes{
    The term symmetric data object is defined in Section \ref{subsec:memory_model}.

    The lock variable must be initialized to zero before any PE performs an
    \openshmem lock operation on the given variable.  Accessing an in-use lock
    variable using any method other than the \openshmem lock API, e.g. using
    local load/store, RMA, or AMO operations, results in undefined behavior.

    Calls to \FUNC{shmem\_ctx\_quiet} can be performed prior to calling the
    \FUNC{shmem\_clear\_lock} routine to ensure completion of operations issued
    on additional contexts.
}

\begin{apiexamples}

\apicexample
    {The following example uses \FUNC{shmem\_lock} in a \Cstd[11] program.}
    {./example_code/shmem_lock_example.c}
    {}

\end{apiexamples}

\end{apidefinition}






\subsection{Cache Management}
All of these routines are deprecated and are provided for backwards
compatibility.  Implementations must include all items in this section, and the
routines should function properly and may notify the user about deprecation of
their use.

\subsubsection{\textbf{SHMEM\_CACHE}}\label{subsec:shmem_cache}
\apisummary{
    Controls data cache utilities.
}

\begin{apidefinition}

\begin{DeprecateBlock}
\begin{Csynopsis}
void @\FuncDecl{shmem\_clear\_cache\_inv}@(void);
void @\FuncDecl{shmem\_set\_cache\_inv}@(void);
void @\FuncDecl{shmem\_clear\_cache\_line\_inv}@(void *dest);
void @\FuncDecl{shmem\_set\_cache\_line\_inv}@(void *dest);
void @\FuncDecl{shmem\_udcflush}@(void);
void @\FuncDecl{shmem\_udcflush\_line}@(void *dest);
\end{Csynopsis}
\end{DeprecateBlock}

\begin{apiarguments}

\apiargument{IN}{dest}{A data object that is local to the \ac{PE}.}

\end{apiarguments}

\apidescription{
    \FUNC{shmem\_set\_cache\_inv} enables automatic cache coherency mode.

    \FUNC{shmem\_set\_cache\_line\_inv} enables automatic cache coherency mode for
    the cache line associated with the address of \VAR{dest} only.

    \FUNC{shmem\_clear\_cache\_inv} disables automatic cache coherency mode
    previously enabled by \FUNC{shmem\_set\_cache\ \_inv} or
    \FUNC{shmem\_set\_cache\_line\_inv}.

    \FUNC{shmem\_udcflush} makes the entire user data cache coherent.

    \FUNC{shmem\_udcflush\_line} makes coherent the cache line that corresponds with
    the address specified by \VAR{dest}.
}

\apireturnvalues{
    None.
}

\apinotes{
    These routines have been retained for improved backward compatibility with
    legacy architectures.  They are not required to be supported by implementing
    them as \VAR{no-ops} and where used, they may have no effect on cache line
    states.
}

\begin{apiexamples}

None.

\end{apiexamples}

\end{apidefinition}


\clearpage
\clearpage %%%%%%%%%%%%%%%%%%%%%%%%%%%%%%%%%%%%%%%%%%%%%%%%%%%%%%%%%%%%

\appendix

%defining pagestyle for annex
%\pagestyle{plain} \withlinenumbers
\pagestyle{fancy} \withlinenumbers
\fancyhf{}
\fancyhead[RE, LO]{\leftmark}
\fancyhead[RO, LE]{\thepage}
\fancyfoot[CE, CO]{\thepage}
\renewcommand{\headrulewidth}{0pt}




\chapter{Writing OpenSHMEM Programs}
\section*{Incorporating OpenSHMEM into Programs}\label{sec:writing_programs}

The following section describes how to write a ``Hello World" \openshmem program.
To write a ``Hello World" \openshmem program, the user must:

\begin{itemize}
\item Include the header file \HEADER{shmem.h} for \Cstd or \HEADER{shmem.fh} for \Fortran.
\item Add the initialization call \hyperref[subsec:shmem_init]{\FUNC{shmem\_init}}.
\item Use \openshmem calls to query the local \ac{PE} number
    (\hyperref[subsec:shmem_my_pe]{\FUNC{shmem\_my\_pe}}) and the total number
    of \acp{PE} (\hyperref[subsec:shmem_n_pes]{\FUNC{shmem\_n\_pes}}).
\item Add the finalization call \hyperref[subsec:shmem_finalize]{\FUNC{shmem\_finalize}}.
\end{itemize}

In \openshmem, the order in which lines appear in the output is not
deterministic because \acp{PE} execute asynchronously in parallel.

\begin{minipage}{\linewidth}
\vspace{0.1in}
\numberedlisting{caption={``Hello World'' example program in \Cstd},label=openshmem-hello,language=OSH2+C}
                {example_code/hello-openshmem.c}
\outputlisting{language=bash,caption={Possible ordering of expected output with 4 \acp{PE} from the program in Listing~\ref{openshmem-hello}}}
                {example_code/hello-openshmem-c.output}
\vspace{0.1in}
\end{minipage}

\clearpage %%%%%%%%%%%%%%%%%%%%%%%%%%%%%%%%%%%%%%%%%%%%%%%%%%%%%%%%%%%%

\begin{deprecate}
\openshmem also provides a \Fortran API. Listing~\ref{openshmem-hello-f90} shows a similar program written in \Fortran.

\begin{minipage}{\linewidth}
\vspace{0.1in}
\numberedlisting{caption={``Hello World'' example program in \Fortran},label=openshmem-hello-f90,language=OSH2+F}
                {example_code/hello-openshmem.f90}
\outputlisting{language=bash,caption={Possible ordering of expected output with 4 \acp{PE} from the program in Listing~\ref{openshmem-hello-f90}}}
                {example_code/hello-openshmem-f90.output}
\vspace{0.1in}
\end{minipage}
\end{deprecate}

\clearpage %%%%%%%%%%%%%%%%%%%%%%%%%%%%%%%%%%%%%%%%%%%%%%%%%%%%%%%%%%%%

The example in Listing~\ref{openshmem-hello-symmetric} shows a more complex
\openshmem program that illustrates the use of symmetric data objects.
Note the declaration of the \VAR{static short dest} array and its use as the
remote destination in \hyperref[subsec:shmem_put]{\FUNC{shmem\_put}}.

The \KEYWORD{static} keyword makes the \VAR{dest} array symmetric on all \acp{PE}.
Each \ac{PE} is able to transfer data to a remote \dest{} array by simply
specifying to an OpenSHMEM routine such as \hyperref[subsec:shmem_put]{\FUNC{shmem\_put}}
the local address of the symmetric data object that will receive the data.
This local address resolution aids programmability because the address of the
\dest{} need not be exchanged with the active side (\ac{PE} \CONST{0}) prior to
the \acf{RMA} routine.

Conversely, the declaration of the \VAR{short source} array is asymmetric
(local only).
The \source{} object does not need to be symmetric because \PUT{} handles the
references to the \VAR{source} array only on the active (local) side.

\begin{minipage}{\linewidth}
\vspace{0.1in}
\numberedlisting{caption={Example program with symmetric data objects},label=openshmem-hello-symmetric,language=OSH2+C}
                {example_code/writing_shmem_example.c}
\outputlisting{language=bash,caption={Possible ordering of expected output with 4 \acp{PE} from the program in Listing~\ref{openshmem-hello-symmetric}}}
                {example_code/writing_shmem_example.output}
\vspace{0.1in}
\end{minipage}




\chapter{Compiling and Running Programs}\label{sec:compiling}
The \openshmem Specification does not specify how
\openshmem programs are compiled, linked, and run. This section shows some
examples of how wrapper programs are utilized in the \openshmem Reference
Implementation to compile and launch programs.

\section{Compilation}
\subsection*{Programs written in \Cstd}

The \openshmem Reference Implementation provides a wrapper program, named
\textbf{oshcc}, to aid in the compilation of \Cstd programs.
The wrapper may be called as follows:

\begin{lstlisting}[language=bash]
oshcc <compiler options> -o myprogram myprogram.c
\end{lstlisting}
Where the $\langle\mbox{compiler options}\rangle$ are options understood by the
underlying \Cstd compiler called by \textbf{oshcc}.


\subsection*{Programs written in \Cpp}

The \openshmem Reference Implementation provides a wrapper program, named
\textbf{oshc++}, to aid in the compilation of \Cpp programs.
The wrapper may be called as follows:

\begin{lstlisting}[language=bash]
oshc++ <compiler options> -o myprogram myprogram.cpp
\end{lstlisting}
Where the $\langle\mbox{compiler options}\rangle$ are options understood by the
underlying \Cpp compiler called by \textbf{oshc++}.


\subsection*{Programs written in \Fortran}

\begin{deprecate}
The \openshmem Reference Implementation provides a wrapper program, named
\textbf{oshfort}, to aid in the compilation of \Fortran programs.
The wrapper may be called as follows:

\begin{lstlisting}[language=bash]
oshfort <compiler options> -o myprogram myprogram.f
\end{lstlisting}
Where the $\langle\mbox{compiler options}\rangle$ are options understood by the
underlying \Fortran compiler called by \textbf{oshfort}.
\end{deprecate}

\section{Running Programs}

The \openshmem Reference Implementation provides a wrapper program, named
\textbf{oshrun}, to launch \openshmem programs.
The wrapper may be called as follows:

\begin{lstlisting}[language=bash]
oshrun <runner options> -np <#> <program> <program arguments>
\end{lstlisting}
The arguments for \textbf{oshrun} are:

\begin{tabular}{p{0.3\textwidth}p{0.6\textwidth}}
$\langle\mbox{runner options}\rangle$ & {Options passed to the underlying launcher.}\tabularnewline
-np $\langle\mbox{\#}\rangle$ & {The number of \acp{PE} to be used in the execution.}\tabularnewline
$\langle\mbox{program}\rangle$ & {The program executable to be launched.}\tabularnewline
$\langle\mbox{program arguments}\rangle$ & {Flags and other parameters to pass to the program.}\tabularnewline
\end{tabular}




\chapter{Undefined Behavior in OpenSHMEM}\label{sec:undefined}

The \openshmem Specification formalizes the expected behavior of
its library routines.  In cases where routines are improperly used
or the input is not in accordance with the Specification, the behavior
is undefined.

\begin{longtable}{|>{\raggedright}p{0.3\textwidth}|>{\raggedright}p{0.6\textwidth}|}
\hline 
\textbf{Inappropriate Usage} & \textbf{Undefined Behavior}\tabularnewline
\hline 
\endhead
Uninitialized library & If the \openshmem library is not initialized,
calls to non-initializing \openshmem routines have undefined
behavior.  For example, an implementation may try to continue or may abort
immediately upon an \openshmem call into the uninitialized library.
\tabularnewline
\hline
Multiple calls to initialization routines & In an \openshmem program where
the initialization routines \FUNC{shmem\_init} or \FUNC{shmem\_init\_thread}
have already been called, any subsequent calls to these initialization routines
result in undefined behavior.
\tabularnewline
\hline
Accessing non-existent \acp{PE} & If a communications routine accesses a
non-existent \ac{PE}, then the \openshmem library may handle this
situation in an implementation-defined way.  For example, the library may report
an error message saying that the \ac{PE} accessed is outside the range of
accessible \acp{PE}, or may exit without a warning.\tabularnewline
\hline 
Use of non-symmetric variables & Some routines require remotely accessible
variables to perform their function.  For example, a \PUT{} to a non-symmetric variable may
be trapped where possible and the library may abort the program.  Another
implementation may choose to continue execution with or without a warning.
\tabularnewline
\hline 
Non-symmetric allocation of symmetric memory & The symmetric memory management routines are
collectives. For example, all \acp{PE} in the program must call
\FUNC{shmem\_malloc} with the same \VAR{size} argument.  Program behavior after a
mismatched \FUNC{shmem\_malloc} call is undefined.\tabularnewline
\hline 
Use of null pointers with non-zero \VAR{len} specified & In any \openshmem routine
that takes a pointer and \VAR{len} describing the number of elements in that
pointer, a null pointer may not be given unless the corresponding \VAR{len} is also
specified as zero. Otherwise, the resulting behavior is undefined.
The following cases summarize this behavior:
\begin{itemize}
    \item \VAR{len} is 0, pointer is null: supported.
    \item \VAR{len} is not 0, pointer is null: undefined behavior.
    \item \VAR{len} is 0, pointer is non-null: supported.
    \item \VAR{len} is not 0, pointer is non-null: supported.
\end{itemize}
\tabularnewline
\hline 
\end{longtable}




\chapter{Interoperability with other Programming Models}\label{sec:mpi}

\section{\ac{MPI} Interoperability}

\begin{sloppypar} % to prevent constants from running into margins.
%
\openshmem routines may be used in conjunction with \ac{MPI} routines in the
same program.  For example, on \ac{SGI} systems, programs that use both \ac{MPI} and
\openshmem routines call \FUNC{MPI\_Init} and \FUNC{MPI\_Finalize} but omit the
call to the \FUNC{shmem\_init} routine.  \openshmem \ac{PE} numbers are equal to
the \ac{MPI} rank within the \CONST{MPI\_COMM\_WORLD} environment variable.
Note that this indexing precludes use of \openshmem routines between processes in
different \CONST{MPI\_COMM\_WORLD}s.  For example, \ac{MPI} processes started using the
\FUNC{MPI\_Comm\_spawn} routine cannot use \openshmem routines to
communicate with their parent \ac{MPI} processes.
%
\end{sloppypar}
%
On \ac{SGI} systems where \ac{MPI} jobs use \ac{TCP}/sockets for inter-host communication,
\openshmem routines may be used to communicate with processes running on the
same host.  The \FUNC{shmem\_pe\_accessible} routine should be used to determine if
a remote \ac{PE} is accessible via \openshmem communication from the local
\ac{PE}. When running an \ac{MPI} program involving multiple executable files,
\openshmem routines may be used to communicate with processes running from the
same or different executable files, provided that the communication is limited
to symmetric data objects.  On these systems, static memory---such as a
\Fortran common block or \Cstd global variable---is symmetric between
processes running from the same executable file, but is not symmetric between
processes running from different executable files.  Data allocated from the
symmetric heap (e.g., \FUNC{shmem\_malloc}, \FUNC{shpalloc}) is symmetric across the
same or different executable files. The \FUNC{shmem\_addr\_accessible} routine
should be used to determine if a local address is accessible via \openshmem
communication from a remote \ac{PE}.

Another important feature of these systems is that the
\FUNC{shmem\_pe\_accessible} routine returns \CONST{TRUE} only if the remote
\ac{PE} is a process running from the same executable file as the local \ac{PE},
indicating that full \openshmem support (static memory and symmetric heap) is
available.  When using \openshmem routines within an \ac{MPI} program, the use
of \ac{MPI} memory-placement environment variables is required when using
non-default memory-placement options.




\chapter{History of OpenSHMEM}\label{sec:openshmem_history}

SHMEM has a long history as a parallel-programming model and has been
extensively used on a number of products since 1993, including the Cray T3D,
Cray X1E, Cray XT3 and XT4, \ac{SGI} Origin, \ac{SGI} Altix, Quadrics-based
clusters, and InfiniBand-based clusters.

\begin{itemize}
\item SHMEM Timeline
  \begin{itemize}
  \item Cray SHMEM
    \begin{itemize}
    \item SHMEM first introduced by Cray Research, Inc.\ in 1993 for Cray T3D
    \item Cray was acquired by \ac{SGI} in 1996
    \item Cray was acquired by Tera in 2000 (MTA)
    \item Platforms: Cray T3D, T3E, C90, J90, SV1, SV2, X1, X2, XE, XMT, XT
    \end{itemize}
  \item \ac{SGI} SHMEM
    \begin{itemize}
    \item \ac{SGI} acquired Cray Research, Inc.\ and SHMEM was integrated into
      \ac{SGI}'s Message Passing Toolkit (MPT)
    \item \ac{SGI} currently owns the rights to SHMEM and \openshmem
    \item Platforms: Origin, Altix 4700, Altix XE, ICE, UV
    \item \ac{SGI} was acquired by Rackable Systems in 2009
    \item \ac{SGI} and \ac{OSSS} signed a
      SHMEM trademark licensing agreement in 2010
    \item \ac{HPE} acquired {SGI} in 2016
    \end{itemize}
  \end{itemize}
\end{itemize}

A listing of \openshmem implementations can be found on
\url{http://www.openshmem.org/}.








\chapter{OpenSHMEM Specification and Deprecated API}\label{sec:dep_api}

\section{Overview}\label{subsec:dep_overview}
\TableIndex{Deprecated API}
For the \openshmem Specification, deprecation is the process of identifying
API that is supported but no longer recommended for use by users.
The deprecated API \textbf{must} be supported until clearly
indicated as otherwise by the Specification.
This chapter records the API or functionality that have been deprecated, the
version of the \openshmem Specification that effected the deprecation, and the
most recent version of the \openshmem Specification in which the feature was
supported before removal.

\begin{center}
\scriptsize
\begin{longtable}{|l|c|c|l|}
    \hline
    \textbf{Deprecated API}
    & \textbf{Deprecated Since}
    & \textbf{Last Version Supported}
    & \textbf{Replaced By} \\
    \hline
    \endhead
    Header Directory: \hyperref[subsec:dep_rationale:mpp]{\HEADER{mpp}} & 1.1 & Current & (none) \\ \hline
    \CorCpp: \hyperref[subsec:start_pes]{\FuncRef{start\_pes}} & 1.2 & Current & \hyperref[subsec:shmem_init]{\FUNC{shmem\_init}} \\ \hline
    \Fortran: \hyperref[subsec:start_pes]{\FuncRef{START\_PES}} & 1.2 & Current & \hyperref[subsec:shmem_init]{\FUNC{SHMEM\_INIT}} \\ \hline
    \hyperref[subsec:start_pes]{Implicit finalization} & 1.2 & Current & \hyperref[subsec:shmem_finalize]{\FUNC{shmem\_finalize}} \\ \hline
    \CorCpp: \FuncRef{\_my\_pe} & 1.2 & Current & \hyperref[subsec:shmem_my_pe]{\FUNC{shmem\_my\_pe}} \\ \hline
    \CorCpp: \FuncRef{\_num\_pes} & 1.2 & Current & \hyperref[subsec:shmem_n_pes]{\FUNC{shmem\_n\_pes}} \\ \hline
    \Fortran: \FuncRef{MY\_PE} & 1.2 & Current & \hyperref[subsec:shmem_my_pe]{\FUNC{SHMEM\_MY\_PE}} \\ \hline
    \Fortran: \FuncRef{NUM\_PES} & 1.2 & Current & \hyperref[subsec:shmem_n_pes]{\FUNC{SHMEM\_N\_PES}} \\ \hline
    \CorCpp: \FuncRef{shmalloc} & 1.2 & Current & \hyperref[subsec:shfree]{\FUNC{shmem\_malloc}} \\ \hline
    \CorCpp: \FuncRef{shfree} & 1.2 & Current & \hyperref[subsec:shfree]{\FUNC{shmem\_free}} \\ \hline
    \CorCpp: \FuncRef{shrealloc} & 1.2 & Current & \hyperref[subsec:shfree]{\FUNC{shmem\_realloc}} \\ \hline
    \CorCpp: \FuncRef{shmemalign} & 1.2 & Current & \hyperref[subsec:shfree]{\FUNC{shmem\_align}} \\ \hline
    \Fortran: \FuncRef{SHMEM\_PUT} & 1.2 & Current & \hyperref[subsec:shmem_put]{\FUNC{SHMEM\_PUT8} or \FUNC{SHMEM\_PUT64}} \\ \hline
    \minitab{\CorCpp: \hyperref[subsec:shmem_cache]{\FuncRef{shmem\_clear\_cache\_inv}}
        \\ \Fortran: \hyperref[subsec:shmem_cache]{\FuncRef{SHMEM\_CLEAR\_CACHE\_INV}}}
        & 1.3 & Current & (none) \\ \hline
    \CorCpp: \hyperref[subsec:shmem_cache]{\FuncRef{shmem\_clear\_cache\_line\_inv}} & 1.3 & Current & (none) \\ \hline
    \minitab{\CorCpp: \hyperref[subsec:shmem_cache]{\FuncRef{shmem\_set\_cache\_inv}}
        \\ \Fortran: \hyperref[subsec:shmem_cache]{\FuncRef{SHMEM\_SET\_CACHE\_INV}}}
        & 1.3 & Current & (none) \\ \hline
    \minitab{\CorCpp: \hyperref[subsec:shmem_cache]{\FuncRef{shmem\_set\_cache\_line\_inv}}
        \\ \Fortran: \hyperref[subsec:shmem_cache]{\FuncRef{SHMEM\_SET\_CACHE\_LINE\_INV}}}
        & 1.3 & Current & (none) \\ \hline
    \minitab{\CorCpp: \hyperref[subsec:shmem_cache]{\FuncRef{shmem\_udcflush}}
        \\ \Fortran: \hyperref[subsec:shmem_cache]{\FuncRef{SHMEM\_UDCFLUSH}}}
        & 1.3 & Current & (none) \\ \hline
    \minitab{\CorCpp: \hyperref[subsec:shmem_cache]{\FuncRef{shmem\_udcflush\_line}}
        \\ \Fortran: \hyperref[subsec:shmem_cache]{\FuncRef{SHMEM\_UDCFLUSH\_LINE}}}
        & 1.3 & Current & (none) \\ \hline
    \LibConstRef{\_SHMEM\_SYNC\_VALUE}         & 1.3 & Current & \hyperref[subsec:library_constants]{\CONST{SHMEM\_SYNC\_VALUE}} \\ \hline
    \LibConstRef{\_SHMEM\_BARRIER\_SYNC\_SIZE} & 1.3 & Current & \hyperref[subsec:library_constants]{\CONST{SHMEM\_BARRIER\_SYNC\_SIZE}} \\ \hline
    \LibConstRef{\_SHMEM\_BCAST\_SYNC\_SIZE}   & 1.3 & Current & \hyperref[subsec:library_constants]{\CONST{SHMEM\_BCAST\_SYNC\_SIZE}} \\ \hline
    \LibConstRef{\_SHMEM\_COLLECT\_SYNC\_SIZE} & 1.3 & Current & \hyperref[subsec:library_constants]{\CONST{SHMEM\_COLLECT\_SYNC\_SIZE}} \\ \hline
    \LibConstRef{\_SHMEM\_REDUCE\_SYNC\_SIZE}  & 1.3 & Current & \hyperref[subsec:library_constants]{\CONST{SHMEM\_REDUCE\_SYNC\_SIZE}} \\ \hline
    \LibConstRef{\_SHMEM\_REDUCE\_MIN\_WRKDATA\_SIZE} & 1.3 & Current & \hyperref[subsec:library_constants]{\CONST{SHMEM\_REDUCE\_MIN\_WRKDATA\_SIZE}} \\ \hline
    \LibConstRef{\_SHMEM\_MAJOR\_VERSION} & 1.3 & Current & \hyperref[subsec:library_constants]{\CONST{SHMEM\_MAJOR\_VERSION}} \\ \hline
    \LibConstRef{\_SHMEM\_MINOR\_VERSION} & 1.3 & Current & \hyperref[subsec:library_constants]{\CONST{SHMEM\_MINOR\_VERSION}} \\ \hline
    \LibConstRef{\_SHMEM\_MAX\_NAME\_LEN} & 1.3 & Current & \hyperref[subsec:library_constants]{\CONST{SHMEM\_MAX\_NAME\_LEN}} \\ \hline
    \LibConstRef{\_SHMEM\_VENDOR\_STRING} & 1.3 & Current & \hyperref[subsec:library_constants]{\CONST{SHMEM\_VENDOR\_STRING}} \\ \hline
    \LibConstRef{\_SHMEM\_CMP\_EQ} & 1.3 & Current & \hyperref[subsec:library_constants]{\CONST{SHMEM\_CMP\_EQ}} \\ \hline
    \LibConstRef{\_SHMEM\_CMP\_NE} & 1.3 & Current & \hyperref[subsec:library_constants]{\CONST{SHMEM\_CMP\_NE}} \\ \hline
    \LibConstRef{\_SHMEM\_CMP\_LT} & 1.3 & Current & \hyperref[subsec:library_constants]{\CONST{SHMEM\_CMP\_LT}} \\ \hline
    \LibConstRef{\_SHMEM\_CMP\_LE} & 1.3 & Current & \hyperref[subsec:library_constants]{\CONST{SHMEM\_CMP\_LE}} \\ \hline
    \LibConstRef{\_SHMEM\_CMP\_GT} & 1.3 & Current & \hyperref[subsec:library_constants]{\CONST{SHMEM\_CMP\_GT}} \\ \hline
    \LibConstRef{\_SHMEM\_CMP\_GE} & 1.3 & Current & \hyperref[subsec:library_constants]{\CONST{SHMEM\_CMP\_GE}} \\ \hline
    \EnvVarRef{SMA\_VERSION}         & 1.4 & Current & \hyperref[subsec:environment_variables]{\ENVVAR{SHMEM\_VERSION}} \\ \hline
    \EnvVarRef{SMA\_INFO}            & 1.4 & Current & \hyperref[subsec:environment_variables]{\ENVVAR{SHMEM\_INFO}} \\ \hline
    \EnvVarRef{SMA\_SYMMETRIC\_SIZE} & 1.4 & Current & \hyperref[subsec:environment_variables]{\ENVVAR{SHMEM\_SYMMETRIC\_SIZE}} \\ \hline
    \EnvVarRef{SMA\_DEBUG}           & 1.4 & Current & \hyperref[subsec:environment_variables]{\ENVVAR{SHMEM\_DEBUG}} \\ \hline
    \minitab{\CorCpp: \FuncRef{shmem\_wait}
        \\ \CorCpp: \FuncRef{shmem\_\FuncParam{TYPENAME}\_wait}}
        & 1.4 & Current & See \textbf{Notes} for \hyperref[subsec:shmem_wait_until]{\FUNC{shmem\_wait\_until}} \\ \hline
    \CorCpp: \FuncRef{shmem\_wait\_until} & 1.4 & Current
        & \Cstd[11]: \hyperref[subsec:shmem_wait_until]{\FUNC{shmem\_wait\_until}}, \CorCpp: \hyperref[subsec:shmem_wait_until]{\FUNC{shmem\_long\_wait\_until}} \\ \hline
    \minitab{\Cstd[11]: \FuncRef{shmem\_fetch}
        \\ \CorCpp: \FuncRef{shmem\_\FuncParam{TYPENAME}\_fetch}}
        & 1.4 & Current & \hyperref[subsec:shmem_atomic_fetch]{\FUNC{shmem\_atomic\_fetch}} \\ \hline
    \minitab{\Cstd[11]: \FuncRef{shmem\_set}
        \\ \CorCpp: \FuncRef{shmem\_\FuncParam{TYPENAME}\_set}}
        & 1.4 & Current & \hyperref[subsec:shmem_atomic_set]{\FUNC{shmem\_atomic\_set}} \\ \hline
    \minitab{\Cstd[11]: \FuncRef{shmem\_cswap}
        \\ \CorCpp: \FuncRef{shmem\_\FuncParam{TYPENAME}\_cswap}}
        & 1.4 & Current & \hyperref[subsec:shmem_atomic_compare_swap]{\FUNC{shmem\_atomic\_compare\_swap}} \\ \hline
    \minitab{\Cstd[11]: \FuncRef{shmem\_swap}
        \\ \CorCpp: \FuncRef{shmem\_\FuncParam{TYPENAME}\_swap}}
        & 1.4 & Current & \hyperref[subsec:shmem_atomic_swap]{\FUNC{shmem\_atomic\_swap}} \\ \hline
    \minitab{\Cstd[11]: \FuncRef{shmem\_finc}
        \\ \CorCpp: \FuncRef{shmem\_\FuncParam{TYPENAME}\_finc}}
        & 1.4 & Current & \hyperref[subsec:shmem_atomic_fetch_inc]{\FUNC{shmem\_atomic\_fetch\_inc}} \\ \hline
    \minitab{\Cstd[11]: \FuncRef{shmem\_inc}
        \\ \CorCpp: \FuncRef{shmem\_\FuncParam{TYPENAME}\_inc}}
        & 1.4 & Current & \hyperref[subsec:shmem_atomic_inc]{\FUNC{shmem\_atomic\_inc}} \\ \hline
    \minitab{\Cstd[11]: \FuncRef{shmem\_fadd}
        \\ \CorCpp: \FuncRef{shmem\_\FuncParam{TYPENAME}\_fadd}}
        & 1.4 & Current & \hyperref[subsec:shmem_atomic_fetch_add]{\FUNC{shmem\_atomic\_fetch\_add}} \\ \hline
    \minitab{\Cstd[11]: \FuncRef{shmem\_add}
        \\ \CorCpp: \FuncRef{shmem\_\FuncParam{TYPENAME}\_add}}
        & 1.4 & Current & \hyperref[subsec:shmem_atomic_add]{\FUNC{shmem\_atomic\_add}} \\ \hline
    Entire \Fortran API & 1.4 & Current & (none) \\ \hline
    \end{longtable}
\end{center}

\section{Deprecation Rationale}\label{subsec:dep_rationale}

\subsection{Header Directory: \HEADER{mpp}}
\label{subsec:dep_rationale:mpp}
In addition to the default system header paths, \openshmem implementations
must provide all \openshmem-specified header files from the \HEADER{mpp}
header directory such that these headers can be referenced in \CorCpp as
\begin{lstlisting}[language=]
#include <mpp/shmem.h>
#include <mpp/shmemx.h>
\end{lstlisting}
and in \Fortran as
\begin{lstlisting}[language=]
include 'mpp/shmem.fh'
include 'mpp/shmemx.fh'
\end{lstlisting}
for backwards compatibility with \ac{SGI} SHMEM.

\subsection{\CorCpp: \FUNC{start\_pes}}
The \CorCpp routine \FUNC{start\_pes} includes an unnecessary initialization
argument that is remnant of historical \emph{SHMEM} implementations and no
longer reflects the requirements of modern \openshmem implementations.
Furthermore, the naming of \FUNC{start\_pes} does not include the standardized
\shmemprefixLC{} naming prefix. This routine has been deprecated and
\openshmem users are encouraged to use \FUNC{shmem\_init} instead.

\subsection{Implicit Finalization}
Implicit finalization was deprecated and replaced with explicit finalization using the
\FUNC{shmem\_finalize} routine.  Explicit finalization improves portability and
also improves interoperability with profiling and debugging tools.

\subsection{\CorCpp: \FUNC{\_my\_pe}, \FUNC{\_num\_pes}, \FUNC{shmalloc},
    \FUNC{shfree}, \FUNC{shrealloc}, \FUNC{shmemalign}}
The \CorCpp routines \FUNC{\_my\_pe}, \FUNC{\_num\_pes}, \FUNC{shmalloc},
\FUNC{shfree}, \FUNC{shrealloc}, and \FUNC{shmemalign} were deprecated in order
to normalize the \openshmem \ac{API} to use \shmemprefixLC{} as the standard
prefix for all routines.

\subsection{\textit{Fortran}: \FUNC{START\_PES}, \FUNC{MY\_PE}, \FUNC{NUM\_PES}} %% WARNING: Issue #66.
The \Fortran routines \FUNC{START\_PES}, \FUNC{MY\_PE}, and \FUNC{NUM\_PES}
were deprecated in order to minimize the API differences from the deprecation
of \CorCpp routines \FUNC{start\_pes}, \FUNC{\_my\_pe}, and \FUNC{\_num\_pes}.

\subsection{\textit{Fortran}: \FUNC{SHMEM\_PUT}} %% WARNING: Issue #66.
The \Fortran routine \FUNC{SHMEM\_PUT} is defined only for the \Fortran
\ac{API} and is semantically identical to \Fortran routines
\FUNC{SHMEM\_PUT8} and \FUNC{SHMEM\_PUT64}.  Since \FUNC{SHMEM\_PUT8} and
\FUNC{SHMEM\_PUT64} have defined equivalents in the \CorCpp interface,
\FUNC{SHMEM\_PUT} is ambiguous and has been deprecated.

\subsection{SHMEM\_CACHE}
The \FUNC{SHMEM\_CACHE} \ac{API}
\begin{center}
\begin{tabular}{ll}
    \CorCpp: & \Fortran: \\
    \FUNC{shmem\_clear\_cache\_inv}     & \FUNC{SHMEM\_CLEAR\_CACHE\_INV} \\
    \FUNC{shmem\_set\_cache\_inv}       & \FUNC{SHMEM\_SET\_CACHE\_INV} \\
    \FUNC{shmem\_set\_cache\_line\_inv} & \FUNC{SHMEM\_SET\_CACHE\_LINE\_INV} \\
    \FUNC{shmem\_udcflush}              & \FUNC{SHMEM\_UDCFLUSH} \\
    \FUNC{shmem\_udcflush\_line}        & \FUNC{SHMEM\_UDCFLUSH\_LINE} \\
    \FUNC{shmem\_clear\_cache\_line\_inv} \\
\end{tabular}
\end{center}
was originally implemented for systems with cache-management instructions.
This API has largely gone unused on cache-coherent system architectures.
\FUNC{SHMEM\_CACHE} has been deprecated.

\subsection{\CONST{\_SHMEM\_*} Library Constants}
The library constants
\begin{center}
\begin{tabular}{ll}
    \CONST{\_SHMEM\_SYNC\_VALUE}         & \CONST{\_SHMEM\_MAX\_NAME\_LEN} \\
    \CONST{\_SHMEM\_BARRIER\_SYNC\_SIZE} & \CONST{\_SHMEM\_VENDOR\_STRING} \\
    \CONST{\_SHMEM\_BCAST\_SYNC\_SIZE}   & \CONST{\_SHMEM\_CMP\_EQ} \\
    \CONST{\_SHMEM\_COLLECT\_SYNC\_SIZE} & \CONST{\_SHMEM\_CMP\_NE} \\
    \CONST{\_SHMEM\_REDUCE\_SYNC\_SIZE}  & \CONST{\_SHMEM\_CMP\_LT} \\
    \CONST{\_SHMEM\_REDUCE\_MIN\_WRKDATA\_SIZE} & \CONST{\_SHMEM\_CMP\_LE} \\
    \CONST{\_SHMEM\_MAJOR\_VERSION}      & \CONST{\_SHMEM\_CMP\_GT} \\
    \CONST{\_SHMEM\_MINOR\_VERSION}      & \CONST{\_SHMEM\_CMP\_GE} \\
\end{tabular}
\end{center}
do not adhere to the \Cstd standard's reserved identifiers and the \Cpp
standard's reserved names.  These constants were deprecated and replaced
with corresponding constants of prefix \shmemprefix{} that adhere to \CorCpp{}
and \Fortran naming conventions.

\subsection{\ENVVAR{SMA\_*} Environment Variables}\label{subsec:deprecate-sma-env}
The environment variables \ENVVAR{SMA\_VERSION}, \ENVVAR{SMA\_INFO},
\ENVVAR{SMA\_SYMMETRIC\_SIZE}, and \ENVVAR{SMA\_DEBUG}
were deprecated in order to normalize the \openshmem \ac{API} to use
\shmemprefix{} as the standard prefix for all environment variables.

\subsection{\CorCpp: \FUNC{shmem\_wait}}
The \CorCpp interface for \FUNC{shmem\_wait} and \FUNC{shmem\_\FuncParam{TYPENAME}\_wait}
was identified as unintuitive with respect to
the comparison operation it performed.  As \FUNC{shmem\_wait} can be trivially
replaced by \FUNC{shmem\_wait\_until} where \VAR{cmp} is
\CONST{SHMEM\_CMP\_NE}, the \FUNC{shmem\_wait} interface was deprecated in
favor of \FUNC{shmem\_wait\_until}, which makes the comparison operation
explicit and better communicates the developer's intent.

\subsection{\CorCpp: \FUNC{shmem\_wait\_until}}
The \CTYPE{long}-typed \CorCpp routine \FUNC{shmem\_wait\_until} was deprecated
in favor of the \Cstd[11] type-generic interface of the same name or the
explicitly typed \CorCpp routine \FUNC{shmem\_long\_wait\_until}.

\subsection{\textit{C11} and \CorCpp: \FUNC{shmem\_fetch}, \FUNC{shmem\_set}, %% Issue #66.
    \FUNC{shmem\_cswap}, \FUNC{shmem\_swap}, \FUNC{shmem\_finc},
    \FUNC{shmem\_inc}, \FUNC{shmem\_fadd}, \FUNC{shmem\_add}}
The \Cstd[11] and \CorCpp interfaces for
\begin{center}
\begin{tabular}{ll}
    \Cstd[11]: & \CorCpp: \\
    \FUNC{shmem\_fetch} & \FUNC{shmem\_\FuncParam{TYPENAME}\_fetch} \\
    \FUNC{shmem\_set}   & \FUNC{shmem\_\FuncParam{TYPENAME}\_set}   \\
    \FUNC{shmem\_cswap} & \FUNC{shmem\_\FuncParam{TYPENAME}\_cswap} \\
    \FUNC{shmem\_swap}  & \FUNC{shmem\_\FuncParam{TYPENAME}\_swap}  \\
    \FUNC{shmem\_finc}  & \FUNC{shmem\_\FuncParam{TYPENAME}\_finc}  \\
    \FUNC{shmem\_inc}   & \FUNC{shmem\_\FuncParam{TYPENAME}\_inc}   \\
    \FUNC{shmem\_fadd}  & \FUNC{shmem\_\FuncParam{TYPENAME}\_fadd}  \\
    \FUNC{shmem\_add}   & \FUNC{shmem\_\FuncParam{TYPENAME}\_add}   \\
\end{tabular}
\end{center}
were deprecated and replaced with
similarly named interfaces within the \FUNC{shmem\_atomic\_*} namespace
in order to more clearly identify these calls as performing atomic operations.
In addition, the abbreviated names ``cswap'', ``finc'', and ``fadd'' were
expanded for clarity to ``compare\_swap'', ``fetch\_inc'', and ``fetch\_add''.

\subsection{\textit{Fortran} API}\label{subsec:deprecate-fortran} %% WARNING: Issue #66.
The entire \openshmem \Fortran API was deprecated because of a general lack of
use and a lack of conformance with legacy \Fortran standards. In lieu of an
extensive update of the \Fortran API, \Fortran users are encouraged to
leverage the \openshmem Specification's \Cstd API through the
\Fortran--\Cstd interoperability initially standardized by \Fortran[2003]%
\footnote{Formally, \Fortran[2003] is known as ISO/IEC~1539-1:2004(E).}.





\chapter{Changes to this Document}\label{sec:changelog}

\section{Version 1.5}
Major changes in \openshmem[1.5] include \dots

The following list describes the specific changes in \openshmem[1.5]:
\begin{itemize}
%
\item This item is a template for changelist entries and should be deleted
    before this document is published.
    \\See Annex~\ref{sec:changelog}.
\end{itemize}

\section{Version 1.4}
Major changes in \openshmem[1.4] include
multithreading support,
\emph{contexts} for communication management,
\FUNC{shmem\_sync},
\FUNC{shmem\_calloc},
expanded type support,
a new namespace for atomic operations,
atomic bitwise operations,
\FUNC{shmem\_test} for nonblocking point-to-point synchronization,
and \Cstd[11] type-generic interfaces for point-to-point synchronization.

The following list describes the specific changes in \openshmem[1.4]:
\begin{itemize}
%
\item New communication management API, including \FUNC{shmem\_ctx\_create};
    \FUNC{shmem\_ctx\_destroy}; and additional RMA, AMO, and memory ordering
    routines that accept \CTYPE{shmem\_ctx\_t} arguments.
\\See Section \ref{sec:ctx}.
%
\item New API \FUNC{shmem\_sync\_all} and \FUNC{shmem\_sync} to provide \ac{PE}
    synchronization without completing pending communication operations.
    \\See Sections \ref{subsec:shmem_sync_all} and \ref{subsec:shmem_sync}.
%
\item Clarified that the \openshmem extensions header files are required, even when empty.
\\See Section~\ref{subsec:bindings}.
%
\item Clarified that the \FUNC{SHMEM\_GET64} and \FUNC{SHMEM\_GET64\_NBI}
    routines are included in the \Fortran language bindings.\\
    See Sections \ref{subsec:shmem_get} and \ref{subsec:shmem_get_nbi}.
%
\item Clarified that \FUNC{shmem\_init} must be matched with a call to
    \FUNC{shmem\_finalize}.
\\See Sections \ref{subsec:shmem_init} and \ref{subsec:shmem_finalize}.
%
\item Added the \CONST{SHMEM\_SYNC\_SIZE} constant.
\\See Section \ref{subsec:library_constants}.
%
\item Added type-generic interfaces for \FUNC{shmem\_wait\_until}.
\\ See Section \ref{subsec:shmem_wait_until}.
%
\item Removed the \VAR{volatile} qualifiers from the \VAR{ivar} arguments to
\FUNC{shmem\_wait} routines and the \VAR{lock} arguments in the lock API.
\emph{Rationale: Volatile qualifiers were added to several API routines in
\openshmem[1.3]; however, they were later found to be unnecessary.}
\\ See Sections \ref{subsec:shmem_wait_until} and \ref{subsec:shmem_lock}.
%
\item Deprecated the \VAR{SMA\_}* environment variables and added equivalent
\VAR{SHMEM\_}* environment variables.
\\ See Section \ref{subsec:environment_variables}.
%
\item Added the \Cstd[11] \CTYPE{\_Noreturn} function specifier to
\FUNC{shmem\_global\_exit}.
\\ See Section \ref{subsec:shmem_global_exit}.
%
\item Clarified ordering semantics of memory ordering, point-to-point synchronization, and collective 
synchronization routines.
%
\item Clarified deprecation overview and added deprecation rationale in Annex F.
\\See Section \ref{sec:dep_api}.
%
\item Deprecated header directory \HEADER{mpp}.
\\See Section \ref{sec:dep_api}.
%
\item Deprecated the \FUNC{shmem\_wait} functions and the \CTYPE{long}-typed \CorCpp \FUNC{shmem\_wait\_until} function.
\\ See Section \ref{subsec:p2p_intro}.
%
\item Added the \FUNC{shmem\_test} functions.
\\ See Section \ref{subsec:p2p_intro}.
%
\item Added the \FUNC{shmem\_calloc} function.
\\ See Section \ref{subsec:shmem_calloc}.
%
\item Introduced the thread safe semantics that define the interaction between
    \openshmem routines and user threads.
\\See Section \ref{subsec:thread_support}.
%
\item Added the new routine \FUNC{shmem\_init\_thread} to initialize the
    \openshmem library with one of the defined thread levels.
\\See Section \ref{subsec:shmem_init_thread}.
%
\item Added the new routine \FUNC{shmem\_query\_thread} to query the thread
    level provided by the \openshmem implementation.
\\See Section \ref{subsec:shmem_query_thread}.
%
\item Clarified the semantics of \FUNC{shmem\_quiet} for a multithreaded
    \openshmem \ac{PE}.
\\See Section \ref{subsec:shmem_quiet}
%
\item Revised the description of \FUNC{shmem\_barrier\_all} for a multithreaded
    \openshmem \ac{PE}.
\\See Section \ref{subsec:shmem_barrier_all}
%
\item Revised the description of \FUNC{shmem\_wait} for a multithreaded
    \openshmem \ac{PE}.
\\See Section \ref{subsec:shmem_wait_until}
%
\item Clarified description for \CONST{SHMEM\_VENDOR\_STRING}.
\\See Section \ref{subsec:library_constants}.
%
\item Clarified description for \CONST{SHMEM\_MAX\_NAME\_LEN}.
\\See Section \ref{subsec:library_constants}.
%
\item Clarified API description for \FUNC{shmem\_info\_get\_name}.
\\See Section \ref{subsec:shmem_info_get_name}.
%
\item Expanded the type support for RMA, AMO, and point-to-point
    synchronization operations.
\\ See Tables \ref{stdrmatypes}, \ref{stdamotypes}, \ref{extamotypes}, and
    \ref{p2psynctypes}
%
\item Renamed AMO operations to use \FUNC{shmem\_atomic\_*} prefix and
      deprecated old AMO routines.
\\ See Section \ref{sec:amo}.
%
\item Added fetching and non-fetching bitwise AND, OR, and XOR atomic
      operations.
\\ See Section \ref{sec:amo}.
%
\item Deprecated the entire \Fortran API.
%
\item Replaced the \CTYPE{complex} macro in complex-typed reductions with the
      \Cstd[99] (and later) type specifier \CTYPE{\_Complex} to remove an
      implicit dependence on \HEADER{complex.h}.
\\ See Section \ref{subsec:shmem_reductions}.
%
\item Clarified that complex-typed reductions in C are optionally supported.
\\ See Section \ref{subsec:shmem_reductions}.
%
\end{itemize}




\section{Version 1.3}
Major changes in \openshmem[1.3] include the addition of
nonblocking \ac{RMA} operations,
atomic \PUT{} and \GET{} operations,
all-to-all collectives,
and \Cstd[11] type-generic interfaces for \ac{RMA} and \ac{AMO} operations.

The following list describes the specific changes in \openshmem[1.3]:
\begin{itemize}
%
\item Clarified implementation of \acp{PE} as threads.
%
\item Added \CTYPE{const} to every read-only pointer argument.
%
\item Clarified definition of \OPR{Fence}.
\\See Section \ref{subsec:programming_model}.
%
\item Clarified implementation of symmetric memory allocation.
\\See Section \ref{subsec:memory_model}.
%
\item Restricted atomic operation guarantees to other atomic operations with the same datatype.
\\See Section \ref{subsec:amo_guarantees}.
%
\item Deprecation of all constants that start with \CONST{\_SHMEM\_*}.
\\See Section \ref{subsec:library_constants}.
%
\item Added a type-generic interface to \openshmem \ac{RMA} and \ac{AMO}
	operations based on \Cstd[11] Generics.
\\See Sections \ref{sec:rma}, \ref{sec:rma_nbi} and \ref{sec:amo}.
%
\item New nonblocking variants of remote memory access, \FUNC{SHMEM\_PUT\_NBI}
	and \FUNC{SHMEM\_GET\_NBI}.
\\See Sections \ref{subsec:shmem_put_nbi} and \ref{subsec:shmem_get_nbi}.
%
\item New atomic elemental read and write operations, \FUNC{SHMEM\_FETCH} and
	\FUNC{SHMEM\_SET}.
\\See Sections \ref{subsec:shmem_atomic_fetch} and \ref{subsec:shmem_atomic_set}
%
\item New alltoall data exchange operations, \FUNC{SHMEM\_ALLTOALL} 
	and \FUNC{SHMEM\_ALLTOALLS}.
\\See Sections \ref{subsec:shmem_alltoall} and \ref{subsec:shmem_alltoalls}.
%
\item Added \CTYPE{volatile} to remotely accessible pointer argument in
	\FUNC{SHMEM\_WAIT} and \FUNC{SHMEM\_LOCK}.
\\See Sections \ref{subsec:shmem_wait_until} and \ref{subsec:shmem_lock}.
%
\item Deprecation of \FUNC{SHMEM\_CACHE}.
\\See Section \ref{subsec:shmem_cache}.
%
\end{itemize}




\section{Version 1.2}
Major changes in \openshmem[1.2] include
a new initialization routine (\FUNC{shmem\_init}),
improvements to the execution model with an explicit
library-finalization routine (\FUNC{shmem\_finalize}),
an early-exit routine (\FUNC{shmem\_global\_exit}),
namespace standardization,
and clarifications to several API descriptions.

The following list describes the specific changes in \openshmem[1.2]:
\begin{itemize}
%
\item Added specification of \VAR{pSync} initialization for all routines that use it.
%
\item Replaced all placeholder variable names \VAR{target} with \VAR{dest} to
      avoid confusion with \Fortran's \KEYWORD{target} keyword.
%
\item New Execution Model for exiting/finishing \openshmem programs.
\\See Section  \ref{subsec:execution_model}.
%
\item New library constants to support API that query version and name information.
\\See Section \ref{subsec:library_constants}.
%
\item New API \FUNC{shmem\_init} to provide mechanism to start an \openshmem
      program and replace deprecated \FUNC{start\_pes}.
\\See Section \ref{subsec:shmem_init}.
%
\item Deprecation of \FUNC{\_my\_pe} and \FUNC{\_num\_pes} routines.
\\See Sections \ref{subsec:shmem_my_pe} and \ref{subsec:shmem_n_pes}.
%
\item New API \FUNC{shmem\_finalize} to provide collective mechanism to cleanly
      exit an \openshmem program and release resources.
\\See Section \ref{subsec:shmem_finalize}.
%
\item New API \FUNC{shmem\_global\_exit} to provide mechanism to exit an
    \openshmem program.
\\See Section \ref{subsec:shmem_global_exit}.
%
\item Clarification related to the address of the referenced object in
    \FUNC{shmem\_ptr}.
\\See Section \ref{subsec:shmem_ptr}.
%
\item New API to query the version and name information. 
\\See Section \ref{subsec:shmem_info_get_version} and \ref{subsec:shmem_info_get_name}.
%
\item \openshmem library API normalization. All \Cstd symmetric memory management
      API begins with  \FUNC{shmem\_}.
\\See Section \ref{subsec:shfree}.
%
\item Notes and clarifications added to \FUNC{shmem\_malloc}.
\\See Section \ref{subsec:shfree}.
%
\item Deprecation of \Fortran API routine \FUNC{SHMEM\_PUT}.
\\See Section \ref{subsec:shmem_put}. 
%
\item Clarification related to \FUNC{shmem\_wait}.
\\See Section \ref{subsec:shmem_wait_until}.
%
\item Undefined behavior for null pointers without zero counts added.
\\See Annex \ref{sec:undefined}
%
\item Addition of new Annex for clearly specifying deprecated API and its
      support across versions of the \openshmem Specification.
\\See Annex \ref{sec:dep_api}.
%
\end{itemize}




\section{Version 1.1}
Major changes from \openshmem[1.0] to \openshmem[1.1] include
the introduction of the \HEADER{shmemx.h} header file for non-standard API
extensions,
clarifications to completion semantics and API descriptions in agreement with
the \ac{SGI} SHMEM specification,
and general readabilty and usability improvements to the document structure.

The following list describes the specific changes in \openshmem[1.1]:
\begin{itemize}
%
\item Clarifications of the completion semantics of memory synchronization 
      interfaces.
\\See Section \ref{subsec:memory_order}.
%
\item Clarification of the completion semantics of memory load and store
      operations in context of \FUNC{shmem\_barrier\_all} and \FUNC{shmem\_barrier}
      routines.
\\See Section \ref{subsec:shmem_barrier_all} and \ref{subsec:shmem_barrier}.
%
\item Clarification of the completion and ordering semantics of
      \FUNC{shmem\_quiet} and \FUNC{shmem\_fence}.
\\See Section \ref{subsec:shmem_quiet} and \ref{subsec:shmem_fence}.
%
\item Clarifications of the completion semantics of \ac{RMA} and \ac{AMO}
      routines.
\\See Sections \ref{sec:rma} and \ref{sec:amo}
%
\item Clarifications of the memory model and the memory alignment requirements
      for symmetric data objects.
\\See Section \ref{subsec:memory_model}.
%
\item Clarification of the execution model and the definition of a \ac{PE}.
\\See Section \ref{subsec:execution_model}
%
\item Clarifications of the semantics of \FUNC{shmem\_pe\_accessible} and
      \FUNC{shmem\_addr\_accessible}.
\\See Section \ref{subsec:shmem_pe_accessible} and \ref{subsec:shmem_addr_accessible}.
%
\item Added an annex on interoperability with \ac{MPI}.
\\See Annex \ref{sec:mpi}.
%
\item Added examples to the different interfaces.
%
\item Clarification of the naming conventions for constant in \Cstd and
      \Fortran.
\\See Section \ref{subsec:library_constants} and \ref{subsec:shmem_wait_until}.
%
\item Added \ac{API} calls: \FUNC{shmem\_char\_p}, \FUNC{shmem\_char\_g}.
\\See Sections \ref{subsec:shmem_p} and \ref{subsec:shmem_g}. 
%
\item Removed \ac{API} calls: \FUNC{shmem\_char\_put},
      \FUNC{shmem\_char\_get}.
\\See Sections \ref{subsec:shmem_put} and \ref{subsec:shmem_get}. 
%
\item The usage of \CTYPE{ptrdiff\_t}, \CTYPE{size\_t}, and \CTYPE{int} in the
      interface signature was made consistent with the description.
\\See Sections \ref{subsec:coll}, \ref{subsec:shmem_iput}, and \ref{subsec:shmem_iget}.
%
\item Revised \FUNC{shmem\_barrier} example.
\\See Section \ref{subsec:shmem_barrier}. 
%
\item Clarification of the initial value of \VAR{pSync} work arrays for
\FUNC{shmem\_barrier}.\\ See Section \ref{subsec:shmem_barrier}. 
%
\item Clarification of the expected behavior when multiple \FUNC{start\_pes}
calls are encountered.
\\See Section \ref{subsec:start_pes}.
%
\item Corrected the definition of atomic increment operation.
\\See Section \ref{subsec:shmem_atomic_inc}.
%
\item Clarification of the size of the symmetric heap and when it is set.
\\See Section \ref{subsec:shfree}.
%
\item Clarification of the integer and real sizes for \Fortran \ac{API}.
\\See Sections \ref{subsec:shmem_atomic_add},
      \ref{subsec:shmem_atomic_compare_swap},
      \ref{subsec:shmem_atomic_swap},
      \ref{subsec:shmem_atomic_fetch_inc},
      \ref{subsec:shmem_atomic_inc}, and
      \ref{subsec:shmem_atomic_fetch_add}.
%
\item Clarification of the expected behavior on program \OPR{exit}.
\\See Section \ref{subsec:execution_model}, Execution Model. 
%
\item More detailed description for the progress of \openshmem operations
provided.
\\See Section \ref{subsec:progress}. 
%
\item Clarification of naming convention for non-standard interfaces and their
inclusion in \HEADER{shmemx.h}.
\\See Section \ref{subsec:bindings}. 
%
\item Various fixes to \openshmem code examples across the Specification to
include appropriate header files. 
%
\item Removing requirement that implementations should detect size mismatch and
return error information for \FUNC{shmalloc} and ensuring consistent
language.
\\See Sections \ref{subsec:shfree} and Annex \ref{sec:undefined}. 
%
\item \Fortran programming fixes for examples.\\ See Sections
\ref{subsec:shmem_reductions} and \ref{subsec:shmem_wait_until}. 
%
\item Clarifications of the reuse \VAR{pSync} and \VAR{pWork} across
collectives.
\\See Sections \ref{subsec:coll}, \ref{subsec:shmem_broadcast},
      \ref{subsec:shmem_collect} and \ref{subsec:shmem_reductions}.
%
\item Name changes for UV and ICE for \ac{SGI} systems.
\\See Annex \ref{sec:openshmem_history}. 
%
\end{itemize}

} %end of setlength command that was started in frontmatter.tex


\clearpage
\phantomsection
\addcontentsline{toc}{chapter}{Index}
\printindex

\end{document}

